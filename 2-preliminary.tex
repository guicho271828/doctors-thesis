\section{Preliminaries and Definitions}

\label{sec:preliminaries}

We first define some notation and the terminology used throughout the
rest of the paper.
$h(n)$ denotes the estimate of the cost from the current node $n$ to the nearest goal node.
$g(n)$ is the current shortest known path cost from the initial node to the current node.
$f(n)=g(n)+h(n)$ is the estimate of the resulting cost of the path to a goal
containing the current node.
We omit the argument $(n)$ unless necessary.

Below, we first present a very basic graph search algorithm that generalizes \astar, Dijkstra \cite{dijkstra1959note}, GBFS and others.
It is built upon two sets, OPEN and CLOSED, where unexpanded nodes are stored in OPEN and expanded nodes are stored in CLOSED. When operations pop$(S)$, push$(n,S)$, remove$(n,S)$ is defined for a node $n$ and a set $S$, it is defined as follows:

\begin{algorithm}                      
\begin{algorithmic}
 \REQUIRE $s_0$, $\text{is\_goal}(\cdot)$, $\text{successors}(\cdot)$ %, strategy
 \STATE Initialize $\text{OPEN}=\emptyset$, $\text{CLOSE}=\emptyset$, $g(s_0)=0$, $\forall s\not=s_0; g(s)=\infty$
 % \STATE \textbf{Local procedure} $\text{open}(s,t,g_{new})$: $g(t) \leftarrow g_{new}$; $\text{parent}(t) \leftarrow s$; $\text{push}(t,\text{OPEN})$
 \STATE $\text{push}(s_0,\text{OPEN})$
 \WHILE{OPEN $\not = \emptyset$}
 \STATE $s = \text{pop}(\text{OPEN}); \text{push}(s,\text{CLOSE})$
 \RETURN $s$ \  \textbf{if} \  $\text{is\_goal}(s)=\text{true}$
 \FORALL{$t \in \text{successors}(s)$}
 \STATE $g_{new} = g(s) + \text{cost}(s,t)$
 \IF{$g_{new} < g(t)$}
 % \STATE $\text{open}(s,t,g_{new})$
 % \STATE $\text{remove}(t,\text{CLOSE})$ 
 % \STATE $\text{CLOSE} \leftarrow \text{CLOSE} \setminus \braces{t}$ 
 \STATE $g(t) \leftarrow g_{new}$;\; $\text{parent}(t) \leftarrow s$;\; $\text{push}(t,\text{OPEN})$;\; $\text{remove}(t,\text{CLOSE})$
 \ENDIF
 \ENDFOR
 \ENDWHILE
\end{algorithmic}
\caption{Graph Search Algorithm using OPEN/CLOSED list}
\label{alg:ocl}
\end{algorithm}

\emph{Best First Search} (BFS) is a subclass of this algorithm where OPEN is sorted according to a \emph{sorting strategy} and pop$(S)$ operation tries to select a single node from the OPEN list following the strategy.
Each sorting strategy is denoted as a vector of several \emph{sorting criteria}, such as
[$\text{criterion}_1$, $\text{criterion}_2$, $\ldots$,
$\text{criterion}_k$], which defines a lexicographic ordering ---
i.e., from the OPEN list, first, select a
set of nodes using $\text{criterion}_1$, and if there are still multiple
nodes remaining in the set, then break ties using $\text{criterion}_2$
and so on, until a single node is selected.  The \emph{first-level
sorting criterion} of a strategy is $\text{criterion}_1$, the
\emph{second-level sorting criterion} is $\text{criterion}_2$, and so on.
%% the word frontier is no longer used in the later text.
% \emph{final frontier} is the set of open nodes with $f^*$.
Note that this corresponds to the command line option format of Fast
Downward \cite{Helmert2006}.

Using this notation, \astar without any tie-breaking strategy can be
denoted as a BFS with $[f]$, and \astar which breaks ties according to $h$
value is denoted as $[f,h]$.
% Similarly, GBFS is denoted as $[h]$.
Unless stated otherwise, we assume the nodes are sorted in the
increasing order of the key value, and a BFS always selects the smallest
key value.

However, a sorting strategy may only provide a partial ordering ---
i.e., the sorting strategy may fail to select a single node because some nodes
may share the same sorting keys.
For such cases, a BFS algorithm must
decide which node to expand by a \emph{default} tie-breaking
criterion $\text{criterion}_k$ such as  \fifo (oldest node first: first-in-first-out), \lifo
(most recently inserted first: last-in-first-out) or \ro (random ordering).
For example, an \astar using $h$ tie-breaking and \fifo default tie-breaking
 is denoted as $[f,h,\fifo]$.
By definition, there is only 1 node which satisfies the default criterion, so
strategies with a default criterion guarantee the total order and
are able to select a single node from the set of nodes.
When the default criterion is irrelevant to the discussion,
we use a wildcard * e.g. $[f,h,*]$ or sometimes omit it.

Given a search algorithm with a sorting strategy, 
a $\plateau{\text{criterion}\ldots}$ is a set of nodes in OPEN whose elements share
the same sort keys according to non-default sorting criteria and therefore
are indistinguishable. In a case of \astar
using tie-breaking with $h$ (sorting strategy $[f,h,*]$), the plateaus are denoted as
$\plateau{f,h}$, the set of nodes with the same $f$ cost and the same $h$ cost.
We can also refer to a specific plateau with $f=f_p$ and $h=h_p$ by $\plateau{f_p,h_p}$.

An \emph{entrance} to a $\plateau{\text{criterion}\ldots}=P$ is
a node $n \in P$, whose current parent is not in
$P$. The \emph{final plateau} is the plateau
containing the solution found by the search algorithm.  In \astar using
admissible heuristics, the final plateau is $\plateau{f^*}$ (without
tie-breaking), or $\plateau{f^*,0}$ (with $h$-based tie-breaking).
