\section{Preliminaries and Definitions}

\label{sec:preliminaries}

We first define some notation and the terminology used throughout the
rest of the paper.
$h(n)$ denotes the estimate of the cost from the current node $n$ to the nearest goal node.
$g(n)$ is the current shortest path cost from the initial node to the
current node.
$f(n)=g(n)+h(n)$ is the estimate of the resulting cost of the path to a goal
containing the current node.
We omit the argument $(n)$ unless necessary.

A \emph{sorting strategy} for a best first search algorithm 
which tries to select a single node from the OPEN list.
Each sorting strategy is denoted as a vector of several \emph{sorting criteria}, such as
[$\text{criterion}_1$, $\text{criterion}_2$, $\ldots$,
$\text{criterion}_k$], which means: From the OPEN list, first, select a
set of nodes using $\text{criterion}_1$.  If there are still multiple
nodes remaining in the set, then break ties using $\text{criterion}_2$
and so on, until a single node is selected.  The \emph{first-level
sorting criterion} of a strategy is $\text{criterion}_1$, the
\emph{second-level sorting criterion} is $\text{criterion}_2$, and so on.
%% the word frontier is no longer used in the later text.
% \emph{final frontier} is the set of open nodes with $f^*$.
Note that this corresponds to the command line option format of Fast
Downward \cite{Helmert2006}.

Using this notation, \astar without any tiebreaking strategy can be
denoted as a Best-First Search (BFS) with $[f]$, and \astar which breaks ties according to $h$
value is denoted as $[f,h]$. Similarly, GBFS is denoted as 
$[h]$.  Unless stated otherwise, we assume the nodes are sorted in the
increasing order of the key value, and a BFS always selects the smallest
key value.

The sorting strategy may fail to select a single node because some nodes
may share the same sorting keys. In such cases, a search algorithm must
decide which node to expand by a \emph{default} tiebreaking
strategy $\text{criterion}_k$ such as  \fifo (first-in-first-out), \lifo
(last-in-first-out) or \ro (random ordering).  
By definition, there is only 1 node which satisfies these criteria, so these
strategies are able to select a single node from the set of
nodes. Although these default strategies may not be a ``sorting''
criterion, {\bf [XXX - why?]}
 we also include them in the vector notation. For
example, an \astar using \fifo tiebreaking is denoted as $[f,h,\fifo]$.


Given a search algorithm which uses a particular sorting strategy, 
a \emph{plateau} is a set of nodes in OPEN whose elements are
indistinguishable according to the sorting strategy. In a case of \astar
using tiebreaking with $h$ ($[f,h]$), this is the set of nodes with the
same $f$ cost and the same $h$ cost.
A plateau whose nodes have $f=f_p$ and $h=h_p$ is denoted as $\plateau{f_p,h_p}$.

An \emph{entrance} to a $\plateau{f_p,h_p}$ is a node $n \in
\plateau{f_p,h_p}$, whose current parent is not a member of
$\plateau{f_p,h_p}$.  The \emph{final plateau}, is the plateau
containing the solution found by the search algorithm.  In \astar using
admissible heuristics, the final plateau is $\plateau{f^*}$ (without
tiebreaking), or $\plateau{f^*,0}$ (with $h$-based tiebreaking).
