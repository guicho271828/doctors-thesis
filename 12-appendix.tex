

\chapter{Appendix: Detailed Data}
\label{sec:appendix}

This Appendix contains some detailed figures and data which are referenced from the text in the previous sections. 

% \setlength{\floatsep}{1mm}
% \setlength{\textfloatsep}{1mm}
% \setlength{\abovecaptionskip}{1mm}
% \setlength{\belowcaptionskip}{1mm}
% \setlength{\abovedisplayskip}{1mm}
% \setlength{\belowdisplayskip}{1mm}
% \setlength{\arraycolsep}{1mm}

\section{Detailed Data for \reftbl{tbl:summary-std}}

\begin{table}[htbp]
 {
 \relsize{-0.5}
 \centering
 \begin{center}
\begin{tabular}{|r|*{2}{ccc|}}
Domain & $[f,\fifo]$ & $[f,\lifo]$ & $[f,\ro]$ & $[f,h,\fifo]$ & $[f,h,\lifo]$ & $[f,h,\ro]$\\
IPC benchmark (1104) & 443 & 558 & 448.9 $\pm$ 1.3 & 558 & \textbf{565} & 558.9 $\pm$ 2.1\\
airport(50) & 18 & 26 & 18 $\pm$ 0 & \textbf{27} & 26 & 25.7 $\pm$ 0.5\\
barman-opt11(20) & 0 & 0 & 0 $\pm$ 0 & 0 & 0 & 0 $\pm$ 0\\
blocks(35) & 26 & 26 & 26 $\pm$ 0 & \textbf{28} & \textbf{28} & \textbf{28} $\pm$ 0\\
\textbf{cybersec(19)} & 0 & 3 & 0 $\pm$ 0 & 2 & 3 & \textbf{3.9} $\pm$ 1.1\\
depot(22) & 5 & 5 & 5 $\pm$ 0 & 6 & 6 & 6 $\pm$ 0\\
driverlog(20) & 12 & 13 & 12 $\pm$ 0 & 13 & 13 & 13 $\pm$ 0\\
elevators-opt11(20) & 14 & 15 & 14 $\pm$ 0 & 15 & 15 & 15 $\pm$ 0\\
floortile-opt11(20) & 6 & 6 & 6 $\pm$ 0 & 6 & 6 & 6 $\pm$ 0\\
freecell(80) & 8 & 9 & 8.7 $\pm$ 0.5 & 9 & 9 & 9 $\pm$ 0\\
grid(5) & 1 & 1 & 1 $\pm$ 0 & 1 & 1 & 1 $\pm$ 0\\
gripper(20) & 6 & 6 & 6 $\pm$ 0 & 6 & 6 & 6 $\pm$ 0\\
hanoi(30) & 12 & 12 & 12 $\pm$ 0 & 12 & 12 & 12 $\pm$ 0\\
logistics00(28) & 16 & 18 & 16 $\pm$ 0 & \textbf{20} & \textbf{20} & \textbf{20} $\pm$ 0\\
miconic(150) & 68 & 140 & 68 $\pm$ 0 & 140 & 140 & 140 $\pm$ 0\\
mprime(35) & 20 & \textbf{22} & 19.9 $\pm$ 0.3 & 21 & 21 & 20.9 $\pm$ 0.3\\
mystery(30) & 15 & 16 & 15 $\pm$ 0 & 16 & 16 & 15.2 $\pm$ 0.4\\
nomystery-opt11(20) & 12 & 13 & 12 $\pm$ 0 & \textbf{14} & \textbf{14} & \textbf{14} $\pm$ 0\\
\textbf{openstacks-opt11(20)} & 11 & \textbf{18} & 11.2 $\pm$ 0.4 & 11 & \textbf{18} & 11.7 $\pm$ 0.5\\
parcprinter-opt11(20) & 12 & 13 & 12 $\pm$ 0 & 13 & 13 & 13 $\pm$ 0\\
parking-opt11(20) & 1 & 1 & 1 $\pm$ 0 & 1 & 1 & 1 $\pm$ 0\\
pathways(30) & 4 & 5 & 4 $\pm$ 0 & 5 & 5 & 5 $\pm$ 0\\
pegsol-opt11(20) & 17 & 17 & 17 $\pm$ 0 & 17 & 17 & 17 $\pm$ 0\\
pipesworld-notankage(50) & 13 & 13 & 13 $\pm$ 0 & 14 & 14 & 14.6 $\pm$ 0.5\\
pipesworld-tankage(50) & 7 & 8 & 8 $\pm$ 0 & 8 & 8 & 8 $\pm$ 0\\
psr-small(50) & 48 & 48 & 48 $\pm$ 0 & 48 & 48 & 48 $\pm$ 0\\
rovers(40) & 7 & 7 & 7 $\pm$ 0 & 7 & 7 & 7 $\pm$ 0\\
scanalyzer-opt11(20) & 4 & \textbf{10} & 5.4 $\pm$ 0.7 & \textbf{10} & \textbf{10} & \textbf{10} $\pm$ 0\\
sokoban-opt11(20) & 19 & 19 & 19 $\pm$ 0 & 19 & 19 & 19 $\pm$ 0\\
storage(30) & 14 & 14 & 14 $\pm$ 0 & 14 & 14 & 14 $\pm$ 0\\
tidybot-opt11(20) & 11 & 12 & 11 $\pm$ 0 & 12 & 12 & 12 $\pm$ 0\\
tpp(30) & 6 & 6 & 6 $\pm$ 0 & 6 & 6 & 6 $\pm$ 0\\
transport-opt11(20) & 6 & 6 & 6 $\pm$ 0 & 6 & 6 & 6 $\pm$ 0\\
visitall-opt11(20) & 9 & 10 & 9.4 $\pm$ 0.5 & 10 & 10 & 10 $\pm$ 0\\
woodworking-opt11(20) & 6 & 9 & 8.2 $\pm$ 0.4 & \textbf{10} & \textbf{10} & \textbf{10} $\pm$ 0\\
zenotravel(20) & 9 & \textbf{11} & 9 $\pm$ 0 & \textbf{11} & \textbf{11} & \textbf{11} $\pm$ 0\\
\end{tabular}
\end{center}

 \caption{
 Coverage comparison (the number of instances solved in 5min, 4GB, LMcut
 heuristics) among
 the standard baseline tie-breaking algorithms. We highlight the
 best results when the difference between the maximum and the minimum coverage exceeds 2.
 }
 \label{tbl:lmcut-ipc-std}
 }
\end{table}

\begin{table}[htbp]
 {
 \relsize{-0.5}
 \centering
 \begin{center}
\begin{tabular}{|r|*{2}{ccc|}}
Domain & $[f,\fifo]$ & $[f,\lifo]$ & $[f,\ro]$ & $[f,h,\fifo]$ & $[f,h,\lifo]$ & $[f,h,\ro]$\\
IPC benchmark (1104) & 460 & 490 & 460.9 $\pm$ 1.6 & 491 & \textbf{496} & 489.4 $\pm$ 1.0\\
airport(50) & 9 & 9 & 9 $\pm$ 0 & 9 & 9 & 9 $\pm$ 0\\
barman-opt11(20) & 4 & 4 & 4 $\pm$ 0 & 4 & 4 & 4 $\pm$ 0\\
blocks(35) & 21 & 22 & 21 $\pm$ 0 & 22 & 22 & 22 $\pm$ 0\\
\textbf{cybersec(19)} & 0 & 0 & 0 $\pm$ 0 & 0 & 0 & 0 $\pm$ 0\\
depot(22) & 5 & 6 & 5 $\pm$ 0 & 6 & 6 & 5 $\pm$ 0\\
driverlog(20) & 12 & 12 & 12 $\pm$ 0 & 12 & 12 & 12 $\pm$ 0\\
elevators-opt11(20) & 13 & 13 & 13 $\pm$ 0 & 13 & 13 & 13 $\pm$ 0\\
floortile-opt11(20) & 5 & 6 & 5 $\pm$ 0 & 6 & 6 & 6 $\pm$ 0\\
freecell(80) & 15 & 16 & 15 $\pm$ 0 & \textbf{17} & \textbf{17} & 16 $\pm$ 0\\
grid(5) & 2 & 2 & 2 $\pm$ 0 & 2 & 2 & 2 $\pm$ 0\\
gripper(20) & 8 & \textbf{20} & 8 $\pm$ 0 & \textbf{20} & \textbf{20} & \textbf{20} $\pm$ 0\\
hanoi(30) & 14 & 14 & 14 $\pm$ 0 & 14 & 14 & 14 $\pm$ 0\\
logistics00(28) & 20 & 20 & 20 $\pm$ 0 & 20 & 20 & 20 $\pm$ 0\\
miconic(150) & 68 & \textbf{73} & 68.3 $\pm$ 0.7 & \textbf{73} & \textbf{73} & \textbf{73.2} $\pm$ 0.4\\
mprime(35) & 23 & 23 & 22 $\pm$ 0 & 23 & \textbf{24} & 23.7 $\pm$ 0.5\\
mystery(30) & 15 & 15 & 15 $\pm$ 0 & 15 & 16 & 15 $\pm$ 0\\
nomystery-opt11(20) & 17 & 18 & 17.8 $\pm$ 0.4 & 18 & 18 & 18 $\pm$ 0\\
\textbf{openstacks-opt11(20)} & 15 & \textbf{19} & 15.4 $\pm$ 0.5 & 15 & \textbf{19} & 15.4 $\pm$ 0.5\\
parcprinter-opt11(20) & 10 & 10 & 10 $\pm$ 0 & 10 & 10 & 10 $\pm$ 0\\
parking-opt11(20) & 1 & 1 & 1 $\pm$ 0 & 1 & 1 & 1 $\pm$ 0\\
pathways(30) & 4 & 4 & 4 $\pm$ 0 & 4 & 4 & 4 $\pm$ 0\\
pegsol-opt11(20) & 17 & \textbf{19} & 17.2 $\pm$ 0.4 & \textbf{19} & \textbf{19} & \textbf{19} $\pm$ 0\\
pipesworld-notankage(50) & 9 & 9 & 8.9 $\pm$ 0.3 & 10 & 10 & 9.9 $\pm$ 0.3\\
pipesworld-tankage(50) & 13 & 13 & 13.1 $\pm$ 0.3 & 13 & 13 & 13.2 $\pm$ 0.4\\
psr-small(50) & 50 & 50 & 50 $\pm$ 0 & 50 & 50 & 50 $\pm$ 0\\
rovers(40) & 6 & \textbf{8} & 6.1 $\pm$ 0.3 & \textbf{8} & \textbf{8} & \textbf{8} $\pm$ 0\\
scanalyzer-opt11(20) & 10 & 10 & 10 $\pm$ 0 & 10 & 10 & 10 $\pm$ 0\\
sokoban-opt11(20) & 20 & 20 & 20 $\pm$ 0 & 20 & 20 & 20 $\pm$ 0\\
storage(30) & 15 & 15 & 15 $\pm$ 0 & 15 & 15 & 15 $\pm$ 0\\
tidybot-opt11(20) & 0 & 0 & 0 $\pm$ 0 & 0 & 0 & 0 $\pm$ 0\\
tpp(30) & 6 & 6 & 6 $\pm$ 0 & 7 & 6 & 6 $\pm$ 0\\
transport-opt11(20) & 7 & 7 & 7 $\pm$ 0 & 7 & 7 & 7 $\pm$ 0\\
visitall-opt11(20) & 9 & 9 & 9 $\pm$ 0 & 9 & 9 & 9 $\pm$ 0\\
woodworking-opt11(20) & 7 & 7 & 7 $\pm$ 0 & 7 & 7 & 7 $\pm$ 0\\
zenotravel(20) & 10 & 10 & 10 $\pm$ 0 & \textbf{12} & \textbf{12} & \textbf{12} $\pm$ 0\\
\end{tabular}
\end{center}

 \caption{
 Coverage comparison (the number of instances solved in 5min, 4GB, M\&S heuristics) among
 the standard baseline tie-breaking algorithms. We highlight the
 best results when the difference between the maximum and the minimum coverage exceeds 2.
 }
 \label{tbl:mands-ipc-std}
 }
\end{table}

% % This comparison can be removed even from the appendix.
% % The result is not essential. Furthermore,
% % these numbers can be extracted from the other tables.
%
% \clearpage
% \section{\mands Data for \reftbl{tbl:lmcut-zerocost-std}}
% 
% \begin{table}[htbp]
%  \centering
%  \begin{center}
\begin{tabular}{|lc|ccr|}
 & $[f,h,\fifo]$ & $[f,h,\fifo]$ & (difference) & \\
depot(22) & 6 & 5 & (-1) & depot-fuel(22)\\
driverlog(20) & 12 & 9 & (-3) & driverlog-fuel(20)\\
elevators-opt11(20) & 13 & 8 & (-5) & elevators-up(20)\\
floortile-opt11(20) & 6 & 8 & (+2) & floortile-ink(20)\\
grid(5) & 2 & 2 &  & grid-fuel(5)\\
gripper(20) & 20 & 20 &  & gripper-move(20)\\
logistics00(28) & 20 & 16 & (-4) & logistics00-fuel(28)\\
mprime(35) & 23 & 21 & (-2) & mprime-succumb(35)\\
nomystery-opt11(20) & 18 & 16 & (-2) & nomystery-fuel(20)\\
parking-opt11(20) & 1 & 0 & (-1) & parking-movecc(20)\\
pathways(30) & 4 & 4 &  & pathways-fuel(30)\\
rovers(40) & 8 & 8 &  & rovers-fuel(40)\\
scanalyzer-opt11(20) & 10 & 11 & (+1) & scanalyzer-analyze(20)\\
sokoban-opt11(20) & 20 & 19 & (-1) & sokoban-pushgoal(20)\\
storage(30) & 15 & 4 & (-11) & storage-lift(20)\\
tidybot-opt11(20) & 0 & 0 &  & tidybot-motion(20)\\
tpp(30) & 7 & 9 & (+2) & tpp-fuel(30)\\
woodworking-opt11(20) & 7 & 7 &  & woodworking-cut(20)\\
zenotravel(20) & 12 & 10 & (-2) & zenotravel-fuel(20)\\
\end{tabular}
\end{center}

%  \caption{
%  Coverage comparison (the number of instances solved) 
%  between the original IPC instances and the modified Zerocost instances,
%  using the same configuration and experimental setting (5min, 4GB, \mands heuristics).
%  This table does not include domains where the total number of instances
%  differ in the Zerocost domain and the original IPC domain. The results in
%  those domains are available in the later sections.
%  }
%  \label{tbl:mands-zerocost-std}
% \end{table}

\clearpage
\section{Detailed Data for \reftbl{tbl:depth-summary}}

\begin{table}[htbp]
 {
 \centering
 \begin{center}
\begin{tabular}{lrrrrrrrrrrrrrrrrrrrrrrrrrrrrrr}
zerocost & cov & rprt & blck & dpt- & drvr & lvtr & flrt & frcl & gd-p & grd- & grpp & hkng & lgst & mcnc & mprm & myst & nmys & prkn & pthw & ppsn & ppsw & psr- & rvrs & scnl & skbn & strg & tdyb & tpp- & wdwr & zntr\\
lm\_$_{\text{F1184}}$ & 227 & 7 & 15 & 4 & 7 & 7 & 8 & 4 & 15 & 1 & 7 & 8 & 15 & 10 & 12 & 5 & 9 & 0 & 4 & 6 & 2 & 19 & 7 & 3 & 18 & 4 & 14 & 7 & 2 & 7\\
lm\_$_{\text{L13955}}$ & 296 & 15 & 17 & 6 & 8 & 13 & 8 & 19 & 15 & 1 & 7 & 9 & 16 & 17 & 14 & 5 & 10 & 0 & 5 & 7 & 4 & 19 & 9 & 9 & 18 & 4 & 16 & 11 & 7 & 7\\
lm\_$_{\text{R15177}}$ & 237 & 7 & 15 & 5 & 7 & 7 & 8 & 5 & 15 & 1 & 7 & 8 & 15 & 10 & 10 & 7 & 9 & 0 & 4 & 8 & 3 & 19 & 7 & 3 & 18 & 4 & 14 & 8 & 6 & 7\\
lm\_$_{\text{R15793}}$ & 238 & 7 & 15 & 5 & 7 & 7 & 8 & 5 & 15 & 1 & 7 & 8 & 15 & 10 & 12 & 5 & 9 & 0 & 4 & 8 & 3 & 19 & 7 & 3 & 18 & 4 & 15 & 8 & 6 & 7\\
lm\_$_{\text{R18410}}$ & 240 & 7 & 15 & 6 & 7 & 7 & 8 & 5 & 15 & 1 & 7 & 8 & 15 & 10 & 12 & 7 & 9 & 0 & 4 & 9 & 2 & 19 & 7 & 3 & 18 & 4 & 15 & 8 & 5 & 7\\
lm$_{\text{iF1184}}$ & 284 & 10 & 17 & 6 & 8 & 7 & 8 & 17 & 13 & 1 & 7 & 9 & 15 & 19 & 21 & 6 & 9 & 0 & 4 & 8 & 4 & 19 & 8 & 6 & 18 & 5 & 15 & 10 & 7 & 7\\
lm$_{\text{iL13955}}$ & 276 & 13 & 18 & 6 & 8 & 9 & 8 & 10 & 15 & 1 & 7 & 9 & 16 & 18 & 14 & 7 & 10 & 0 & 5 & 6 & 3 & 19 & 9 & 5 & 18 & 5 & 15 & 10 & 5 & 7\\
lm$_{\text{iR15177}}$ & 298 & 10 & 18 & 6 & 8 & 9 & 8 & 17 & 15 & 1 & 7 & 9 & 15 & 20 & 19 & 7 & 10 & 0 & 4 & 10 & 6 & 19 & 9 & 5 & 18 & 5 & 16 & 11 & 9 & 7\\
lm$_{\text{iR15793}}$ & 292 & 10 & 18 & 6 & 8 & 9 & 8 & 16 & 15 & 1 & 7 & 9 & 15 & 20 & 19 & 6 & 9 & 0 & 5 & 10 & 5 & 19 & 9 & 5 & 17 & 5 & 16 & 11 & 7 & 7\\
lm$_{\text{iR18410}}$ & 294 & 11 & 17 & 6 & 8 & 8 & 8 & 17 & 14 & 1 & 7 & 9 & 15 & 18 & 21 & 6 & 9 & 0 & 4 & 10 & 5 & 19 & 9 & 5 & 18 & 5 & 16 & 11 & 10 & 7\\
lmh$_{\text{F1184}}$ & 271 & 15 & 17 & 6 & 8 & 7 & 8 & 4 & 15 & 1 & 7 & 9 & 16 & 16 & 15 & 7 & 10 & 0 & 5 & 8 & 3 & 19 & 8 & 9 & 18 & 4 & 16 & 8 & 5 & 7\\
lmh$_{\text{L13955}}$ & 294 & 13 & 17 & 6 & 8 & 13 & 8 & 19 & 15 & 1 & 7 & 9 & 16 & 17 & 14 & 5 & 10 & 0 & 5 & 8 & 4 & 19 & 8 & 9 & 18 & 4 & 16 & 11 & 7 & 7\\
lmh$_{\text{R15177}}$ & 276 & 13 & 17 & 6 & 8 & 7 & 9 & 5 & 15 & 1 & 7 & 9 & 16 & 16 & 17 & 7 & 10 & 0 & 5 & 8 & 4 & 19 & 8 & 9 & 18 & 4 & 16 & 8 & 7 & 7\\
lmh$_{\text{R15793}}$ & 278 & 14 & 17 & 6 & 8 & 7 & 8 & 5 & 15 & 1 & 7 & 9 & 16 & 17 & 17 & 8 & 10 & 0 & 5 & 8 & 4 & 19 & 8 & 9 & 18 & 4 & 16 & 8 & 7 & 7\\
lmh$_{\text{R18410}}$ & 276 & 14 & 17 & 6 & 8 & 7 & 8 & 5 & 15 & 1 & 7 & 9 & 16 & 17 & 16 & 8 & 10 & 0 & 4 & 9 & 3 & 19 & 8 & 9 & 18 & 4 & 16 & 8 & 7 & 7\\
lmhiF1184 & 299 & 14 & 17 & 6 & 8 & 7 & 8 & 17 & 15 & 1 & 7 & 9 & 16 & 19 & 22 & 6 & 10 & 0 & 5 & 8 & 3 & 19 & 8 & 9 & 18 & 5 & 16 & 11 & 8 & 7\\
lmhiL13955 & 279 & 13 & 17 & 6 & 8 & 9 & 8 & 10 & 15 & 1 & 7 & 9 & 16 & 18 & 14 & 5 & 10 & 0 & 5 & 8 & 3 & 19 & 8 & 10 & 18 & 4 & 16 & 10 & 5 & 7\\
lmhiR15177 & 305 & 14 & 17 & 6 & 8 & 9 & 9 & 17 & 15 & 1 & 7 & 9 & 16 & 20 & 20 & 6 & 10 & 0 & 5 & 10 & 5 & 19 & 8 & 10 & 18 & 4 & 16 & 11 & 8 & 7\\
lmhiR15793 & 304 & 14 & 17 & 6 & 8 & 9 & 8 & 16 & 15 & 1 & 7 & 9 & 15 & 21 & 20 & 9 & 10 & 0 & 4 & 10 & 5 & 19 & 8 & 9 & 18 & 4 & 16 & 11 & 8 & 7\\
lmhiR18410 & 302 & 14 & 17 & 6 & 8 & 8 & 8 & 16 & 15 & 1 & 7 & 9 & 15 & 20 & 21 & 7 & 10 & 0 & 4 & 10 & 5 & 19 & 8 & 9 & 18 & 4 & 16 & 11 & 9 & 7\\
\end{tabular}
\end{center}

 % 
  \caption{ Coverage comparison (the number of instances solved in 5min, 4GB, \lmcut heuristics) on {620
 Zerocost instances}.  We highlight the best results when the difference between the best and the worst coverages
 is greater than 2.  }
 % 
 \label{tbl:lmcut-zerocost-full}}
\end{table}

\begin{table}[htbp]
 {
 \centering
 \begin{center}
\begin{tabular}{|c|cccHHH|cccHHH|cccHHH|cccHHH|}
zerocost & mn\_$_{\text{F27958}}$ & mn\_$_{\text{L28267}}$ & mn\_$_{\text{R10848}}$ & mn\_$_{\text{R2894}}$ & mn\_$_{\text{R7102}}$ & mn$_{\text{iF27958}}$ & mn$_{\text{iL28267}}$ & mn$_{\text{iR10848}}$ & mn$_{\text{iR2894}}$ & mn$_{\text{iR7102}}$ & mnh$_{\text{F27958}}$ & mnh$_{\text{L28267}}$ & mnh$_{\text{R10848}}$ & mnh$_{\text{R2894}}$ & mnh$_{\text{R7102}}$ & mnhiF27958 & mnhiL28267 & mnhiR10848 & mnhiR2894 & mnhiR7102\\
cov & 250 & 315 & 269 & 271 & 270 & 310 & 289 & 317 & 314 & 317 & 295 & 316 & 304 & 304 & 304 & 317 & 303 & 326 & 322 & 326\\
rprt & 5 & 5 & 5 & 5 & 5 & 5 & 5 & 5 & 5 & 5 & 5 & 5 & 5 & 5 & 5 & 5 & 5 & 5 & 5 & 5\\
blck & 20 & 20 & 20 & 20 & 20 & 20 & 20 & 20 & 20 & 20 & 20 & 20 & 20 & 20 & 20 & 20 & 20 & 20 & 20 & 20\\
dpt- & 5 & 5 & 6 & 6 & 6 & 6 & 5 & 6 & 6 & 6 & 5 & 5 & 6 & 6 & 6 & 6 & 5 & 6 & 6 & 6\\
drvr & 8 & 9 & 8 & 8 & 8 & 9 & 9 & 9 & 9 & 9 & 9 & 9 & 9 & 9 & 9 & 9 & 9 & 9 & 9 & 9\\
lvtr & 8 & 14 & 8 & 8 & 9 & 9 & 13 & 12 & 12 & 10 & 8 & 14 & 8 & 8 & 9 & 9 & 13 & 12 & 12 & 10\\
flrt & 8 & 8 & 8 & 8 & 8 & 7 & 8 & 7 & 7 & 8 & 8 & 8 & 8 & 8 & 8 & 7 & 7 & 7 & 7 & 6\\
frcl & 5 & 17 & 8 & 7 & 6 & 17 & 15 & 17 & 17 & 17 & 5 & 17 & 8 & 7 & 6 & 17 & 15 & 17 & 17 & 17\\
gd-p & 15 & 15 & 15 & 15 & 15 & 15 & 15 & 15 & 15 & 15 & 15 & 15 & 15 & 15 & 15 & 15 & 15 & 15 & 15 & 15\\
grd- & 2 & 2 & 2 & 2 & 2 & 2 & 2 & 2 & 2 & 2 & 2 & 2 & 2 & 2 & 2 & 2 & 2 & 2 & 2 & 2\\
grpp & 8 & 20 & 8 & 8 & 8 & 20 & 10 & 18 & 18 & 19 & 20 & 20 & 20 & 20 & 20 & 20 & 20 & 20 & 20 & 20\\
hkng & 12 & 13 & 12 & 12 & 13 & 13 & 12 & 12 & 12 & 12 & 13 & 13 & 13 & 13 & 13 & 13 & 12 & 12 & 13 & 12\\
lgst & 16 & 16 & 16 & 16 & 16 & 16 & 16 & 16 & 16 & 16 & 16 & 16 & 16 & 16 & 16 & 16 & 16 & 16 & 16 & 16\\
mcnc & 19 & 30 & 19 & 20 & 20 & 30 & 30 & 30 & 30 & 30 & 29 & 30 & 30 & 30 & 30 & 30 & 30 & 30 & 30 & 30\\
mprm & 14 & 19 & 15 & 16 & 15 & 24 & 15 & 22 & 20 & 22 & 21 & 19 & 20 & 19 & 20 & 25 & 15 & 24 & 22 & 25\\
myst & 4 & 4 & 6 & 6 & 6 & 4 & 4 & 6 & 6 & 6 & 4 & 4 & 6 & 6 & 6 & 4 & 4 & 6 & 6 & 6\\
nmys & 15 & 16 & 16 & 16 & 16 & 15 & 16 & 16 & 16 & 16 & 16 & 16 & 16 & 16 & 16 & 16 & 16 & 16 & 16 & 16\\
prkn & 0 & 0 & 0 & 0 & 0 & 0 & 0 & 0 & 0 & 0 & 0 & 0 & 0 & 0 & 0 & 0 & 0 & 0 & 0 & 0\\
pthw & 4 & 4 & 4 & 4 & 4 & 4 & 4 & 4 & 4 & 4 & 4 & 4 & 4 & 4 & 4 & 4 & 4 & 4 & 4 & 4\\
ppsn & 3 & 3 & 3 & 4 & 3 & 5 & 3 & 5 & 5 & 5 & 3 & 3 & 3 & 4 & 3 & 5 & 3 & 5 & 5 & 5\\
ppsw & 3 & 9 & 7 & 8 & 8 & 4 & 4 & 9 & 8 & 9 & 5 & 9 & 8 & 8 & 8 & 5 & 6 & 9 & 8 & 10\\
psr- & 19 & 19 & 19 & 19 & 19 & 19 & 19 & 19 & 19 & 19 & 19 & 19 & 19 & 19 & 19 & 19 & 19 & 19 & 19 & 19\\
rvrs & 8 & 8 & 8 & 8 & 8 & 8 & 8 & 8 & 8 & 8 & 8 & 8 & 8 & 8 & 8 & 8 & 8 & 8 & 8 & 8\\
scnl & 9 & 11 & 9 & 9 & 9 & 9 & 9 & 9 & 9 & 8 & 11 & 11 & 11 & 11 & 11 & 11 & 11 & 11 & 11 & 11\\
skbn & 18 & 18 & 19 & 18 & 18 & 18 & 18 & 17 & 17 & 17 & 19 & 19 & 18 & 18 & 19 & 18 & 18 & 18 & 18 & 18\\
strg & 4 & 4 & 4 & 4 & 4 & 4 & 4 & 4 & 4 & 4 & 4 & 4 & 4 & 4 & 4 & 4 & 4 & 4 & 4 & 4\\
tdyb & 0 & 0 & 0 & 0 & 0 & 0 & 0 & 0 & 0 & 0 & 0 & 0 & 0 & 0 & 0 & 0 & 0 & 0 & 0 & 0\\
tpp- & 8 & 10 & 8 & 8 & 8 & 11 & 10 & 11 & 11 & 11 & 9 & 10 & 9 & 10 & 9 & 11 & 10 & 11 & 11 & 11\\
wdwr & 2 & 7 & 7 & 7 & 7 & 7 & 6 & 9 & 9 & 9 & 7 & 7 & 8 & 8 & 8 & 8 & 7 & 10 & 8 & 11\\
zntr & 8 & 9 & 9 & 9 & 9 & 9 & 9 & 9 & 9 & 10 & 10 & 9 & 10 & 10 & 10 & 10 & 9 & 10 & 10 & 10\\
\end{tabular}
\end{center}

  \caption{
 Coverage comparison (the number of instances solved in 5min, 4GB, \mands heuristics)
 on {620 Zerocost instances}. We highlight the
 best results when the difference between the maximum and the minimum coverage exceeds 2.
 }
 \label{tbl:mands-zerocost-full}
 }
\end{table}

\begin{table}[htbp]
 {
 \centering
 \begin{center}
\begin{tabular}{lrrrrrrrrrrrrrrrrrrrr}
benchmark & lm\_$_{\text{F7947}}$ & lm\_$_{\text{L23076}}$ & lm\_$_{\text{R25103}}$ & lm\_$_{\text{R4668}}$ & lm\_$_{\text{R6506}}$ & lm$_{\text{iF7947}}$ & lm$_{\text{iL23076}}$ & lm$_{\text{iR25103}}$ & lm$_{\text{iR4668}}$ & lm$_{\text{iR6506}}$ & lmh$_{\text{F7947}}$ & lmh$_{\text{L23076}}$ & lmh$_{\text{R25103}}$ & lmh$_{\text{R4668}}$ & lmh$_{\text{R6506}}$ & lmhiF7947 & lmhiL23076 & lmhiR25103 & lmhiR4668 & lmhiR6506\\
cov & 443 & 558 & 449 & 451 & 450 & 533 & 549 & 560 & 562 & 563 & 558 & 565 & 561 & 560 & 561 & 571 & 575 & 573 & 571 & 573\\
rprt & 18 & 26 & 18 & 18 & 18 & 21 & 23 & 21 & 22 & 21 & 27 & 26 & 26 & 26 & 26 & 27 & 26 & 26 & 26 & 26\\
brmn & 0 & 0 & 0 & 0 & 0 & 0 & 0 & 0 & 0 & 0 & 0 & 0 & 0 & 0 & 0 & 0 & 0 & 0 & 0 & 0\\
blck & 26 & 26 & 26 & 26 & 26 & 27 & 26 & 26 & 26 & 27 & 28 & 28 & 28 & 28 & 28 & 28 & 28 & 28 & 28 & 28\\
cybr & 0 & 3 & 0 & 0 & 0 & 5 & 12 & 7 & 9 & 8 & 2 & 3 & 4 & 5 & 4 & 8 & 12 & 11 & 9 & 10\\
dpt & 5 & 5 & 5 & 5 & 5 & 6 & 6 & 6 & 6 & 6 & 6 & 6 & 6 & 6 & 6 & 6 & 6 & 6 & 6 & 6\\
drvr & 12 & 13 & 12 & 12 & 12 & 12 & 13 & 12 & 12 & 13 & 13 & 13 & 13 & 13 & 13 & 13 & 13 & 13 & 13 & 13\\
lvtr & 14 & 15 & 14 & 14 & 14 & 14 & 15 & 14 & 15 & 14 & 15 & 15 & 15 & 15 & 15 & 15 & 15 & 15 & 15 & 15\\
flrt & 6 & 6 & 6 & 6 & 6 & 6 & 6 & 6 & 6 & 6 & 6 & 6 & 6 & 6 & 6 & 6 & 6 & 6 & 6 & 6\\
frcl & 8 & 9 & 9 & 9 & 9 & 9 & 9 & 9 & 9 & 9 & 9 & 9 & 9 & 9 & 9 & 9 & 9 & 9 & 9 & 9\\
grd & 1 & 1 & 1 & 1 & 1 & 1 & 1 & 1 & 1 & 1 & 1 & 1 & 1 & 1 & 1 & 1 & 1 & 1 & 1 & 1\\
grpp & 6 & 6 & 6 & 6 & 6 & 6 & 6 & 6 & 6 & 6 & 6 & 6 & 6 & 6 & 6 & 6 & 6 & 6 & 6 & 6\\
hn & 12 & 12 & 12 & 12 & 12 & 12 & 12 & 12 & 12 & 12 & 12 & 12 & 12 & 12 & 12 & 12 & 12 & 12 & 12 & 12\\
lgst & 16 & 18 & 16 & 16 & 16 & 20 & 20 & 20 & 20 & 20 & 20 & 20 & 20 & 20 & 20 & 20 & 20 & 20 & 20 & 20\\
mcnc & 68 & 140 & 68 & 68 & 68 & 125 & 121 & 139 & 139 & 139 & 140 & 140 & 140 & 140 & 140 & 140 & 140 & 140 & 140 & 140\\
mprm & 20 & 22 & 20 & 20 & 20 & 22 & 22 & 21 & 21 & 21 & 21 & 21 & 21 & 21 & 21 & 21 & 21 & 21 & 21 & 21\\
myst & 15 & 16 & 15 & 15 & 15 & 16 & 16 & 16 & 15 & 16 & 16 & 16 & 16 & 15 & 16 & 16 & 16 & 16 & 16 & 16\\
nmys & 12 & 13 & 12 & 12 & 12 & 12 & 13 & 13 & 14 & 13 & 14 & 14 & 14 & 14 & 14 & 14 & 14 & 14 & 14 & 14\\
pnst & 11 & 18 & 11 & 12 & 11 & 17 & 18 & 18 & 18 & 18 & 11 & 18 & 12 & 12 & 12 & 18 & 18 & 18 & 18 & 18\\
prcp & 12 & 13 & 12 & 12 & 12 & 12 & 13 & 13 & 13 & 13 & 13 & 13 & 13 & 13 & 13 & 13 & 13 & 13 & 13 & 13\\
prkn & 1 & 1 & 1 & 1 & 1 & 1 & 1 & 1 & 1 & 1 & 1 & 1 & 1 & 1 & 1 & 1 & 1 & 1 & 1 & 1\\
pthw & 4 & 5 & 4 & 4 & 4 & 5 & 5 & 5 & 5 & 5 & 5 & 5 & 5 & 5 & 5 & 5 & 5 & 5 & 5 & 5\\
pgsl & 17 & 17 & 17 & 17 & 17 & 17 & 17 & 17 & 17 & 17 & 17 & 17 & 17 & 17 & 17 & 17 & 17 & 17 & 17 & 17\\
ppsw & 13 & 13 & 13 & 13 & 13 & 13 & 13 & 14 & 13 & 14 & 14 & 14 & 15 & 14 & 15 & 14 & 15 & 14 & 14 & 15\\
ppsw & 7 & 8 & 8 & 8 & 8 & 8 & 8 & 8 & 8 & 8 & 8 & 8 & 8 & 8 & 8 & 8 & 8 & 8 & 8 & 8\\
psr- & 48 & 48 & 48 & 48 & 48 & 48 & 48 & 48 & 48 & 48 & 48 & 48 & 48 & 48 & 48 & 48 & 48 & 48 & 48 & 48\\
rvrs & 7 & 7 & 7 & 7 & 7 & 7 & 7 & 7 & 7 & 7 & 7 & 7 & 7 & 7 & 7 & 7 & 7 & 7 & 7 & 7\\
scnl & 4 & 10 & 5 & 6 & 6 & 8 & 9 & 9 & 8 & 9 & 10 & 10 & 10 & 10 & 10 & 10 & 10 & 10 & 10 & 10\\
skbn & 19 & 19 & 19 & 19 & 19 & 19 & 19 & 19 & 19 & 19 & 19 & 19 & 19 & 19 & 19 & 19 & 19 & 19 & 19 & 19\\
strg & 14 & 14 & 14 & 14 & 14 & 14 & 14 & 15 & 15 & 15 & 14 & 14 & 14 & 14 & 14 & 14 & 14 & 14 & 14 & 14\\
tdyb & 11 & 12 & 11 & 11 & 11 & 11 & 12 & 12 & 12 & 12 & 12 & 12 & 12 & 12 & 12 & 12 & 12 & 12 & 12 & 12\\
tpp & 6 & 6 & 6 & 6 & 6 & 6 & 6 & 6 & 6 & 6 & 6 & 6 & 6 & 6 & 6 & 6 & 6 & 6 & 6 & 6\\
trns & 6 & 6 & 6 & 6 & 6 & 6 & 6 & 6 & 6 & 6 & 6 & 6 & 6 & 6 & 6 & 6 & 6 & 6 & 6 & 6\\
vstl & 9 & 10 & 10 & 9 & 10 & 10 & 10 & 10 & 10 & 10 & 10 & 10 & 10 & 10 & 10 & 10 & 10 & 10 & 10 & 10\\
wdwr & 6 & 9 & 8 & 9 & 8 & 6 & 11 & 12 & 12 & 12 & 10 & 10 & 10 & 10 & 10 & 10 & 10 & 10 & 10 & 10\\
zntr & 9 & 11 & 9 & 9 & 9 & 11 & 11 & 11 & 11 & 11 & 11 & 11 & 11 & 11 & 11 & 11 & 11 & 11 & 11 & 11\\
\end{tabular}
\end{center}

  \caption{
 Coverage comparison (the number of instances solved in 5min, 4GB, LMcut
 heuristics) on {1104 standard IPC benchmark instances}. We highlight the
 best results when the difference between the maximum and the minimum coverage exceeds 2.
 }
 \label{tbl:lmcut-ipc-full}
 }
\end{table}

\begin{table}[htbp]
 {
 \centering
 \begin{center}
\begin{tabular}{lrrrrrrrrrrrrrrrrrrrr}
benchmark & mn\_$_{\text{F4499}}$ & mn\_$_{\text{L19052}}$ & mn\_$_{\text{R18939}}$ & mn\_$_{\text{R30213}}$ & mn\_$_{\text{R9559}}$ & mn$_{\text{iF4499}}$ & mn$_{\text{iL19052}}$ & mn$_{\text{iR18939}}$ & mn$_{\text{iR30213}}$ & mn$_{\text{iR9559}}$ & mnh$_{\text{F4499}}$ & mnh$_{\text{L19052}}$ & mnh$_{\text{R18939}}$ & mnh$_{\text{R30213}}$ & mnh$_{\text{R9559}}$ & mnhiF4499 & mnhiL19052 & mnhiR18939 & mnhiR30213 & mnhiR9559\\
cov & 460 & 490 & 464 & 462 & 460 & 483 & 484 & 483 & 484 & 483 & 491 & 496 & 491 & 490 & 489 & 487 & 487 & 487 & 484 & 486\\
rprt & 9 & 9 & 9 & 9 & 9 & 9 & 9 & 9 & 9 & 9 & 9 & 9 & 9 & 9 & 9 & 9 & 9 & 9 & 9 & 9\\
brmn & 4 & 4 & 4 & 4 & 4 & 4 & 4 & 4 & 4 & 4 & 4 & 4 & 4 & 4 & 4 & 4 & 4 & 4 & 4 & 4\\
blck & 21 & 22 & 21 & 21 & 21 & 21 & 22 & 21 & 22 & 21 & 22 & 22 & 22 & 22 & 22 & 22 & 21 & 22 & 21 & 22\\
cybr & 0 & 0 & 0 & 0 & 0 & 0 & 0 & 0 & 0 & 0 & 0 & 0 & 0 & 0 & 0 & 0 & 0 & 0 & 0 & 0\\
dpt & 5 & 6 & 5 & 5 & 5 & 5 & 5 & 5 & 5 & 5 & 6 & 6 & 5 & 5 & 5 & 5 & 5 & 5 & 5 & 5\\
drvr & 12 & 12 & 12 & 12 & 12 & 12 & 12 & 12 & 12 & 12 & 12 & 12 & 12 & 12 & 12 & 12 & 12 & 12 & 12 & 12\\
lvtr & 13 & 13 & 13 & 13 & 13 & 11 & 11 & 12 & 12 & 12 & 13 & 13 & 13 & 13 & 13 & 12 & 12 & 12 & 12 & 12\\
flrt & 5 & 6 & 5 & 5 & 5 & 5 & 5 & 5 & 5 & 5 & 6 & 6 & 6 & 6 & 6 & 6 & 6 & 6 & 6 & 6\\
frcl & 15 & 16 & 15 & 15 & 15 & 16 & 16 & 16 & 16 & 16 & 17 & 17 & 16 & 16 & 16 & 16 & 16 & 16 & 16 & 16\\
grd & 2 & 2 & 2 & 2 & 2 & 2 & 2 & 2 & 2 & 2 & 2 & 2 & 2 & 2 & 2 & 2 & 2 & 2 & 2 & 2\\
grpp & 8 & 20 & 8 & 8 & 8 & 20 & 20 & 20 & 20 & 20 & 20 & 20 & 20 & 20 & 20 & 20 & 20 & 20 & 20 & 20\\
hn & 14 & 14 & 14 & 14 & 14 & 14 & 14 & 14 & 14 & 14 & 14 & 14 & 14 & 14 & 14 & 14 & 14 & 14 & 14 & 14\\
lgst & 20 & 20 & 20 & 20 & 20 & 20 & 20 & 20 & 20 & 20 & 20 & 20 & 20 & 20 & 20 & 20 & 20 & 20 & 20 & 20\\
mcnc & 68 & 73 & 70 & 68 & 68 & 73 & 73 & 73 & 72 & 74 & 73 & 73 & 73 & 73 & 74 & 73 & 73 & 73 & 72 & 74\\
mprm & 23 & 23 & 22 & 22 & 23 & 23 & 23 & 23 & 24 & 23 & 23 & 24 & 24 & 23 & 24 & 23 & 24 & 24 & 23 & 24\\
myst & 15 & 15 & 15 & 15 & 15 & 15 & 15 & 15 & 15 & 15 & 15 & 16 & 15 & 15 & 15 & 15 & 16 & 15 & 15 & 15\\
nmys & 17 & 18 & 18 & 18 & 18 & 18 & 18 & 18 & 18 & 18 & 18 & 18 & 18 & 18 & 18 & 18 & 18 & 18 & 18 & 18\\
pnst & 15 & 19 & 16 & 16 & 15 & 19 & 19 & 19 & 19 & 19 & 15 & 19 & 16 & 16 & 15 & 19 & 19 & 19 & 19 & 19\\
prcp & 10 & 10 & 10 & 10 & 10 & 10 & 10 & 10 & 10 & 10 & 10 & 10 & 10 & 10 & 10 & 10 & 10 & 10 & 10 & 10\\
prkn & 1 & 1 & 1 & 1 & 1 & 1 & 1 & 1 & 1 & 1 & 1 & 1 & 1 & 1 & 1 & 1 & 1 & 1 & 1 & 1\\
pthw & 4 & 4 & 4 & 4 & 4 & 4 & 4 & 4 & 4 & 4 & 4 & 4 & 4 & 4 & 4 & 4 & 4 & 4 & 4 & 4\\
pgsl & 17 & 19 & 17 & 18 & 17 & 18 & 19 & 19 & 19 & 19 & 19 & 19 & 19 & 19 & 19 & 19 & 19 & 19 & 19 & 19\\
ppsw & 9 & 9 & 9 & 9 & 8 & 10 & 9 & 9 & 9 & 8 & 10 & 10 & 10 & 10 & 9 & 10 & 9 & 10 & 10 & 9\\
ppsw & 13 & 13 & 14 & 13 & 13 & 13 & 13 & 13 & 13 & 13 & 13 & 13 & 14 & 14 & 13 & 13 & 13 & 13 & 13 & 13\\
psr- & 50 & 50 & 50 & 50 & 50 & 50 & 50 & 50 & 50 & 50 & 50 & 50 & 50 & 50 & 50 & 50 & 50 & 50 & 50 & 50\\
rvrs & 6 & 8 & 6 & 6 & 6 & 8 & 8 & 7 & 7 & 7 & 8 & 8 & 8 & 8 & 8 & 8 & 8 & 7 & 7 & 7\\
scnl & 10 & 10 & 10 & 10 & 10 & 10 & 10 & 10 & 10 & 11 & 10 & 10 & 10 & 10 & 10 & 10 & 10 & 10 & 10 & 10\\
skbn & 20 & 20 & 20 & 20 & 20 & 19 & 19 & 19 & 19 & 18 & 20 & 20 & 20 & 20 & 20 & 19 & 19 & 19 & 19 & 18\\
strg & 15 & 15 & 15 & 15 & 15 & 15 & 15 & 15 & 15 & 15 & 15 & 15 & 15 & 15 & 15 & 15 & 15 & 15 & 15 & 15\\
tdyb & 0 & 0 & 0 & 0 & 0 & 0 & 0 & 0 & 0 & 0 & 0 & 0 & 0 & 0 & 0 & 0 & 0 & 0 & 0 & 0\\
tpp & 6 & 6 & 6 & 6 & 6 & 6 & 6 & 6 & 6 & 6 & 7 & 6 & 6 & 6 & 6 & 6 & 6 & 6 & 6 & 6\\
trns & 7 & 7 & 7 & 7 & 7 & 6 & 6 & 6 & 6 & 6 & 7 & 7 & 7 & 7 & 7 & 6 & 6 & 6 & 6 & 6\\
vstl & 9 & 9 & 9 & 9 & 9 & 9 & 9 & 9 & 9 & 9 & 9 & 9 & 9 & 9 & 9 & 9 & 9 & 9 & 9 & 9\\
wdwr & 7 & 7 & 7 & 7 & 7 & 7 & 7 & 7 & 7 & 7 & 7 & 7 & 7 & 7 & 7 & 7 & 7 & 7 & 7 & 7\\
zntr & 10 & 10 & 10 & 10 & 10 & 10 & 10 & 10 & 10 & 10 & 12 & 12 & 12 & 12 & 12 & 10 & 10 & 10 & 10 & 10\\
\end{tabular}
\end{center}

  \caption{
 Coverage comparison (the number of instances solved in 5min, 4GB, M\&S
 heuristics) on {1104 standard IPC benchmark instances}. We highlight the
 best results when the difference between the maximum and the minimum coverage exceeds 2.
 }
 \label{tbl:mands-ipc-full}
 }
\end{table}

\clearpage
\section{Additional Figures for \refig{fig:expansion-ratio}: $\lifo$ Default Tiebreaking}

\begin{figure}[htbp]
 \centering
 % \begin{tabular}{cccc}
 %  nodes/sec                  & LMcut      & M\&S       & M\&S slowdown\\
 %  \hline
 %  $[f,h,\lifo]$              & 8.86$\times 10^3$ & 1.37$\times 10^5$ & 100\%\\
 %  $[f,h,\depth,\lifo]$ & 9.37$\times 10^3$ & 1.13$\times 10^5$ & 82\%\\
 %  \hline
 %  $[f,h,\fifo]$              & 9.65$\times 10^3$ & 1.41$\times 10^5$ & 100\%\\
 %  $[f,h,\depth,\fifo]$ & 9.62$\times 10^3$ & 1.24$\times 10^5$ & 87\%\\
 %  \hline
 % \end{tabular}
 % \caption{Comparison of the average node expansion ratio (node/sec) between
 % standard tie-breaking and depth-based tie-breaking on \lmcut and \mands
 % heuristics. Numbers are averaged over the problem instances solved by
 % all 4 configurations. Since the node evaluation of \mands is an order of
 % magnitude faster than \lmcut, the overhead of managing depth-based
 % tie-breaking queue is non-negligible on \mands.}
 % \includegraphics{img/node-sec/lmhiF-lmh_F.pdf}
 % % \includegraphics{img/node-sec/lmhiL-lmh_L.pdf}
 % \includegraphics{img/node-sec/mnhiF-mnh_F.pdf}
 % % \includegraphics{img/node-sec/mnhiL-mnh_L.pdf}
 % \includegraphics{img/node-sec/lmhiF-lmh_F-hist.pdf}
 \includegraphics{img/node-sec/lmhiL-lmh_L-hist.pdf}
 % \includegraphics{img/node-sec/mnhiF-mnh_F-hist.pdf}
 \includegraphics{img/node-sec/mnhiL-mnh_L-hist.pdf}
 % 
 \caption{Histogram comparing the node evaluation ratio (node/sec) between standard tie-breaking ($[f,h,\lifo]$) and
 depth-based tie-breaking ($[f,h,\depth,\lifo]$) on \lmcut and \mands heuristics.
 On \mands, compared to \lmcut, node evaluation rate more often becomes
 slower when depth is enabled. This is because the node evaluation of \mands is an order of
 magnitude faster than \lmcut, and the overhead of managing depth-based tie-breaking queue becomes significant.
 }
 % 
 \label{fig:expansion-ratio-lifo}
\end{figure}

% \clearpage
% \section{Additional Figures for \refig{fig:eval-comparison}: $\lifo$ Default Tiebreaking}
% 
% \begin{figure}[htbp]
%  \centering
%  % \includegraphics{img/node-sec/lmhiF-lmh_F-eval.pdf}
%  \includegraphics{img/node-sec/lmhiL-lmh_L-eval.pdf}
%  % \includegraphics{img/node-sec/mnhiF-mnh_F-eval.pdf}
%  \includegraphics{img/node-sec/mnhiL-mnh_L-eval.pdf}
%  % \includegraphics{img/node-sec/lmhiF-lmh_F-exp.pdf}
%  % \includegraphics{img/node-sec/lmhiL-lmh_L-exp.pdf}
%  % \includegraphics{img/node-sec/mnhiF-mnh_F-exp.pdf}
%  % \includegraphics{img/node-sec/mnhiL-mnh_L-exp.pdf}
%  % 
%  \caption{Comparison of the number of evaluated nodes on IPC domains, between standard and depth-diversified search algorithms.
%  \pddl{Cybersec} and \pddl{Openstacks} shows that the evaluation is reduced by depth.
%  There are also slight effect on \pddl{pegsol-opt11} because it also contains zero-cost actions.
%  However, the two 0-cost actions (\pddl{jump-continue-move, end-move}) are the necessary ``finalization'' actions that should always be executed after the unit-cost action (\pddl{jump-start-move}).
%  Due to the characteristics of these actions, the effect of depth diversification on 0-cost actions are limited.
%  }
%  % 
%  \label{fig:eval-comparison-lifo}
% \end{figure}


\clearpage
\section{Additional Figures for \refig{fig:depth-histogram}: More Histograms for the Size of Final Plateaus}

These includes 12 additional histograms for the size of final plateaus on
more variety of domains and instances.

\begin{figure}[htbp]
\includegraphics[width=0.49\linewidth]{img/output-lmcut/miconic-up/p79-0.pdf}
\includegraphics[width=0.49\linewidth]{img/output-lmcut/pathways-fuel/p03-0.pdf}
\includegraphics[width=0.49\linewidth]{img/output-lmcut/pipesnt-pushstart/p06-0.pdf}
\includegraphics[width=0.49\linewidth]{img/output-lmcut/pipesworld-pushend/p06-0.pdf}
\includegraphics[width=0.49\linewidth]{img/output-lmcut/psr-small-open/p46-0.pdf}
\includegraphics[width=0.49\linewidth]{img/output-lmcut/rovers-fuel/p07-0.pdf}
 \caption{(Page 1/2) Number of nodes ($y$-axis) expanded per depth ($x$-axis) in
 the final plateau with different tie-breaking strategies. Both axes are in logarithmic scale.
 }
 \label{fig:depth-histogram2}
\end{figure}

\begin{figure}[htbp]
\includegraphics[width=0.49\linewidth]{img/output-lmcut/scanalyzer-analyze/p04-0.pdf}
\includegraphics[width=0.49\linewidth]{img/output-lmcut/storage-lift/p11-0.pdf}
\includegraphics[width=0.49\linewidth]{img/output-lmcut/tidybot-motion/p16-0.pdf}
\includegraphics[width=0.49\linewidth]{img/output-lmcut/tpp-fuel/p08-0.pdf}
\includegraphics[width=0.49\linewidth]{img/output-lmcut/woodworking-cut/p04-0.pdf}
\includegraphics[width=0.49\linewidth]{img/output-lmcut/zenotravel-fuel/p07-0.pdf}
 \caption{(Page 2/2) Number of nodes ($y$-axis) expanded per depth ($x$-axis) in
 the final plateau with different tie-breaking strategies. Both axes are in logarithmic scale.
 }
 \label{fig:depth-histogram3}
\end{figure}

\clearpage
\section{Additional Figures for \refig{fig:depth-histogram4}: More Histograms for the Size of Non-final Plateaus}

These are the additional histograms for the size of non-final plateaus on
more variety of domains and instances.

\begin{figure}[htbp]
\includegraphics[width=0.49\linewidth]{img/output-lmcut/depot-fuel/p07-1.pdf}
\includegraphics[width=0.49\linewidth]{img/output-lmcut/driverlog-fuel/p04-1.pdf}
\includegraphics[width=0.49\linewidth]{img/output-lmcut/zenotravel-fuel/p07-1.pdf}
\includegraphics[width=0.49\linewidth]{img/output-lmcut/floortile-ink/p02-14.pdf}
\includegraphics[width=0.49\linewidth]{img/output-lmcut/mprime-succumb/p12-1.pdf}
\includegraphics[width=0.49\linewidth]{img/output-lmcut/storage-lift/p12-2.pdf}
 \caption{Depth distribution in the non-final plateaus ($\plateau{f^*,h}, h\not=0$): Other domains.}
 \label{fig:depth-histogram5}
\end{figure}

\clearpage
\section{Detailed Data for \reftbl{tbl:dtg-summary}}

\setlength{\tabcolsep}{0.1em}

\begin{table}[htbp]
 {
 \relsize{-0.5}
 \centering
 \begin{center}
\begin{tabular}{|r|*{4}{ccc|}}
 & \rb{$[f,\ffo,\fifo]$} & \rb{$[f,\ffo,\lifo]$} & \rb{$[f,\ffo,\ro]$} & \rb{$[f,\ffo,\depth,\fifo]$} & \rb{$[f,\ffo,\depth,\lifo]$} & \rb{$[f,\ffo,\depth,\ro]$} & \rb{$[f,h,\hh,\fifo]$} & \rb{$[f,h,\hh,\lifo]$} & \rb{$[f,h,\hh,\ro]$} & \rb{$[f,\hh,\fifo]$} & \rb{$[f,\hh,\lifo]$} & \rb{$[f,\hh,\ro]$}\\
Zerocost (620) & 337 & 340 & 341 \(\pm\) 2.2 & 340 & 342 & 344.3 \(\pm\) 1.8 & 305 & 309 & 305.9 \(\pm\) 2.1 & 295 & 303 & 301.0\\
airport-fuel(20) & 13 & 11 & 11.7 \(\pm\) 0.5 & 13 & 11 & 11.7 \(\pm\) 0.5 & 14 & 12 & 12.8 \(\pm\) 0.8 & 13 & 12 & 12.7\\
blocks-stack(20) & 17 & 17 & 17 \(\pm\) 0 & 17 & 17 & 17 \(\pm\) 0 & 15 & 15 & 15 \(\pm\) 0 & 15 & 15 & 15.0\\
depot-fuel(22) & 6 & 6 & 6 \(\pm\) 0 & 6 & 6 & 6 \(\pm\) 0 & 6 & 6 & 6 \(\pm\) 0 & 6 & 6 & 6.0\\
driverlog-fuel(20) & 8 & 8 & 8 \(\pm\) 0 & 8 & 8 & 8 \(\pm\) 0 & 8 & 8 & 8 \(\pm\) 0 & 8 & 8 & 8.0\\
elevators-up(20) & 20 & 20 & 20 \(\pm\) 0 & 20 & 20 & 20 \(\pm\) 0 & 20 & 20 & 20 \(\pm\) 0 & 20 & 20 & 19.9\\
floortile-ink(20) & 9 & 8 & 8.7 \(\pm\) 0.5 & 9 & 8 & 8.7 \(\pm\) 0.5 & 8 & 8 & 8 \(\pm\) 0 & 8 & 8 & 8.0\\
freecell-move(20) & 17 & 18 & 17.9 \(\pm\) 0.8 & 17 & 18 & 18.3 \(\pm\) 0.9 & 12 & 14 & 13.2 \(\pm\) 0.4 & 12 & 14 & 13.3\\
grid-fuel(5) & 1 & 1 & 1 \(\pm\) 0 & 1 & 1 & 1 \(\pm\) 0 & 1 & 1 & 1 \(\pm\) 0 & 1 & 1 & 1.0\\
gripper-move(20) & 6 & 6 & 6 \(\pm\) 0 & 6 & 6 & 6 \(\pm\) 0 & 6 & 6 & 6 \(\pm\) 0 & 6 & 6 & 6.0\\
hiking-fuel(20) & 9 & 9 & 9 \(\pm\) 0 & 9 & 9 & 9 \(\pm\) 0 & 8 & 8 & 8 \(\pm\) 0 & 8 & 8 & 8.0\\
logistics00-fuel(28) & 15 & 15 & 15 \(\pm\) 0 & 15 & 15 & 15 \(\pm\) 0 & 15 & 15 & 15 \(\pm\) 0 & 15 & 15 & 15.0\\
miconic-up(30) & 15 & 21 & 17.9 \(\pm\) 1.2 & 15 & 21 & 18 \(\pm\) 1.2 & 14 & 17 & 15.1 \(\pm\) 0.9 & 14 & 17 & 15.1\\
mprime-succumb(35) & 30 & 23 & 28.3 \(\pm\) 0.9 & 30 & 27 & 29.3 \(\pm\) 0.7 & 20 & 16 & 20.1 \(\pm\) 0.6 & 19 & 16 & 19.1\\
mystery-feast(20) & 8 & 8 & 8 \(\pm\) 0 & 8 & 8 & 8 \(\pm\) 0 & 6 & 5 & 5.9 \(\pm\) 0.3 & 7 & 6 & 6.9\\
nomystery-fuel(20) & 10 & 10 & 10 \(\pm\) 0 & 10 & 10 & 10 \(\pm\) 0 & 10 & 10 & 10 \(\pm\) 0 & 10 & 10 & 10.0\\
parking-movecc(20) & 20 & 20 & 20 \(\pm\) 0 & 20 & 20 & 20 \(\pm\) 0 & 13 & 15 & 14.4 \(\pm\) 1.5 & 13 & 14 & 14.3\\
pathways-fuel(30) & 5 & 5 & 5 \(\pm\) 0 & 5 & 5 & 5 \(\pm\) 0 & 5 & 5 & 4 \(\pm\) 0 & 5 & 5 & 4.1\\
pipesnt-pushstart(20) & 9 & 9 & 9 \(\pm\) 0 & 9 & 9 & 9 \(\pm\) 0 & 8 & 8 & 7.8 \(\pm\) 0.4 & 7 & 8 & 7.7\\
pipesworld-pushend(20) & 7 & 8 & 7.1 \(\pm\) 0.3 & 7 & 7 & 7.7 \(\pm\) 0.5 & 5 & 5 & 5 \(\pm\) 0 & 5 & 6 & 5.1\\
psr-small-open(20) & 19 & 19 & 19 \(\pm\) 0 & 19 & 19 & 19 \(\pm\) 0 & 19 & 19 & 19 \(\pm\) 0 & 19 & 19 & 19.0\\
rovers-fuel(40) & 8 & 9 & 8 \(\pm\) 0 & 8 & 8 & 8 \(\pm\) 0 & 7 & 7 & 7 \(\pm\) 0 & 7 & 7 & 7.0\\
scanalyzer-analyze(20) & 15 & 15 & 15 \(\pm\) 0 & 15 & 15 & 15 \(\pm\) 0 & 16 & 18 & 15.3 \(\pm\) 0.9 & 8 & 11 & 10.1\\
sokoban-pushgoal(20) & 17 & 17 & 17 \(\pm\) 0 & 17 & 17 & 17 \(\pm\) 0 & 16 & 16 & 16 \(\pm\) 0 & 16 & 16 & 16.0\\
storage-lift(20) & 4 & 4 & 4.3 \(\pm\) 0.5 & 4 & 4 & 4.8 \(\pm\) 0.4 & 4 & 4 & 4 \(\pm\) 0 & 4 & 4 & 4.0\\
tidybot-motion(20) & 15 & 16 & 16 \(\pm\) 0 & 16 & 16 & 15.9 \(\pm\) 0.3 & 14 & 14 & 14 \(\pm\) 0 & 14 & 14 & 14.0\\
tpp-fuel(30) & 8 & 10 & 9.1 \(\pm\) 0.3 & 10 & 10 & 10 \(\pm\) 0 & 8 & 10 & 8.2 \(\pm\) 0.4 & 8 & 10 & 8.7\\
woodworking-cut(20) & 19 & 20 & 20 \(\pm\) 0 & 19 & 20 & 20 \(\pm\) 0 & 20 & 20 & 20 \(\pm\) 0 & 20 & 20 & 20.0\\
zenotravel-fuel(20) & 7 & 7 & 7 \(\pm\) 0 & 7 & 7 & 7 \(\pm\) 0 & 7 & 7 & 7 \(\pm\) 0 & 7 & 7 & 7.0\\
\end{tabular}
\end{center}

 \caption{
 Coverage results with { \lmcut for computing $f$ and inadmissible distance-to-go heuristics for tie-breaking, on 620 Zerocost instances}. We highlight the best results when the difference between the maximum and the minimum coverage exceeds 2, over all configurations \emph{including \reftbl{tbl:lmcut-zerocost-full}}.
 }
 \label{tbl:dtg-lmcut-zero}
 }
\end{table}

\begin{table}[htbp]
 {
 \relsize{-0.5}
 \centering
 \begin{center}
\begin{tabular}{|r|HHHHHHHHHHHHHHHHHHcccHHH|cccHHH|}
 & $[f,\ffo,\fifo]$ & $[f,\ffo,\lifo]$ & $[f,\ffo,\ro]$ & R & R & R & $[f,\ffo,\depth,\fifo]$ & $[f,\ffo,\depth,\lifo]$ & $[f,\ffo,\depth,\ro]$ & R & R & R & $[f,\gco,\fifo]$ & $[f,\gco,\lifo]$ & $[f,\gco,\ro]$ & R & R & R & $[f,h,\hat{h},\depth,\fifo]$ & $[f,h,\hat{h},\depth,\lifo]$ & $[f,h,\hat{h},\depth,\ro]$ & R & R & R & $[f,\hat{h},\depth,\fifo]$ & $[f,\hat{h},\depth,\lifo]$ & $[f,\hat{h},\depth,\ro]$ & R & R & R\\
\hline
 & 351 & 346 & 351 $\pm$ 1 & 352 & 351 & 350 & 352 & 348 & 352.7 $\pm$ 0.6 & 352 & 353 & 353 & 351 & 357 & 354 $\pm$ 2.6 & 356 & 351 & 355 & 322 & 321 & 323.3 $\pm$ 1.5 & 325 & 322 & 323 & 323 & 320 & 321.3 $\pm$ 1.5 & 323 & 321 & 320\\
\hline
airport-fuel(20) & 5 & 5 & 5 $\pm$ 0 & 5 & 5 & 5 & 5 & 5 & 5 $\pm$ 0 & 5 & 5 & 5 & 5 & 5 & 5 $\pm$ 0 & 5 & 5 & 5 & 1 & 1 & 1 $\pm$ 0 & 1 & 1 & 1 & 1 & 1 & 1 $\pm$ 0 & 1 & 1 & 1\\
blocks-stack(20) & 20 & 19 & 20 $\pm$ 0 & 20 & 20 & 20 & 20 & 20 & 20 $\pm$ 0 & 20 & 20 & 20 & 20 & 20 & 20 $\pm$ 0 & 20 & 20 & 20 & 20 & 20 & 20 $\pm$ 0 & 20 & 20 & 20 & 20 & 20 & 20 $\pm$ 0 & 20 & 20 & 20\\
depot-fuel(22) & 4 & 4 & 4 $\pm$ 0 & 4 & 4 & 4 & 4 & 4 & 4 $\pm$ 0 & 4 & 4 & 4 & 6 & 6 & 6 $\pm$ 0 & 6 & 6 & 6 & 6 & 6 & 6 $\pm$ 0 & 6 & 6 & 6 & 6 & 6 & 6 $\pm$ 0 & 6 & 6 & 6\\
driverlog-fuel(20) & 9 & 9 & 9 $\pm$ 0 & 9 & 9 & 9 & 9 & 9 & 9 $\pm$ 0 & 9 & 9 & 9 & 9 & 9 & 9 $\pm$ 0 & 9 & 9 & 9 & 9 & 9 & 9 $\pm$ 0 & 9 & 9 & 9 & 9 & 9 & 9 $\pm$ 0 & 9 & 9 & 9\\
elevators-up(20) & 20 & 20 & 20 $\pm$ 0 & 20 & 20 & 20 & 20 & 20 & 20 $\pm$ 0 & 20 & 20 & 20 & 20 & 20 & 20 $\pm$ 0 & 20 & 20 & 20 & 19 & 19 & 19 $\pm$ 0 & 19 & 19 & 19 & 19 & 19 & 19 $\pm$ 0 & 19 & 19 & 19\\
floortile-ink(20) & 9 & 8 & 8.3 $\pm$ 0.6 & 8 & 9 & 8 & 9 & 8 & 9 $\pm$ 0 & 9 & 9 & 9 & 8 & 8 & 8 $\pm$ 0 & 8 & 8 & 8 & 8 & 8 & 8 $\pm$ 0 & 8 & 8 & 8 & 8 & 8 & 8 $\pm$ 0 & 8 & 8 & 8\\
freecell-move(20) & 17 & 17 & 17.3 $\pm$ 0.6 & 17 & 18 & 17 & 17 & 17 & 17.7 $\pm$ 0.6 & 17 & 18 & 18 & 12 & 17 & 13 $\pm$ 1 & 14 & 12 & 13 & 13 & 13 & 12.7 $\pm$ 0.6 & 12 & 13 & 13 & 13 & 14 & 12.7 $\pm$ 0.6 & 12 & 13 & 13\\
 & 15 & 15 & 15 $\pm$ 0 & 15 & 15 & 15 & 15 & 15 & 15 $\pm$ 0 & 15 & 15 & 15 & 15 & 15 & 15 $\pm$ 0 & 15 & 15 & 15 & 15 & 15 & 15 $\pm$ 0 & 15 & 15 & 15 & 15 & 15 & 15 $\pm$ 0 & 15 & 15 & 15\\
grid-fuel(5) & 2 & 2 & 2 $\pm$ 0 & 2 & 2 & 2 & 2 & 2 & 2 $\pm$ 0 & 2 & 2 & 2 & 2 & 2 & 2 $\pm$ 0 & 2 & 2 & 2 & 2 & 2 & 2 $\pm$ 0 & 2 & 2 & 2 & 2 & 2 & 2 $\pm$ 0 & 2 & 2 & 2\\
gripper-move(20) & 20 & 20 & 20 $\pm$ 0 & 20 & 20 & 20 & 20 & 20 & 20 $\pm$ 0 & 20 & 20 & 20 & 20 & 20 & 20 $\pm$ 0 & 20 & 20 & 20 & 20 & 20 & 20 $\pm$ 0 & 20 & 20 & 20 & 20 & 20 & 20 $\pm$ 0 & 20 & 20 & 20\\
hiking-fuel(20) & 11 & 11 & 11 $\pm$ 0 & 11 & 11 & 11 & 11 & 11 & 11 $\pm$ 0 & 11 & 11 & 11 & 13 & 13 & 12.7 $\pm$ 0.6 & 13 & 12 & 13 & 13 & 13 & 12.3 $\pm$ 0.6 & 13 & 12 & 12 & 13 & 13 & 12 $\pm$ 0 & 12 & 12 & 12\\
logistics00-fuel(28) & 16 & 16 & 16 $\pm$ 0 & 16 & 16 & 16 & 16 & 16 & 16 $\pm$ 0 & 16 & 16 & 16 & 16 & 16 & 16 $\pm$ 0 & 16 & 16 & 16 & 16 & 16 & 16 $\pm$ 0 & 16 & 16 & 16 & 16 & 16 & 16 $\pm$ 0 & 16 & 16 & 16\\
miconic-up(30) & 30 & 30 & 30 $\pm$ 0 & 30 & 30 & 30 & 30 & 30 & 30 $\pm$ 0 & 30 & 30 & 30 & 30 & 30 & 30 $\pm$ 0 & 30 & 30 & 30 & 22 & 22 & 22.3 $\pm$ 0.6 & 23 & 22 & 22 & 22 & 22 & 22 $\pm$ 0 & 22 & 22 & 22\\
mprime-succumb(35) & 28 & 23 & 27 $\pm$ 1 & 28 & 26 & 27 & 28 & 25 & 27.3 $\pm$ 0.6 & 27 & 28 & 27 & 21 & 19 & 19 $\pm$ 0 & 19 & 19 & 19 & 21 & 17 & 20.7 $\pm$ 0.6 & 21 & 20 & 21 & 21 & 17 & 20.7 $\pm$ 0.6 & 21 & 20 & 21\\
mystery-feast(20) & 3 & 3 & 3 $\pm$ 0 & 3 & 3 & 3 & 3 & 3 & 3 $\pm$ 0 & 3 & 3 & 3 & 6 & 4 & 6 $\pm$ 0 & 6 & 6 & 6 & 5 & 5 & 5 $\pm$ 0 & 5 & 5 & 5 & 5 & 5 & 5 $\pm$ 0 & 5 & 5 & 5\\
nomystery-fuel(20) & 15 & 15 & 15 $\pm$ 0 & 15 & 15 & 15 & 15 & 15 & 15 $\pm$ 0 & 15 & 15 & 15 & 16 & 16 & 16 $\pm$ 0 & 16 & 16 & 16 & 16 & 16 & 16 $\pm$ 0 & 16 & 16 & 16 & 16 & 16 & 16 $\pm$ 0 & 16 & 16 & 16\\
parking-movecc(20) & 10 & 10 & 10 $\pm$ 0 & 10 & 10 & 10 & 10 & 10 & 10 $\pm$ 0 & 10 & 10 & 10 & 10 & 10 & 11 $\pm$ 1 & 11 & 10 & 12 & 2 & 2 & 2 $\pm$ 0 & 2 & 2 & 2 & 2 & 2 & 2 $\pm$ 0 & 2 & 2 & 2\\
pathways-fuel(30) & 4 & 4 & 4 $\pm$ 0 & 4 & 4 & 4 & 4 & 4 & 4 $\pm$ 0 & 4 & 4 & 4 & 4 & 4 & 4 $\pm$ 0 & 4 & 4 & 4 & 4 & 4 & 4 $\pm$ 0 & 4 & 4 & 4 & 4 & 4 & 4 $\pm$ 0 & 4 & 4 & 4\\
pipesnt-pushstart(20) & 5 & 5 & 5 $\pm$ 0 & 5 & 5 & 5 & 5 & 5 & 5 $\pm$ 0 & 5 & 5 & 5 & 5 & 5 & 5 $\pm$ 0 & 5 & 5 & 5 & 1 & 2 & 2 $\pm$ 1 & 3 & 1 & 2 & 1 & 2 & 1.7 $\pm$ 1.2 & 3 & 1 & 1\\
pipesworld-pushend(20) & 5 & 5 & 5 $\pm$ 0 & 5 & 5 & 5 & 5 & 5 & 5.3 $\pm$ 0.6 & 5 & 5 & 6 & 10 & 10 & 10 $\pm$ 0 & 10 & 10 & 10 & 8 & 8 & 8 $\pm$ 0 & 8 & 8 & 8 & 8 & 7 & 7.7 $\pm$ 0.6 & 7 & 8 & 8\\
psr-small-open(20) & 19 & 19 & 19 $\pm$ 0 & 19 & 19 & 19 & 19 & 19 & 19 $\pm$ 0 & 19 & 19 & 19 & 19 & 19 & 19 $\pm$ 0 & 19 & 19 & 19 & 19 & 19 & 19 $\pm$ 0 & 19 & 19 & 19 & 19 & 19 & 19 $\pm$ 0 & 19 & 19 & 19\\
rovers-fuel(40) & 8 & 8 & 8 $\pm$ 0 & 8 & 8 & 8 & 8 & 8 & 8 $\pm$ 0 & 8 & 8 & 8 & 8 & 8 & 8 $\pm$ 0 & 8 & 8 & 8 & 8 & 8 & 8 $\pm$ 0 & 8 & 8 & 8 & 8 & 8 & 8 $\pm$ 0 & 8 & 8 & 8\\
scanalyzer-analyze(20) & 15 & 16 & 15.3 $\pm$ 0.6 & 16 & 15 & 15 & 15 & 15 & 15.3 $\pm$ 0.6 & 16 & 15 & 15 & 17 & 19 & 18.3 $\pm$ 0.6 & 19 & 18 & 18 & 14 & 15 & 14.7 $\pm$ 0.6 & 15 & 15 & 14 & 15 & 14 & 14.3 $\pm$ 1.2 & 15 & 15 & 13\\
sokoban-pushgoal(20) & 18 & 18 & 18 $\pm$ 0 & 18 & 18 & 18 & 18 & 18 & 18 $\pm$ 0 & 18 & 18 & 18 & 18 & 19 & 18 $\pm$ 0 & 18 & 18 & 18 & 17 & 17 & 17 $\pm$ 0 & 17 & 17 & 17 & 17 & 17 & 17 $\pm$ 0 & 17 & 17 & 17\\
storage-lift(20) & 4 & 4 & 4 $\pm$ 0 & 4 & 4 & 4 & 4 & 4 & 4 $\pm$ 0 & 4 & 4 & 4 & 4 & 4 & 4 $\pm$ 0 & 4 & 4 & 4 & 4 & 4 & 4 $\pm$ 0 & 4 & 4 & 4 & 4 & 4 & 4 $\pm$ 0 & 4 & 4 & 4\\
tidybot-motion(20) & 0 & 0 & 0 $\pm$ 0 & 0 & 0 & 0 & 0 & 0 & 0 $\pm$ 0 & 0 & 0 & 0 & 0 & 0 & 0 $\pm$ 0 & 0 & 0 & 0 & 0 & 0 & 0 $\pm$ 0 & 0 & 0 & 0 & 0 & 0 & 0 $\pm$ 0 & 0 & 0 & 0\\
tpp-fuel(30) & 10 & 11 & 11 $\pm$ 0 & 11 & 11 & 11 & 11 & 11 & 11 $\pm$ 0 & 11 & 11 & 11 & 8 & 10 & 9 $\pm$ 0 & 9 & 9 & 9 & 9 & 10 & 9.7 $\pm$ 0.6 & 9 & 10 & 10 & 9 & 10 & 9.3 $\pm$ 0.6 & 10 & 9 & 9\\
woodworking-cut(20) & 20 & 20 & 20 $\pm$ 0 & 20 & 20 & 20 & 20 & 20 & 20 $\pm$ 0 & 20 & 20 & 20 & 19 & 19 & 20 $\pm$ 0 & 20 & 20 & 20 & 20 & 20 & 20 $\pm$ 0 & 20 & 20 & 20 & 20 & 20 & 20 $\pm$ 0 & 20 & 20 & 20\\
zenotravel-fuel(20) & 9 & 9 & 9 $\pm$ 0 & 9 & 9 & 9 & 9 & 9 & 9 $\pm$ 0 & 9 & 9 & 9 & 10 & 10 & 10 $\pm$ 0 & 10 & 10 & 10 & 10 & 10 & 10 $\pm$ 0 & 10 & 10 & 10 & 10 & 10 & 10 $\pm$ 0 & 10 & 10 & 10\\
\end{tabular}
\end{center}

 \caption{
 Coverage results with \mands for computing $f$ and inadmissible distance-to-go heuristics for tie-breaking, on 620 Zerocost instances. We highlight the best results when the difference between the maximum and the minimum coverage exceeds 2, over all configurations \emph{including \reftbl{tbl:mands-zerocost-full}}.
 }
 \label{tbl:dtg-mands-zero}
 }
\end{table}

\begin{table}[htbp]
 {
 \relsize{-0.5} 
 \centering
 \begin{center}
\begin{tabular}{|r|cccHHH|cccHHH|cccHHH|cccHHHHHHHHH|}
Domain & \rb{$[f,h,\hh,\depth,\fifo]$} & \rb{$[f,h,\hh,\depth,\lifo]$} & \rb{$[f,h,\hh,\depth,\ro]$} & R & R & R & \rb{$[f,\hh,\depth,\fifo]$} & \rb{$[f,\hh,\depth,\lifo]$} & \rb{$[f,\hh,\depth,\ro]$} & R & R & R & \rb{$[f,\ffo,\fifo]$} & \rb{$[f,\ffo,\lifo]$} & \rb{$[f,\ffo,\ro]$} & R & R & R & \rb{$[f,\ffo,\depth,\fifo]$} & \rb{$[f,\ffo,\depth,\lifo]$} & \rb{$[f,\ffo,\depth,\ro]$} & R & R & R & \rb{$[f,\gco,\fifo]$} & \rb{$[f,\gco,\lifo]$} & \rb{$[f,\gco,\ro]$} & R & R & R\\
\hline
IPC benchmark (1104) & 536 & 535 & 534.7 \(\pm\) 2.1 & 537 & 533 & 534 & 534 & 534 & 534.7 \(\pm\) 1.5 & 536 & 533 & 535 & 564 & 562 & 564.3 \(\pm\) 1.5 & 563 & 564 & 566 & 563 & 560 & 563.3 \(\pm\) 1.5 & 562 & 563 & 565 & 558 & 566 & 560.7 \(\pm\) 0.6 & 561 & 560 & 561\\
\hline
airport(50) & 24 & 24 & 24 \(\pm\) 0 & 24 & 24 & 24 & 24 & 25 & 24 \(\pm\) 0 & 24 & 24 & 24 & 25 & 24 & 25 \(\pm\) 0 & 25 & 25 & 25 & 25 & 24 & 24.7 \(\pm\) 0.6 & 25 & 24 & 25 & 23 & 26 & 24 \(\pm\) 1 & 23 & 24 & 25\\
barman-opt11(20) & 0 & 0 & 0 \(\pm\) 0 & 0 & 0 & 0 & 0 & 0 & 0 \(\pm\) 0 & 0 & 0 & 0 & 0 & 0 & 0 \(\pm\) 0 & 0 & 0 & 0 & 0 & 0 & 0 \(\pm\) 0 & 0 & 0 & 0 & 0 & 0 & 0 \(\pm\) 0 & 0 & 0 & 0\\
blocks(35) & 27 & 27 & 27 \(\pm\) 0 & 27 & 27 & 27 & 27 & 27 & 27 \(\pm\) 0 & 27 & 27 & 27 & 27 & 27 & 27 \(\pm\) 0 & 27 & 27 & 27 & 27 & 27 & 27 \(\pm\) 0 & 27 & 27 & 27 & 28 & 28 & 28 \(\pm\) 0 & 28 & 28 & 28\\
\textbf{cybersec(19)} & 6 & 4 & 5 \(\pm\) 1 & 6 & 4 & 5 & 5 & 3 & 5.7 \(\pm\) 1.2 & 7 & 5 & 5 & 6 & 6 & 5.3 \(\pm\) 0.6 & 5 & 5 & 6 & 6 & 5 & 6 \(\pm\) 0 & 6 & 6 & 6 & 0 & 1 & 0.7 \(\pm\) 0.6 & 1 & 1 & 0\\
depot(22) & 5 & 5 & 5 \(\pm\) 0 & 5 & 5 & 5 & 5 & 5 & 5 \(\pm\) 0 & 5 & 5 & 5 & 6 & 6 & 6 \(\pm\) 0 & 6 & 6 & 6 & 6 & 6 & 6 \(\pm\) 0 & 6 & 6 & 6 & 6 & 6 & 6 \(\pm\) 0 & 6 & 6 & 6\\
driverlog(20) & 12 & 12 & 12 \(\pm\) 0 & 12 & 12 & 12 & 12 & 12 & 12 \(\pm\) 0 & 12 & 12 & 12 & 13 & 13 & 13 \(\pm\) 0 & 13 & 13 & 13 & 13 & 13 & 13 \(\pm\) 0 & 13 & 13 & 13 & 13 & 13 & 13 \(\pm\) 0 & 13 & 13 & 13\\
elevators-opt11(20) & 12 & 12 & 12 \(\pm\) 0 & 12 & 12 & 12 & 12 & 12 & 12 \(\pm\) 0 & 12 & 12 & 12 & 15 & 15 & 15 \(\pm\) 0 & 15 & 15 & 15 & 14 & 15 & 14 \(\pm\) 0 & 14 & 14 & 14 & 15 & 15 & 15 \(\pm\) 0 & 15 & 15 & 15\\
floortile-opt11(20) & 6 & 6 & 6 \(\pm\) 0 & 6 & 6 & 6 & 6 & 6 & 6 \(\pm\) 0 & 6 & 6 & 6 & 6 & 6 & 6 \(\pm\) 0 & 6 & 6 & 6 & 6 & 6 & 6 \(\pm\) 0 & 6 & 6 & 6 & 6 & 6 & 6 \(\pm\) 0 & 6 & 6 & 6\\
freecell(80) & 8 & 8 & 8 \(\pm\) 0 & 8 & 8 & 8 & 8 & 8 & 8 \(\pm\) 0 & 8 & 8 & 8 & 9 & 9 & 9 \(\pm\) 0 & 9 & 9 & 9 & 9 & 9 & 9 \(\pm\) 0 & 9 & 9 & 9 & 9 & 9 & 9 \(\pm\) 0 & 9 & 9 & 9\\
grid(5) & 1 & 1 & 1 \(\pm\) 0 & 1 & 1 & 1 & 1 & 1 & 1 \(\pm\) 0 & 1 & 1 & 1 & 1 & 1 & 1 \(\pm\) 0 & 1 & 1 & 1 & 1 & 1 & 1 \(\pm\) 0 & 1 & 1 & 1 & 1 & 1 & 1 \(\pm\) 0 & 1 & 1 & 1\\
gripper(20) & 6 & 6 & 6 \(\pm\) 0 & 6 & 6 & 6 & 6 & 6 & 6 \(\pm\) 0 & 6 & 6 & 6 & 6 & 6 & 6 \(\pm\) 0 & 6 & 6 & 6 & 6 & 6 & 6 \(\pm\) 0 & 6 & 6 & 6 & 6 & 6 & 6 \(\pm\) 0 & 6 & 6 & 6\\
hanoi(30) & 11 & 11 & 11 \(\pm\) 0 & 11 & 11 & 11 & 11 & 11 & 11 \(\pm\) 0 & 11 & 11 & 11 & 12 & 12 & 12 \(\pm\) 0 & 12 & 12 & 12 & 12 & 12 & 12 \(\pm\) 0 & 12 & 12 & 12 & 12 & 12 & 12 \(\pm\) 0 & 12 & 12 & 12\\
logistics00(28) & 17 & 17 & 17 \(\pm\) 0 & 17 & 17 & 17 & 17 & 17 & 17 \(\pm\) 0 & 17 & 17 & 17 & 20 & 20 & 20 \(\pm\) 0 & 20 & 20 & 20 & 20 & 20 & 20 \(\pm\) 0 & 20 & 20 & 20 & 20 & 20 & 20 \(\pm\) 0 & 20 & 20 & 20\\
miconic(150) & 140 & 140 & 140 \(\pm\) 0 & 140 & 140 & 140 & 140 & 140 & 140 \(\pm\) 0 & 140 & 140 & 140 & 140 & 140 & 140 \(\pm\) 0 & 140 & 140 & 140 & 140 & 140 & 140 \(\pm\) 0 & 140 & 140 & 140 & 140 & 140 & 140 \(\pm\) 0 & 140 & 140 & 140\\
mprime(35) & 20 & 21 & 20.3 \(\pm\) 0.6 & 21 & 20 & 20 & 20 & 21 & 20.3 \(\pm\) 0.6 & 21 & 20 & 20 & 22 & 22 & 22 \(\pm\) 0 & 22 & 22 & 22 & 22 & 22 & 22 \(\pm\) 0 & 22 & 22 & 22 & 20 & 22 & 20.3 \(\pm\) 0.6 & 21 & 20 & 20\\
mystery(30) & 15 & 15 & 15 \(\pm\) 0 & 15 & 15 & 15 & 15 & 15 & 15 \(\pm\) 0 & 15 & 15 & 15 & 16 & 16 & 16 \(\pm\) 0 & 16 & 16 & 16 & 16 & 16 & 16 \(\pm\) 0 & 16 & 16 & 16 & 15 & 16 & 15 \(\pm\) 0 & 15 & 15 & 15\\
nomystery-opt11(20) & 13 & 13 & 13 \(\pm\) 0 & 13 & 13 & 13 & 13 & 13 & 13 \(\pm\) 0 & 13 & 13 & 13 & 14 & 14 & 14 \(\pm\) 0 & 14 & 14 & 14 & 14 & 14 & 14 \(\pm\) 0 & 14 & 14 & 14 & 14 & 14 & 14 \(\pm\) 0 & 14 & 14 & 14\\
\textbf{openstacks-opt11(20)} & 10 & 10 & 10 \(\pm\) 0 & 10 & 10 & 10 & 10 & 10 & 10 \(\pm\) 0 & 10 & 10 & 10 & 17 & 17 & 17 \(\pm\) 0 & 17 & 17 & 17 & 17 & 17 & 17 \(\pm\) 0 & 17 & 17 & 17 & 18 & 18 & 18 \(\pm\) 0 & 18 & 18 & 18\\
parcprinter-opt11(20) & 13 & 13 & 13 \(\pm\) 0 & 13 & 13 & 13 & 13 & 13 & 13 \(\pm\) 0 & 13 & 13 & 13 & 13 & 13 & 13 \(\pm\) 0 & 13 & 13 & 13 & 13 & 13 & 13 \(\pm\) 0 & 13 & 13 & 13 & 13 & 13 & 13 \(\pm\) 0 & 13 & 13 & 13\\
parking-opt11(20) & 1 & 1 & 1 \(\pm\) 0 & 1 & 1 & 1 & 1 & 1 & 1 \(\pm\) 0 & 1 & 1 & 1 & 1 & 1 & 1 \(\pm\) 0 & 1 & 1 & 1 & 1 & 1 & 1 \(\pm\) 0 & 1 & 1 & 1 & 1 & 1 & 1 \(\pm\) 0 & 1 & 1 & 1\\
pathways(30) & 5 & 5 & 5 \(\pm\) 0 & 5 & 5 & 5 & 5 & 5 & 5 \(\pm\) 0 & 5 & 5 & 5 & 5 & 5 & 5 \(\pm\) 0 & 5 & 5 & 5 & 5 & 5 & 5 \(\pm\) 0 & 5 & 5 & 5 & 5 & 5 & 5 \(\pm\) 0 & 5 & 5 & 5\\
pegsol-opt11(20) & 16 & 16 & 16 \(\pm\) 0 & 16 & 16 & 16 & 16 & 16 & 16 \(\pm\) 0 & 16 & 16 & 16 & 17 & 17 & 17 \(\pm\) 0 & 17 & 17 & 17 & 17 & 17 & 17 \(\pm\) 0 & 17 & 17 & 17 & 17 & 17 & 17 \(\pm\) 0 & 17 & 17 & 17\\
pipesworld-notankage(50) & 12 & 12 & 12 \(\pm\) 0 & 12 & 12 & 12 & 12 & 12 & 12 \(\pm\) 0 & 12 & 12 & 12 & 13 & 13 & 13 \(\pm\) 0 & 13 & 13 & 13 & 13 & 13 & 13 \(\pm\) 0 & 13 & 13 & 13 & 14 & 15 & 14.7 \(\pm\) 0.6 & 15 & 14 & 15\\
pipesworld-tankage(50) & 7 & 7 & 7 \(\pm\) 0 & 7 & 7 & 7 & 7 & 7 & 7 \(\pm\) 0 & 7 & 7 & 7 & 8 & 8 & 8 \(\pm\) 0 & 8 & 8 & 8 & 8 & 8 & 8 \(\pm\) 0 & 8 & 8 & 8 & 8 & 8 & 8 \(\pm\) 0 & 8 & 8 & 8\\
psr-small(50) & 48 & 48 & 48 \(\pm\) 0 & 48 & 48 & 48 & 48 & 48 & 48 \(\pm\) 0 & 48 & 48 & 48 & 48 & 48 & 48 \(\pm\) 0 & 48 & 48 & 48 & 48 & 48 & 48 \(\pm\) 0 & 48 & 48 & 48 & 48 & 48 & 48 \(\pm\) 0 & 48 & 48 & 48\\
rovers(40) & 7 & 7 & 7 \(\pm\) 0 & 7 & 7 & 7 & 7 & 7 & 7 \(\pm\) 0 & 7 & 7 & 7 & 7 & 7 & 7 \(\pm\) 0 & 7 & 7 & 7 & 7 & 7 & 7 \(\pm\) 0 & 7 & 7 & 7 & 7 & 7 & 7 \(\pm\) 0 & 7 & 7 & 7\\
scanalyzer-opt11(20) & 10 & 10 & 10 \(\pm\) 0 & 10 & 10 & 10 & 8 & 10 & 8.7 \(\pm\) 0.6 & 9 & 8 & 9 & 10 & 10 & 10 \(\pm\) 0 & 10 & 10 & 10 & 10 & 10 & 10 \(\pm\) 0 & 10 & 10 & 10 & 10 & 10 & 10 \(\pm\) 0 & 10 & 10 & 10\\
sokoban-opt11(20) & 17 & 17 & 17 \(\pm\) 0 & 17 & 17 & 17 & 17 & 17 & 17 \(\pm\) 0 & 17 & 17 & 17 & 19 & 19 & 19 \(\pm\) 0 & 19 & 19 & 19 & 19 & 19 & 19 \(\pm\) 0 & 19 & 19 & 19 & 19 & 19 & 19 \(\pm\) 0 & 19 & 19 & 19\\
storage(30) & 14 & 14 & 14 \(\pm\) 0 & 14 & 14 & 14 & 14 & 14 & 14 \(\pm\) 0 & 14 & 14 & 14 & 14 & 14 & 14 \(\pm\) 0 & 14 & 14 & 14 & 14 & 14 & 14 \(\pm\) 0 & 14 & 14 & 14 & 15 & 15 & 15 \(\pm\) 0 & 15 & 15 & 15\\
tidybot-opt11(20) & 11 & 11 & 10.3 \(\pm\) 0.6 & 11 & 10 & 10 & 10 & 11 & 10 \(\pm\) 0 & 10 & 10 & 10 & 11 & 11 & 11 \(\pm\) 0 & 11 & 11 & 11 & 11 & 11 & 11 \(\pm\) 0 & 11 & 11 & 11 & 12 & 12 & 12 \(\pm\) 0 & 12 & 12 & 12\\
tpp(30) & 6 & 6 & 6 \(\pm\) 0 & 6 & 6 & 6 & 6 & 6 & 6 \(\pm\) 0 & 6 & 6 & 6 & 6 & 6 & 6 \(\pm\) 0 & 6 & 6 & 6 & 6 & 6 & 6 \(\pm\) 0 & 6 & 6 & 6 & 6 & 6 & 6 \(\pm\) 0 & 6 & 6 & 6\\
transport-opt11(20) & 6 & 6 & 6 \(\pm\) 0 & 6 & 6 & 6 & 6 & 6 & 6 \(\pm\) 0 & 6 & 6 & 6 & 6 & 6 & 6 \(\pm\) 0 & 6 & 6 & 6 & 6 & 6 & 6 \(\pm\) 0 & 6 & 6 & 6 & 6 & 6 & 6 \(\pm\) 0 & 6 & 6 & 6\\
visitall-opt11(20) & 10 & 10 & 10 \(\pm\) 0 & 10 & 10 & 10 & 10 & 10 & 10 \(\pm\) 0 & 10 & 10 & 10 & 10 & 10 & 10 \(\pm\) 0 & 10 & 10 & 10 & 10 & 10 & 10 \(\pm\) 0 & 10 & 10 & 10 & 10 & 10 & 10 \(\pm\) 0 & 10 & 10 & 10\\
woodworking-opt11(20) & 9 & 9 & 9 \(\pm\) 0 & 9 & 9 & 9 & 11 & 8 & 10 \(\pm\) 1 & 9 & 10 & 11 & 10 & 9 & 11 \(\pm\) 1 & 10 & 11 & 12 & 10 & 8 & 10.7 \(\pm\) 1.5 & 9 & 11 & 12 & 10 & 10 & 10 \(\pm\) 0 & 10 & 10 & 10\\
zenotravel(20) & 11 & 11 & 11 \(\pm\) 0 & 11 & 11 & 11 & 11 & 11 & 11 \(\pm\) 0 & 11 & 11 & 11 & 11 & 11 & 11 \(\pm\) 0 & 11 & 11 & 11 & 11 & 11 & 11 \(\pm\) 0 & 11 & 11 & 11 & 11 & 11 & 11 \(\pm\) 0 & 11 & 11 & 11\\
\end{tabular}
\end{center}

 \caption{
 Coverage results with \lmcut for computing $f$ and inadmissible distance-to-go heuristics for tie-breaking, on 1104 standard IPC benchmark instances.
 }
 \label{tbl:dtg-lmcut-ipc}
 }
\end{table}

\begin{table}[htbp]
 {
 \relsize{-0.5}
 \centering
 \let\hline\midrule
\begin{center}
\begin{tabular}{|r|*{4}{ccc|}}
 & \rb{$[f,\hh,\fifo]$} & \rb{$[f,\hh,\lifo]$} & \rb{$[f,\hh,\ro]$} & \rb{$[f,h,\hh,\fifo]$} & \rb{$[f,h,\hh,\lifo]$} & \rb{$[f,h,\hh,\ro]$} & \rb{$[f,\ffo,\fifo]$} & \rb{$[f,\ffo,\lifo]$} & \rb{$[f,\ffo,\ro]$} & \rb{$[f,\ffo,\depth,\fifo]$} & \rb{$[f,\ffo,\depth,\lifo]$} & \rb{$[f,\ffo,\depth,\ro]$}\\
IPC benchmark (1104) & 477 & 475 & 470.4 $\pm$ 0.9 & 476 & 475 & 470.9 $\pm$ 0.9 & 458 & 457 & 457 $\pm$ 1.3 & 457 & 457 & 456.8 $\pm$ 1.2\\
\hline
airport(50) & 7 & 7 & 7 $\pm$ 0 & 7 & 7 & 7 $\pm$ 0 & 9 & 9 & 9 $\pm$ 0 & 9 & 9 & 9 $\pm$ 0\\
barman-opt11(20) & 4 & 4 & 4 $\pm$ 0 & 4 & 4 & 4 $\pm$ 0 & 4 & 4 & 4 $\pm$ 0 & 4 & 4 & 4 $\pm$ 0\\
blocks(35) & 22 & 21 & 21 $\pm$ 0 & 21 & 21 & 21 $\pm$ 0 & 21 & 20 & 20.1 $\pm$ 0.3 & 20 & 20 & 20 $\pm$ 0\\
cybersec(19) & 0 & 0 & 0 $\pm$ 0 & 0 & 0 & 0 $\pm$ 0 & 0 & 0 & 0 $\pm$ 0 & 0 & 0 & 0 $\pm$ 0\\
depot(22) & 5 & 5 & 5 $\pm$ 0 & 5 & 5 & 5 $\pm$ 0 & 4 & 4 & 4 $\pm$ 0 & 4 & 4 & 4 $\pm$ 0\\
driverlog(20) & 12 & 12 & 12 $\pm$ 0 & 12 & 12 & 12 $\pm$ 0 & 11 & 11 & 11 $\pm$ 0 & 11 & 11 & 11 $\pm$ 0\\
elevators-opt11(20) & 13 & 13 & 12 $\pm$ 0 & 13 & 13 & 12 $\pm$ 0 & 10 & 10 & 10 $\pm$ 0 & 10 & 10 & 10 $\pm$ 0\\
floortile-opt11(20) & 6 & 6 & 6 $\pm$ 0 & 6 & 6 & 6 $\pm$ 0 & 7 & 7 & 7 $\pm$ 0 & 7 & 7 & 7 $\pm$ 0\\
freecell(80) & 15 & 15 & 15 $\pm$ 0 & 15 & 15 & 15 $\pm$ 0 & 14 & 14 & 14 $\pm$ 0 & 14 & 14 & 14 $\pm$ 0\\
grid(5) & 2 & 2 & 2 $\pm$ 0 & 2 & 2 & 2 $\pm$ 0 & 2 & 2 & 2 $\pm$ 0 & 2 & 2 & 2 $\pm$ 0\\
gripper(20) & 20 & 20 & 20 $\pm$ 0 & 20 & 20 & 20 $\pm$ 0 & 20 & 20 & 20 $\pm$ 0 & 20 & 20 & 20 $\pm$ 0\\
hanoi(30) & 14 & 14 & 14 $\pm$ 0 & 14 & 14 & 14 $\pm$ 0 & 13 & 13 & 13 $\pm$ 0 & 13 & 13 & 13 $\pm$ 0\\
logistics00(28) & 20 & 20 & 20 $\pm$ 0 & 20 & 20 & 20 $\pm$ 0 & 20 & 20 & 20 $\pm$ 0 & 20 & 20 & 20 $\pm$ 0\\
miconic(150) & 72 & 72 & 72 $\pm$ 0.5 & 72 & 72 & 72 $\pm$ 0.5 & 69 & 69 & 69.2 $\pm$ 0.4 & 69 & 69 & 69.2 $\pm$ 0.4\\
mprime(35) & 19 & 19 & 19.3 $\pm$ 0.5 & 20 & 19 & 19.3 $\pm$ 0.5 & 21 & 21 & 21.1 $\pm$ 0.8 & 21 & 21 & 21.2 $\pm$ 0.7\\
mystery(30) & 15 & 15 & 15 $\pm$ 0 & 15 & 15 & 15 $\pm$ 0 & 15 & 15 & 15 $\pm$ 0 & 15 & 15 & 15 $\pm$ 0\\
nomystery-opt11(20) & 18 & 18 & 18 $\pm$ 0 & 18 & 18 & 18 $\pm$ 0 & 16 & 16 & 16 $\pm$ 0 & 16 & 16 & 16 $\pm$ 0\\
openstacks-opt11(20) & 18 & 19 & 18 $\pm$ 0 & 18 & 19 & 18 $\pm$ 0 & 18 & 18 & 18 $\pm$ 0 & 18 & 18 & 17.7 $\pm$ 0.5\\
parcprinter-opt11(20) & 10 & 10 & 10 $\pm$ 0 & 10 & 10 & 10 $\pm$ 0 & 11 & 11 & 11 $\pm$ 0 & 11 & 11 & 11 $\pm$ 0\\
parking-opt11(20) & 1 & 1 & 0.6 $\pm$ 0.5 & 1 & 1 & 0.8 $\pm$ 0.4 & 0 & 0 & 0 $\pm$ 0 & 0 & 0 & 0 $\pm$ 0\\
pathways(30) & 4 & 4 & 4 $\pm$ 0 & 4 & 4 & 4 $\pm$ 0 & 4 & 4 & 4 $\pm$ 0 & 4 & 4 & 4 $\pm$ 0\\
pegsol-opt11(20) & 19 & 19 & 19 $\pm$ 0 & 19 & 19 & 19 $\pm$ 0 & 17 & 17 & 17 $\pm$ 0 & 17 & 17 & 17 $\pm$ 0\\
pipesworld-notankage(50) & 6 & 5 & 5.7 $\pm$ 0.7 & 6 & 5 & 5.9 $\pm$ 0.8 & 9 & 9 & 8.7 $\pm$ 0.5 & 9 & 9 & 8.8 $\pm$ 0.4\\
pipesworld-tankage(50) & 12 & 12 & 12 $\pm$ 0 & 12 & 12 & 12 $\pm$ 0 & 9 & 9 & 9 $\pm$ 0 & 9 & 9 & 9 $\pm$ 0\\
psr-small(50) & 50 & 50 & 50 $\pm$ 0 & 50 & 50 & 50 $\pm$ 0 & 50 & 50 & 50 $\pm$ 0 & 50 & 50 & 50 $\pm$ 0\\
rovers(40) & 8 & 8 & 6 $\pm$ 0 & 7 & 8 & 6.1 $\pm$ 0.3 & 6 & 6 & 6 $\pm$ 0 & 6 & 6 & 6 $\pm$ 0\\
scanalyzer-opt11(20) & 10 & 10 & 9.9 $\pm$ 0.3 & 10 & 10 & 9.8 $\pm$ 0.4 & 7 & 7 & 6.8 $\pm$ 0.4 & 7 & 7 & 6.8 $\pm$ 0.4\\
sokoban-opt11(20) & 18 & 18 & 18 $\pm$ 0 & 18 & 18 & 18 $\pm$ 0 & 19 & 19 & 19 $\pm$ 0 & 19 & 19 & 19 $\pm$ 0\\
storage(30) & 15 & 15 & 15 $\pm$ 0 & 15 & 15 & 15 $\pm$ 0 & 14 & 14 & 14 $\pm$ 0 & 14 & 14 & 14 $\pm$ 0\\
tidybot-opt11(20) & 0 & 0 & 0 $\pm$ 0 & 0 & 0 & 0 $\pm$ 0 & 0 & 0 & 0 $\pm$ 0 & 0 & 0 & 0 $\pm$ 0\\
tpp(30) & 6 & 6 & 6 $\pm$ 0 & 6 & 6 & 6 $\pm$ 0 & 6 & 6 & 6 $\pm$ 0 & 6 & 6 & 6 $\pm$ 0\\
transport-opt11(20) & 7 & 7 & 6 $\pm$ 0 & 7 & 7 & 6 $\pm$ 0 & 6 & 6 & 6 $\pm$ 0 & 6 & 6 & 6 $\pm$ 0\\
visitall-opt11(20) & 9 & 9 & 9 $\pm$ 0 & 9 & 9 & 9 $\pm$ 0 & 9 & 9 & 9 $\pm$ 0 & 9 & 9 & 9 $\pm$ 0\\
woodworking-opt11(20) & 8 & 8 & 8.1 $\pm$ 0.3 & 8 & 8 & 8.1 $\pm$ 0.3 & 7 & 7 & 7.1 $\pm$ 0.3 & 7 & 7 & 7.1 $\pm$ 0.3\\
zenotravel(20) & 12 & 11 & 10.9 $\pm$ 0.3 & 12 & 11 & 10.9 $\pm$ 0.3 & 10 & 10 & 10 $\pm$ 0 & 10 & 10 & 10 $\pm$ 0\\
\end{tabular}
\end{center}

 \caption{
 Coverage results with \mands for computing $f$ and inadmissible distance-to-go heuristics for tie-breaking, on 1104 standard IPC benchmark instances.
 }
 \label{tbl:dtg-mands-ipc} }
\end{table}
