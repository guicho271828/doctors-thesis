
\chapter{Search Diversification for Satisficing Search Algorithms: Intra-vs-Inter plateau Diversification}


%% trying not to mention ``h-value'' or ``g-value'' until it is defined

% \begin{abstract}
% Recent enhancements to greedy best-first search (GBFS) such as  DBFS, $\epsilon$-GBFS, Type-GBFS improve performance by occasionally
% introducing exploratory behavior which occasionally expands non-greedy nodes.
% However, most of these exploratory mechanisms do not address exploration within the space sharing the same heuristic estimate (plateau).
% In this paper, we show these two modes of exploration, which work across (\emph{inter-}) and within (\emph{intra-}) plateau, are complementary, and can be combined to yield superior performance. 
% % 
% We then introduces a new fractal-inspired scheme called Invasion-Percolation diversification,
% % We also introduce IP-diversification, a method combining Minimum Spanning Tree and randomization,
% which addresses  ``breadth''-bias instead of the ``depth''-bias addressed by the existing diversification methods.
% We evaluate IP-diversification for  both intra- and inter-plateau exploration,  and show that it  significantly improves  performance  in several domains.
% % 
% Finally, we show that combining diversification methods results in a planner which is competitive to the state-of-the-art for satisficing planning.
% \end{abstract}

\section{Analysis of Diversification Strategies for Greedy Best First Search}

Many search problems in AI are too difficult to solve optimally, and finding even one satisficing solution is challenging. 
Greedy Best-First Search (GBFS) is a best-first search variant where $f(n)$, the expansion priority of node $n$ is based only on a heuristic estimate of the node, i.e., $f(n) = h(n)$  (in contrast with \astar search, where the node priority also considers $g(n)$, the cost from the start state to $n$, and $f(n) = g(n)+h(h)$).
GBFS has been shown to be quite useful when it is necessary to find some  satisficing solution quickly, and GBFS 
has been the basis for \lsota domain-independent planners. % \cite{domshlak2015red}. % let's intentionally leave this ambiguous and not explicitly cite current sota planners, since we don't want our claims of ``improving upon sota'', ``competitive w. sota'' to be misunderstood as, e.g., ``better than gbfs/redblack''

Despite the ubiquitous use of GBFS for satisficing search,
previous work has shown  that GBFS is susceptible to being 
easily trapped by undetected dead ends and huge search plateaus.
On infinite graphs, GBFS is not even complete \cite{Valenzano2016}
because it could be misdirected by the heuristic guidance forever.
These pathological behaviors are caused by the fact that 
the search behavior of GBFS strongly depends on the
quality of the heuristic function.

%Existing methods tackle this issue from two directions.
Previous approaches to this problem can be classified into two classes.
The first class of methods focuses on an issue arising with inadmissible heuristics,
which can incorrectly label nodes which are close to the goal (low $h^*$, optimal cost to goal) as unpromising (overestimation: $h>h^*$), 
causing GBFS to delay expanding them until all other open nodes with smaller $h$-values have been expanded.
Several approaches have been proposed for alleviating this problem, including 
DBFS \cite{imai2011novel}, $\epsilon$-GBFS \cite{valenzano2014comparison} and Type-GBFS \cite{xie14type}.
These approaches \emph{diversify} the search by occasionally expand nodes which do not have the lowest $h$-value, 
and provide an opportunity to expand nodes that are mistakenly overlooked due to  heuristic errors.

The second class of methods focuses on a different issue which arises in both admissible and inadmissible heuristics:
A node that is far from the goal (high $h^*$) can be mislabeled as promising (underestimation: $h<h^*$), 
causing GBFS to have larger plateaus and expand unnecessary nodes.
Techniques which address this issue include  \emph{plateau escaping} \cite{Coles07}, \emph{local exploration} \cite{XieH14gbfsle} or \emph{tiebreaking} \cite{Asai2016}.

%% this paragraph is important because diversity, exploration, and
%% ``removing bias'' might not immediately connect with each other
All of these methods share the objective of removing some \emph{bias},
thereby encouraging  \emph{exploration} by the search process
and adding \emph{diversity} in decision making process.
In this paper, we use the terms ``exploration'', ``diversity'', and ``bias removal'' interchangeably.
Previous work lacked a common framework which unified these various approaches to diversification/exploration/bias removal.
Furthermore, as shown later, the current \lsota methods are based on diversification with respect to search depth (distance from the start / goal / plateau entrance),
so the bias among the set of nodes with the same search depth is not removed.

In this chapter, we first show that  the above two classes of approaches to diversification are orthogonal and should be combined for better performance.
We show that a recently proposed depth-based tie-breaking strategy for \astar \cite{Asai2016} also improves the performance of GBFS by diversifying the depth within each h-plateau.
Both depth diversification strategy and Type-GBFS are shown to be instances of a \emph{type}-based diversification strategy \cite{xie14type}: Depth diversification applies type-based diversification within a plateau, and Type-GBFS applies it between plateaus.
We compare their empirical performance and show that their improvements are complementary -- 
They improve the performance in different domains, and a configuration using both methods, achieves the best overall coverage.
This effectively shows that \emph{inter}-plateau and \emph{intra}-plateau diversification are two orthogonal usages of diversification, and both modes should be used if possible.
 
Next, we propose and evaluate a new diversification strategy called IP-diversification which addresses diversity with respect to \emph{breadth}.
We evaluate this new diversification strategy both for intra-plateau and inter-plateau exploration.
Complementary effects on intra/inter-plateau exploration were similarly observed. In addition, IP-diversification outperforms the Type-based diversification strategy.
Finally, we show that by combining several intra/inter plateau exploration strategies, we can improve upon \lsota planners in terms of coverage.
% The FD implementation of "Type-LAMA" is not the same as the Jasper code which competed in IPC2014, so can't say anything aobut Type-LAMA IPC performance rank.
%(LAMA, the IPC2011 winner,  and Type-LAMA, the non-portfolio planner with the highest coverage in IPC2014) in terms of coverage.

