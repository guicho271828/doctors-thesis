\section{Intra- and Inter-plateau Diversification}
\label{sec:intra-inter}

Previous works on exploration for GBFS  address the problem of heuristic errors
by occasionally expanding nodes with high $h$.
Since this type of diversification operates  across different search plateaus,
we refer to these as \emph{inter-plateau} exploration.
However, we propose another type of exploration,
which we call \emph{intra-plateau} exploration, which works within a particular plateau.
This type of exploration only changes expansion order among the nodes within a plateau.
We use this new term rather than tiebreaking in order to emphasize its relationship to plateaus.

Existing inter-plateau exploration can be understood as a diversification applied to $h^*$ plateau.
Consider a hypothetical  2-dimensional histogram (\refig{fig:h-hstar}) of the number of nodes for each pair $h,h^*$.
If both axes were $h^*$ (i.e., $h$ is a perfect heuristic), all nodes would be on the diagonal line $x=y$.
However, in reality, $h$ has errors relative to  $h^*$, as would be shown if we projected the histogram to the $x$-axis.
Since low-$h^*$ nodes may have high-$h$ values,
it is sometimes reasonable to expand
high-$h$ nodes depending on the distribution defined by the problem characteristics and the heuristic function.
% To our knowledge, all previous work on exploration for GBFS has addressed exploration along this dimension by ignoring the best-first ordering wrto $h$.

\begin{figure}[bt]
 \centering
 \includegraphics{img/h-hstar.pdf}
 \caption{A conceptual view of the node distribution wrto $h^*$ and inadmissible $h$.
 The peak line on the surface is on $x=y$.
 Projection to $x$-axis shows the distribution of $h$ values, while 
 projection to $y$-axis shows the distribution of $h^*$ values.
 }
 \label{fig:h-hstar}
\end{figure}

However, the converse can also be true --
not only can a single $h^*$-plateau consists of nodes with different $h$ values,
a single $h$-plateau consists of nodes with different $h^*$ values,
as would be shown by projecting the histogram  to the $y$-axis in \refig{fig:h-hstar}.
This leads to an observation that
\emph{in the worst case, a naive algorithm may keep expanding bad (high-$h^*$) nodes within an $h$-value plateau}.

More precisely, the node selection algorithm of a diversified GBFS variant can be described as follows:
\begin{defi}
 An \emph{inter}-plateau diversification strategy for GBFS is a method for selecting the next $h$-value.
\end{defi}
\begin{defi}
 An \emph{intra}-plateau diversification strategy for GBFS is a method for selecting the next node in the plateau selected by an \emph{inter}-plateau strategy.
\end{defi}
This view cleanly separates the effects of two strategies, providing a firm basis for the observation that their effects are orthogonal and should be combined for the better performance.
It is straightforward to see that,
given an OPEN list state and an inter-plateau diversification strategy,
the next h-plateau to select a node from is independent of intra-plateau strategy.
Likewise, given a set of nodes with the same $h$-value and an intra-plateau (tiebreaking) strategy,
the next expanded node is independent of inter-plateau strategy.

Intra-plateau diversification is similar to \emph{local exploration} \cite{XieH14gbfsle,XieMH15}, but is more restrictive. \emph{Local exploration} is targeted for uninformative heuristic region (UHR), which includes both plateaus and local minima. In fact, GBFS-LS does not restrict the local exploration by the $h$-value, thus it may eventually expand a different plateau as the side effect.
Similarly, some existing inter-plateau strategies such as $\varepsilon$-greedy GBFS have some intra-plateau side-effects because they may distort the expansion order within a plateau.

%% this explanation is ambiguous, so I left it out.
% \emph{Unlike inter-plateau exploration, which addresses \textbf{incorrect information for h$^*$-plateau},
% intra-plateau exploration addresses the problem of \textbf{insufficient information for h-plateau.}}
% %An intra-plateau diversification tries to expand low-$h^*$ nodes within an $h$-value plateau
% %with a sufficient probability.
% %% ``trap'' may refer to specific structure defined in ``Traps, invariant and dead-ends'' lipo/muise
% 
%% below is just repeating the previous statements, so left out in favor of space.
% An intra-plateau exploration strategy tries to avoid the aforementioned pathology of continually expanding high-$h^*$ nodes within an $h$-value plateau.
% Since we do not know \emph{a priori} which nodes in an $h$-plateau are better (low-$h^*$) than the other nodes in the same $h$-plateau, by
% an  adversary argument, the safest strategy is to avoid biased choices  -- 
% in the absences of useful heuristic knowledge which differentiates among a set of nodes, 
% an expansion policy which is biased to expand some particular group of states within a plateau can be exploited by an adversary which seeks to hide better (low-$h^*$) nodes.

\subsection{Type-Based Diversification}

The notions of inter-vs-intra plateau exploration allows us to discuss and compare depth diversification \cite{Asai2016} and Type-GBFS \cite{xie14type} within a unified framework
 -- it turns out that
these are essentially the same basic idea,  applied to different contexts (inter-vs-intra plateau, satisficing-vs-optimal search), using different parameters (type systems).

\citeauthor{lelis2013stratified} \citeyear{lelis2013stratified} define a general framework for adding exploration to search using ``type systems'':

%% for future reference, to see how much our definition drifts from the original
%\begin{defi} %Definition, copied from Xie's TypeGBFS paper (2014). This is almost the same as the original def in Lelis et al 2013, except the original def was for a search tree.
%  Let $S$ be the set of nodes in the search space. $T = \{t_1,...,t_n \}$ is a type system for $S$  if $T$ is a disjoint partitioning of $S$. For every $s \in S$, $T(s)$ denotes the unique $t \in T$ with $s \in S$.
%\end{defi}
%They define a type bucket $tb$ which partitions nodes according to their type, and they define type bucket-based node selection as: pick a bucket $b$ uniformly at random from all non-empty buckets. Then, pick a node uniformly at random from $b$.
%Type-GBFS alternatively expands a node from the regular open list $O$ and from this type bucket, where new nodes are added to both $O$ and $tb$.

\begin{defi} 
A \emph{Type system} \cite{lelis2013stratified} is a 
function from a node to a vector,
$T: \text{node} \rightarrow \mathbb{Z}^k, T(n)=\brackets{t_1(n) \ldots t_k(n)} $, where each function $t_i(n)$ returns an integer for each node $n$.
\end{defi}

%% These functions are assumed to represent the characteristic feature of the state.

\citeauthor{xie14type} proposed a node selection technique based on type systems.

\begin{defi}
% \emph{Type-Based Diversification} 
 \emph{Type-Based Node Selection} \cite{xie14type}
with a type system $T(\cdot)$ of $k$ types maintains a $k$-dimensional matrix of sets of nodes,
 where each set $S_v$ is associated with a vector $v=\brackets{v_1,\ldots,v_k}$.
 Each node $n$ is stored in $S_{T(n)}$.
 For dequeueing, it randomly selects a non-empty set from all sets,
 and a random node in the set is dequeued.
\end{defi}

The reason for selecting a set at random is to try to allocate the search effort among a diverse set of nodes.
Some sets could contain a large number of nodes while others are only scarcely populated.
Type-based node selection tries to remove this cardinality bias among buckets.
Because type-based node selection  has this diversification as an explicit goal and is best understood as a diversification strategy, we call it \textbf{\emph{type-based diversification}} in the rest of this paper.

% describe Type-GBFS first to keep the description of all the type-based previous work together.
Type-GBFS \cite{xie14type} uses type-based diversification with type system $\brackets{g,h}$ for inter-plateau
exploration. Their inter-plateau exploration is implemented by queue alternation \cite{RogerH10} between
standard Best-First queue and type-based diversification queue.

Depth diversification \cite{Asai2016} originally addressed the issue of zero-cost actions in admissible search with \astar,
and the configuration was denoted as $[f,h,\depth]$ using the type system notation, where $f=g+h$ and $d$ is a number of steps from the current node to the nearest ancestor that has the different $h$-value.
In order to use $\depth$ for GBFS, the resulting configuration is $[h,\depth]$.
This configuration is considered an instance of intra-plateau type-based diversification because
it uses type-based diversification $\depth$ for diversifying the search within plateaus defined by $h$.

%% removed Type-GBFS with <h>. Type-GBFS with <g,h> turns out to be just sufficient for discussion.
