\section{Background}

\subsection{Exploration Mechanisms}

One class of improvements to GBFS seeks to introduce exploration (diversity) to the search process, as exemplified by DBFS \cite{imai2011novel}, $\epsilon$-GBFS \cite{valenzano2014comparison}, Type-GBFS
\cite{xie14type}.
These algorithms address the problem of GBFS getting stuck due to heuristic errors.
GBFS will not expande a node $n$  until it expands all nodes with a lower $h$-value than $n$.
Thus, search progress can be delayed when a good (low-$h^*$) node is mistakenly assigned a poor (high) $h$-value (overestimation), or bad (high-$h^*$) nodes are assigned promising $h$-values (low-$h$, underestimation).
These exploration strategies allow the search to escape local minima by relaxing the $h$-based best-first node expansion order.

KBFS(k) \cite{felner2003kbfs} attempts to address this problem by expanding $k$ nodes at a time.
% in order to prevent the search algorithm from getting stuck in a heuristic trap.
%  KBFS provided an
% interesting observation to GBFS --- KBFS(1) is equivalent to GBFS, and
% KBFS($\infty$) is equivalent to Breadth-First Search.
% 
% 
$\epsilon$-GBFS \cite{valenzano2014comparison} selects a random node from OPEN with some fixed probability $\epsilon <1$.
This is a randomized, weighted alternating 
OPEN list using $[h,*]$ and $[\ro]$ (no sorting criteria): $\mit{alt}([h,*],[\ro])$.

While $\epsilon$-GBFS relies on  a pure randomization strategy to escape traps and introduce exploration, 
Type-GBFS \cite{xie14type} explicitly seeks to remove bias and diversify the search  by categorizing OPEN according to several key values, such as $[g,h]$ for each state.
Each node is assigned to a bucket according to its key value.
The search then selects a random node in a random
bucket, avoiding the cardinality bias among buckets.
Since Type-GBFS does not sort the buckets according to the key vector, we use a different notation $\brackets{\ldots}$,
such as $\brackets{g,h}$ denoting type buckets whose key values are $g$ and $h$.
In the implementation evaluated by \citeauthor{xie14type} (\citeyear{xie14type}),
Type-GBFS alternates the exploitative (standard best-first order) expansion and the exploratory (randomized) expansion. We denote this 
 as $\mit{alt}([h,*], [\brackets{g,h},\ro])$.

DBFS \cite{imai2011novel} diversifies the
search based on $g$ and $h$ values, but with several key differences from
the above two algorithms: First, the exploratory selection is not uniformly
random, but is subject to a particular distribution function based on 
$h, g, h_{min}$ and $g_{max}$. Second, it uses a local search with
a bounded number of expansions equal to $h$, which dynamically balances the exploration
and exploitation --- it does more GBFS when $h$ is large (far
from the goal), and less GBFS near the goal ($h$ is small).

GBFS with Local Exploration (GBFS-LE) introduces a 2-level search architecture which runs
GBFS until it detects that no improvements have been made for a while, and
then runs a local GBFS (GBFS-LS) or random walk (GBFS-LRW)
in order to find an exit to a more promising region of the search space \cite{XieH14gbfsle}.

\subsection{Tiebreaking}

% To date, there has been relatively little work on 
% tie-breaking policies for BFS.
% 
% Recently, \citeauthor{Asai2016} (\citeyear{Asai2016}) performed an in-depth investigation
% of tie-breaking strategies for \astar, in which
% the tie-breaking policy was found to have a
% significant effect on the performance when the search plateau is huge.
% In the most commonly used  sorting strategies, $[f,h,\fifo]$ or $[f,h,\lifo]$, the search has a strong bias to focus on  either the regions of smaller (\fifo) or larger (\lifo) search depth of each $\plateau{f,h}$, which can cause a failure to find a solution within a given time limit.
% 
% To address the issue caused by the search bias within a plateau,
% they proposed a notion of \emph{depth} and diversified the search over different depths
% within a plateau. The depth $d(n)$ of a state $n$ is an integer representing 
% the step-wise distance from the \emph{entrance} of the plateau (the most recent state which
% entered the plateau, along the path from the initial state). 
% % 
% $d(n)=d(m)+1$ when $n$ and the parent node $m$ are on the same plateau.
% For example, with strategy $[f,h,*]$, 
% $\plateau{f(n),h(n)}=\plateau{f(m),h(m)}$, therefore
% $f(n) = f(m) \land h(n) = h(m)$.
% $d(n)$ is 0 otherwise.
% The nodes are stored in buckets indexed by depth, and expansions are allocated across different buckets with equal probability at every iteration.
% The resulting sorting strategy is denoted as $[f,h,\depth,*]$.

For GBFS, to our knowledge, there is currently no
well-established tie-breaking policy analogous to $h$-based tie-breaking for \astar.
Presumably, this is  because
while \astar has access to three cost values ($f$, $g$, and $h$),
%% better to keep this footnote even without citations, just so we acknowledge that there are some "obvious" tiebreaking strategies.
GBFS is guided solely by the heuristic value $h$.\footnote{Tie-breaking based on  $g$ is sometimes used, but this is motivated as a means to find higher-quality solutions. To our knowledge, in a satisficing context, tie-breaking strategies for reducing search effort have not been explicitly motivated or evaluated.} 
As a consequence,
improvements to  GBFS have been primarily achieved by addressing other aspects, such as
modifying the evaluation scheme \cite[lazy evaluation]{richter2010lama}, queue alternation
(multiple heuristic functions), preferred operators \cite{hoffmann01}, and diversification.

