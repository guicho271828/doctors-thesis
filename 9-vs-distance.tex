% \clearpage
\section{Tie-Breaking Using Distance-to-Go Estimates}

\label{sec:distance-to-go}
In the previous section, we proposed a framework which views cost-optimal \astar search as a series of satisficing searches on each $f$-cost plateau, and argued that 
the problem of tie-breaking can be reduced to a satisficing search.
We showed that the depth diversification tie-breaking criterion which is highly effective on Zerocost domains is in fact a case where a previously studied technique for satisficing search (type-based exploration) turns out to be highly effective when applied to tie-breaking.
In this section, we push this insight further and propose a more general approach to improving the
search performance in plateaus produced by Zerocost domains --
using inadmissible \emph{distance-to-go} estimates (heuristics) as a tie-breaking criterion within an admissible \astar search.

% ; This allows the heuristics to be
% inadmissible when it is only used for tie-breaking strategy.

% The obvious
% drawback of this strategy is the cost of additional heuristic
% computation.

% We next compare our depth diversification to the distance-to-go
% estimates in the plateau.  
Distance-to-go estimates are a class of
heuristics which treat all actions as if they have the unit cost. Even under
the presence of 0-cost actions, those estimates can predict the
number of operations required to reach a goal.
In general, the estimates are inadmissible (unless the estimates are guaranteed to underestimate the number of required actions and all actions in the original domain have unit cost).
Previous work on distance-to-go-heuristics has focused on 
their use for satisficing planning.


% Except for cases where the target domains are unit-cost by origin (e.g. puzzle domains), in which case \astar is already using unit cost estimates without any modification anyways,
$A^*_\epsilon$ \cite{pearl1982studies} is one of
the earliest algorithms that combines distance-to-go estimates with the cost estimates. It is a bounded-suboptimal
search which expands from the \emph{focal} list, the set of nodes with $f(n)\leq w\cdot f_{\mit{min}}$ where weight $w$ serves as a suboptimality bound, similar to weighted \astar, 
 and $f_{\mit{min}}$ is the minimum $f$ value in the OPEN list.  While $f$
is based on an admissible heuristic function, the nodes in the focal list are expanded in increasing order of
an inadmissible distance-to-go estimate $\hh$. Since the search does not follow the best-first order according to $f$, it is 
not admissible, and is instead $w$-admissible. One exception is the case of $w=1$ where the focal list is equivalent
to the $f$ plateau and the expansion order in the focal list corresponds to the tie-breaking on plateaus. In our
notation, this algorithm can be written\footnote{
However, an actual implementation may differ due to dynamic updates to $f_{\mit{min}}$.}
as a BFS with
\[
 [ \lceil \frac{f}{w\cdot f_{\mit{min}}} \rceil, \hh, \mit{default}]
\]
where \textit{default} is one of \fifo, \lifo or \ro.
This notation derives from the fact that the focal list ``blur''s $f$ up to $w\cdot f_{\mit{min}}$.
For example, when $w=2, f_{\mit{min}}=5$ and
$f(n)=5,9,11$, then $\lceil \frac{f}{w\cdot f_{\mit{min}}} \rceil=1,1,2$ respectively. 

Continuing this line of work, \shortciteauthor{thayer2009using} \citeyear{thayer2009using,thayer2011bounded}
evaluated various distance-to-go configurations of Weighted
\astar, Dynamically Weighted \astar \cite{pohl1973avoidance} and $A^*_\epsilon$.
Some configurations use distance-to-go as part of
tie-breaking while they focus on bounded-suboptimal search rather than cost-optimal search.
% 
\shortciteauthor{cushing2010cost} \citeyear{cushing2010cost} pointed out the danger of relying  % neutral tone
on cost estimates in a satisficing search by investigating ``$\varepsilon$-cost traps'' and other pitfalls caused by
cost estimators for search guidance. % Again, this work targets satisficing search. %''satisficing search'' appears in previous sentence.
% 
Finally, a \sota satisficing planner FD/LAMA2011 incorporates distance-to-go estimates in its iterated search
framework \cite{richter2011lama}. The first iteration of LAMA uses distance-to-go estimates combined with various satisficing
search enhancements.

In temporal planning, \shortciteauthor{benton2010g} proposed an inadmissible technique where short actions are
hidden behind long actions and do not increase makespan \cite{benton2010g}. Such actions cause a ``g-value
plateaus'', which are similar to the large plateaus caused by 0-cost actions in sequential planning.  They implemented an inadmissible
heuristic function combined with distance-to-go estimates on top of Temporal Fast Downward system
\cite{eyerich2009using}.  % As stated, their method is not applicable to the cost-optimal search.%true, but they might claim that adapting their technique to admissible search is ``trivial'', so let's just leave this out.

%% I couldnt find this in the paper now... 
%  and showing that (non-admissibly) treating all actions as unit cost sometimes finds an optimal plan quickly.
% Although it could find an optimal plan by chance, this does not guarantee the optimality of the solution.

\subsection{Embedding Distance-to-Go Estimates in Admissible Search}

Although previous work on distance-to-go estimates assume a satisficing context,
we show that distance-to-go estimates can be useful for cost-optimal search.
Since the admissibility of the sorting strategy and the optimality of the solution are not affected by the
second or later levels of sorting criteria, it is possible to use an inadmissible distance-to-go estimate
in these subsequent sorting criteria without sacrificing the optimality of the solution found.
This means inadmissible heuristics can be used for tie-breaking.


Let $h$ be an admissible heuristic function, and
$\hat{h}$ be a distance-to-go variation of $h$, i.e., $\hat{h}$ uses essentially the same algorithm as $h$, except that while $h$ uses the actual action costs for the problem domain, $\hat{h}$ replaces all action costs with 1.
Since $h$ is admissible, multi-heuristic sorting strategies such as $[g+h,h,\hat{h}]$ or $[g+h,\hat{h}]$
are admissible.

Moreover, we can even use a multi-heuristic strategy which uses an inadmissible heuristic for tie-breaking which is unrelated to the primary, admissible heuristic $h$.
 For example, $[g+h^{\lmcut},\ffo]$ is an admissible sorting strategy
because the first sorting criterion $f=g+h^{\lmcut}$ uses an admissible
\lmcut heuristic. Its second sorting criterion, the distance-to-go FF
heuristic \cite{Hoffmann01}, does not affect the admissibility of this entire sorting strategy.

A potential problem with sorting strategies which use multiple heuristics is the cost of computing additional
heuristic estimates. For example, $[g+h^{\lmcut},\ffo]$ requires more time to evaluate each node compared to a standard tie-breaking strategy such as $[g+h^{\lmcut},h^{\lmcut}]$ because computing the $\ffo$ heuristic incurs significant overhead per node while the results of $h^{\lmcut}$ can be reused by a caching mechanism.
When the inadmissible heuristic for tie-breaking is $\hat{h}$, i.e. a distance-to-go (unit cost) variant of the primary, admissible heuristic $h$, it may be possible to reduce this overhead to some extent by implementing $h$ and $\hat{h}$ so that they share some of the computation  -- this is a direction for future work.

\subsubsection{Combining Distance-to-go Estimates with Default Tie-Breaking and Depth Diversification} % added subsection to highlight the new combination of depth + d2go

Tie-breaking using distance-to-go estimates can still leave a set of nodes which are equivalent up to the distance-to-go criterion (multiple nodes can have the same $f$, $h$, and $\hh$ values), so additional level(s) of tie-breaking are necessary in order to select a single node. By adding a standard default criterion such as \fifo, \lifo, \ro, we obtain a sorting strategy that imposes a total order. For example, 
$[f^{\lmcut},\ffo,\fifo]$
 applies \fifo after the distance-to-go estimate $\ffo$.

Furthermore, it is possible to combine depth diversity based tie-breaking with distance-to-go estimates by applying the depth-diversity criterion after the distance-to-go estimate. For example, 
$[f^{\lmcut},\ffo,\depth,\fifo]$ applies depth diversification criterion after the $\ffo$ distance-to-go estimate.
As we shall see below, a sorting strategy which performs tie-breaking using both distance-to-go estimates and depth diversity results in the best performance overall.



\subsection{Evaluation of Distance-to-go Estimates as Tie-Breaking Criteria for Admissible Search}

We tested various admissible sorting strategies on IPC domains and Zerocost domains.
% on Zerocost domains  where tie-breaking strategies have the large impact.
The configurations are listed in \reftbl{list:distance-configs}. 
In all configurations, the first sorting criterion is the $f=g+h$ value
where $h$ is an admissible heuristic (either \lmcut or \mands) using the actual action-cost based  cost calculation.
As the second (and third) criteria,
we used $\hat{h}$, the distance-to-go version tested  of the original heuristic function $h$, as well as a distance-to-go variation of FF
heuristic ($\hat{h}^{\ff}$).
% , and a distance-to-go variation of
% GoalCount heuristic ($\hat{h}^{\text{GoalCount}}$) which is added to
% represent an uninformative but fast inadmissible heuristic function with
% least additional overhead.
We also added configurations with the depth metric within
$\plateau{f,\hat{h}^{\text{\ff}}}$.
A summary of the results is shown in \reftbl{tbl:dtg-summary}.
Detailed per-domain results are shown in \reftbls{tbl:dtg-lmcut-zero}{tbl:dtg-mands-ipc}.

\begin{table}[htbp]
 \centering
 \[
 \begin{array}{lcllcllcl}
  (1)\, [h+g, & h,                      &\mit{default}] & % \\\relax
  (2)\, [h+g, & h,     \quad   \hat{h}, &\mit{default}] & % \\\relax
  (3)\, [h+g, & \hat{h},                &\mit{default}]     \\\relax
  (4)\, [h+g, & \hat{h}^{\ff},          &\mit{default}] & % \\\relax
  % not include it unless the reviewer complained. 
  % (5)\, [h+g, & \hat{h}^{\text{GoalCount}},  &\mit{default}]     \\\relax
  (5)\, [h+g,   & h, \depth,                   &\mit{default}] & % \\\relax
  (6)\, [h+g,   & \hat{h}^{\text{FF}}, \depth, &\mit{default}]     \\\relax
 \end{array}
 \]
 \vspace{-3em}
 \caption{Configurations compared in this section. $h$ is
 either \lmcut or \mands, and $\mit{default}$ is one of the default
 tie-breaking strategies $\fifo,\lifo,$ or $\ro$. }
 \label{list:distance-configs}
\end{table}

\begin{table}[htbp]
 \centering
 
\begin{center}
\begin{tabular}{|l|l|l|l|l|}
\hline
 & Zerocost(620) & Zerocost(620) & IPC (1104) & IPC (1104)\\
 & $h=\lmcut$ & $h=\mands$ & $h=\lmcut$ & $h=\mands$\\
\hline
Non-Depth &  &  &  & \\
$[f,h,\fifo]$ & 271 & 295 & 558 & 491\\
$[f,h,\lifo]$ & 294 & 316 & 565 & \textbf{496}\\
$[f,h,\ro]$ & 276.7 $\pm$ 1.2 & 304 & 560.7 $\pm$ 0.6 & 490 $\pm$ 1\\
$[f,\hat{h},\fifo]$ & 308 & 323 & 534 & 477\\
$[f,\hat{h},\lifo]$ & 316 & 320 & 534 & 475\\
$[f,\hat{h},\ro]$ & 313 $\pm$ 1 & 321.3 $\pm$ 1.5 & 534.7 $\pm$ 1.5 & 471 $\pm$ 1\\
$[f,h,\hat{h},\fifo]$ & 318 & 322 & 536 & 476\\
$[f,h,\hat{h},\lifo]$ & 322 & 321 & 535 & 475\\
$[f,h,\hat{h},\ro]$ & 319.3 $\pm$ 3.2 & 323.3 $\pm$ 1.5 & 534.7 $\pm$ 2.1 & 471.3 $\pm$ 0.6\\
$[f,\ffo,\fifo]$ & 350 & 351 & 564 & 458\\
$[f,\ffo,\lifo]$ & 353 & 346 & 562 & 457\\
$[f,\ffo,\ro]$ & 352 $\pm$ 1 & 351 $\pm$ 1 & 564.3 $\pm$ 1.5 & 456.3 $\pm$ 0.6\\
 &  &  &  & \\
Depth-based &  &  &  & \\
$[f,h,\depth,\fifo]$ & 299 & 317 & 571 & 487\\
$[f,h,\depth,\lifo]$ & 279 & 303 & \textbf{575} & 487\\
$[f,h,\depth,\ro]$ & 303.7 $\pm$ 1.5 & 324.7 $\pm$ 2.3 & 572.3 $\pm$ 1.2 & 485.7 $\pm$ 1.5\\
$[f,\ffo,\depth,\fifo]$ & 353 & 352 & 563 & 457\\
$[f,\ffo,\depth,\lifo]$ & 355 & 348 & 560 & 457\\
$[f,\ffo,\depth,\ro]$ & \textbf{356.7 $\pm$ 1.5} & \textbf{352.7 $\pm$ 0.6} & 563.3 $\pm$ 1.5 & 456 $\pm$ 1\\
\hline
\end{tabular}
\end{center}

 \caption{
 Summary Results: Coverage comparison (the number
 of instances solved in 5min, 4GB) between several sorting strategies.
 For comparison, we also include the results of configurations evaluated in the previous sections.
 }
 \label{tbl:dtg-summary}
\end{table}

\subsubsection{Evaluation on Zerocost Domains}

In Zerocost domains, we see that $\hat{h}$ tie-breaking outperforms $h$ tie-breaking for both \lmcut (e.g. $256\rightarrow 295$ with \fifo) and \mands (e.g. $280\rightarrow 308$ with \fifo).
Also, combining $h$ and $\hat{h}$ can further improve performance when the heuristic is \lmcut (e.g. $295\rightarrow 305$ with \fifo).
The results of combining $h$ and $\hh$ was comparable to $\hh$ when the main heuristic function $h$ is \mands.
% 
Yet more surprisingly, using $\ffo$ further improved the performance for both \lmcut
(e.g. $[f,h,\hh,\fifo]:305 \rightarrow [f,\ffo,\fifo]:337$) and \mands 
(e.g. $[f,h,\hh,\fifo]:307 \rightarrow [f,\ffo,\fifo]:336$).
% 
Thus, when the depth diversity criterion  is not used, the best configurations are those
which use $\ffo$.

% let's avoid using \sota to describe relatively ``old'' heuristics like ff (where ``relatively old'': 10yrs).
The reason for the good performance of $[f^{\lmcut},\ffo,\mit{default}]$ is not surprising:
$\ffo$ is by itself known to be a powerful inadmissible heuristic
function for satisficing GBFS, and 
if we ignore the first sorting criterion, $[f^{\lmcut},\ffo,\mit{default}]$ is a GBFS with $[\ffo,\mit{default}]$.

In detail, following domains are improved by each change:
\begin{itemize}
 \item \textbf{$h^\lmcut \to \hh^\lmcut$}: \pddl{elevator-up} (all), \pddl{freecell-move} (\fifo,\ro), \pddl{mprime-succumb} (all), \pddl{mystery-feast} (\lifo,\ro) \pddl{parking-movecc} (all), \pddl{pipesworld-pushend} (all), \pddl{scanalyzer-analyze} (\lifo,\ro), \pddl{wood\-working-cut} (all). (Compare \reftbl{tbl:lmcut-zerocost-full} and \reftbl{tbl:dtg-lmcut-zero}.)
 \item \textbf{$h^\mands \to \hh^\mands$}: elevator-up (all), \pddl{freecell-move} (\fifo,\ro), \pddl{mprime-succumb} (\ro), \pddl{mystery-feast} (\fifo,\lifo) \pddl{parking-movecc} (all), \pddl{pipesworld-pushend} (all), \pddl{scanalyzer-analyze} (all), wood\-working-cut (all), \pddl{zenotravel-fuel} (\lifo,\ro). (Compare \reftbl{tbl:mands-zerocost-full} and \reftbl{tbl:dtg-mands-zero}.)
 \item \textbf{$\hh^\lmcut \to h^\lmcut,\hh^\lmcut$}: \pddl{airport-fuel} (\fifo,\ro), \pddl{mprime-succumb} (\fifo,\ro), \pddl{parking-movecc} (\lifo), \pddl{pipesworld-pushend} (\fifo), \pddl{scanalyzer-analyze} (all). (See \reftbl{tbl:dtg-lmcut-zero}.)
 \item \textbf{$\hh^\mands \to h^\mands,\hh^\mands$}: comparable. (See \reftbl{tbl:dtg-mands-zero}.)
 \item \textbf{$h^\lmcut,\hh^\lmcut \to \ffo$}: \pddl{blocks-stack} (all), \pddl{freecell-move} (all), \pddl{hiking-fuel} (all), \pddl{miconic-up} (all), \pddl{mprime-succumb} (all), \pddl{parking-movecc} (all), \pddl{pipesworld-pushend} (all), \pddl{sokoban-pushgoal} (all), \pddl{tidybot-motion} (all), \pddl{tpp-fuel} (\ro). (See \reftbl{tbl:dtg-lmcut-zero}.)
 \item \textbf{$h^\mands,\hh^\mands \to \ffo$}: \pddl{airport-fuel} (all), elevator-up (all), \pddl{freecell-move} (all), \pddl{miconic-up} (all), \pddl{mprime-succumb} (all), \pddl{parking-movecc} (all), \pddl{pipesnt-pushstart} (all), \pddl{scanalyzer-analyze} (all), \pddl{sokoban-pushgoal} (all), \pddl{tpp-fuel} (\ro). (See \reftbl{tbl:dtg-mands-zero}.)
\end{itemize}
 
Adding the depth diversity criterion further improves the performance of the $\ffo$-based strategies,
 although the impact was small.
The coverage increased in both
 $h=h^\lmcut$ (\fifo: $337\rightarrow 340$, \lifo: $340\rightarrow 342$, \ro: $341\rightarrow 344.3$) and
 $h=h^\mands$ (\fifo: $336\rightarrow 337$, \lifo: $331\rightarrow 333$).
Improvement was observed in the following domains:
\begin{itemize}
 \item \textbf{\lmcut}: \pddl{mprime-succumb} (\lifo, \ro), \pddl{storage-lift} (\ro), \pddl{tidybot-motion} (\fifo), \pddl{tpp-fuel} (\fifo, \ro). (See \reftbl{tbl:dtg-lmcut-zero}.)
 \item \textbf{\mands}: \pddl{mprime-succumb} (\lifo, \ro), \pddl{tpp-fuel} (\fifo). (See \reftbl{tbl:dtg-mands-zero}.)
\end{itemize}
When the default tie-breaking was \ro and the heuristic is \mands, $[f,\ffo,\depth,\ro]$ performed slightly worse than 
$[f,\ffo,\ro]$, but the difference was very small  ($337.9\rightarrow 337.6$) and $\depth$ made the performance slightly more robust (smaller standard deviation: $2.1\rightarrow 1.3$).

\subsubsection{Evaluation on Standard IPC Domains}

For the standard IPC benchmark instances, the overhead due to the additional computation of
$\hat{h}$ or $\ffo$ tends to harm the overall performance.
% The only domains consistently enhanced by distance-to-go estimates are \pddl{mprime} using \lmcut,
% \pddl{parcprinter-opt11} using \mands and \pddl{woodworking-opt11} using \mands. Moreover, their improvements are small.
Therefore, the best configuration using \lmcut was
$[f,h,\depth,\lifo]$ which uses depth and does not impose the cost of
additional heuristics, and the best result using \mands
was $[f,h,\lifo]$ which imposes no overhead including the depth.

If we look further into the detail, we observed the following:
In \pddl{Cybersec}, distance-to-go variants (e.g. $[f^\lmcut,\ffo,\lifo]$:5) improve upon the standard strategy (e.g. $[f^\lmcut,h^\lmcut,\lifo]$:3), but does not improve upon depth (e.g. $[f,h,\depth,\lifo]$: 12). When $h=h^\mands$, all coverages are zero. Overheads by $\ffo$ also slightly degrade the performance in \pddl{Openstacks} (e.g. $[f^\lmcut,h^\lmcut,\lifo]$:18, $[f^\lmcut,\ffo,\lifo]$:17, $[f^\lmcut,h^\lmcut,\depth,\lifo]$: 18; Also, $[f^\mands,h^\mands,\lifo]$:19, $[f^\mands,\ffo,\lifo]$:18, $[f^\mands,h^\mands,\depth,\lifo]$: 19). Thus, in these two domains, although there are some improvements in search efficiency due to the guidance by $\ffo$ or $\hh$, the runtime overhead of computing the  distance-to-go heuristics outweighed the benefit.
 
In the domains with only positive cost actions (all IPC domains except \pddl{Openstacks} and \pddl{Cybersec}), $\hat{h}$ or $\ffo$
only harm the overall performance due to the overhead.
When the primary heuristics is \lmcut, we do not observe a significant difference between single-heuristics strategies except for the fractional difference in the configurations using \ro.
When the primary heuristic is  \mands, $[f^\mands,h^\mands,\lifo]$ performs slightly better than  other default tie-breaking strategies; It also outperforms the depth-based variants as we already discussed in \refsec{sec:depth-based-evaluation}.
%  (\reftbl{tbl:dtg-lmcut-ipc} and \reftbl{tbl:dtg-mands-ipc}).

\subsubsection{Summary of the Evaluation}
% \pagebreak[3]

Thus, to recapitulate, our conclusions can be summarized in \reftbl{tbl:summary}. We conclude that, although the performance gain by depth diversification and distance-to-go heuristics depend on the domain characteristics, they provide a promising overall performance enhancement.

\begin{table}[htb]
\centering
\begin{tabular}{|p{5em}|p{8em}||p{10em}|p{10em}|}
\hline
 Primary Heuristics & Zerocost domains
                    & Zerocost IPC domains (\pddl{Cybersec}, \pddl{Openstacks})
                    & Positive-cost IPC domains \\
\hline
\lmcut & $[f,\ffo,\depth,\ro]$
       & $[f,h,\depth,\lifo]$
       & $[f,h, \mit{default}]$ or $[f,h, \depth, \mit{default}]$ (any \textit{default} tie-breaking) \\[0.1em]
\hline
\mands & $[f,\ffo,\ro]$ or $[f,\ffo,\depth,\ro]$, but the latter has a smaller variance. 
       & $[f,h,\lifo]$ or $[f,h,\depth,\mit{default}]$ (any \textit{default} tie-breaking)
       & $[f,h,\lifo]$ \\
\hline
\end{tabular}
\caption{Summary of the performance evaluation: Best tie-breaking strategy for each group of domains and each primary heuristic function.}
\label{tbl:summary}
\end{table}

\subsection{Simple Dynamic Configuration for Overall Performance}
\label{sec:dynamic-configuration}

In practice, the performance degradation when using multi-heuristic strategy in domains with only positive cost actions does not pose a problem.
We can easily avoid the overhead incurred by the distance-to-go heuristics in those domains by applying the following simple policy:
If there are any 0-cost actions, use a multi-heuristic strategy; Otherwise, use a single-heuristic strategy.

Since such a check has negligible impact on the total runtime, we can extrapolate the result of applying this rule based on the previously obtained results.
Coverage results in \reftbl{tbl:dtg-summary-sum} show the total coverage of
Zerocost and IPC benchmark domains. The bottom two rows, labeled as \emph{dynamic configuration},
are the extrapolated results when the switching policy is applied -- this dynamic configuration achieves the highest overall coverage.

\begin{table}[htbp]
 \centering \begin{center}
\begin{tabular}{|l|l|l|}
\hline
 & IPC+Zerocost (1724) & IPC+Zerocost (1724)\\
 & \(h=\lmcut\) & \(h=\mands\)\\
\hline
Baselines &  & \\
\([f,h,\fifo]\) & 829 & 786\\
\([f,h,\lifo]\) & 859 & 812\\
\([f,h,\ro]\) & 837.4 & 794\\
\([f,h,\depth,\fifo]\) & 870 & 804\\
\([f,h,\depth,\lifo]\) & 854 & 790\\
\([f,h,\depth,\ro]\) & 876 & 810.4\\
 &  & \\
Distance-to-Go &  & \\
\([f,\hat{h},\fifo]\) & 842 & 800\\
\([f,\hat{h},\lifo]\) & 850 & 795\\
\([f,\hat{h},\ro]\) & 847.7 & 792.3\\
\([f,h,\hat{h},\fifo]\) & 854 & 798\\
\([f,h,\hat{h},\lifo]\) & 857 & 796\\
\([f,h,\hat{h},\ro]\) & 854 & 794.6\\
\([f,\ffo,\fifo]\) & 914 & 809\\
\([f,\ffo,\lifo]\) & 915 & 803\\
\([f,\ffo,\ro]\) & 916.3 & 807.3\\
\([f,\ffo,\depth,\fifo]\) & 916 & 809\\
\([f,\ffo,\depth,\lifo]\) & 915 & 805\\
\([f,\ffo,\depth,\ro]\) & 920 & 808.7\\
 &  & \\
If Zerocost &  & \\
Then \([f,\ffo,\depth,\ro]\) &  & \\
Else \([f,h,\lifo]\) & \textbf{923.7} & \textbf{847.7}\\
\hline
\end{tabular}
\end{center}

 \caption{
 % 
 Summary Results: Coverage comparison, the total of IPC domains and Zerocost domains (the number of instances
 solved in 5min, 4GB) between several sorting strategies, plus a dynamic configuration strategy.  $[f,h,\fifo],
 [f,h,\ro], [f,\hh,\mit{default}], [f,h, \hh,\mit{default}], [f,\ffo,\mit{default}]$ are not shown because they achieve smaller coverage.
 % 
 }
 \label{tbl:dtg-summary-sum}
\end{table}

When the configuration rule is applied to standard IPC instances, the domains with 0-cost actions are \pddl{Cybersec} and \pddl{Openstacks} only. They are solved using a multi-heuristic strategy while other domains are solved in the best performing single-heuristic strategy. In Zerocost instances, all domains are solved using the multi-heuristic strategy.
\todo*{Maybe we should show dynamic configuration results in the detailed per-domain tables, just so that it's obvious that we can get ``dominant'', state-of-the-art  behavior --- results are simulated, not actually implemented. Thus the total results should be fine. However, I added exactly which domains were 0-cost.}

% While these multi-heuristic strategies did not improve the perfomance in
% the positive cost domains because the final plateaus are small, a simple
% method which 


Overall, these results also strengthen our claim that one should not necessarily rely upon $h$-based
tie-breaking in some
domains, as already discussed in \refsec{sec:noh}. In Zerocost domains,
using a distance-to-go version of an inadmissible heuristic function for
tie-breaking is more effective. Also, combining the depth metric with
such an inadmissible heuristics is also effective.

We only tested this relatively simple dynamic configuration that switches between two strategies based on the presence of 0-cost operators. However, as noted in \refsec{sec:depth-based-evaluation}, domain-specific solvers (compared to domain-independent solver, which is our main interest) can benefit from fine-tuning the tiebreaking strategy so that it is most suited to the target domain.

