\subsection{Comparison to Distance-to-Go Estimates}

We next compare our depth diversification to the distance-to-go
estimates in the plateau.  Distance-to-go estimates are the type of
heuristics which treat actions as if they have the unit cost. Even under
the presence of zero-cost actions, those estimates can predict the
number of operations required to reach a goal.
While several papers advocates the benefit of this technique, many of
these claims took place in the context of satisficing planning.

\shortciteauthor{benton2010g} proposed an inadmissible technique for temporal
planning where short actions are hidden behind long actions and do not
increase makespan \cite{benton2010g}. Such actions cause so-called
``g-value plateau'', a similar situation caused by 0-cost actions in
sequential planning.  They implemented an inadmissible heuristic
function combined with distance-to-go estimates on top of
Temporal Fast Downward system.  As stated, their method is not
applicable to the optimising search.
 
% ,wilt2011cost
Similarly, in sequencial satisficing planning,
\shortciteauthor{cushing2010cost} showed that
(non-admissibly) treating all actions as unit cost sometimes finds an
optimal plan quickly \cite{cushing2010cost}.
Although it could find an optimal plan by chance, this does not
guarantee the optimality of the solution.
 
Also, a \sota satisficing planner FD/LAMA2011 incorpolates
distance-to-go estimates in its iterated search framework. The first
iteration of LAMA uses distance-to-go estimates combined with various
satisficing search enhancements.

Still, there is a number of possibilities to adopt the idea of distance-to-go in optimal search.
One key idea is that, as long as the first sorting criteria $f$ uses an ordinary
admissible heuristics, any sorting strategy returns an optimal solution.
Therefore, it is possible to use the
distance-to-go estimates in the sorting criteria of optimal search as
long as it is not the first criteria.
 
Let $\hat{h}$ be a distance-to-go variation of an original heuristic function $h$.
%% redundunt
% Upon computing the estimates, $\hat{h}$ treats all
% actions to have unit costs, while original heuristic function $h$ uses the
% standard action cost.
% 
A multi-heuristic strategy such as $[g+h,h,\hat{h}]$ or $[g+h,\hat{h}]$
returns an optimal solution because the first criteria $f=g+h$ is computed with
a normal cost function.
Moreover, it is even possible to use any inadmissible heuristic function to break
ties. For example, one can use inadmissible FF heuristics to break ties
among $f$ which uses admissible \lmcut heuristics: $[g+h^{\lmcut},\hat{h}^{\ff}]$.

However, an obvious flaw in such a strategy is the cost of computing additional
heuristic estimates. Using different heuristics obviously causes such an
overhead. Using distance-to-go estimates $\hat{h}$ along with
original heuristics $h$ could be optimized so that they share part of
the computation, but still should incur some overhead.

We tested various sorting strategies on IPC domains and zerocost domains.
% on zerocost domains  where tiebreaking strategies have the large impact.
The configurations are listed in \reftbl{list:distance-configs}. 
In all configurations, the first sorting criterion is the $f=g+h$ value
where $h$ is either admissible \lmcut or \mands heuristics.
As the second or later sorting criteria,
we tested $\hat{h}$ of original $h$, as well as a distance-to-go variation of FF
heuristic ($\hat{h}^{\ff}$).
% , and a distance-to-go variation of
% GoalCount heuristic ($\hat{h}^{\text{GoalCount}}$) which is added to
% represent an uninformative but fast inadmissible heuristic function with
% least additional overhead.
We also added configurations with the depth metric within
$\plateau{\hat{h}}$ or $\plateau{\hat{h}^{\text{\ff}}}$.

\begin{table}[htbp]
 \centering
 \[
 \begin{array}{lcll}
  (1)\, [h+g, & h,                           &L] \\\relax
  (2)\, [h+g, & h,     \quad   \hat{h},      &L] \\\relax
  (3)\, [h+g, & \hat{h},                     &L] \\\relax
  (4)\, [h+g, & \hat{h}^{\ff},               &L] \\\relax
  (5)\, [h+g, & \hat{h}^{\text{GoalCount}},  &L] \\\relax
  (6)\, [h+g, & \hat{h}^{\text{FF}}, \depth, &L] \\\relax
 \end{array}  
 \]
 \caption{Configurations which uses the distance-to-go estimates. $h$ is
 one of $\braces{\lmcut, \mands}$, and $L$ is one of last-resort
 tiebreaking strategy $\braces{\fifo,\lifo,\ro}$. }
 \label{list:distance-configs}
\end{table}

% $[h^{\lmcut}+g,\hat{h}^{\mbox{LMcount}},\fifo]$,

The results are shown in \reftbl{tbl:distance-to-go}. We also show the
same results as in the previous sections in the same table. From the
results, we can see that the distance-to-go FF heuristics 
outperforms the standard tiebreaking strategy $[f,h,L]$, as well as
other configurations (1-5). Furthermore, a depth metric further enhanced
the performance of this configuration.

\begin{table}[htbp]
 \centering
 \caption{
 Summary Results: Coverage comparison (the number
 of instances solved in 5min, 2GB) between several sorting strategies.
 }
 \label{tbl:distance-to-go}
\end{table}

