\subsection{Comparison to Distance-to-Go Estimates}

We next compare our depth diversification to the distance-to-go
estimates in the plateau.  Distance-to-go estimates are the type of
heuristics which treat actions as if they have the unit cost. Even under
the presence of zero-cost actions, those estimates can predict the
number of operations required to reach a goal.
While several papers advocates the benefit of this technique, many of
these claims took place in the context of satisficing planning.

\shortciteauthor{benton2010g} proposed an inadmissible technique for temporal
planning where short actions are hidden behind long actions and do not
increase makespan \cite{benton2010g}. Such actions cause so-called
``g-value plateau'', a similar situation caused by 0-cost actions in
sequential planning.  They implemented an inadmissible heuristic
function combined with distance-to-go estimates on top of
Temporal Fast Downward system.  As stated, their method is not
applicable to the optimising search.
 
% ,wilt2011cost
Similarly, in sequencial satisficing planning,
\shortciteauthor{cushing2010cost} showed that
(non-admissibly) treating all actions as unit cost sometimes finds an
optimal plan quickly \cite{cushing2010cost}.
Although it could find an optimal plan by chance, this does not
guarantee the optimality of the solution.
 
Also, a \sota satisficing planner FD/LAMA2011 incorpolates
distance-to-go estimates in its iterated search framework. The first
iteration of LAMA uses distance-to-go estimates combined with various
satisficing search enhancements.

Still, there is a number of possibilities to adopt the idea of
distance-to-go in optimal search. One such example is to use the
distance-to-go estimates as part of sorting criteria.
% 
Let $\hat{h}$ be a distance-to-go variation of an original heuristic function $h$.
%% redundunt
% Upon computing the estimates, $\hat{h}$ treats all
% actions to have unit costs, while original heuristic function $h$ uses the
% standard action cost.
% 
Now, as long as the first sorting criteria is based on $f=g+h$, any
sorting strategy returns an optimal solution. Thus, it is possible to
use a multi-heuristic strategy such as $[f,h,\hat{h}]$ or $[f,\hat{h}]$
which uses inadmissible $\hat{h}$ as part of sorting criteria.
Furthermore, it is even possible to use any inadmissible heuristic function to break
ties. For example, one can use inadmissible FF heuristics to break ties
among $f$ which uses admissible \lmcut heuristic function: $[h^{\lmcut}+g,\hat{h}^{\ff}]$.

There are number of possible combinations here, but the obvious flaw in
such a strategy is the cost of computing additional heuristic
estimates. Using different heuristics obviously causes such an
overhead. Using distance-to-go estimates $\hat{h}$ along with original
heuristics $h$ could be optimized so that it shares part of the
computation, but still should incur some overhead.

We tested various combinations of heuristics on zerocost domains where
tiebreaking strategies have the large impact.
We tested the following configurations:
$[h^{\lmcut}+g,h^{\lmcut},\fifo]$, 
$[h^{\lmcut}+g,h^{\lmcut},\depth,\fifo]$, 
$[h^{\lmcut}+g,h^{\lmcut},\hat{h}^{\lmcut},\fifo]$, 
$[h^{\lmcut}+g,\hat{h}^{\lmcut},\fifo]$, 
$[h^{\lmcut}+g,\hat{h}^{\ff},\fifo]$,
$[h^{\lmcut}+g,\hat{h}^{\mbox{GC}},\fifo]$.

% $[h^{\lmcut}+g,\hat{h}^{\mbox{LMcount}},\fifo]$,
