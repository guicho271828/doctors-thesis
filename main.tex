%%%%%%%%%%%%%%%%%%%%%%%%%%%%%%%%%%%%%%%%%%%%%%%%%%%%%%%%%%%%%%%%
\begin{hidden}
 
\end{hidden}

%%%%%%%%%%%%%%%%%%%%%%%%%%%%%%%%%%%%%%%%%%%%%%%%%%%%%%%%%%%%%%%%

\begin{abstract}
Since the beginning of AI research, the study on one of its core
component, graph search, has long focused on how to guide it using the
heuristic estimate and how to improve the estimates.  One thing that has
been out of focus is the study on tiebreaking, or how to \emph{directly}
improve the search performance on a plateau without changing the
heuristics nor search algorithms.  Such techniques should be done
without help of additional information from the heuristics because
theoretically there is some point that the heuristics cannot be improved
any more while the plateau is still exponentially large.

In this paper, we investigate and improve tiebreaking strategies for
 optimising search using A* and satisficing search using GBFS.
For A*, we first experimentally analyze the performance of common tiebreaking strategies that break ties according to the heuristic value of the nodes.
We find that tiebreaking has a significant impact on search algorithm performance when there are zero-cost operators that induce large plateau regions in the search space.
Next we develop a new framework for tiebreaking based on a depth metric which
 measures the distance from the entrance to the plateau, and propose a new, diversifying strategy which significantly outperforms standard strategies on domains with zero-cost actions. 

We also apply the same strategy to GBFS for satisficing search, where
 currently no obvious tiebreaking rules are proposed. Our depth
 diviersification resulted in an improvement orthogonal to the other
 search enhancements such as Lazy Evaluation, Preferred Operators,
 Multi-heuristic search, and recently proposed Type-Based
 diversification tequeniques.
%both existing planning benchmark domains and newly generated domains with zero-cost actions. %XXX this can be misinterpreted to mean that there is a significant improvement on all kinds of domains, not restricted to domains with 0-cost edges.
%%  % 
%% We investigate various existing myth on tiebreaking
%% strategies and propose simple yet effective methods for improving the
%% search performance within plateau.
%%  % 
%%  % 
%%  They do not depend on any particular heuristic, nor
%%  on multi-heuristic portfolio.
%%  They work even if the heuristic
%%  function no longer provides useful information.
%%  % Moreover, they do not even try to obtain any further information from
%%  % the domain.
%%  We empirically evaluate our strategies against state-of-the-art admissible planner.
\end{abstract}

\section{Introduction}
\subsubparagraph{hidden topic}

aaa

\cite{Asai2016}
