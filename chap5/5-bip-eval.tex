\section{Evaluation of IP-Diversification} 

% in order to minimize the effect of domain configuration problem
% \cite{vallati2015effective} (e.g. successor ordering caused by the way actions are described in PDDL).

% \subsubsection{Evaluation under Eager GBFS}

Given the performance of blind search,
IP-diversification is a good candidate for improving the performance of diversified heuristic search.
% In this section, we evaluate the effects of type-based diversification and IP-diversification on Eager GBFS, where nodes are evaluated as soon as they are generated.
% Since the effect of diversification could be heuristic-dependent, we tested $\ff$ and $\cg$.
% Following the previous work \cite{valenzano2014comparison,xie14type}, all configurations are evaluated under unit cost transformation
%  because we are solely focused on the coverage and the first solution.
% This is natural because GBFS has no guarantee on the plan cost.
% Also, according to \cite[p6, left, middle]{xie14type}, Type-GBFS does not incur significant loss in solution quality.
% 
% Finally, since depth diversification works only within plateaus, it does not change the best-first order wrto $h$
% value and it is unlikely to have significant change in solution quality except for some noise from the randomness
% in tiebreaking.
% 
%% not included --- its is a pseudo heuristics anyways
% , Landmark-Count (LC) heuristics\cite{richter2008landmarks}
% 
We compared the performance of 
(h), the standard GBFS, with
the combined Type-based diversification (hdD) from \refsec{sec:gbfs-comparison}
as well as intra-plateau  IP-diversification (hb:$[h,\bip]$), inter-plateau IP-diversification (hB:$alt([h],[\bip])$), and combined intra/inter-plateau IP diversification (hbB:$alt([h,\bip],[\bip])$).
% All configurations uses \fifo default tie-breaking which is the default configuration in Fast Downward,
% except Type-bucket in Type-GBFS (which uses \ro, as specified in their paper).
%(\textbf{See \reftbl{tbl:lazy-supplemental} in supplements for Lazy-GBFS results, where heuristic value of parents are used.})
% (\textbf{See Supplemental \reftbl{tbl:eager-supplemental} for comparisons to the other default tie-breaking methods (\lifo and \ro). \reftbl{tbl:lazy-supplemental} also contains Lazy-GBFS results, where heuristic value of parents are used}.)
\todo*{\ro GBFS evaluation}

\begin{table}[htbp] 
\setlength{\tabcolsep}{0.2em}
\centering
\begin{tabular}{|ll|r|rrr|r|r|rrr|r|}
% \begin{tabularx}{\linewidth}{|ll|l|l|lll|l|l|lll|}
\hline
& & \multicolumn{ 5}{c|}{$\cg$} & \multicolumn{ 5}{c|}{$\ff$} \\ 
   &       & h   & {hb}           & {hB}           & {hbB}                & {hdD}                & h   & {hb}           & {hB}           & {hbB}          & {hdD}         \\ 
   &       &     & {intra}        & {inter}        & {both}               & {both}               &     & {intra}        & {inter}        & {both}         & {both}        \\ \hline
%  &total  & 228 & 232.5          & 261.8          & \textbf{268.5}       & 263.2                & 231 & 261.5          & 297.7          & \textbf{307.4} & 280.6         \\ \hline \multirow{14}{0.4em}{\rotatebox{90}{\textbf{\relsize{-1}IPC2011}}}
   &total  & 187 & 187.2          & 206.8          & \textit{208.7}       & \textbf{215.8}       & 192 & 207.8          & 232.9          & \textbf{237.7} & 223.9         \\ \hline \multirow{8}{1em}{\rotatebox{90}{\textbf{\relsize{-1}IPC11 w/o duplicates}}}
%  &barman & 0   & 0              & 0              & 0                    & 0                    & 8   & 8.1            & \bred{16.2}    & \bred{15.2}    &         10.3  \\ 
   &elevators & 9   & 9.2            & \bred{12.6}    & \bred{13.3}          & \textbf{9.7}         & 19  & 18.2           & 18.5           & 19.4           & 13.7          \\ 
 % &floortile & 0   & 0              & \bred{0.6}     & \bred{0.5}           & \textbf{2}           & 6   & 4.6            & 5.6            & 5.2            & \textbf{6.9}  \\ 
   &nomystery & 7   & 6.4            & 5.5            & 5.6                  & \textbf{15.1}        & 9   & 6.6            & 7.6            & 6.6            & \textbf{17}   \\ 
 % &openstacks & 10  & \borange{12}   & \borange{13.5} & \borange{13.6}       & 12.8                 & 11  & \borange{19.2} & \borange{17.7} & \borange{19}   & 15.8          \\ 
   &parcprinter & 20  & 19.6           & 13.7           & 12.4                 & 18.7                 & 20  & 20             & 19.9           & 18.9           & 20            \\ 
 % &parking & 18  & \borange{19.2} & \borange{19.4} & \borange{19.9}       & 14.1                 & 11  & \borange{16.8} & \borange{13.3} & \borange{16.9} & \textbf{17.2} \\ 
   &pegsol & 20  & 20             & 19.7           & 19.8                 & 20                   & 20  & 20             & 20             & 20             & 20            \\ 
   &scanayzer & 20  & 20             & 20             & 20                   & 20                   & 15  & 16.6           & \bred{19.1}    & \bred{19.1}    & \textbf{18.6} \\ 
   &sokoban & 16  & 15.9           & 15.8           & 15.2                 & \textbf{17}          & 19  & 18.6           & 18.5           & 18.4           & 17.4          \\ 
   &tidybot & 16  & \borange{17.3} & \borange{17.5} & \borange{17.5}       & \textbf{18.6}        & 16  & 15             & 16.4           & {16.3}         & \textbf{16.7} \\ 
 % &transport & 10  & 10.2           & \bred{11.5}    & \bred{15.6}          & \textbf{12.1}        & 0   & 0              & {0.2}          & \textbf{1.3}   & 0.2           \\ 
 % &visitall & 3   & 3.9            & \bred{10}      & \bred{10.2}          &         6.4          & 3   & \borange{5}    & \borange{11.8} & \borange{12.1} & 6.3           \\ 
   &woodworking & 2   & 1.8            & \bred{14}      & \bred{12.8}          &         \textbf{7.7} & 2   & 1.5            & \bred{14.8}    & \bred{15.7}    & \textbf{7.2}           \\ \hline \multirow{14}{1em}{\rotatebox{90}{\textbf{\relsize{-1}IPC14}}}
   &barman & 0   & 0              & 0              & 0                    & 0                    & 0   & 0              & \bred{7.6}     & \bred{6.5}     & \textbf{1}             \\ 
   &cavediving & 7   & 7.1            & 7              & 6.9                  & 7                    & 7   & 7              & 7              & 7              & 7.2           \\ 
   &childsnack & 1   & 0              & 0.1            & 0                    & \textbf{1.5}         & 0   & 0              & 0.1            & 0              & 0.3           \\ 
   &citycar & 0   & {0.2}          & \textbf{1.1}   & {0.4}                & \textbf{4.7}         & 0   & 0              & \bred{3}       & \bred{3.8}     & \textbf{7.1}  \\ 
   &floortile & 0   & 0              & \bred{0.5}     & \textcolor{red}{0.2} & \textbf{2}           & 2   & 2              & 2.1            & 2              & 2.1           \\ 
   &ged    & 0   & 0              & \bred{4.8}     & \bred{4.6}           & \textbf{9.7}         & 19  & 19.2           & 12.8           & 13             & 13.8          \\ 
   &hiking & 18  & 15.9           & \bred{18.7}    & \bred{18.8}          & \textbf{19.7}        & 20  & 17.6           & 19.9           & 20             & 20            \\ 
   &maintainance & 16  & 14.6           & 14.9           & 14.1                 & 15.8                 & 11  & 6.7            & 10             & 5.8            & 11.1          \\ 
   &openstacks & 0   & 0.1            & \bred{2.5}     & \bred{2.4}           &         \textbf{0.5} & 0   & \borange{15.7} & \borange{11.7} & \borange{14.5} & \textbf{7}             \\ 
   &parking & 7   & \bblue{10.4}   & 7.6            & \bblue{10.9}         & 4.1                  & 4   & \bblue{5.4}    & 2.3            & \bblue{4.8}    & \textbf{5.7}  \\ 
   &tetris & 18  & \bblue{19.7}   & 17.6           & \bblue{19.4}         & 14.3                 & 1   & \borange{8.6}  & \borange{7}    & \borange{11.1} & \textbf{4.9}           \\ 
   &thoughtful & 5   & 4.9            & 5.2            & 5.2                  & 5                    & 8   & \borange{9.1}  & \borange{11.2} & \borange{11}   & \textbf{13.1} \\ 
   &transport & 5   & 4.1            & \bred{6}       & \bred{7.1}           & 4.7                  & 0   & 0              & 0              & 0              & 0             \\ 
   &visitall & 0   & 0              & \bred{2}       & \bred{2.1}           & 0                    & 0   & 0              & \bred{3.4}     & \bred{3.8}     & 0             \\ 
\hline
% \end{tabularx}
\end{tabular}
\caption{
Number of solved instances (5 min, 4Gb RAM), mean of 10 runs.
\textbf{h}: baseline GBFS.
\textbf{hb/hB}: intra / inter-plateau IP diversification $[h,\bip]$ and $alt([h],[\bip])$,
\textbf{hbB}: A combined IP configuration $alt([h,\bip],[\bip])$,
\textbf{hdD}: $alt([h,\brackets{d}],[\brackets{g,h},\ro])$ (same as hdD from \reftbl{tbl:results})  . % confusing because hdD is not IP-based
The same highlighting/coloring rules as \reftbl{tbl:results} are applied, showing that
% Results follow the same observation made in \reftbl{tbl:results}.  
intra/inter-plateau schemes based on IP are complementary.
\textbf{bold} shows the improvements by \textbf{hdD}.
Although \textbf{hbB} and \textbf{hdD} are comparable overall,
per-domain comparison shows \textbf{hbB} and \textbf{hdD} are complementary.
}
\label{tbl:results2}
\end{table}


% In order to remove the effect of randomness of the algorithm, we
% implemented a deterministic version of Type-based queue for Type-GBFS
% which, instead of selecting a bucket at random, iterates over the
% buckets in a reverse order that each bucket is introduced.
% Similarly, the diversification based on the depth is not randomized but
% is implemented as a loop-based implementation.

Results are shown in \reftbl{tbl:results2}. IP-diversification, applied to both
intra- and inter-plateau exploration, resulted in improvements on both the $\ff$ and $\cg$ heuristics.
Complementary effects similar to \reftbl{tbl:results} are observed between hb and hB, and  hbB outperforms both hb and hB.
This provides additional empirical evidence for the hypothesis that intra/inter-plateau exploration  are complementary, and that they can be combined to yield superior performance.

Overall, hbB performs comparably to hdD. However, note that some domains were improved by Type-based but not by IP (e.g. \pddl{nomystery, sokoban, childsnack})
or vise versa (\pddl{transport, visitall}).
% 
% For example, with $\cg$, (hb) outperforms (hd) on \pddl{parking14} and \pddl{tetris14}, while
% \pddl{openstacks14} and \pddl{childsnack14} are solved only by (hd) and other configurations have 0 coverage.
% Similarly with $\ff$, while (hb) improves performance on \pddl{visitall11} ($3.0\rightarrow 5.3$), (hd) does not, and
%  and vice versa on \pddl{childsnack14} ($0.0\rightarrow 4.0$ by (hd)).
% 
These results indicate that Type-based and IP diversification are orthogonal,
addressing different diversity criteria (depth vs breadth).
