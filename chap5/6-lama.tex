\section{Intra- and Inter-Plateau Diversification on a State-of-the-Art Planner} % \sota kind of risky, but this time, I think we justfified usage of \sota  better by relating it to Jasper and Probe
% separating this out to once again remind readers of the intra-inter contribution

Up to this point, we have evaluated intra/inter-plateau exploration on greedy best-first search in order to cleanly isolate their effect.
Next, we evaluate the combined effect of intra/inter-plateau exploration when applied to a state-of-the-art planner,
% (\textbf{See Supplemental \reftbl{tbl:lama-supplemental} for configurations not listed in \reftbl{tbl:results}.})
% 
the LAMA2011 configuration in the current version of FastDownward,
which incorporates a number of search enhancement techniques such as lazy evaluation, multi-heuristic search and preferred operators.
In order to focus on coverage, we only run the first iteration (unit-cost GBFS) of LAMA, 
denoted as $alt([\ff],\pref{\ff},[\lc],\pref{\lc})$, where
% $\ff$ denotes the FF heuristics,
$\lc$ denotes the landmark-count heuristic
and $\pref{X}$ denotes the preferred operator queue with sorting strategy $X$.


% As additional baselines, we also include coverages for Probe \cite{LipovetzkyG11} and Jasper \cite{xie14ipc}, which are \sota, non-portfolio planners from the IPC'14 agile track, 
% Jasper combines Type-LAMA with GBFS-LS \cite{XieH14gbfsle}, which starts a local GBFS when 
% the number of expansion in the current local minima exceeds a certain threshold.
% %While this method address the problem of large plateau, it does not address the diversity
% %within plateau.
% %Also, since Jasper is based on an older LAMA, %
% We also include the original LAMA'11 implementation.
% %In our environment, Probe and Jasper did not outperform the original LAMA'11.
% %This is presumably due to the difference in the CPU speed (AMD@2.36GHz in IPC14 vs our Intel Haswell@2.9GHz), resulting in more search being performed in 5 minutes on our machines vs. the machines used in the IPC14 agile track competition.
% The latest FastDownward LAMA configuration performs better significantly better than the other baselines.


% Since our version is based on the latest FastDownward, we also tested the original version used in IPC2011.
% 
% Type-LAMA \cite{xie14type} 
% adds Type-based buckets $\brackets{g,h}$ to the list of alternating queues in LAMA, and is denoted as: 
% $alt\left([\ff],\ldots,\pref{\lc},\brackets{g,\ff}\right)$.

% We followed the best configuration suggested by \citeauthor{xie14type} (\citeyear{xie14type})
% for Type-GBFS, which uses the type vector $\brackets{g,\ff}$.

We apply the methods proposed in this paper incrementally.
We first add a single exploration strategy to LAMA.
(d, b) augments $[h]$ with type-based and IP diversification for intra-plateau exploration ($[h,\depth]$ and $[h,\bip]$), respectively.
(D, B) incorporates inter-plateau exploration by adding $\brackets{g,\ff}$ and $[\bip]$ to LAMA's alternation queue, respectively.
LAMA+D is equivalent to Type-LAMA \cite{xie14type}.
Next, we combine intra/inter-plateau diversification methods:
(dD) applies both changes in (d) and (D), and similarly (bB) applies both changes in (b) and (B).

Finally, (db$^2$DB) incorporates all 4 methods into LAMA.
Let $db$ denote $alt(\depth,\bip)$, alternation between depth and IP based diversification for intra-plateau exploration,
and let $DB$ denote $alt(\brackets{g,\ff},\bip)$, alternation between type-based and IP based diversification for inter-plateau exploration.
The resulting configuration,
LAMA-db$^2$DB, incorporates all of the ideas proposed in this paper: {\small $alt\big([\ff,db], \pref{\ff}, [\lc,db], \pref{\lc}, DB\big)$.}
This configuration alternates between type-based and IP diversification in each iteration.
It allocates 1/5 of the entire search time to inter-plateau exploration 
(same as the frequency with which Type-LAMA selects from $\brackets{g,\ff}$),
so it spends 1/10 of the time on $[\bip]$ and 1/10 of the time on $\brackets{g,\ff})$.
Adopting more sophisticated approaches for determining exploration frequency \cite{schulte2014balancing,nakhost2009monte} is a direction for future work.
%\citeauthor{schulte2014balancing} (\citeyear{schulte2014balancing}) combined Tree-search framework
%\cite{keller2013trial} with UCT \cite{kocsis2006bandit} to automatically adjust the ratio of exploration %in Classical Planning.
%Similarly, Arvand \cite{nakhost2009monte} uses Random Walk Local Search for exploration,
%with sophisticated mechanism for balancing the ratio. %Since our LAMA-Btdb uses a trivial approach for determining the exploration ratio (fixed to 1/5),
%Adopting these exploration-ration tuning approaches is a direction for future work.

% supplement?
%Since this experimental setting (satisficing, 5min limit) is similar to the Agile track of the IPC, 
% we also include results for Probe \cite{LipovetzkyG11}, which had the highest coverage in the 2014 IPC Agile track, as another baseline.


%% removed ``once every 5 expansions to be the same as TGBFS'' both because it was unclear what it meant, and also, it wasn't clear why it's desirable to match the Type-GBFS exploration rate.
%in order to explore at the same rate as Type-GBFS, we
%(once for every 5 expansions).

% \begin{center}
\begin{tabular}{lr}
 & coverage\\
\hline
Probe & 319\\
LAMA & 386.2\\
Type-LAMA & 387.3\\
LAMA-bB & \\
LAMA-dD & \\
LAMA-dbBD & 397.5\\
LAMA-bdbdBD & 398.3\\
\end{tabular}
\end{center}

\begin{table*}[htbp]
\setlength{\tabcolsep}{0.4em}
\centering
\begin{tabular}{ll|llllllll}
% \begin{tabularx}{\linewidth}{ll|ccc|llllllll}
\hline
 &            & \multicolumn{8}{c}{Planners Based on the Latest FastDownward} \\
 % 
 &            & LAMA   & +{d}           & +{D}             & +{dD}            & +{b}           & +{B}             & +{bB}          & +{db$^2$DB}      \\ 
 &total       &{293.2} & {296.5}        & {294.3}          & {295.4}          & {293.3}        & {287.6}          & {297.6}        & {\textbf{304.5}} \\ \hline \multirow{8}{1em}{\rotatebox{90}{\textbf{IPC11 w/o duplicates}}}
 &elevators   &20      & 19.3           & 19               & 19.2             & 20             & 19.4             & 19.9           & 19.6             \\ 
 &nomystery   &10      & 9.9            & \bred{17.4}      & \bred{16.4}      & 9.8            & 10.4             & 9.7            & \bred{16.1}      \\ 
 &parcprinter &20      & 18.4           & 19.9             & 19.7             & 18.2           & 19.5             & 18.3           & 19.3             \\ 
 &pegsol      &20      & 19             & 20               & 20               & 19.4           & 20               & 20             & 20               \\ 
 &scanalyzer  &19      & 19.3           & 19.1             & 19.2             & \borange{19.5} & \borange{19.6}   & \borange{19.5} & 19.2             \\ 
 &sokoban     &17      & 16.9           & 16.9             & 16.6             & 16.4           & 17               & 16.9           & 16.2             \\ 
 &tidybot     &16      & \textbf{17}    & 15.8             & 15.8             & 14.8           & 15.7             & \textbf{16.5}  & \textbf{16.5}    \\ 
 &woodwork    &20      & 20             & 20               & 20               & 20             & 20               & 20             & 20               \\ \hline \multirow{14}{1em}{\rotatebox{90}{\textbf{IPC14}}}
 &barman      &15      & 13.6           & 9.5              & 10.4             & 12.1           & \textbf{16}      & 14.2           & 14               \\ 
 &cavediving  &7       & 7              & 7.1              & 7.1              & 6.8            & 6.9              & 6.7            & 7                \\ 
 &childsnack  &0       & \textbf{9.3}   & 0.1              & 0                & 0.2            & 0.3              & 0.1            & 0                \\ 
 &citycar     &2       & 1              & \borange{5.5}    & \borange{4.4}    & \borange{4.5}  & \borange{4.2}    & \borange{4.1}  & \borange{4.4}    \\ 
 &floortile   &2       & 2              & 2.1              & 2                & 2              & 2                & 2              & 2                \\ 
 &ged         &20      & 20             & 20               & 20               & 20             & 20               & 20             & 20               \\ 
 &hiking      &18.5    & 18.7           & 17.5             & 18.7             &  \bblue{19.1}  & 17.5             &  \bblue{19.6}  & 18.8             \\ 
 &maintenance &1       & 1              &   \bred{5.5}     &   \bred{5.6}     & 1              & 1                & 1              &   \bred{3.6}     \\ 
 &openstacks  &20      & 20             & 20               & 20               & 20             & 20               & 20             & 20               \\ 
 &parking     &19.1    & \textbf{19.8}  & 16.7             & 18.7             & \textbf{19.6}  & 18.1             & 18.7           & \textbf{19.6}    \\ 
 &tetris      &9.3     & 7.1            & 7.4              & 7.1              &  \bblue{12.4}  & 4.7              &  \bblue{15.3}  &  \bblue{14.2}    \\ 
 &thoughtful  &14      & \borange{14.5} &   \borange{15.1} &   \borange{15.4} & 13.1           &   \borange{14.5} & 12.9           &   \borange{14.6} \\ 
 &transport   &3.3     &  \bblue{3.8}   & 2.6              &  \bblue{3.8}     &  \bblue{4.4}   & 3.7              &  \bblue{3.8}   & 3.5              \\ 
 &visitall    &20      & 18.9           & 17.1             & 15.3             & 20             & 17.1             & 18.4           & 15.9             \\ \hline
\end{tabular}
\caption{
Number of solved instances in 5min,4GB RAM.
 % is based on a newer version of FastDownward and its
LAMA's sorting strategy is $alt([\ff],\pref{\ff},[\lc],\pref{\lc})$.
For each heuristic $h=\ff$ and $h=\lc$ in LAMA,
 % +d, +D, +dD, +b, +B, +bB adds the exploration:
(d,b) augments $[h]$ with type-based and IP diversification for intra-plateau exploration ($[h,\depth]$ and $[h,\bip]$, respectively).
%(D=Type-LAMA \cite{xie14type}, B) uses respective diversification for inter-plateau exploration by adding $\brackets{g,\ff}$ and $[\bip]$ to LAMA's alternation queue.
(D,B) applies inter-plateau exploration by adding $\brackets{g,\ff}$ and $[\bip]$ to LAMA's alternation queue, respectively. D corresponds to Type-LAMA \cite{xie14type}.
(dD) includes both changes in (d) and (D) (similarly for (bB), (b) and (B)).
Finally, (db$^2$DB) combines all methods:
{\small $alt\big([\ff,alt(\depth,\bip)], \pref{\ff}, [\lc,alt(\depth,\bip)], \pref{\lc}, alt(\brackets{g,\ff},\bip)\big)$.}
The same highlighting rules as \reftbl{tbl:results} are applied.
LAMA+db$^2$DB combines improvements from 4 diversification strategies and achieved the best overall coverage.
% In the right columns, Jasper'14, Probe'14, LAMA'11 are the original versions used in IPC'14/'11.
}
\label{tbl:lama}
\end{table*}
% & Jasper'14 & Probe'14 & LAMA'11 
% & 248       & 241      & 271     
% & 20        & 16       & 19      
% & 4         & 6        & 11      
% & 6         & 14       & 20      
% & 20        & 20       & 19      
% & 16        & 17       & 20      
% & 16        & 13       & 16      
% & 18        & 17       & 16      
% & 19        & 20       & 20      
% & 20        & 20       & 20      
% & 6         & 1        & 7       
% & 0         & 0        & 0       
% & 5         & 9        & 1       
% & 0         & 2        & 2       
% & 18        & 19       & 20      
% & 19        & 20       & 6       
% & 12        & 14       & 6       
% & 18        & 0        & 20      
% & 0         & 6        & 4       
% & 6         & 1        & 7       
% & 15        & 16       & 16      
% & 0         & 6        & 8       
% & 10        & 4        & 13      

\reftbl{tbl:lama} shows the number of solved instances.
Each single diversification improved the overall performance of LAMA except LAMA+B.
For combinations of two methods (dD and bB),
complementary effects by intra-/inter-plateau diversification similar to \reftbl{tbl:results} are observed.
Although LAMA+B did not result in improvement, adding B to LAMA+b resulted in larger coverage in LAMA+bB.
Finally, bd$^2$BD outperformed all other methods.
We observed complementary effects from dD and bB, each addressing different diversity criteria.
% Type-based exploration sometimes significantly degrades LAMA performance when the heuristic is correct but ignored in favor of the type-based queue. 

%\section{Related Work} % let's remove the Related Work section because right now, the contents are 100% IP-related, and makes it look as though IP should be viewed as the most important contribution.

% Valenzano14 baseline discussion moved to IP section.

% too dangerous because there's no clear example of a "continuous space" which is clearly better suited for IP than novelty. replaced below in conclusions+future work with a more positive type of mention.
% % "novelty metric" itself needs to be mentioned, although we cant cite BWFS ...
% While novelty metric used in Probe \cite{LipovetzkyG11} or IW \cite{lipovetzky2012width} also addresses diversity of a new node from existing nodes, one drawback is that it assumes a propositional representation of the problem. As such, currently there is no straightforward way to compute the novelty metric on domains where no such representation is available. In contrast, IP-diversification only assumes the graph-based representation of the explored problem space and therefore, for example, the same method can be applied to the sampled nodes in a continuous space without modification, given the background theories on Percolation Physics on continuous space.
% The analysis in this direction is future work.

% removed 30 years
%  (\emph{real world is continuous})

% \section{Related Work}
% 
% Other techniques that can be viewed as addressing breadth-biases are
% symmetry breaking \cite{Fox1998,pochter2011exploiting,domshlak2013symmetry}
% and partial order reduction \cite{hall2013faster,wehrle2013relative}.
% While these are usually described as ``pruning techniques'',
% they can be viewed as diversification techniques, in that 
% \emph{they aim to remove the bias for a set of redundant nodes.}
% % 
% Suppose we have a set of nodes $S=\{a_1, a_2, b, c\}$ where the subset
% $A=\{a_1, a_2\}$ is ``redundant'' according to some measure (e.g. by symmetry, partial order). 
% A search algorithm which selects a node randomly from $S$ for expansion is clearly biased -- 
% group $A$ is twice as likely to be expanded than either $b$ or $c$, although $a_1$ and $a_2$ are ``same'' nodes.
% By eliminating either $a_1$ or $a_2$, symmetry and partial-order reduction eliminates this bias.


% Seems safer to remove this, because Lelis et al 2013 is applied to sliding tiles, blocksworld, pancake, so claiming their work is for trees doesn't look right.
%% The previous work on Type system \cite{lelis2013stratified} has also used the number of children with particular
%% heuristic value. Although this could be another direction to address the breadth-related bias, our work is
%% significantly different --- First, we focus on the search on a general graph rather than a tree. Second, they claim
%% ``type system can use any information about a node to define its type'', without further explanation why this
%% metric works.

