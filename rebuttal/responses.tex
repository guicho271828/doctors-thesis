% Created 2016-11-12 土 21:24
\documentclass{article}
\usepackage[utf8]{inputenc}
\usepackage[T1]{fontenc}
\usepackage{textcomp}
\usepackage{marvosym}
\usepackage{amsmath}
\usepackage{amssymb}
\usepackage[dvipdfmx]{hyperref,graphicx}
\usepackage{helvet,times,courier}
\tolerance=1000
\usepackage[pxjahyper]{}
\usepackage[margin=20mm]{geometry}
\author{Masataro Asai}
\date{}
\title{Responses}
\begin{document}

\maketitle
First, thanks very much to all of the reviewers for their detailed and helpful comments.
To address the reviewers' comments, we have modified the paper as described below.

To reduce the length of the main portion of the paper, we moved many tables to the Appendix, 
eliminated some tables and consolidated some figures.
Now the paper length excluding references and appendix is 39 pages.
The numberings of the sections, figures, and tables have changed due to changes made in response to the reviewers' comments.

In this response, when disambiguation is required, we use [Revised:X] to indicate that we refer to those figures/tables/sections using
the new numbering system, and [Original:X] to indicate the numbers in the original submitted manuscript.

\section{Reviewer 1}
\label{sec:orgheadline13}

\subsection{[1.1] 1) The definition of plateau can be improved.}
\label{sec:orgheadline1}

\begin{quote}
The definition of plateau can be improved. The formal definition of
a plateau should include a plateau with respect to some cost
functions. So, you have a plateau with respect to f, plateau(f), a
plateau with respect both g and h, plateau(g,h) etc. This should be
defined at the beginning and this terminology should be later used all
over. Currently, you are not fully formal and consistent on this. As
you say on page 18. a plateau is related to a sorting strategy. Define
this in the beginning. In fact you write the following sentence which
is in fact a definition: "Having the same key values means that n and
m are on the same plateau". Move this to be a definition in one of the
first sections."
\end{quote}

Plateaus are formally defined in the 2nd-to-last paragraph of Section 2, Preliminaries.
However, as the reviewer noted, there were some informal usages of the term "plateau" throughout the paper.
In the revision, we increased the usage of the more precise notation "plateau(f,h)" throughout the paper.

\subsection{[1.2] 2) page 18: "In order to diversify the expansion\ldots{} " --> This paragraph is very hard to understand}
\label{sec:orgheadline2}

\begin{quote}
page 18: "In order to diversify the expansion\ldots{} " --> This
paragraph is very hard to understand but it is a very important
paragraph as it gives the pseudo code for your new technique. Please
rewrite it. What is Dc? Is this a counter? Is there one Dc or one for
each depth? Please clarify. Maybe even give an example.

page 20: "round-robin sampling from the available depth buckets as
described above." --> This is a key sentence that might help
understanding what exactly is the diversifying method. I think you
mean you that do a round robin from the deepest available depth to the
shallowest available depth. You must clarify this.
\end{quote}

We rewrote the description and added the pseudocode (Algorithm 2) for depth diversification
in Section [Original:6,Revised:5].
d\(_{\text{c}}\) is a counter assigned to each plateau.

\subsection{[1.3] 3) It will be very interesting to see what happens if you factor away the constant time per node.}
\label{sec:orgheadline3}

\begin{quote}
page 22: Table 7.5. It will be very interesting to see what
happens if you factor away the constant time per node. Just compare
the number of nodes expanded for a set of instances that can be solved
by all methods (as you did in the 4.1-4.5 tables). This will tell you
if indeed this is the reason for the negative behavior. This is a
rather major comment. You do not always have to use the 30 minutes
limit in your experiments.
\end{quote}

Thanks for the suggestion. In order to factor away the constant low-level overhead of depth bucket management
which degrades the performance of M\&S on IPC domains,
we compared the number of node evaluations between the depth-diversified and standard search method.
We added Figure [Revised:6.2] comparing the cumulative coverage over the number of evaluations.
This shows that the evaluations are almost identical for most IPC instances, meaning that search efficiency in positive cost domains was largely unaffected by depth diversification, as expected.
There were differences in Openstacks (this is also as expected, since Openstacks has zero-cost actions).

\subsection{[1.4a] 4) Section 5: Zerocost domains. I buy all your arguments on zero cost}
\label{sec:orgheadline4}

\begin{quote}
Section 5: Zerocost domains. I buy all your arguments on zero cost
domains. You spend too much effort to validate them. I suggest to
shorten this entire section and the related tables. All your claims
seem reasonable and you do not have to necessarily show all the
numbers.
\end{quote}

As suggested, we shortened Section [Revised:4].
We moved one of the tables to Appendix. XXX Which table in the Appendix? -- maybe it was removed entirely?

\subsection{[1.4b]  4)  The claims about FIFO in infinite graphs (section 5.3) is trivial.}
\label{sec:orgheadline5}

\begin{quote}
The claims about FIFO in infinite graphs (section 5.3) is trivial. You can shorten it or even delete it.
\end{quote}

In the revised version, we moved the section [Original:5.3] to the end of Section [Original:8,Revised:7],
where it fits better (using the A*-as-sequence-of-SAT framework)

In response to item [2.XXX] below from Reviewer 2, 
we formalized some additional completeness results on infinite graphs, which was also added to this revised section [Original:5.3,Revised7.2].

\subsection{[1.5] 5) Section 6.1 is trivial.}
\label{sec:orgheadline6}

\begin{quote}
Section 6.1 is trivial. It is easy to see that different "depth"
values only occur in zero domains. I would shorten it or even omit it.
\end{quote}

We formalized and significantly shortened the proof (Theorem 1),
and  Section [Original:6.1] has been folded into into Section [Revised:6].


\subsection{[1.6] 6) Section 7.1: this section repeats what you said above and \ldots{}}
\label{sec:orgheadline7}

\begin{quote}
Section 7.1: this section repeats what you said above and I was
convinced when you said it. You can just report that you observed this
in your experiments and I do not need to see all the exact results.
Consider to omit these experiments and just mention that you have
results that support this trend.
\end{quote}

We moved several non-critical figures [Original:XXX,XXX]to the appendix [Revised:Appendix XXX,XXX].

\subsection{[1.7] 7) The beginning of Section 8 is also rather trivial.}
\label{sec:orgheadline8}

\begin{quote}
The beginning of Section 8 is also rather trivial. This is the main
rational behind IDA* as you say in the end. I would significantly
shorten it but it should get a subsection index if it stays. It is not
an introduction to your later section 8.1 which I find quite
interesting and more deep and should certainly be kept.
\end{quote}

We shortened the beginning of Section [Original:8,Revised:7], compared "A*-as-series-of-satisficing-search" to IDA*, and added a paragraph connecting this introduction better as an introduction to  subsections 8.1



\subsection{minor 1) -- should be "current shortest known path"}
\label{sec:orgheadline9}

\begin{quote}
page 4: "g(n) is the current shortest path cost from the initial node
to the current node." -- should be "current shortest known path"
\end{quote}

Fixed.

\subsection{minor 2) -- I did not like this syntax. Give the reference and\ldots{}}
\label{sec:orgheadline10}

\begin{quote}
page 5: "Holte, 2010, note that since f = g+h\ldots{}.) I did not like this
syntax. Give the reference and then give your comment but not in the
same parenthesis.
\end{quote}

Fixed as suggested.

\subsection{minor 3) -- Calling it the  third is misleading\ldots{}}
\label{sec:orgheadline11}

\begin{quote}
page 21: "the third, depth-diversification criteria." Calling it the
third is misleading. It is actually the second which comes before the
default criterion.
\end{quote}

Fixed as you suggested.  XXX - not fixed???

\subsection{minor 4) -- The first sections are very short. Maybe they can be one large section\ldots{}}
\label{sec:orgheadline12}

\begin{quote}
The first sections are very short. Maybe they can be one large section
with different subsections.
\end{quote}

We merged section [Original:2] and [Original:3] into section [Revised:2],

\section{Reviewer 2}
\label{sec:orgheadline25}

\subsection{[2.1] 1) Maxim Likhachev's ARA* paper\ldots{}}
\label{sec:orgheadline14}

\begin{quote}
 Maxim Likhachev's ARA* paper presents an elegant solution to
avoid the final plateau problem for non zero-cost domains. His
algorithm notes the cost of the goal, whenever a new path to goal is
discovered, and concludes the search when the minimum cost of any
state in OPEN becomes greater than or equal to the current goal cost
(f = f*). While this approach is not applicable for 0-cost domains, I
think this merits a discussion and probable inclusion of results in
case of other domains used.
\end{quote}

We added a paragraph describing the relationship to ARA* in Related Work (Section [Revised:9]).

ARA* could largely avoid the problem of final plateau if the previous suboptimal searches happen to
have found the optimal solution already (and thus pruning most nodes on f=f*). 
However, ARA* is based on an iterated anytime framework, whereas our work is based on the standard (A*) admissible search.
We point out this difference.


\subsection{[2.2] 2) the amount of data is a bit too much\ldots{}}
\label{sec:orgheadline15}

\begin{quote}
 While I appreciate the in-depth experimental investigation
presented in this work, i think the amount of data is a bit too much.
For example, 26 plots for number of nodes vs depth is rather
confusing. I like the summarization done for most tables, which points
to the key take-aways. I think the experimental results should be
presented in a more compact fashion, and if needed the detailed
results can be pushed to an appendix (even there, i believe some
compaction will be good). This will also help to reduce the length of
the paper. Currently, it seems too long for the content.
\end{quote}

We moved many tables and plots to the appendix, so the length of the main portion of the paper has been reduced to 39 pages (excluding references and appendix).

\subsection{\label{orgtarget1} 3) the theory and analysis part\ldots{} Section 5.3}
\label{sec:orgheadline16}

\begin{quote}
 While the paper presents experimental results in detail, the theory
and analysis part looks weak in my opinion. Most of the analytical
results are presented in an informal manner. For example, 5.3
discusses the completeness of search strategies on ZeroCost domains. I
would suggest that such results should be presented using formal
statements with proofs.
\end{quote}

We moved Section [Original:5.3]
to the end of Section [Original:8,Revised:7] and added more formal statements regarding the completeness 
on infinite graphs. This material was moved because the analysis is most natural 
using the A*-as-sequence-of-SAT framework introduced in Section [Revised:7].

\subsection{[2.3b] 3) the theory and analysis part\ldots{} Section 6.1}
\label{sec:orgheadline17}

\begin{quote}
Similarly, the analysis in 6.1 can be more
precise, results in 6.1 can be presented in terms of theorems.
\end{quote}

We have formalized the result (Theorem 1) and made it more precise.

\subsection{[2.4] 4) Sec6, "more nodes will tend to have shallower depth" vs disjoint forest model}
\label{sec:orgheadline18}

\begin{quote}
 In the last paragraph of section 6, it is stated that "more nodes
will tend to have shallower depth than deeper depth" whereas the
analysis in 6.3 assumes a disjoint forest model which i guess
increases the number of nodes with depth. These two assumptions seems
to be in contrast to each other. I think a more formal treatment of
the analysis can allay such confusions for a reader.
\end{quote}

To clarify: According to the \emph{no-exhaustion assumption} , no depth bucket exhausts due to the expansion.
This implies that there are sufficiently large number of nodes in depth \(d=0\) so that
 depth 0 does not exhaust as a result of expansion.
If FIFO default tiebreaking is used,
it tries to expand all those nodes with depth 0 before expanding any nodes in depth d >= 1.
Similar situation happens at every depth.
Thus, even if the entire graph is a forest model, FIFO causes a heavy bias to expanding nodes with shallow depth.

It's true that there are surely more nodes with larger depth if \emph{all} nodes in the entire plateau are expanded, which is the case for \(f<f^*\).
However, in the final plateau of A*, FIFO expands only a fraction of nodes with depth \(d\leq d^*\),
where \(d^*\) is the \emph{minimum solution depth}, the smallest depth of the solutions.
Entire nodes above the solution depths (\(d>d^*\)) are not expanded due to the breadth-first behavior.
During this process the expanded nodes are biased to the shallower region.

This has been clarified in the text (Section [Revised:5.2]), and 
for further clarity, we also added Figures 5.2 and 5.3 which illustrate the scenarios.


\subsection{[2.5] 5) I think it will be helpful if the authors include pseudocodes for\ldots{}}
\label{sec:orgheadline19}

\begin{quote}
 All the strategies proposed are explained in text only. I think it
will be helpful if the authors include pseudocodes for their
algorithms. In fact, i think it will be helpful if the authors present
a basic A* algorithm with default tie-breaking and build upon that for
their strategies. It will create a nice flow in my opinion, and use of
pseudocode will also remove any chance of mis-interpreting the
strategies.
\end{quote}

As suggested, we added pseudo-code for  Best-First search (Algorithm 1), and depth-based tiebreaking (Algorithm 2).


\subsection{[2.6] 6) state/prove the properties of each of these algorithms, especially important ones like completeness}
\label{sec:orgheadline20}

\begin{quote}
 Tied to point 6, i think it would be good to state/prove the
properties of each of these algorithms/strategies, especially
important ones like completeness. The current format leaves a lot of
un-answered questions like does depth-diversification ensure
completeness (for infinite spaces). The answers may be obvious in many
cases, however, i would still prefer if they are explicitly
stated/proved.
\end{quote}

We proved the completeness and its conditions as requested in Section [Revised:8] (See also the response to Question \ref{orgtarget1}).

\subsection{7) I like the idea of representing A* as a series of satisficing search. Here also, i would suggest inclusion of pseudocode.}
\label{sec:orgheadline21}

\begin{quote}
 I like the idea of representing A* as a series of satisficing
search. Here also, i would suggest inclusion of pseudocode. For
example, A* exhausts an f-plateau before moving on to the next one.
While this is expressed in text, highlighting such properties through
pseudocode may improve a reader's understanding. Similar to earlier
cases, here also the authors can start with a basic pseudocode (for A*
as a series of satisficing searches), and present their strategies on
top of that with formal discussion about the properties.
\end{quote}

Added pseudo code of A*-as-sequence-of-SAT-search, as you suggested.

\subsection{8) distribution of goal depth in the final frontier}
\label{sec:orgheadline22}

\begin{quote}
 I think it will be interesting to find out what is distribution of
goal depth in the final frontier. I believe there will be a strong
correlation between the goal depth and the relative performance of the
strategies (which the authors mention), and it would be good to
analyze this statistically. Similarly, for strategies in section 9, it
would be interesting to find out the correlation between the
performance of different strategies with the accuracy of the
distance-to-go estimates.
\end{quote}

We agree that goal depth distribution
and distance-to-goal heuristic accuracy might be strongly correlated with tie-breaking strategy performance.
This poses interesting avenues for future work, 
and may be very useful, for example, in an extension of this work which 
seeks to automatically select tie-breaking strategy.  

Thanks for these suggestions. 


\subsection{9) Finally, I think it would be nice if we have some infinite spaces in the ZeroCost domains}
\label{sec:orgheadline23}

\begin{quote}
 Finally, I think it would be nice if we have some infinite spaces
in the ZeroCost domains, and understand the impact of different
strategies on them. My hunch is that in many cases people use
fifo/breadth-first exploration to avoid completeness problems, i
believe inclusion of such graphs (or some domains that closely
approximate such behavior) will enhance the analysis.
\end{quote}


In this paper, we focused on domain-independent planning in the classical planning framework (specifically, in the STRIPS/SAS+ framework), for which the search spaces are finite.
Zerocost domains were created as variations of standard IPC benchmarks (which are all in this finite-space framework).

Empirical evaluation of tie-breaking strategies on infinite search
spaces is an interesting avenue for future work, but 
since infinite search spaces are beyond the scope of classical planning, this will require
careful design of interesting/practical benchmark domains and solvers.

We agree that completeness can be one good reason for choosing a  fifo tie-breaking strategy.
However, in our survey of papers mentinoing tie-breaking strategies, we couldn't find any work 
which specifically mentioned fifo tie-breaking and also handled infinite spaces -- 
the use of fifo which we cite in the paper is by Fast Downward, a classical planner, and as mentioned above, 
infinite search spaces are beyond the scope of the standard clasical planning framework, so it's unlikely that the use of FIFO
tie-breaking in FD was motivated by completeness concerns.


\subsection{minor comments}
\label{sec:orgheadline24}

\begin{quote}
 I think the abstract needs to be re-written to precisely state the
contribution. In particular i would suggest changing the sentences
after "With this in mind, ..". Somehow it seems that the depth
diversification is the second strategy, which is not the case. Also,
"We proposes" -> "We propose".
\end{quote}


\begin{quote}
 page 27, claim 1 "A Last-In-First-Out ..". Is this a general claim,
or is it tied to the domains you tested on. I think this should be
made clear.
\end{quote}


\begin{quote}
 Section 2, the 4th paragraph can probably be combined with the
second. Also, may be it would be better if you present exact formal
definitions of the terms.
\end{quote}

\begin{quote}
 I would suggest that you include some pictorial representation of
your analysis in section 6.3. There are several illustrations of A*
layers in other places that are helpful, some such illustration of
your model would be nice.
\end{quote}

\begin{quote}
 There are a number of typos and grammar mistakes, please correct
them. For example, "did not modified" -> "did not modify", "new
current parent" -> "current parent", and others.
\end{quote}

Thank you for the detailed comments, they are all fixed according to your suggestions.
\section{Reviewer 3}
\label{sec:orgheadline34}

\subsection{1) There are a large number of colourful scatterplots in the paper, most of which would probably be better presented in a different form.}
\label{sec:orgheadline26}

\begin{quote}
 There are a large number of colourful scatterplots in the paper, most
of which would probably be better presented in a different form. For
example, the data in Figure 1.1 is essentially 1-dimensional: what we
are interested in is the distribution or frequency of ratios between
the size of the final plateau and the search space; a histogram or a
cumulative distribution would show this more clearly. Whether colour-
coding it for domains is useful is questionable; there's only a few
points that can be distinguished well enough to identify what domain
they belong to (and even those do not tell the full story, since there
is no way to see where other instances from the same domain fall).

The data in Figures in 4.1, 4.2, 5.2 and 7.1 would similarly benefit
from a more thought-through visual presentation.
\end{quote}

The figure [Original,Revised:4.1] and [Original,Revised:4.2] should be in the present form.
The role of these figures is to identify which domain is affected by the different default criteria.

The figure [Original,Revised:1.1] is paired with [Original,Revised:4.2].
If we change the format of [Original,Revised:1.1] from the current one to the histogram,
then it loses the consistency with [Original,Revised:4.2].

Separating the figure into per-domain analyses would further increase the paper length.

However, we indeed benefit from converting [Original,Revised:7.1] into a histograms
comparing the node evaluation ratio, because the domain characteristics is not important
in this figure. Thank you for the suggestion.

\subsection{2) The description in the early part of the paper (Sections 1, 3, 4, 5) somewhat convey the false impression that there has been no previous recognition of the challenge that plateaus can create for A* search}
\label{sec:orgheadline27}

\begin{quote}
 The description in the early part of the paper (Sections 1, 3, 4, 5)
somewhat convey the false impression that there has been no previous
recognition of the challenge that plateaus can create for A* search,
in particular in the presence of zero cost transitions, or attempts to
address it. There are a number of relevant related works, for example,
those by Benton et al., and Cushing et al., which are cited somewhere
in the paper, but do not appear anywhere in the initial discussion nor
in the related works section. (The SoCS 2011 paper "Cost-Based
Heuristic Search Is Sensitive to the Ratio of Operator Costs", by
Christopher Wilt and Wheeler Ruml, may also be relevant.) This should
be rectified; the previous state of knowledge should be clearly
established early in the paper.
\end{quote}

Previously, the issues of zero cost transitions were not directly associated with 
a failure in tie-breaking. Thus, previous work focused on how to modify the main 
evaluation functions (use of distance-to-go functions, inflating the heuristic value)
or to modify the expansion order (e.g. Thayer and Ruml, ICAPS08).

Considering the flow of the paper,
which start by examining the standard tiebreaking strategies on optimal search,
then proceed to identify and connect the source of the problem with 0-cost transitions,
inserting additional section around the beginning that discuss the suboptimal search
would be out of place and unnatural.

\subsection{3) This applies also to the summary of the authors earlier conference paper.}
\label{sec:orgheadline28}

\begin{quote}
 This applies also to the summary of the authors earlier conference
paper. Rather than the "note" at the end of the introduction (which I
assume the authors intend to remove from the published version of the
paper), the summary of that paper, and the novel contributions this
article makes over it, should be integrated in the presentation.
\end{quote}

We fixed it as you suggested.

\subsection{4) The argument in the last paragraph before Section 5.1 and the second paragraph of Section 5.1 do not make sense.}
\label{sec:orgheadline29}

\begin{quote}
 The argument in the last paragraph before Section 5.1 and the second
paragraph of Section 5.1 do not make sense. First, the authors say
they selected subsets of instances of some domains in order to avoid
skewing the results by uneven instance set sizes; but then, these
domains are excluded from the following analysis.
\end{quote}

The analyses from which these instances are excluded are Section [Original:5.1,Revised:4.1] only.
They are still evaluated in the later sections.

\subsection{5) Furthermore in Section 5.1, why is the comparison done using the [f,h,fifo] strategy}
\label{sec:orgheadline30}

\begin{quote}
 Furthermore in Section 5.1, why is the comparison done using the
[f,h,fifo] strategy, given that the experiment in Section 4 showed
tie-breaking using "lifo" to be much more efficient?
\end{quote}

The aim of this experiment is to show that there can be some performance difference for some planner,
and we consider this is sufficient.
Being the planner Fast Downward, which is currently the most successful state-of-the-art planner
and by default uses the FIFO default tiebreaking,
we consider using FIFO as a representative would be a reasonable choice.

Also, you can extract the numbers for [f,h,lifo] experiments from
Table [Original:7.2, Revised:12.3] and Table [Original:7.4, Revised:12.5].
We obtained the same results using these numbers:
The coverages in the original and Zerocost domains are similarly different.

\subsection{6) In Section 6.2, the authors argue that \ldots{} pruning methods \ldots{} are somehow equivalent to tie-breaking. This is not accurate.}
\label{sec:orgheadline31}

\begin{quote}
 In Section 6.2, the authors argue that pruning methods such as
symmetry or partial order reduction are somehow equivalent to
tie-breaking. This is not accurate. Although a bias towards some
states may be created by the presence of, for example, symmetries, as
the authors argue, pruning the symmetric states does \underline{more} than just
"remove the bias". If the states in question have f-values that are
less than the cost of the optimal solution, no form of tie-breaking
will prevent A* from expanding all of them, but symmetry pruning will.
\end{quote}

In the revised version, we clarified that pruning is a stronger technique
than diversification.

\subsection{7) In Section 7, Table 7.1 shows that there is little consistency in the results}
\label{sec:orgheadline32}

\begin{quote}
 In Section 7, Table 7.1 shows that there is little consistency in the
results, particularly on the benchmark set in which only a few domains
have zero cost actions. Table 7.2 shows that this is the case even on
the Zerocost problem set, when considerd by domain. This is worth more
emphasis in the discussion. While the experiment shows that
depth-based tie-breaking \textbf{can} be advantageous, it is by no means
always the case.
\end{quote}

The inconsistency is natural considering
that the aim of diversifying the depth is to choose the \textbf{safest} practice in a domain-independent
manner. Depending on the domain, the \textbf{best} practice may vary -- for example, fifo is the best in
airport-fuel with LMcut, while lifo is the best in freecell-move with LMcut.
However, although these two default strategies may work well in some domains,
it does more harm than good in many other domains,
encountering the worst case pathological behavior.

This is previously addressed in section 6 in the original version:

\begin{quote}
"In the former case, fifo should perform well because\ldots{} However, in the latter case, exhaustively
searching the shallower depths can result in \ldots{} because \ldots{}"
\end{quote}

In the revised version,
we added a paragraph in the end of section 7
emphasizing and explaining the inconsistency you suggested.

\subsection{8) I'm somewhat sceptical about the value of these figures\ldots{}which of the examples are showing the failure of depth-based tie-breaking strategies.}
\label{sec:orgheadline33}

\begin{quote}
 I'm somewhat sceptical about the value of these figures. They show
only examples of what can happen on isolated instances. Although such
deep-dives may be useful to explain what is happening in different
cases (particularly given the variance in the results), the volume and
unclear selection of the examples make them less informative. (For
instance, it is not clear which of the examples are showing the
failure of depth-based tie-breaking compared to default tie-breaking
strategies.)
\end{quote}

The purpose of these figures is not to show the performance,
but how depth diversification and other strategies follow the expected depth distribution.
(Sec.7.1, "To understand the behavior of depth-based policies\ldots{}")

In terms of performance measured by the number of expanded nodes,
freecell-move p04 in Figure [Original:7.2, Revised:7.3], mid-right,
is an instance on which lifo solved the problem
with much smaller expansions than depth diversification.
This can also be seen as the coverage difference in Table [Original:7.2, Revised:12.2].
\end{document}