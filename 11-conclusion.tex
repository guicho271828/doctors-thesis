
\section{Conclusions and Future Work}

In this paper, we investigated the tiebreaking strategies for cost-optimal search using \astar.  
\begin{comment} % the goals are unimportant
We sought to (1) shed some light on the importance of tiebreaking in \astar,  %one of the most popular best-first search algorithms, 
(2) improve \astar without modifying its main heuristic function at all, and (3) to
improve \astar by introducing inadmissible techniques. We reached all of these goals successfully: We sought
various possible enhancements and achieved significant performance improvements solely by the tiebreaking
techniques. In detail, the contributions in this paper are the following:
\end{comment}
Our contributions are as follows:
% \begin{enumerate} % It looks bad when an entire section is just an enumerated list, 
 First, we showed that tiebreaking has a significant role in cost-optimal
       search using \astar. We empirically showed that most instances in the IPC
       benchmark have large plateaus with regard to $f$, and most of the
       search effort is spent in the final plateau with $f=f^*$.

 We then showed that  the commonly used tie-breaking policy based on $h$ value was shown to fail to
       provide guidance in the plateau when the instances have zero-cost
       actions and have large plateaus with regard to $h$.
       We empirically showed that most of the search effort can be spent in
       the final plateau with $f=f^*, h=0$ in some domains. We pointed out that in such
       a plateau, the search is controlled solely by the
       default tiebreaking \fifo, \lifo or \ro.

 We proposed and analyzed a new set of benchmark instances for cost-optimal planning, called Zerocost domains.
       They tend to have larger final plateaus with $f=f^*, h=0$ and pose a new challenge to the traditional search
       algorithms.

 As one approach to improving search performance in Zerocost domains, we proposed a depth metric
       which measures the distance from the entrance to the
       plateau. Using this metric, we described the pathological
       behaviors of \fifo, \lifo and \ro, proposed a new diversification
       strategy, theoretically and empirically showed that it avoids the
       pathological behavior and achieves a better performance.

We then introduced a new interpretation of cost-optimal \astar search as a series of satisficing
       searches among $f$-cost plateaus of an increasing order of $f$. 
%This opens  many opportunities for unifying work on satisficing search and cost-optimal search, as 
%many techniques which have been developed for 
%       satisficing search can be directly applied to plateau search in   cost-optimal search.
This perpsecitve led to another, novel approach for effective tie-breaking in Zerocost domains, we proposed the use of
       inadmissible heuristics as part of a multi-heuristics tiebreaking strategy.
       Depth metric further improves the performance of those
       heuristics. Although there is an additional cost to compute
       multiple heuristic values, the cost can be eliminated by a simple
       case-based configuration which checks for zero-cost actions.

Our reformulating of A* as a sequence of satisficing searches  points to  an interesting direction for future work.
Although we evaluated only one relatively simple, satisficing configuration ($\ffo$) in
the experiments, many techniques which have previously been developed for satisficing planning can be applied to enhance tiebreaking (plateau-search) in cost-optimal search, including
lazy evaluation \cite{richter2010lama}, alternating/pareto open
list \cite{RogerH10}, helpful actions (preferred operators) \cite{Hoffmann01},
random walk local search \cite{nakhost2009monte}, macro operators
\cite{Botea2005,ChrpaVM15}, factored planning
\cite{amir2003factored,brafman2006factored,Asai2015} and
exploration-based search enhancements
\cite{valenzano2014comparison,xie14type,Valenzano2016}.

% Another direction for future work is implementing tiebreaking strategies for IDA* and similar linear-space
% strategies like RBFS.
% For example, how do we implement or simulate depth-based tiebreaking with a linear space usage?
% \todo*{In fact, exploring the application of ``IDA* = series of satisficing seraches'' seems like an interesting
% direction for future work, e.g., how to simulate depth-based tiebreaking (node ordering) in IDA*? IDA* is useful
% for domains where memory limitations are the bottleneck, so keeping the depth buckets in memory may not be
% possible. More broadly, linear space, best-first search including RBFS}
