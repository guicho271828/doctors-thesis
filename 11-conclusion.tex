
\section{Conclusion}

We investigated and improved the tiebreaking strategies for optimizing search using A*.
The contributions in this paper are the following:

\begin{enumerate}
 \item We showed that tiebreaking has a significant role in the optimal
       search. We empirically showed that most instances in the IPC
       benchmark have large plateaus with regard to $f$, and most of the
       search effort is spent in the final plateau with $f=f^*$.
 \item Standard tiebreaking rule using $h$ value was shown to fail to
       provide guidance in the plateau when the instances have zero-cost
       actions and have large plateaus with regard to $h$.
       We empirically showed that most of the search effort is spent in
       the final plateau with $f=f^*, h=0$. We pointed out that in such
       a plateau, search is controlled solely by the
       last-resort tiebreaking \fifo, \lifo and \ro.
 \item We proposed and analyzed a new set of benchmark
       instances for cost-optimal planing, called Zerocost domains.
 \item As one solution to Zerocost domains, we proposed a depth metric
       which measures the distance from the entrance to the
       plateau. Using this metric, we described the pathological
       behaviors of \fifo, \lifo and \ro, proposed a new diversification
       strategy, theoretically and empirically showed that it avoids the
       pathological behavior and achieves a better performance.
 \item As another solution to Zerocost domains, we proposed the use of
       inadmissible heuristics as part of multi-heuristics tiebreaking.
       Depth metric further improves the performance of those
       heuristics. Although there is an additional cost to compute
       multiple heuristic values, the cost can be eliminated by a simple
       case-based configuration which checks for the zero-cost actions.
 \item We introduced a new view to optimal \astar search:
       \textbf{optimizing search can be seen as a series of
       satisficing search in the plateaus of an increasing order of
       $f$}. This view introduces more opportunity to using the
       satisficing techniques for optimal search and vice versa, as
       shown in the preliminary results that depth metric increases the
       performance of GBFS using various heuristics.
\end{enumerate}

Future work expands the last view by introducing much more variety of
existing satisficing search techniques to optimal planning. Examples
include: Lazy evaluation \cite{richter2010lama}, Alternating/Pareto open
list \cite{RogerH10}, helpful actions (preferred operators) \cite{Hoffmann01},
random walk local search \cite{nakhost2009monte}, macro operators
\cite{Botea2005,ChrpaVM15}, factored planning
\cite{amir2003factored,brafman2006factored,Asai2015} and
exploration-based search enhancements
\cite{valenzano2014comparison,xie14type,Valenzano2016}.
