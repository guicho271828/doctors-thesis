\chapter{Introduction}

For years, heuristic search on automated planning has achieved a
significant success and shown its ability to scale to larger and larger
problems. Much of the success are attributed to the development of
more and more sophisticated heuristic functions, while relatively less
attention was paid on the base search algorithms.

In fact, despite the number of papers which try to push the state of the art of the
optimal planning by improving the admissible heuristic functions and developing their theory, much
less attention was paid on the common underlying algorithm, A*, until \cite{Asai2016}.  Similarly, while
there is a large body of work on satisficing planning algorithms, many algorithms tend to be ad-hoc
and lack theoretical foundation other than completeness. \sota LAMA planner
\cite{richter2010lama} incorporates 5 improvements at once and the reason of its success in the
planning competition is hard to discuss.

The contribution in this dissertation is a unified view for search
algorithms and proposals of new algorithms based on the new
understanding. This dissertation proceeds as follows.
 
After the introduction and the preliminary backgrounds, we first analyze
and discuss the search space topology of various domain-independent
planning problems with regard to $f$, the admissible lower-bound for the
solution cost (\refchap{chap:plateaus}).  We show that, contrary to the
conventional wisdom, these combinatorial problems contain huge
\emph{final plateaus}, a set of nodes which has the same $f$ value as the
optimal solution cost $f^*$.
We next investigate the behavior of existing
tiebreaking strategy for \astar algorithm and identify
an important class of problems called \emph{Zerocost domains}, which are
characterized by the huge number of zero-cost edges and makes the
existing tiebreaking strategies useless.

In the next chapter (\refchap{chap:tiebreaking}),
we propose a notion of \emph{depth} in a plateau that explains the behavior of existing tiebreaking strategies
in Zerocost domains and 
propose a new strategy called \emph{depth diversification} strategy
which significantly outperforms the existing strategies in several
zero-cost domains.
We analyze the behavior of depth diversification under some assumptions and
verify that the expectation matches the empirical behavior.
% 
As a result, we show that there are still plenty room for improvements
in the base search algorithms that can impact the search performance,
despite the recent trend in improving the search performance solely by
the enhancements of heuristic functions and pruning techniques.

In \refchap{chap:opt}, based upon the findings in the previous section,
we proceed to show that optimal search can be reduced to satisficing
search. We reformulate the traditional understanding of optimizing
best-first search algorithms (such as A*) by dividing the search space
into \emph{plateaus} of increasing $f$-value, then characterizing A* as a
sequence of satisficing searches on each plateau in the increasing order
of $f$-value.

The previous chapter effectively shows that the performance of optimal
search algorithms can be improved by improving the satisficing search
algorithm performance instead.
Thus, we focus on the satisficing search algorithms hereafter.
To obtain a deeper understanding of satisficing search algorithm,
in \refchap{chap:sat}, we investigate two notions in search algorithms,
tiebreaking and exploration,
and reformulate them as the orthogonal dimensions in the space
of errors between a heuristic function $h$ and the true cost to goal $h^*$.
We empirically verify this hypothesis by comparing the search performance between algorithms
that has the same diversification mechanism applied to tiebreaking and $h$-value selection.

Since the diversification mechanism in both tiebreaking and exploration
are based on \emph{knowledge-free}, blind search algorithms, we further
conclude that satisficing search algorithms can be ultimately improved
by developing the more sophisticated blind search algorithms.
In \refchap{chap:bip}, we do exactly this: We propose a new
diversification mechanism called \emph{Invation Percolation}, which is
based on a fractal structure resulted by the minimum spanning tree on a
search graph.

We conclude the final chapter discussing the relationship to existing
algorithms and the future directions.

This thesis is based on author's past conference and journal publications.
\refchap{chap:topology}-\refchap{chap:opt} are published in \cite{Asai2016} and \cite{asai2017tie}.
\refchap{chap:sat}-\refchap{chap:bip} are published in \cite{Asai2017b}.
