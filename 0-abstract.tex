
\begin{abstract}
In this paper, we investigate and improve tiebreaking strategies for
 optimising search using A* and satisficing search using GBFS.  For A*,
 we first experimentally analyze the performance of common tiebreaking
 strategies that break ties according to the heuristic value of the
 nodes.  We find that tiebreaking has a significant impact on search
 algorithm performance when there are zero-cost operators that induce
 large plateau regions in the search space. With this in mind, we
 develop a new kind of tiebreaking based on a depth metric which
 measures the distance from the entrance to the plateau, and propose a
 new, diversifying strategy which significantly outperforms standard
 strategies on domains with zero-cost actions.

We also apply the same strategy to GBFS for satisficing search, where
 currently no obvious tiebreaking rules are proposed. Our depth
 diviersification resulted in an orthogonal improvement to the other
 search enhancements such as Lazy Evaluation, Preferred Operators,
 Multi-heuristic search, and recently proposed Type-Based
 diversification tequeniques.

Finally, we provide a theoretical analysis on why the depth-based
diversification improves the performance of several algorithms. By doing
this, it ensure the same technique can applied to various best first
search algorithms that are not investigated in this paper such as
Weighted A*.

This paper is a significantly extended version of the AAAI-16 paper by
the same authors.
\end{abstract}
