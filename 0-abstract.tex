
\begin{abstract}
In this paper, we investigate and improve tiebreaking strategies for
 cost-optimal search using A*.
 We first experimentally analyze the performance of common tiebreaking
 strategies that break ties according to the heuristic value of the
 nodes.  We find that tiebreaking strategy has a significant impact on search
 algorithm performance when there are zero-cost operators that induce
 large plateau regions in the search space. With this in mind, we
 develop two new classes of tiebreaking strategy.
 The first class of strategy we propose is based on a depth metric which
 measures the distance from the entrance to the plateau. We  proposes a
 new, depth diversification strategy which significantly outperforms standard
 strategies on domains with zero-cost actions.
 We give both a theoretical analysis and an empirical analysis
 supporting these results.
 The second class of strategy we investigate is 
 an admissible multi-heuristic tiebreaking strategy
 based on inadmissible, distance-to-go variations of various heuristics.
 This is shown to further improve the performance
 in combination with the depth metric.
 Finally, we open a new interpretation to the cost-optimal search by 
 pointing out that the multi-heuristic strategy is a rare attempt to apply a satisficing
 technique to cost-optimal search.{\bf [TODO: preceding sentence doesn't make sense - fix later]}
 This new interpretation is additionally supported by
 the empirical results of the satisficing search using Greedy Best First Search
 enhanced by the depth diversification.
 % 
 \textbf{Note to Reviewers: This paper is a significantly extended
version of an AAAI-16 paper by the same authors. The differences are
summarized at the end of \refsec{sec:introduction}.}
\end{abstract}
