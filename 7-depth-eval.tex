
\section{Evaluating Depth-Based Tiebreaking}
\label{sec:depth-based-evaluation}
We evaluated our depth-based diversifying tiebreaking strategies against standard
tiebreaking strategies.
In addition to the 35 IPC benchmark domains with 1104 instances used in
the previous set of experiments, we used 28 zerocost domains with 620
instances.

\subsection{Evaluating Depth-Based Tiebreaking with $h$-tiebreaking}

We compared the performance of standard tiebreaking methods $[f,h,\fifo]$,
$[f,h,\lifo]$ to $[f,h,\depth,\fifo]$ and
$[f,h,\depth,\lifo]$.  These all use $h$ as the first-level
tiebreaking and either \fifo or \lifo as the last-resort tiebreaking.

First two experiments are conducted on \textbf{1104 standard IPC
benchmark instances}, and the latter two experiments are conducted on
\textbf{620 zerocost instances}.  Each experiment uses either \lmcut
heuristics or \mands heuristics.  For \mands heuristics, we used the
settings recommended by Fast Downward website (bisimulation-based shrink
strategy, DFP merge strategy and exact label reduction).

We first show the summary results of these experiments.
Overall, depth-based tiebreaking tends to show larger coverages than the
standard tiebreaking strategies. In the following, we describe the
details of each experiments.

\begin{table}[htb]
 {
 \centering
\begin{tabular}{|*{5}{c|}}
\hline
 & \multicolumn{4}{|c|}{\lmcut Coverages (\# problems solved)}\\
\hline                                    
 Domain               &  $[f,h,\fifo]$ &  $[f,h,\lifo]$ &  $[f,h,\depth,\fifo]$ &  $[f,h,\depth,\lifo]$ \\ \hline
 IPC,\lmcut(1104)     &558             &565             &\textbf{570.6\spm{}1.5}      &560.0\spm{}0.9               \\ 
 IPC,\mands(1104)     &479             &\textbf{488}    &484.0\spm{}0.0               &481.4\spm{}1.4               \\ \hline
 Zerocost,\lmcut(620) &256             &279             &\textbf{287.2\spm{}2.4}      &280.2\spm{}4.2               \\ 
 Zerocost,\mands(620) &276             &290             &\textbf{310.2\spm{}2.1}      &303.2\spm{}1.7               \\ \hline
\end{tabular}
 \caption{
 Summary Results: Coverage comparison (the number of instances solved in 5min, 2GB, \lmcut/\mands
 heuristics) between standard tiebreaking and depth-based tiebreaking ($\depth$). }
 \label{tbl:lmcut-ipc-full}
 }
\end{table}

\reftbl{tbl:lmcut-ipc-full} shows the number of \textbf{1104 standard
IPC benchmark instances} solved by \lmcut heuristics with various
tiebreaking strategies, under 5min, 2GB experiments. We highlight the
best results when the difference between the maximum and the mininum
coverage exceeds 2.  Depth-based tiebreaking ($\depth$) shows
impressive results on Openstacks and Cybersec domains because these
domains contain many instances of zero-cost edges and the final plateau
$\plateau{f,h}$ is huge (See \refig{fig:plateau}).  Most other instances
are unaffected by depth-based tiebreaking.  Thus, our method offers a
better performance in the domains of interest for free, i.e. without
losing performance in other domains.

\begin{table}[htbp]
 {
 \centering
 \begin{tabular}{|c|c|c|c|c|c|c|c|c|c||c|c|c|}
\hline
 & \multicolumn{4}{|c|}{Coverages (\# problems solved)} \\ \hline
 Domain                               &  $[f,h,\fifo]$ &  $[f,h,\lifo]$ &  $[f,h,\rd,\ro]$       &  $[\rd,\ro]$          \\ \hline
 sum(620)                             &276             &290             &\textbf{310.2\spm{}2.1} &303.2\spm{}1.7         \\ \hline
 {\relsize{-1}airport-fuel(20)}       &5               &5               &5.0\spm{}0.0            &5.0\spm{}0.0           \\
 {\relsize{-1}blocks-stack(20)}       &20              &20              &20.0\spm{}0.0           &19.8\spm{}0.4          \\
 {\relsize{-1}depot-fuel(22)}         &5               &3               &\textbf{6.0\spm{}0.0}   &\textbf{6.0\spm{}0.0}  \\
 {\relsize{-1}driverlog-fuel(20)}     &9               &8               &9.0\spm{}0.0            &9.0\spm{}0.0           \\
 {\relsize{-1}elevators-up(20)}       &7               &\textbf{13}     &11.4\spm{}1.5           &10.4\spm{}0.8          \\
 {\relsize{-1}floortile-ink(20)}      &7               &6               &6.6\spm{}0.5            &7.6\spm{}0.5           \\
 {\relsize{-1}freecell-move(20)}      &5               &18              &18.4\spm{}0.5           &18.4\spm{}0.5          \\
 {\relsize{-1}grid-fuel(5)}           &2               &2               &2.0\spm{}0.0            &2.0\spm{}0.0           \\
 {\relsize{-1}gripper-move(20)}       &\textbf{20}     &\textbf{20}     &\textbf{20.0\spm{}0.0}  &18.0\spm{}1.1          \\
 {\relsize{-1}hiking-fuel(20)}        &13              &13              &12.4\spm{}0.5           &12.2\spm{}0.4          \\
 {\relsize{-1}logistics00-fuel(28)}   &16              &16              &16.0\spm{}0.0           &16.0\spm{}0.0          \\
 {\relsize{-1}miconic-up(30)}         &29              &\textbf{30}     &\textbf{30.0\spm{}0.0}  &\textbf{30.0\spm{}0.0} \\
 {\relsize{-1}mprime-succumb(35)}     &20              &19              &\textbf{23.0\spm{}0.9}  &22.0\spm{}1.4          \\
 {\relsize{-1}mystery-feast(20)}      &4               &4               &6.0\spm{}0.0            &6.0\spm{}0.0           \\
 {\relsize{-1}nomystery-fuel(20)}     &16              &16              &16.0\spm{}0.0           &16.0\spm{}0.0          \\
 {\relsize{-1}parking-movecc(20)}     &0               &0               &0.0\spm{}0.0            &0.0\spm{}0.0           \\
 {\relsize{-1}pathways-fuel(30)}      &4               &4               &4.0\spm{}0.0            &4.0\spm{}0.0           \\
 {\relsize{-1}pipesnt-pushstart(20)}  &3               &2               &\textbf{4.8\spm{}0.4}   &\textbf{4.8\spm{}0.4}  \\
 {\relsize{-1}pipesworld-pushend(20)} &6               &9               &\textbf{10.0\spm{}0.0}  &9.2\spm{}0.7           \\
 {\relsize{-1}psr-small-open(20)}     &19              &19              &19.0\spm{}0.0           &19.0\spm{}0.0          \\
 {\relsize{-1}rovers-fuel(40)}        &8               &8               &8.0\spm{}0.0            &8.0\spm{}0.0           \\
 {\relsize{-1}scanalyzer-analyze(20)} &11              &9               &11.0\spm{}0.0           &10.4\spm{}0.5          \\
 {\relsize{-1}sokoban-pushgoal(20)}   &17              &15              &\textbf{17.6\spm{}0.5}  &15.4\spm{}0.5          \\
 {\relsize{-1}storage-lift(20)}       &4               &4               &4.0\spm{}0.0            &4.0\spm{}0.0           \\
 {\relsize{-1}tidybot-motion(20)}     &0               &0               &0.0\spm{}0.0            &0.0\spm{}0.0           \\
 {\relsize{-1}tpp-fuel(30)}           &9               &10              &\textbf{11.0\spm{}0.0}  &\textbf{11.0\spm{}0.0} \\
 {\relsize{-1}woodworking-cut(20)}    &7               &7               &9.0\spm{}1.1            &\textbf{9.8\spm{}0.7}  \\
 {\relsize{-1}zenotravel-fuel(20)}    &\textbf{10}     &\textbf{10}     &\textbf{10.0\spm{}0.0}  &9.2\spm{}0.4           \\\hline
\end{tabular}

 \caption{
 Coverage comparison (the number of instances solved in 5min, 2GB, LMcut
 heuristics) on \textbf{1104 standard IPC benchmark instances}. We highlight the
 best results when the difference between the maximum and the mininum coverage exceeds 2.
 }
 \label{tbl:lmcut-ipc-full}
 }
\end{table}


\reftbl{tbl:mands-ipc-full} shows the results by \mands heuristics.
In this configuration, depth-based tiebreaking negatively affects the performance.
As we show in \reftbl{tbl:expansion-ratio}, this is because
the low-level overhead of depth-based tiebreaking decreases the high
node processing speed of \mands. Evaluation of \mands heuristics is
very efficiently implemented as a table lookup, and it is able to
evaluate an order of magnutude larger number of nodes
compared to \lmcut heuristics.

\reftbl{tbl:mands-evaluations} shows that if we instead compare the
number of evaluations on problems solved by both, depth-based
tiebreaking significantly outperforms the standard tiebreaking
strategies. Moreover, in the next \textbf{zerocost} domain experiments,
depth-based tiebreaking ourperforms the standard tiebreaking
overall. Besides, the coverages by \mands is less than that of \lmcut.

\begin{table}[htbp]
 {
 \centering
 \begin{tabular}{|*{5}{c|}}
\hline
 & \multicolumn{4}{|c|}{\mands Coverages (\# problems solved)}\\
\hline                                    
 Domain                                 &  $[f,h,\fifo]$ &  $[f,h,\lifo]$ &  $[f,h,\brackets{d},\fifo]$       &  $[f,h,\brackets{d},\lifo]$        \\ \hline                                    
 sum(1104)                              &479             &\textbf{488}    &484.0\spm{}0.0         &481.4\spm{}1.4          \\ \hline
 {\relsize{-1}airport(50)}              &9               &9               &9.0\spm{}0.0           &9.0\spm{}0.0            \\
 {\relsize{-1}barman-opt11(20)}         &4               &4               &4.0\spm{}0.0           &4.0\spm{}0.0            \\
 {\relsize{-1}blocks(35)}               &22              &21              &21.6\spm{}0.5          &21.8\spm{}0.4           \\
 {\relsize{-1}cybersec(19)}             &0               &0               &0.0\spm{}0.0           &0.0\spm{}0.0            \\
 {\relsize{-1}depot(22)}                &5               &6               &5.0\spm{}0.0           &5.0\spm{}0.0            \\
 {\relsize{-1}driverlog(20)}            &12              &12              &12.0\spm{}0.0          &12.0\spm{}0.0           \\
 {\relsize{-1}elevators-opt11(20)}      &12              &12              &12.0\spm{}0.0          &12.0\spm{}0.0           \\
 {\relsize{-1}floortile-opt11(20)}      &6               &6               &6.0\spm{}0.0           &5.2\spm{}0.4            \\
 {\relsize{-1}freecell(80)}             &17              &17              &16.0\spm{}0.0          &15.6\spm{}0.5           \\
 {\relsize{-1}grid(5)}                  &2               &2               &2.0\spm{}0.0           &2.0\spm{}0.0            \\
 {\relsize{-1}gripper(20)}              &20              &20              &20.0\spm{}0.0          &20.0\spm{}0.0           \\
 {\relsize{-1}hanoi(30)}                &14              &14              &14.0\spm{}0.0          &14.0\spm{}0.0           \\
 {\relsize{-1}logistics00(28)}          &20              &20              &20.0\spm{}0.0          &20.0\spm{}0.0           \\
 {\relsize{-1}miconic(150)}             &\textbf{73}     &\textbf{73}     &\textbf{73.0\spm{}0.0} &72.4\spm{}0.5           \\
 {\relsize{-1}mprime(35)}               &23              &24              &23.4\spm{}0.5          &23.2\spm{}0.7           \\
 {\relsize{-1}mystery(30)}              &15              &16              &15.0\spm{}0.0          &15.0\spm{}0.0           \\
 {\relsize{-1}nomystery-opt11(20)}      &18              &18              &18.0\spm{}0.0          &18.0\spm{}0.0           \\
 {\relsize{-1}openstacks-opt11(20)}     &13              &\textbf{19}     &\textbf{19.0\spm{}0.0} &\textbf{19.0\spm{}0.0}  \\
 {\relsize{-1}parcprinter-opt11(20)}    &9               &9               &9.0\spm{}0.0           &9.0\spm{}0.0            \\
 {\relsize{-1}parking-opt11(20)}        &1               &1               &1.0\spm{}0.0           &1.0\spm{}0.0            \\
 {\relsize{-1}pathways(30)}             &4               &4               &4.0\spm{}0.0           &4.0\spm{}0.0            \\
 {\relsize{-1}pegsol-opt11(20)}         &19              &19              &19.0\spm{}0.0          &18.8\spm{}0.4           \\
 {\relsize{-1}pipesworld-notankage(50)} &8               &9               &8.0\spm{}0.0           &8.0\spm{}0.0            \\
 {\relsize{-1}pipesworld-tankage(50)}   &13              &13              &13.0\spm{}0.0          &13.0\spm{}0.0           \\
 {\relsize{-1}psr-small(50)}            &50              &50              &50.0\spm{}0.0          &50.0\spm{}0.0           \\
 {\relsize{-1}rovers(40)}               &8               &8               &8.0\spm{}0.0           &7.6\spm{}0.5            \\
 {\relsize{-1}scanalyzer-opt11(20)}     &10              &10              &10.0\spm{}0.0          &10.4\spm{}0.5           \\
 {\relsize{-1}sokoban-opt11(20)}        &19              &19              &19.0\spm{}0.0          &18.4\spm{}0.5           \\
 {\relsize{-1}storage(30)}              &15              &15              &15.0\spm{}0.0          &15.0\spm{}0.0           \\
 {\relsize{-1}tidybot-opt11(20)}        &0               &0               &0.0\spm{}0.0           &0.0\spm{}0.0            \\
 {\relsize{-1}tpp(30)}                  &6               &6               &6.0\spm{}0.0           &6.0\spm{}0.0            \\
 {\relsize{-1}transport-opt11(20)}      &6               &6               &6.0\spm{}0.0           &6.0\spm{}0.0            \\
 {\relsize{-1}visitall-opt11(20)}       &9               &9               &9.0\spm{}0.0           &9.0\spm{}0.0            \\
 {\relsize{-1}woodworking-opt11(20)}    &7               &7               &7.0\spm{}0.0           &7.0\spm{}0.0            \\
 {\relsize{-1}zenotravel(20)}           &10              &10              &10.0\spm{}0.0          &10.0\spm{}0.0           \\\hline
\end{tabular}

 \caption{
 Coverage comparison (the number of instances solved in 5min, 2GB, M\&S
 heuristics) on \textbf{1104 standard IPC benchmark instances}. We highlight the
 best results when the difference between the maximum and the mininum coverage exceeds 2.
 }
 \label{tbl:mands-ipc-full}
 }
\end{table}

\begin{table}[htb]
 \centering
 \begin{tabular}{cccc}
  nodes/sec                  & LMcut      & M\&S       & M\&S slowdown\\
  \hline
  $[f,h,\lifo]$              & 8.86$\times 10^3$ & 1.37$\times 10^5$ & 100\%\\
  $[f,h,\depth,\lifo]$ & 9.37$\times 10^3$ & 1.13$\times 10^5$ & 82\%\\
  \hline
  $[f,h,\fifo]$              & 9.65$\times 10^3$ & 1.41$\times 10^5$ & 100\%\\
  $[f,h,\depth,\fifo]$ & 9.62$\times 10^3$ & 1.24$\times 10^5$ & 87\%\\
  \hline
 \end{tabular}
 \caption{Comparison of the average node expansion ratio (node/sec) between
 standard tiebreaking and depth-based tiebreaking on \lmcut and \mands
 heuristics. Numbers are averaged over the problem instances solved by
 all 4 configurations. Since the node evaluation of \mands is an order of
 magnitude faster than \lmcut, the overhead of managing depth-based
 tiebreaking queue is non-negligeble on \mands.}
 \label{tbl:expansion-ratio}
\end{table}

\begin{figure}[htb]
 \centering
 \caption{Comparison of the total number of nodes generated by \mands
 heuristics, with vs without depth-based tiebreaking.}
 \label{tbl:mands-evaluations}
\end{figure}

In zerocost domains, our proposed method outperforms the traditional
tiebreaking methods in both \lmcut and \mands heuristics
(\reftbl{tbl:lmcut-zerocost-full} and \reftbl{tbl:mands-zerocost-full}).
Significant improvements were observed in X domains when using \lmcut,
and in X domains when using \mands.

\begin{table}[htbp]
 {
 \centering
 \begin{tabular}{|*{5}{c|}}
\hline
 & \multicolumn{4}{|c|}{Coverages (\# problems solved)} \\ \hline
 Domain                               &  $[h,\fifo]$ &  $[h,\lifo]$ &  $[h,\rd,\ro]$         &  $[\rd,\ro]$          \\ \hline
 sum(620)                             &256           &279           &\textbf{287.2\spm{}2.4} &280.2\spm{}4.2         \\ \hline
 {\relsize{-1}airport-fuel(20)}       &\textbf{15}   &13            &14.4\spm{}0.7           &10.4\spm{}0.5          \\
 {\relsize{-1}blocks-stack(20)}       &17            &17            &17.0\spm{}0.0           &16.0\spm{}0.0          \\
 {\relsize{-1}depot-fuel(22)}         &6             &6             &6.0\spm{}0.0            &6.0\spm{}0.0           \\
 {\relsize{-1}driverlog-fuel(20)}     &8             &8             &8.0\spm{}0.0            &8.0\spm{}0.0           \\
 {\relsize{-1}elevators-up(20)}       &7             &\textbf{13}   &9.4\spm{}1.1            &8.2\spm{}0.7           \\
 {\relsize{-1}floortile-ink(20)}      &8             &8             &8.1\spm{}0.3            &8.0\spm{}0.0           \\
 {\relsize{-1}freecell-move(20)}      &4             &19            &16.5\spm{}0.7           &16.6\spm{}0.8          \\
 {\relsize{-1}grid-fuel(5)}           &1             &1             &1.0\spm{}0.0            &1.0\spm{}0.0           \\
 {\relsize{-1}gripper-move(20)}       &7             &7             &6.0\spm{}0.0            &7.0\spm{}0.0           \\
 {\relsize{-1}hiking-fuel(20)}        &9             &9             &9.0\spm{}0.0            &9.0\spm{}0.0           \\
 {\relsize{-1}logistics00-fuel(28)}   &16            &16            &15.0\spm{}0.0           &16.0\spm{}0.0          \\
 {\relsize{-1}miconic-up(30)}         &16            &17            &\textbf{19.8\spm{}1.0}  &20.4\spm{}1.0          \\
 {\relsize{-1}mprime-succumb(35)}     &15            &14            &\textbf{20.1\spm{}0.7}  &18.6\spm{}2.0          \\
 {\relsize{-1}mystery-feast(20)}      &7             &5             &7.2\spm{}0.4            &7.2\spm{}0.7           \\
 {\relsize{-1}nomystery-fuel(20)}     &10            &10            &10.0\spm{}0.0           &9.4\spm{}0.5           \\
 {\relsize{-1}parking-movecc(20)}     &0             &0             &0.0\spm{}0.0            &0.0\spm{}0.0           \\
 {\relsize{-1}pathways-fuel(30)}      &5             &5             &4.4\spm{}0.5            &4.8\spm{}0.4           \\
 {\relsize{-1}pipesnt-pushstart(20)}  &8             &8             &\textbf{9.8\spm{}0.4}   &\textbf{9.8\spm{}0.4}  \\
 {\relsize{-1}pipesworld-pushend(20)} &3             &4             &4.5\spm{}0.8            &\textbf{5.4\spm{}0.8}  \\
 {\relsize{-1}psr-small-open(20)}     &19            &19            &19.0\spm{}0.0           &19.0\spm{}0.0          \\
 {\relsize{-1}rovers-fuel(40)}        &8             &8             &8.0\spm{}0.0            &9.0\spm{}0.0           \\
 {\relsize{-1}scanalyzer-analyze(20)} &9             &9             &9.1\spm{}0.3            &7.4\spm{}1.0           \\
 {\relsize{-1}sokoban-pushgoal(20)}   &18            &18            &17.9\spm{}0.3           &17.0\spm{}0.0          \\
 {\relsize{-1}storage-lift(20)}       &4             &4             &4.4\spm{}0.5            &4.6\spm{}0.5           \\
 {\relsize{-1}tidybot-motion(20)}     &16            &16            &16.0\spm{}0.0           &15.6\spm{}0.5          \\
 {\relsize{-1}tpp-fuel(30)}           &8             &\textbf{11}   &\textbf{11.0\spm{}0.0}  &\textbf{11.0\spm{}0.0} \\
 {\relsize{-1}woodworking-cut(20)}    &5             &7             &\textbf{8.6\spm{}0.9}   &7.8\spm{}0.7           \\
 {\relsize{-1}zenotravel-fuel(20)}    &7             &7             &7.0\spm{}0.0            &7.0\spm{}0.0           \\\hline
\end{tabular}

 \caption{
 Coverage comparison (the number of instances solved in 5min, 2GB, \lmcut heuristics) 
 on \textbf{620 zerocost instances}. We highlight the
 best results when the difference between the maximum and the mininum coverage exceeds 2.
 }
 \label{lmcut-zerocost-full}
 }
\end{table}

\begin{table}[htbp]
 {
 \centering
 \begin{tabular}{|c|c|c|c|c|c|c|c|c|c||c|c|c|}
\hline
 & \multicolumn{4}{|c|}{Coverages (\# problems solved)} \\ \hline
 Domain                               &  $[f,h,\fifo]$ &  $[f,h,\lifo]$ &  $[f,h,\rd,\ro]$       &  $[\rd,\ro]$          \\ \hline
 sum(620)                             &276             &290             &\textbf{310.2\spm{}2.1} &303.2\spm{}1.7         \\ \hline
 {\relsize{-1}airport-fuel(20)}       &5               &5               &5.0\spm{}0.0            &5.0\spm{}0.0           \\
 {\relsize{-1}blocks-stack(20)}       &20              &20              &20.0\spm{}0.0           &19.8\spm{}0.4          \\
 {\relsize{-1}depot-fuel(22)}         &5               &3               &\textbf{6.0\spm{}0.0}   &\textbf{6.0\spm{}0.0}  \\
 {\relsize{-1}driverlog-fuel(20)}     &9               &8               &9.0\spm{}0.0            &9.0\spm{}0.0           \\
 {\relsize{-1}elevators-up(20)}       &7               &\textbf{13}     &11.4\spm{}1.5           &10.4\spm{}0.8          \\
 {\relsize{-1}floortile-ink(20)}      &7               &6               &6.6\spm{}0.5            &7.6\spm{}0.5           \\
 {\relsize{-1}freecell-move(20)}      &5               &18              &18.4\spm{}0.5           &18.4\spm{}0.5          \\
 {\relsize{-1}grid-fuel(5)}           &2               &2               &2.0\spm{}0.0            &2.0\spm{}0.0           \\
 {\relsize{-1}gripper-move(20)}       &\textbf{20}     &\textbf{20}     &\textbf{20.0\spm{}0.0}  &18.0\spm{}1.1          \\
 {\relsize{-1}hiking-fuel(20)}        &13              &13              &12.4\spm{}0.5           &12.2\spm{}0.4          \\
 {\relsize{-1}logistics00-fuel(28)}   &16              &16              &16.0\spm{}0.0           &16.0\spm{}0.0          \\
 {\relsize{-1}miconic-up(30)}         &29              &\textbf{30}     &\textbf{30.0\spm{}0.0}  &\textbf{30.0\spm{}0.0} \\
 {\relsize{-1}mprime-succumb(35)}     &20              &19              &\textbf{23.0\spm{}0.9}  &22.0\spm{}1.4          \\
 {\relsize{-1}mystery-feast(20)}      &4               &4               &6.0\spm{}0.0            &6.0\spm{}0.0           \\
 {\relsize{-1}nomystery-fuel(20)}     &16              &16              &16.0\spm{}0.0           &16.0\spm{}0.0          \\
 {\relsize{-1}parking-movecc(20)}     &0               &0               &0.0\spm{}0.0            &0.0\spm{}0.0           \\
 {\relsize{-1}pathways-fuel(30)}      &4               &4               &4.0\spm{}0.0            &4.0\spm{}0.0           \\
 {\relsize{-1}pipesnt-pushstart(20)}  &3               &2               &\textbf{4.8\spm{}0.4}   &\textbf{4.8\spm{}0.4}  \\
 {\relsize{-1}pipesworld-pushend(20)} &6               &9               &\textbf{10.0\spm{}0.0}  &9.2\spm{}0.7           \\
 {\relsize{-1}psr-small-open(20)}     &19              &19              &19.0\spm{}0.0           &19.0\spm{}0.0          \\
 {\relsize{-1}rovers-fuel(40)}        &8               &8               &8.0\spm{}0.0            &8.0\spm{}0.0           \\
 {\relsize{-1}scanalyzer-analyze(20)} &11              &9               &11.0\spm{}0.0           &10.4\spm{}0.5          \\
 {\relsize{-1}sokoban-pushgoal(20)}   &17              &15              &\textbf{17.6\spm{}0.5}  &15.4\spm{}0.5          \\
 {\relsize{-1}storage-lift(20)}       &4               &4               &4.0\spm{}0.0            &4.0\spm{}0.0           \\
 {\relsize{-1}tidybot-motion(20)}     &0               &0               &0.0\spm{}0.0            &0.0\spm{}0.0           \\
 {\relsize{-1}tpp-fuel(30)}           &9               &10              &\textbf{11.0\spm{}0.0}  &\textbf{11.0\spm{}0.0} \\
 {\relsize{-1}woodworking-cut(20)}    &7               &7               &9.0\spm{}1.1            &\textbf{9.8\spm{}0.7}  \\
 {\relsize{-1}zenotravel-fuel(20)}    &\textbf{10}     &\textbf{10}     &\textbf{10.0\spm{}0.0}  &9.2\spm{}0.4           \\\hline
\end{tabular}

 \caption{
 Coverage comparison (the number of instances solved in 5min, 2GB, \mands heuristics)
 on \textbf{620 zerocost instances}. We highlight the
 best results when the difference between the maximum and the mininum coverage exceeds 2.
 }
 \label{mands-zerocost-full}
 }
\end{table}



\subsection{Depth-Based Tiebreaking Without Considering $h$}

In \refsec{sec:h-necessary}, we showed that $[f,\lifo]$ tiebreaking
(without considering $h$) is sufficient for the standard IPC benchmarks
-- the performance of $[f,\lifo]$, $[f,h,\lifo]$, and $[f,h,\fifo]$ are
comparable.  In order to see if it also holds for depth-based
tiebreaking, we evaluated the performance of $[f,\depth,\fifo]$
and $[f,\depth,\lifo]$.

\reftbl{tbl:lmcut-zerocost-noh} and \reftbl{tbl:mands-zerocost-noh}
 shows that
 $[f,h,\depth,\fifo]$ and  $[f,h,\depth,\lifo]$
 perform comparably to $[f,h,\fifo]$ and $[f,h,\lifo]$.

% Although $[\rd,\ro]$ behaves in a less greedy/depth-first manner than $[f,\lifo]$, 
% it explores nodes with high depth sufficiently often so that even if \lifo behavior (seeking nodes that are far from the plateau entrance) is required, $[\rd,\ro]$ will eventually find the solution.
% Moreover, there are some domains (\pddl{pipesworld-pushend} and \pddl{woodworking-opt11}) where a the more randomized behavior of $[\rd,\ro]$ is advantageous.
% Thus, overall, $[\rd,\ro]$ performs moderately well, and 
% \emph{neither $h$ nor \lifo-behavior is necessary in order to obtain performance that is competitive with the standard
% tiebreaking strategies}.

\begin{table}[htbp]
 {
 \centering
 \begin{tabular}{|c|c|c|c|c|c|c|c|c|c||c|c|c|}
\hline
 & \multicolumn{4}{|c|}{Coverages (\# problems solved)} \\ \hline
 Domain                               &  $[f,h,\fifo]$ &  $[f,h,\lifo]$ &  $[f,h,\rd,\ro]$       &  $[\rd,\ro]$          \\ \hline
 sum(620)                             &276             &290             &\textbf{310.2\spm{}2.1} &303.2\spm{}1.7         \\ \hline
 {\relsize{-1}airport-fuel(20)}       &5               &5               &5.0\spm{}0.0            &5.0\spm{}0.0           \\
 {\relsize{-1}blocks-stack(20)}       &20              &20              &20.0\spm{}0.0           &19.8\spm{}0.4          \\
 {\relsize{-1}depot-fuel(22)}         &5               &3               &\textbf{6.0\spm{}0.0}   &\textbf{6.0\spm{}0.0}  \\
 {\relsize{-1}driverlog-fuel(20)}     &9               &8               &9.0\spm{}0.0            &9.0\spm{}0.0           \\
 {\relsize{-1}elevators-up(20)}       &7               &\textbf{13}     &11.4\spm{}1.5           &10.4\spm{}0.8          \\
 {\relsize{-1}floortile-ink(20)}      &7               &6               &6.6\spm{}0.5            &7.6\spm{}0.5           \\
 {\relsize{-1}freecell-move(20)}      &5               &18              &18.4\spm{}0.5           &18.4\spm{}0.5          \\
 {\relsize{-1}grid-fuel(5)}           &2               &2               &2.0\spm{}0.0            &2.0\spm{}0.0           \\
 {\relsize{-1}gripper-move(20)}       &\textbf{20}     &\textbf{20}     &\textbf{20.0\spm{}0.0}  &18.0\spm{}1.1          \\
 {\relsize{-1}hiking-fuel(20)}        &13              &13              &12.4\spm{}0.5           &12.2\spm{}0.4          \\
 {\relsize{-1}logistics00-fuel(28)}   &16              &16              &16.0\spm{}0.0           &16.0\spm{}0.0          \\
 {\relsize{-1}miconic-up(30)}         &29              &\textbf{30}     &\textbf{30.0\spm{}0.0}  &\textbf{30.0\spm{}0.0} \\
 {\relsize{-1}mprime-succumb(35)}     &20              &19              &\textbf{23.0\spm{}0.9}  &22.0\spm{}1.4          \\
 {\relsize{-1}mystery-feast(20)}      &4               &4               &6.0\spm{}0.0            &6.0\spm{}0.0           \\
 {\relsize{-1}nomystery-fuel(20)}     &16              &16              &16.0\spm{}0.0           &16.0\spm{}0.0          \\
 {\relsize{-1}parking-movecc(20)}     &0               &0               &0.0\spm{}0.0            &0.0\spm{}0.0           \\
 {\relsize{-1}pathways-fuel(30)}      &4               &4               &4.0\spm{}0.0            &4.0\spm{}0.0           \\
 {\relsize{-1}pipesnt-pushstart(20)}  &3               &2               &\textbf{4.8\spm{}0.4}   &\textbf{4.8\spm{}0.4}  \\
 {\relsize{-1}pipesworld-pushend(20)} &6               &9               &\textbf{10.0\spm{}0.0}  &9.2\spm{}0.7           \\
 {\relsize{-1}psr-small-open(20)}     &19              &19              &19.0\spm{}0.0           &19.0\spm{}0.0          \\
 {\relsize{-1}rovers-fuel(40)}        &8               &8               &8.0\spm{}0.0            &8.0\spm{}0.0           \\
 {\relsize{-1}scanalyzer-analyze(20)} &11              &9               &11.0\spm{}0.0           &10.4\spm{}0.5          \\
 {\relsize{-1}sokoban-pushgoal(20)}   &17              &15              &\textbf{17.6\spm{}0.5}  &15.4\spm{}0.5          \\
 {\relsize{-1}storage-lift(20)}       &4               &4               &4.0\spm{}0.0            &4.0\spm{}0.0           \\
 {\relsize{-1}tidybot-motion(20)}     &0               &0               &0.0\spm{}0.0            &0.0\spm{}0.0           \\
 {\relsize{-1}tpp-fuel(30)}           &9               &10              &\textbf{11.0\spm{}0.0}  &\textbf{11.0\spm{}0.0} \\
 {\relsize{-1}woodworking-cut(20)}    &7               &7               &9.0\spm{}1.1            &\textbf{9.8\spm{}0.7}  \\
 {\relsize{-1}zenotravel-fuel(20)}    &\textbf{10}     &\textbf{10}     &\textbf{10.0\spm{}0.0}  &9.2\spm{}0.4           \\\hline
\end{tabular}

 \caption{
 Coverage comparison (the number of instances solved in 5min, 2GB, LMcut
 heuristics) on \textbf{1104 standard IPC benchmark instances}. We highlight the
 best results when the difference between the maximum and the mininum coverage exceeds 2.
 }
 \label{tbl:lmcut-ipc-noh}
 }
\end{table}
\begin{table}[htbp]
 {
 \centering
 \begin{tabular}{|c|c|c|c|c|c|c|c|c|c||c|c|c|}
\hline
 & \multicolumn{4}{|c|}{Coverages (\# problems solved)} \\ \hline
 Domain                               &  $[f,h,\fifo]$ &  $[f,h,\lifo]$ &  $[f,h,\rd,\ro]$       &  $[\rd,\ro]$          \\ \hline
 sum(620)                             &276             &290             &\textbf{310.2\spm{}2.1} &303.2\spm{}1.7         \\ \hline
 {\relsize{-1}airport-fuel(20)}       &5               &5               &5.0\spm{}0.0            &5.0\spm{}0.0           \\
 {\relsize{-1}blocks-stack(20)}       &20              &20              &20.0\spm{}0.0           &19.8\spm{}0.4          \\
 {\relsize{-1}depot-fuel(22)}         &5               &3               &\textbf{6.0\spm{}0.0}   &\textbf{6.0\spm{}0.0}  \\
 {\relsize{-1}driverlog-fuel(20)}     &9               &8               &9.0\spm{}0.0            &9.0\spm{}0.0           \\
 {\relsize{-1}elevators-up(20)}       &7               &\textbf{13}     &11.4\spm{}1.5           &10.4\spm{}0.8          \\
 {\relsize{-1}floortile-ink(20)}      &7               &6               &6.6\spm{}0.5            &7.6\spm{}0.5           \\
 {\relsize{-1}freecell-move(20)}      &5               &18              &18.4\spm{}0.5           &18.4\spm{}0.5          \\
 {\relsize{-1}grid-fuel(5)}           &2               &2               &2.0\spm{}0.0            &2.0\spm{}0.0           \\
 {\relsize{-1}gripper-move(20)}       &\textbf{20}     &\textbf{20}     &\textbf{20.0\spm{}0.0}  &18.0\spm{}1.1          \\
 {\relsize{-1}hiking-fuel(20)}        &13              &13              &12.4\spm{}0.5           &12.2\spm{}0.4          \\
 {\relsize{-1}logistics00-fuel(28)}   &16              &16              &16.0\spm{}0.0           &16.0\spm{}0.0          \\
 {\relsize{-1}miconic-up(30)}         &29              &\textbf{30}     &\textbf{30.0\spm{}0.0}  &\textbf{30.0\spm{}0.0} \\
 {\relsize{-1}mprime-succumb(35)}     &20              &19              &\textbf{23.0\spm{}0.9}  &22.0\spm{}1.4          \\
 {\relsize{-1}mystery-feast(20)}      &4               &4               &6.0\spm{}0.0            &6.0\spm{}0.0           \\
 {\relsize{-1}nomystery-fuel(20)}     &16              &16              &16.0\spm{}0.0           &16.0\spm{}0.0          \\
 {\relsize{-1}parking-movecc(20)}     &0               &0               &0.0\spm{}0.0            &0.0\spm{}0.0           \\
 {\relsize{-1}pathways-fuel(30)}      &4               &4               &4.0\spm{}0.0            &4.0\spm{}0.0           \\
 {\relsize{-1}pipesnt-pushstart(20)}  &3               &2               &\textbf{4.8\spm{}0.4}   &\textbf{4.8\spm{}0.4}  \\
 {\relsize{-1}pipesworld-pushend(20)} &6               &9               &\textbf{10.0\spm{}0.0}  &9.2\spm{}0.7           \\
 {\relsize{-1}psr-small-open(20)}     &19              &19              &19.0\spm{}0.0           &19.0\spm{}0.0          \\
 {\relsize{-1}rovers-fuel(40)}        &8               &8               &8.0\spm{}0.0            &8.0\spm{}0.0           \\
 {\relsize{-1}scanalyzer-analyze(20)} &11              &9               &11.0\spm{}0.0           &10.4\spm{}0.5          \\
 {\relsize{-1}sokoban-pushgoal(20)}   &17              &15              &\textbf{17.6\spm{}0.5}  &15.4\spm{}0.5          \\
 {\relsize{-1}storage-lift(20)}       &4               &4               &4.0\spm{}0.0            &4.0\spm{}0.0           \\
 {\relsize{-1}tidybot-motion(20)}     &0               &0               &0.0\spm{}0.0            &0.0\spm{}0.0           \\
 {\relsize{-1}tpp-fuel(30)}           &9               &10              &\textbf{11.0\spm{}0.0}  &\textbf{11.0\spm{}0.0} \\
 {\relsize{-1}woodworking-cut(20)}    &7               &7               &9.0\spm{}1.1            &\textbf{9.8\spm{}0.7}  \\
 {\relsize{-1}zenotravel-fuel(20)}    &\textbf{10}     &\textbf{10}     &\textbf{10.0\spm{}0.0}  &9.2\spm{}0.4           \\\hline
\end{tabular}

 \caption{
 Coverage comparison (the number of instances solved in 5min, 2GB, LMcut
 heuristics) on \textbf{1104 standard IPC benchmark instances}. We highlight the
 best results when the difference between the maximum and the mininum coverage exceeds 2.
 }
 \label{tbl:mands-ipc-noh}
 }
\end{table}


%% moved to Table 2



\subsection{Search Behavior Within a Plateau}

To understand the behavior of depth-based policies, we plotted the histogram of
the depths of search nodes opened by
the most successful depth-based strategy, $[f,h,\rd,\ro]$, as well as 
the standard $[f,h,\fifo]$, $[f,h,\lifo]$ strategies
in the final plateau, plateau($f^*,0$) until the solution is found.
% 
Although $[f,h,\fifo]$ and $[f,h,\lifo]$ do not operate with an explicit notion of ``depth'', 
they are equivalent to $[f,h,\fd,\fifo]$ and $[f,h,\ld,\lifo]$, respectively,
%so we recorded the depths according to these depth-based equivalents of $[f,h,\fifo]$ and $[f,h,\lifo]$.
so we recorded and plotted the depths according to  $[f,h,\fd,\fifo]$ and $[f,h,\ld,\lifo]$.


\refig{fig:depth-histogram} shows the result
on
 \pddl{Openstacks-opt11} p10 (left) and
 \pddl{Woodworking-cut} p04 (right).
 % and \pddl{psr-small-open} p48 (right)
In both instances, 
% \pddl{openstacks} and \pddl{Woodworking-cut}, 
we observed that the depth-first behavior of $[f,h,\lifo]$ results in 
deeper search, missing the key branch at intermediate depths.
On the other hand, the breadth-first behavior of $[f,h,\fifo]$ often gets stuck spending an excessive amount of time searching around the plateau entrance.
$[f,h,\rd,\ro]$ is balancing the search at various depths, which results in successfully solving more problems within the time limit (\reftbl{tbl:depth}). %to avoid confusion, avoiding word ``coverage'' here because coverage could refer to search algorithm ``covering'' the search space.

% The figure also illustrates the behavior of RandomDepth: Although it
% randomly selects from the buckets,
% a new depth is found only when the largest-depth bucket is selected,
% resulting in a decreasing curves.
% Finding the optimal balance is an interesting avenue of
% future work.

\begin{figure}[tb]
 \centering \relsize{-3}
 \hfill
 \includegraphics{tables/aaai16-log-rd/aaai16prelim3/depth-histogram-openstacks-opt11-strips-p10.pdf}
 \hfill
 \includegraphics{tables/aaai16-log-rd/2zerocost/depth-histogram-woodworking-cut-p04.pdf}
 \hfill
 % \includegraphics{tables/aaai16-log-rd/2zerocost/depth-histogram-psr-small-open-p48.pdf}
 \caption{Number of nodes ($y$-axis) expanded per depth ($x$-axis) in
 the final plateau for 
 % (in left-to-right order)
 \pddl{Openstacks} p10 
 (left)
 and
 \pddl{Woodworking-cut} p04
 (right)
 %  and
 % \pddl{psr-small-open} p48
 with different tiebreakings. Both axes are in logarithmic scale.
 % Axes are logarithmic.
 }
 \label{fig:depth-histogram}
\end{figure}


\subsection{Comparison to Distance-to-Go Estimates}


\subsection{Comparison to Multi-Heuristic Tiebreaking}

