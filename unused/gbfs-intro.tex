\subsubparagraph{gbfs}

After \refsec{sec:gbfs}, we apply the same depth-based diversifying tiebreaking
method to Greedy Best First Search (GBFS) for satisficing planning.

Despite the ubiquitous use of GBFS for satisficing search,
various research have suggested that GBFS can be
easily trapped by undetected dead ends and a huge search plateau.
On infinite graphs, GBFS is not even complete \cite{Valenzano2016}
because it can be misdirected by the heuristic guidance forever.
These pathological behaviors are caused by the fact that 
the search process of GBFS heavily relies upon the
quality of the heuristic function.

The problem is amplified by the fact that GBFS is usually combined
with inadmissible heuristic functions, such as FF
heuristics\cite{Hoffmann01}, Causal Graph heuristics \cite{Helmert2006}, Context Enhanced
Additive (CEA) heuristics\cite{helmert2008unifying}.
% , Landmark-count heuristics\cite{richter2008landmarks}
These functions may overestimate the true distance to the goal,
which can causes some important nodes to be incorrectly labelled as unpromising
(very far from the goal). Best-first search algorithms like GBFS thus
ignores these nodes until after expanding all other nodes with the smaller estimates.

Recently, several approaches
\cite{imai2011novel,valenzano2014comparison,xie14type} have been
proposed for alleviating those problems in GBFS. They improve the search
performance by occasionally ignoring the heuristic guidance, which
provides a chance to recover from the past wrong decisions made by the
heuristics, and makes the algorithm more robust against the heuristic
errors.  These diversified algorithms are probabilistically complete on
the inifinite graphs \cite{Valenzano2016}: The probability of finding a
solution approaches to 1 as the runtime approaches to infinity; in other
words, it eventually finds a solution.

% by relaxing its dependency to the heuristic function.

%% this paragraph is important because diversity, exploration, and
%% ``removing bias'' might not immediately connect with each other
One common theme in these algorithms is to remove the \emph{bias} and
add a \emph{diversity} and \emph{exploration} to the search process.
Exploration, an opposite word to exploitation, is an act of removing
the focus or the bias, thus is equivalent to an act of introducing the
diversity. Hereafter, we use these three key words interchangeably.

In these sections, we show that there is much more room for diversifying
the behavior of GBFS, and show that the use of our depth-based
tiebreaking strategy improves the performance of GBFS by
diversifying the search within the same $h$-plateau.
 
GBFS variants mentioned above all employ the exploration by occasionally
expanding the nodes with bad heuristic estimates. While this can avoid
the problem of large bias of the search algorithm to the nodes with the
least estimates, it cannot detect the bias among the nodes with the
\emph{same} estimates. Imagine what happens if a heuristic function
always returns a constant value $c$. Since all nodes expanded by GBFS
has the same evaluation $c$, above-mentioned GBFS variants fail to
identify their difference based on the estimates. \footnote{Type-GBFS
and DBFS2 can also diversify the search based on the $g$ value, the
current shortest path cost from the initial state.}

To our knowledge, there are currently no well-established tiebreaking
policy for GBFS.  Compared to \astar, GBFS does not use $f$ and $g$
value during the search process.  The search by GBFS is solely guided by
the heuristic value $h$: It always expands the node with the smallest
$h$ value, and hence the analogy from \astar e.g.\ sorting the nodes
with $f$ and breaking ties according to $h$ is not possible.

% We propose the use of recently proposed tiebreaking strategy based on
% the \emph{depth} of each node from the entrance in the plateau, originally
% designed for \astar, also improves the performance of GBFS by
% diversifying the search within the same h-plateau.
Our depth-based tiebreaking can be implemented as an extension to GBFS, and
can also be incorpolated as a part of type-bucket generation criteria in
Type-Based GBFS.  We empirically evaluate the performance of the
proposed approach, comparing numerous search enhancement techniques such
as Lazy Evaluation, PLUS\-ONE/ONE cost type, Preferred Operators and
Multi-queue heuristic search.
