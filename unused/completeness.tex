
\subsection{Completeness of Tie-Breaking Strategies on Zerocost Domains}

% The presence of 0-cost edges in the search space affects the completeness of tie-breaking strategies for \astar. 
As long as the search graph is finite, \astar is complete regardless of the tie-breaking strategy.
However, if the search graph is infinite, the completeness of the algorithm could be affected.

Let's assume the branching factor is finite, which means only the depth of the search graph is infinite.
Also, assume that solutions have finite depths and a finite $f^*$ value.

\begin{theo}
 If any of $\plateau{f}$ for $f<f^*$ are infinite, then \astar does not terminate in a finite length of time.
\end{theo}

\astar requires expanding entire nodes in $\plateau{f}$ before expanding $\plateau{f+1}$, and it never finishes.

So we next examine the case where all of $\plateau{f}$ for $f<f^*$ are finite.
Since \astar does not expand nodes in $\plateau{f}$ for $f>f^*$,
they do not affect the completeness of \astar.
Thus, we can assume $\plateau{f^*}$ is infinite.

\begin{propo}
 If $\plateau{f^*}$ is infinite,
 \begin{enumerate}
  \item \astar with FIFO tiebreaking always terminate with a solution.
        \label{item:fifo-complete}
  \item \astar with LIFO tiebreaking is not guaranteed to terminate.
        \label{item:lifo-complete}
  \item \astar with Random Order tiebreaking is probabilistically complete.
        \label{item:ro-complete}
  \item \astar with Depth Diversification terminates with a solution
        regardless of further tiebreaking.
        \label{item:depth-complete}
 \end{enumerate}
\end{propo}

\begin{proof}
\begin{enumerate}
 \item Since the solution have finite length and thus finite depth in the plateau, the search space expanded by FIFO (Breadth-first search) in infinite $\plateau{f^*}$ is finite.  Thus FIFO returns the solution in a finite time.
 \item LIFO tiebreaking (Depth-first search), in contrast,
       may explore into infinitely large depth and never returns to the solution depth.
 \item (Proof for \refstep{item:ro-complete})
       Let $d_{sol}$ be the smallest solution depth of all solutions.
       It is sufficient to show that the probability of expanding all nodes in $d_{sol}$
       approaches to 1 as the runtime increases.
       Let $b$ be the maximum branching factor (which is finite).
       The size of OPEN list is upper bounded by $(b-1)^t$ after $t$ expansions.
\end{enumerate}
 
In Random Order tiebreaking,


Since the solution have finite length and thus finite depth in the plateau,
the search space expanded by FIFO (Breadth-first search) in infinite $\plateau{f^*}$ is finite.
Thus FIFO returns the solution in a finite time.

Since the solution have finite length and thus finite depth in the plateau,
the search space expanded by FIFO (Breadth-first search) in infinite $\plateau{f^*}$ is finite.
Thus FIFO returns the solution in a finite time.
\end{proof}

Since the solution have finite length and thus finite depth in the plateau,
The search space expanded by FIFO (Breadth-first search) in infinite $\plateau{f^*}$ is finite.

% If the search graph is infinite, and the branching factor is finite, then 
% \fifo will always return a solution if one exists.
% 
% However, some solvable, infinite search graphs with finite branching factors can contain traps for a \lifo tie-breaking policy -- there may be infinite regions of the search space consisting of nodes connected by 0-cost edges which cause \astar with \lifo tie-breaking to endlessly search this region, never returning to another region of the graph which contains a solution.
% Note that this holds in general regardless of whether $h$-based tie-breaking is applied prior to $\lifo$ tie-breaking, since all nodes in such a trap would have the same $h$-value of zero if $h$ is admissible.
% The randomized, $\ro$ tie-breaking method is probabilistically complete for solvable problems in the presence of such traps -- since $\ro$ will expand nodes outside the trap with some non-zero probability, it will eventually (in the limit) find the solution.

