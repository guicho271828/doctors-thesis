
%% does not seem important!

\subsection{Depth-Based Tiebreaking Without Considering $h$}

\label{sec:depth-noh}

In \refsec{sec:noh}, we showed that $[f,\lifo]$ tiebreaking
(without considering $h$) is sufficient for the standard IPC benchmarks
-- the performance of $[f,\lifo]$, $[f,h,\lifo]$, and $[f,h,\fifo]$ are
comparable.  In order to see if it also holds for depth-based
tiebreaking, we evaluated the performance of $[f,\depth,\fifo]$
and $[f,\depth,\lifo]$.
\reftbl{tbl:lmcut-zero-noh} and \reftbl{tbl:mands-zero-noh}
 shows that
 $[f,h,\depth,\fifo]$ and  $[f,h,\depth,\lifo]$
 perform comparably to $[f,h,\fifo]$ and $[f,h,\lifo]$.

% Although $[\rd,\ro]$ behaves in a less greedy/depth-first manner than $[f,\lifo]$, 
% it explores nodes with high depth sufficiently often so that even if \lifo behavior (seeking nodes that are far from the plateau entrance) is required, $[\rd,\ro]$ will eventually find the solution.
% Moreover, there are some domains (\pddl{pipesworld-pushend} and \pddl{woodworking-opt11}) where a the more randomized behavior of $[\rd,\ro]$ is advantageous.
% Thus, overall, $[\rd,\ro]$ performs moderately well, and 
% \emph{neither $h$ nor \lifo-behavior is necessary in order to obtain performance that is competitive with the standard
% tiebreaking strategies}.

\begin{table}[htbp]
 {
 \centering
 \setlength{\tabcolsep}{0.1em}
 \begin{center}
\begin{tabular}{|rcccHHH|cccHHHHHHHHHHHHHHH|}
\hline
 & $[f,\fifo]$ & $[f,\lifo]$ & $[f,\ro]$ & R & R & R & $[f,\depth,\fifo]$ & $[f,\depth,\lifo]$ & $[f,\depth,\ro]$ & R & R & R & $[f,h,\fifo]$ & $[f,h,\lifo]$ & $[f,h,\ro]$ & R & R & R & $[f,h,\depth,\fifo]$ & $[f,h,\depth,\lifo]$ & $[f,h,\depth,\ro]$ & R & R & R\\
\hline
IPC benchmark (1104) & 443 & 558 & 450 $\pm$ 1 & 449 & 451 & 450 & 533 & 549 & 561.7 $\pm$ 1.5 & 560 & 562 & 563 & 558 & 565 & 560.7 $\pm$ 0.6 & 561 & 560 & 561 & 571 & 575 & 572.3 $\pm$ 1.2 & 573 & 571 & 573\\
\hline
airport(50) & 18 & 26 & 18 $\pm$ 0 & 18 & 18 & 18 & 21 & 23 & 21.3 $\pm$ 0.6 & 21 & 22 & 21 & 27 & 26 & 26 $\pm$ 0 & 26 & 26 & 26 & 27 & 26 & 26 $\pm$ 0 & 26 & 26 & 26\\
barman-opt11(20) & 0 & 0 & 0 $\pm$ 0 & 0 & 0 & 0 & 0 & 0 & 0 $\pm$ 0 & 0 & 0 & 0 & 0 & 0 & 0 $\pm$ 0 & 0 & 0 & 0 & 0 & 0 & 0 $\pm$ 0 & 0 & 0 & 0\\
blocks(35) & 26 & 26 & 26 $\pm$ 0 & 26 & 26 & 26 & 27 & 26 & 26.3 $\pm$ 0.6 & 26 & 26 & 27 & 28 & 28 & 28 $\pm$ 0 & 28 & 28 & 28 & 28 & 28 & 28 $\pm$ 0 & 28 & 28 & 28\\
\textbf{cybersec(19)} & 0 & 3 & 0 $\pm$ 0 & 0 & 0 & 0 & 5 & 12 & 8 $\pm$ 1 & 7 & 9 & 8 & 2 & 3 & 4.3 $\pm$ 0.6 & 4 & 5 & 4 & 8 & 12 & 10 $\pm$ 1 & 11 & 9 & 10\\
depot(22) & 5 & 5 & 5 $\pm$ 0 & 5 & 5 & 5 & 6 & 6 & 6 $\pm$ 0 & 6 & 6 & 6 & 6 & 6 & 6 $\pm$ 0 & 6 & 6 & 6 & 6 & 6 & 6 $\pm$ 0 & 6 & 6 & 6\\
driverlog(20) & 12 & 13 & 12 $\pm$ 0 & 12 & 12 & 12 & 12 & 13 & 12.3 $\pm$ 0.6 & 12 & 12 & 13 & 13 & 13 & 13 $\pm$ 0 & 13 & 13 & 13 & 13 & 13 & 13 $\pm$ 0 & 13 & 13 & 13\\
elevators-opt11(20) & 14 & 15 & 14 $\pm$ 0 & 14 & 14 & 14 & 14 & 15 & 14.3 $\pm$ 0.6 & 14 & 15 & 14 & 15 & 15 & 15 $\pm$ 0 & 15 & 15 & 15 & 15 & 15 & 15 $\pm$ 0 & 15 & 15 & 15\\
floortile-opt11(20) & 6 & 6 & 6 $\pm$ 0 & 6 & 6 & 6 & 6 & 6 & 6 $\pm$ 0 & 6 & 6 & 6 & 6 & 6 & 6 $\pm$ 0 & 6 & 6 & 6 & 6 & 6 & 6 $\pm$ 0 & 6 & 6 & 6\\
freecell(80) & 8 & 9 & 9 $\pm$ 0 & 9 & 9 & 9 & 9 & 9 & 9 $\pm$ 0 & 9 & 9 & 9 & 9 & 9 & 9 $\pm$ 0 & 9 & 9 & 9 & 9 & 9 & 9 $\pm$ 0 & 9 & 9 & 9\\
grid(5) & 1 & 1 & 1 $\pm$ 0 & 1 & 1 & 1 & 1 & 1 & 1 $\pm$ 0 & 1 & 1 & 1 & 1 & 1 & 1 $\pm$ 0 & 1 & 1 & 1 & 1 & 1 & 1 $\pm$ 0 & 1 & 1 & 1\\
gripper(20) & 6 & 6 & 6 $\pm$ 0 & 6 & 6 & 6 & 6 & 6 & 6 $\pm$ 0 & 6 & 6 & 6 & 6 & 6 & 6 $\pm$ 0 & 6 & 6 & 6 & 6 & 6 & 6 $\pm$ 0 & 6 & 6 & 6\\
hanoi(30) & 12 & 12 & 12 $\pm$ 0 & 12 & 12 & 12 & 12 & 12 & 12 $\pm$ 0 & 12 & 12 & 12 & 12 & 12 & 12 $\pm$ 0 & 12 & 12 & 12 & 12 & 12 & 12 $\pm$ 0 & 12 & 12 & 12\\
logistics00(28) & 16 & 18 & 16 $\pm$ 0 & 16 & 16 & 16 & 20 & 20 & 20 $\pm$ 0 & 20 & 20 & 20 & 20 & 20 & 20 $\pm$ 0 & 20 & 20 & 20 & 20 & 20 & 20 $\pm$ 0 & 20 & 20 & 20\\
miconic(150) & 68 & 140 & 68 $\pm$ 0 & 68 & 68 & 68 & 125 & 121 & 139 $\pm$ 0 & 139 & 139 & 139 & 140 & 140 & 140 $\pm$ 0 & 140 & 140 & 140 & 140 & 140 & 140 $\pm$ 0 & 140 & 140 & 140\\
mprime(35) & 20 & 22 & 20 $\pm$ 0 & 20 & 20 & 20 & 22 & 22 & 21 $\pm$ 0 & 21 & 21 & 21 & 21 & 21 & 21 $\pm$ 0 & 21 & 21 & 21 & 21 & 21 & 21 $\pm$ 0 & 21 & 21 & 21\\
mystery(30) & 15 & 16 & 15 $\pm$ 0 & 15 & 15 & 15 & 16 & 16 & 15.7 $\pm$ 0.6 & 16 & 15 & 16 & 16 & 16 & 15.7 $\pm$ 0.6 & 16 & 15 & 16 & 16 & 16 & 16 $\pm$ 0 & 16 & 16 & 16\\
nomystery-opt11(20) & 12 & 13 & 12 $\pm$ 0 & 12 & 12 & 12 & 12 & 13 & 13.3 $\pm$ 0.6 & 13 & 14 & 13 & 14 & 14 & 14 $\pm$ 0 & 14 & 14 & 14 & 14 & 14 & 14 $\pm$ 0 & 14 & 14 & 14\\
\textbf{openstacks-opt11(20)} & 11 & 18 & 11.3 $\pm$ 0.6 & 11 & 12 & 11 & 17 & 18 & 18 $\pm$ 0 & 18 & 18 & 18 & 11 & 18 & 12 $\pm$ 0 & 12 & 12 & 12 & 18 & 18 & 18 $\pm$ 0 & 18 & 18 & 18\\
parcprinter-opt11(20) & 12 & 13 & 12 $\pm$ 0 & 12 & 12 & 12 & 12 & 13 & 13 $\pm$ 0 & 13 & 13 & 13 & 13 & 13 & 13 $\pm$ 0 & 13 & 13 & 13 & 13 & 13 & 13 $\pm$ 0 & 13 & 13 & 13\\
parking-opt11(20) & 1 & 1 & 1 $\pm$ 0 & 1 & 1 & 1 & 1 & 1 & 1 $\pm$ 0 & 1 & 1 & 1 & 1 & 1 & 1 $\pm$ 0 & 1 & 1 & 1 & 1 & 1 & 1 $\pm$ 0 & 1 & 1 & 1\\
pathways(30) & 4 & 5 & 4 $\pm$ 0 & 4 & 4 & 4 & 5 & 5 & 5 $\pm$ 0 & 5 & 5 & 5 & 5 & 5 & 5 $\pm$ 0 & 5 & 5 & 5 & 5 & 5 & 5 $\pm$ 0 & 5 & 5 & 5\\
pegsol-opt11(20) & 17 & 17 & 17 $\pm$ 0 & 17 & 17 & 17 & 17 & 17 & 17 $\pm$ 0 & 17 & 17 & 17 & 17 & 17 & 17 $\pm$ 0 & 17 & 17 & 17 & 17 & 17 & 17 $\pm$ 0 & 17 & 17 & 17\\
pipesworld-notankage(50) & 13 & 13 & 13 $\pm$ 0 & 13 & 13 & 13 & 13 & 13 & 13.7 $\pm$ 0.6 & 14 & 13 & 14 & 14 & 14 & 14.7 $\pm$ 0.6 & 15 & 14 & 15 & 14 & 15 & 14.3 $\pm$ 0.6 & 14 & 14 & 15\\
pipesworld-tankage(50) & 7 & 8 & 8 $\pm$ 0 & 8 & 8 & 8 & 8 & 8 & 8 $\pm$ 0 & 8 & 8 & 8 & 8 & 8 & 8 $\pm$ 0 & 8 & 8 & 8 & 8 & 8 & 8 $\pm$ 0 & 8 & 8 & 8\\
psr-small(50) & 48 & 48 & 48 $\pm$ 0 & 48 & 48 & 48 & 48 & 48 & 48 $\pm$ 0 & 48 & 48 & 48 & 48 & 48 & 48 $\pm$ 0 & 48 & 48 & 48 & 48 & 48 & 48 $\pm$ 0 & 48 & 48 & 48\\
rovers(40) & 7 & 7 & 7 $\pm$ 0 & 7 & 7 & 7 & 7 & 7 & 7 $\pm$ 0 & 7 & 7 & 7 & 7 & 7 & 7 $\pm$ 0 & 7 & 7 & 7 & 7 & 7 & 7 $\pm$ 0 & 7 & 7 & 7\\
scanalyzer-opt11(20) & 4 & 10 & 5.7 $\pm$ 0.6 & 5 & 6 & 6 & 8 & 9 & 8.7 $\pm$ 0.6 & 9 & 8 & 9 & 10 & 10 & 10 $\pm$ 0 & 10 & 10 & 10 & 10 & 10 & 10 $\pm$ 0 & 10 & 10 & 10\\
sokoban-opt11(20) & 19 & 19 & 19 $\pm$ 0 & 19 & 19 & 19 & 19 & 19 & 19 $\pm$ 0 & 19 & 19 & 19 & 19 & 19 & 19 $\pm$ 0 & 19 & 19 & 19 & 19 & 19 & 19 $\pm$ 0 & 19 & 19 & 19\\
storage(30) & 14 & 14 & 14 $\pm$ 0 & 14 & 14 & 14 & 14 & 14 & 15 $\pm$ 0 & 15 & 15 & 15 & 14 & 14 & 14 $\pm$ 0 & 14 & 14 & 14 & 14 & 14 & 14 $\pm$ 0 & 14 & 14 & 14\\
tidybot-opt11(20) & 11 & 12 & 11 $\pm$ 0 & 11 & 11 & 11 & 11 & 12 & 12 $\pm$ 0 & 12 & 12 & 12 & 12 & 12 & 12 $\pm$ 0 & 12 & 12 & 12 & 12 & 12 & 12 $\pm$ 0 & 12 & 12 & 12\\
tpp(30) & 6 & 6 & 6 $\pm$ 0 & 6 & 6 & 6 & 6 & 6 & 6 $\pm$ 0 & 6 & 6 & 6 & 6 & 6 & 6 $\pm$ 0 & 6 & 6 & 6 & 6 & 6 & 6 $\pm$ 0 & 6 & 6 & 6\\
transport-opt11(20) & 6 & 6 & 6 $\pm$ 0 & 6 & 6 & 6 & 6 & 6 & 6 $\pm$ 0 & 6 & 6 & 6 & 6 & 6 & 6 $\pm$ 0 & 6 & 6 & 6 & 6 & 6 & 6 $\pm$ 0 & 6 & 6 & 6\\
visitall-opt11(20) & 9 & 10 & 9.7 $\pm$ 0.6 & 10 & 9 & 10 & 10 & 10 & 10 $\pm$ 0 & 10 & 10 & 10 & 10 & 10 & 10 $\pm$ 0 & 10 & 10 & 10 & 10 & 10 & 10 $\pm$ 0 & 10 & 10 & 10\\
woodworking-opt11(20) & 6 & 9 & 8.3 $\pm$ 0.6 & 8 & 9 & 8 & 6 & 11 & 12 $\pm$ 0 & 12 & 12 & 12 & 10 & 10 & 10 $\pm$ 0 & 10 & 10 & 10 & 10 & 10 & 10 $\pm$ 0 & 10 & 10 & 10\\
zenotravel(20) & 9 & 11 & 9 $\pm$ 0 & 9 & 9 & 9 & 11 & 11 & 11 $\pm$ 0 & 11 & 11 & 11 & 11 & 11 & 11 $\pm$ 0 & 11 & 11 & 11 & 11 & 11 & 11 $\pm$ 0 & 11 & 11 & 11\\
\hline
\end{tabular}
\end{center}

 \caption{
 Coverage comparison \textbf{with \lmcut, without $h$-tiebreaking, on 1104 standard IPC benchmark instances}. We highlight the
 best results when the difference between the maximum and the mininum coverage exceeds 2.
 }
 \label{tbl:lmcut-ipc-noh}
 }
\end{table}
\begin{table}[htbp]
 {
 \centering
 \setlength{\tabcolsep}{0.1em}
 \begin{center}
\begin{tabular}{|r|cccHHH|cccHHHHHHHHHHHHHHH|}
\hline
Domain & $[f,\fifo]$ & $[f,\lifo]$ & $[f,\ro]$ & R & R & R & $[f,\depth,\fifo]$ & $[f,\depth,\fifo]$ & $[f,\depth,\ro]$ & R & R & R & $[f,h,\fifo]$ & $[f,h,\fifo]$ & $[f,h,\ro]$ & R & R & R & $[f,h,\depth,\fifo]$ & $[f,h,\depth,\fifo]$ & $[f,h,\depth,\ro]$ & R & R & R\\
\hline
IPC benchmark (1104) & 460 & 490 & 462 $\pm$ 2 & 464 & 462 & 460 & 483 & 484 & 483.3 $\pm$ 0.6 & 483 & 484 & 483 & 491 & 496 & 490 $\pm$ 1 & 491 & 490 & 489 & 487 & 487 & 485.7 $\pm$ 1.5 & 487 & 484 & 486\\
\hline
airport(50) & 9 & 9 & 9 $\pm$ 0 & 9 & 9 & 9 & 9 & 9 & 9 $\pm$ 0 & 9 & 9 & 9 & 9 & 9 & 9 $\pm$ 0 & 9 & 9 & 9 & 9 & 9 & 9 $\pm$ 0 & 9 & 9 & 9\\
barman-opt11(20) & 4 & 4 & 4 $\pm$ 0 & 4 & 4 & 4 & 4 & 4 & 4 $\pm$ 0 & 4 & 4 & 4 & 4 & 4 & 4 $\pm$ 0 & 4 & 4 & 4 & 4 & 4 & 4 $\pm$ 0 & 4 & 4 & 4\\
blocks(35) & 21 & 22 & 21 $\pm$ 0 & 21 & 21 & 21 & 21 & 22 & 21.3 $\pm$ 0.6 & 21 & 22 & 21 & 22 & 22 & 22 $\pm$ 0 & 22 & 22 & 22 & 22 & 21 & 21.7 $\pm$ 0.6 & 22 & 21 & 22\\
\textbf{cybersec(19)} & 0 & 0 & 0 $\pm$ 0 & 0 & 0 & 0 & 0 & 0 & 0 $\pm$ 0 & 0 & 0 & 0 & 0 & 0 & 0 $\pm$ 0 & 0 & 0 & 0 & 0 & 0 & 0 $\pm$ 0 & 0 & 0 & 0\\
depot(22) & 5 & 6 & 5 $\pm$ 0 & 5 & 5 & 5 & 5 & 5 & 5 $\pm$ 0 & 5 & 5 & 5 & 6 & 6 & 5 $\pm$ 0 & 5 & 5 & 5 & 5 & 5 & 5 $\pm$ 0 & 5 & 5 & 5\\
driverlog(20) & 12 & 12 & 12 $\pm$ 0 & 12 & 12 & 12 & 12 & 12 & 12 $\pm$ 0 & 12 & 12 & 12 & 12 & 12 & 12 $\pm$ 0 & 12 & 12 & 12 & 12 & 12 & 12 $\pm$ 0 & 12 & 12 & 12\\
elevators-opt11(20) & 13 & 13 & 13 $\pm$ 0 & 13 & 13 & 13 & 11 & 11 & 12 $\pm$ 0 & 12 & 12 & 12 & 13 & 13 & 13 $\pm$ 0 & 13 & 13 & 13 & 12 & 12 & 12 $\pm$ 0 & 12 & 12 & 12\\
floortile-opt11(20) & 5 & 6 & 5 $\pm$ 0 & 5 & 5 & 5 & 5 & 5 & 5 $\pm$ 0 & 5 & 5 & 5 & 6 & 6 & 6 $\pm$ 0 & 6 & 6 & 6 & 6 & 6 & 6 $\pm$ 0 & 6 & 6 & 6\\
freecell(80) & 15 & 16 & 15 $\pm$ 0 & 15 & 15 & 15 & 16 & 16 & 16 $\pm$ 0 & 16 & 16 & 16 & 17 & 17 & 16 $\pm$ 0 & 16 & 16 & 16 & 16 & 16 & 16 $\pm$ 0 & 16 & 16 & 16\\
grid(5) & 2 & 2 & 2 $\pm$ 0 & 2 & 2 & 2 & 2 & 2 & 2 $\pm$ 0 & 2 & 2 & 2 & 2 & 2 & 2 $\pm$ 0 & 2 & 2 & 2 & 2 & 2 & 2 $\pm$ 0 & 2 & 2 & 2\\
gripper(20) & 8 & 20 & 8 $\pm$ 0 & 8 & 8 & 8 & 20 & 20 & 20 $\pm$ 0 & 20 & 20 & 20 & 20 & 20 & 20 $\pm$ 0 & 20 & 20 & 20 & 20 & 20 & 20 $\pm$ 0 & 20 & 20 & 20\\
hanoi(30) & 14 & 14 & 14 $\pm$ 0 & 14 & 14 & 14 & 14 & 14 & 14 $\pm$ 0 & 14 & 14 & 14 & 14 & 14 & 14 $\pm$ 0 & 14 & 14 & 14 & 14 & 14 & 14 $\pm$ 0 & 14 & 14 & 14\\
logistics00(28) & 20 & 20 & 20 $\pm$ 0 & 20 & 20 & 20 & 20 & 20 & 20 $\pm$ 0 & 20 & 20 & 20 & 20 & 20 & 20 $\pm$ 0 & 20 & 20 & 20 & 20 & 20 & 20 $\pm$ 0 & 20 & 20 & 20\\
miconic(150) & 68 & 73 & 68.7 $\pm$ 1.2 & 70 & 68 & 68 & 73 & 73 & 73 $\pm$ 1 & 73 & 72 & 74 & 73 & 73 & 73.3 $\pm$ 0.6 & 73 & 73 & 74 & 73 & 73 & 73 $\pm$ 1 & 73 & 72 & 74\\
mprime(35) & 23 & 23 & 22.3 $\pm$ 0.6 & 22 & 22 & 23 & 23 & 23 & 23.3 $\pm$ 0.6 & 23 & 24 & 23 & 23 & 24 & 23.7 $\pm$ 0.6 & 24 & 23 & 24 & 23 & 24 & 23.7 $\pm$ 0.6 & 24 & 23 & 24\\
mystery(30) & 15 & 15 & 15 $\pm$ 0 & 15 & 15 & 15 & 15 & 15 & 15 $\pm$ 0 & 15 & 15 & 15 & 15 & 16 & 15 $\pm$ 0 & 15 & 15 & 15 & 15 & 16 & 15 $\pm$ 0 & 15 & 15 & 15\\
nomystery-opt11(20) & 17 & 18 & 18 $\pm$ 0 & 18 & 18 & 18 & 18 & 18 & 18 $\pm$ 0 & 18 & 18 & 18 & 18 & 18 & 18 $\pm$ 0 & 18 & 18 & 18 & 18 & 18 & 18 $\pm$ 0 & 18 & 18 & 18\\
\textbf{openstacks-opt11(20)} & 15 & 19 & 15.7 $\pm$ 0.6 & 16 & 16 & 15 & 19 & 19 & 19 $\pm$ 0 & 19 & 19 & 19 & 15 & 19 & 15.7 $\pm$ 0.6 & 16 & 16 & 15 & 19 & 19 & 19 $\pm$ 0 & 19 & 19 & 19\\
parcprinter-opt11(20) & 10 & 10 & 10 $\pm$ 0 & 10 & 10 & 10 & 10 & 10 & 10 $\pm$ 0 & 10 & 10 & 10 & 10 & 10 & 10 $\pm$ 0 & 10 & 10 & 10 & 10 & 10 & 10 $\pm$ 0 & 10 & 10 & 10\\
parking-opt11(20) & 1 & 1 & 1 $\pm$ 0 & 1 & 1 & 1 & 1 & 1 & 1 $\pm$ 0 & 1 & 1 & 1 & 1 & 1 & 1 $\pm$ 0 & 1 & 1 & 1 & 1 & 1 & 1 $\pm$ 0 & 1 & 1 & 1\\
pathways(30) & 4 & 4 & 4 $\pm$ 0 & 4 & 4 & 4 & 4 & 4 & 4 $\pm$ 0 & 4 & 4 & 4 & 4 & 4 & 4 $\pm$ 0 & 4 & 4 & 4 & 4 & 4 & 4 $\pm$ 0 & 4 & 4 & 4\\
pegsol-opt11(20) & 17 & 19 & 17.3 $\pm$ 0.6 & 17 & 18 & 17 & 18 & 19 & 19 $\pm$ 0 & 19 & 19 & 19 & 19 & 19 & 19 $\pm$ 0 & 19 & 19 & 19 & 19 & 19 & 19 $\pm$ 0 & 19 & 19 & 19\\
pipesworld-notankage(50) & 9 & 9 & 8.7 $\pm$ 0.6 & 9 & 9 & 8 & 10 & 9 & 8.7 $\pm$ 0.6 & 9 & 9 & 8 & 10 & 10 & 9.7 $\pm$ 0.6 & 10 & 10 & 9 & 10 & 9 & 9.7 $\pm$ 0.6 & 10 & 10 & 9\\
pipesworld-tankage(50) & 13 & 13 & 13.3 $\pm$ 0.6 & 14 & 13 & 13 & 13 & 13 & 13 $\pm$ 0 & 13 & 13 & 13 & 13 & 13 & 13.7 $\pm$ 0.6 & 14 & 14 & 13 & 13 & 13 & 13 $\pm$ 0 & 13 & 13 & 13\\
psr-small(50) & 50 & 50 & 50 $\pm$ 0 & 50 & 50 & 50 & 50 & 50 & 50 $\pm$ 0 & 50 & 50 & 50 & 50 & 50 & 50 $\pm$ 0 & 50 & 50 & 50 & 50 & 50 & 50 $\pm$ 0 & 50 & 50 & 50\\
rovers(40) & 6 & 8 & 6 $\pm$ 0 & 6 & 6 & 6 & 8 & 8 & 7 $\pm$ 0 & 7 & 7 & 7 & 8 & 8 & 8 $\pm$ 0 & 8 & 8 & 8 & 8 & 8 & 7 $\pm$ 0 & 7 & 7 & 7\\
scanalyzer-opt11(20) & 10 & 10 & 10 $\pm$ 0 & 10 & 10 & 10 & 10 & 10 & 10.3 $\pm$ 0.6 & 10 & 10 & 11 & 10 & 10 & 10 $\pm$ 0 & 10 & 10 & 10 & 10 & 10 & 10 $\pm$ 0 & 10 & 10 & 10\\
sokoban-opt11(20) & 20 & 20 & 20 $\pm$ 0 & 20 & 20 & 20 & 19 & 19 & 18.7 $\pm$ 0.6 & 19 & 19 & 18 & 20 & 20 & 20 $\pm$ 0 & 20 & 20 & 20 & 19 & 19 & 18.7 $\pm$ 0.6 & 19 & 19 & 18\\
storage(30) & 15 & 15 & 15 $\pm$ 0 & 15 & 15 & 15 & 15 & 15 & 15 $\pm$ 0 & 15 & 15 & 15 & 15 & 15 & 15 $\pm$ 0 & 15 & 15 & 15 & 15 & 15 & 15 $\pm$ 0 & 15 & 15 & 15\\
tidybot-opt11(20) & 0 & 0 & 0 $\pm$ 0 & 0 & 0 & 0 & 0 & 0 & 0 $\pm$ 0 & 0 & 0 & 0 & 0 & 0 & 0 $\pm$ 0 & 0 & 0 & 0 & 0 & 0 & 0 $\pm$ 0 & 0 & 0 & 0\\
tpp(30) & 6 & 6 & 6 $\pm$ 0 & 6 & 6 & 6 & 6 & 6 & 6 $\pm$ 0 & 6 & 6 & 6 & 7 & 6 & 6 $\pm$ 0 & 6 & 6 & 6 & 6 & 6 & 6 $\pm$ 0 & 6 & 6 & 6\\
transport-opt11(20) & 7 & 7 & 7 $\pm$ 0 & 7 & 7 & 7 & 6 & 6 & 6 $\pm$ 0 & 6 & 6 & 6 & 7 & 7 & 7 $\pm$ 0 & 7 & 7 & 7 & 6 & 6 & 6 $\pm$ 0 & 6 & 6 & 6\\
visitall-opt11(20) & 9 & 9 & 9 $\pm$ 0 & 9 & 9 & 9 & 9 & 9 & 9 $\pm$ 0 & 9 & 9 & 9 & 9 & 9 & 9 $\pm$ 0 & 9 & 9 & 9 & 9 & 9 & 9 $\pm$ 0 & 9 & 9 & 9\\
woodworking-opt11(20) & 7 & 7 & 7 $\pm$ 0 & 7 & 7 & 7 & 7 & 7 & 7 $\pm$ 0 & 7 & 7 & 7 & 7 & 7 & 7 $\pm$ 0 & 7 & 7 & 7 & 7 & 7 & 7 $\pm$ 0 & 7 & 7 & 7\\
zenotravel(20) & 10 & 10 & 10 $\pm$ 0 & 10 & 10 & 10 & 10 & 10 & 10 $\pm$ 0 & 10 & 10 & 10 & 12 & 12 & 12 $\pm$ 0 & 12 & 12 & 12 & 10 & 10 & 10 $\pm$ 0 & 10 & 10 & 10\\
\hline
\end{tabular}
\end{center}

 \caption{
 Coverage comparison \textbf{with \mands, without $h$-tiebreaking, on 1104 standard IPC benchmark instances}. We highlight the
 best results when the difference between the maximum and the mininum coverage exceeds 2.
 }
 \label{tbl:mands-ipc-noh}
 }
\end{table}

\begin{table}[htbp]
 {
 \centering
 \setlength{\tabcolsep}{0.1em}
 \begin{center}
\begin{tabular}{|rcccHHH|cccHHHHHHHHHHHHHHH|}
 & \([f,\fifo]\) & \([f,\lifo]\) & \([f,\ro]\) & R & R & R & \([f,\depth,\fifo]\) & \([f,\depth,\lifo]\) & \([f,\depth,\ro]\) & R & R & R & \([f,h,\fifo]\) & \([f,h,\lifo]\) & \([f,h,\ro]\) & R & R & R & \([f,h,\depth,\fifo]\) & \([f,h,\depth,\lifo]\) & \([f,h,\depth,\ro]\) & R & R & R\\
\hline
 & 227 & 296 & 238.3 \(\pm\) 1.5 & 237 & 238 & 240 & 284 & 276 & 294.7 \(\pm\) 3.1 & 298 & 292 & 294 & 271 & 294 & 276.7 \(\pm\) 1.2 & 276 & 278 & 276 & 299 & 279 & 303.7 \(\pm\) 1.5 & 305 & 304 & 302\\
\hline
airport-fuel(20) & 7 & 15 & 7 \(\pm\) 0 & 7 & 7 & 7 & 10 & 13 & 10.3 \(\pm\) 0.6 & 10 & 10 & 11 & 15 & 13 & 13.7 \(\pm\) 0.6 & 13 & 14 & 14 & 14 & 13 & 14 \(\pm\) 0 & 14 & 14 & 14\\
blocks-stack(20) & 15 & 17 & 15 \(\pm\) 0 & 15 & 15 & 15 & 17 & 18 & 17.7 \(\pm\) 0.6 & 18 & 18 & 17 & 17 & 17 & 17 \(\pm\) 0 & 17 & 17 & 17 & 17 & 17 & 17 \(\pm\) 0 & 17 & 17 & 17\\
depot-fuel(22) & 4 & 6 & 5.3 \(\pm\) 0.6 & 5 & 5 & 6 & 6 & 6 & 6 \(\pm\) 0 & 6 & 6 & 6 & 6 & 6 & 6 \(\pm\) 0 & 6 & 6 & 6 & 6 & 6 & 6 \(\pm\) 0 & 6 & 6 & 6\\
driverlog-fuel(20) & 7 & 8 & 7 \(\pm\) 0 & 7 & 7 & 7 & 8 & 8 & 8 \(\pm\) 0 & 8 & 8 & 8 & 8 & 8 & 8 \(\pm\) 0 & 8 & 8 & 8 & 8 & 8 & 8 \(\pm\) 0 & 8 & 8 & 8\\
elevators-up(20) & 7 & 13 & 7 \(\pm\) 0 & 7 & 7 & 7 & 7 & 9 & 8.7 \(\pm\) 0.6 & 9 & 9 & 8 & 7 & 13 & 7 \(\pm\) 0 & 7 & 7 & 7 & 7 & 9 & 8.7 \(\pm\) 0.6 & 9 & 9 & 8\\
floortile-ink(20) & 8 & 8 & 8 \(\pm\) 0 & 8 & 8 & 8 & 8 & 8 & 8 \(\pm\) 0 & 8 & 8 & 8 & 8 & 8 & 8.3 \(\pm\) 0.6 & 9 & 8 & 8 & 8 & 8 & 8.3 \(\pm\) 0.6 & 9 & 8 & 8\\
freecell-move(20) & 4 & 19 & 5 \(\pm\) 0 & 5 & 5 & 5 & 17 & 10 & 16.7 \(\pm\) 0.6 & 17 & 16 & 17 & 4 & 19 & 5 \(\pm\) 0 & 5 & 5 & 5 & 17 & 10 & 16.3 \(\pm\) 0.6 & 17 & 16 & 16\\
 & 15 & 15 & 15 \(\pm\) 0 & 15 & 15 & 15 & 13 & 15 & 14.7 \(\pm\) 0.6 & 15 & 15 & 14 & 15 & 15 & 15 \(\pm\) 0 & 15 & 15 & 15 & 15 & 15 & 15 \(\pm\) 0 & 15 & 15 & 15\\
grid-fuel(5) & 1 & 1 & 1 \(\pm\) 0 & 1 & 1 & 1 & 1 & 1 & 1 \(\pm\) 0 & 1 & 1 & 1 & 1 & 1 & 1 \(\pm\) 0 & 1 & 1 & 1 & 1 & 1 & 1 \(\pm\) 0 & 1 & 1 & 1\\
gripper-move(20) & 7 & 7 & 7 \(\pm\) 0 & 7 & 7 & 7 & 7 & 7 & 7 \(\pm\) 0 & 7 & 7 & 7 & 7 & 7 & 7 \(\pm\) 0 & 7 & 7 & 7 & 7 & 7 & 7 \(\pm\) 0 & 7 & 7 & 7\\
hiking-fuel(20) & 8 & 9 & 8 \(\pm\) 0 & 8 & 8 & 8 & 9 & 9 & 9 \(\pm\) 0 & 9 & 9 & 9 & 9 & 9 & 9 \(\pm\) 0 & 9 & 9 & 9 & 9 & 9 & 9 \(\pm\) 0 & 9 & 9 & 9\\
logistics00-fuel(28) & 15 & 16 & 15 \(\pm\) 0 & 15 & 15 & 15 & 15 & 16 & 15 \(\pm\) 0 & 15 & 15 & 15 & 16 & 16 & 16 \(\pm\) 0 & 16 & 16 & 16 & 16 & 16 & 15.3 \(\pm\) 0.6 & 16 & 15 & 15\\
miconic-up(30) & 10 & 17 & 10 \(\pm\) 0 & 10 & 10 & 10 & 19 & 18 & 19.3 \(\pm\) 1.2 & 20 & 20 & 18 & 16 & 17 & 16.7 \(\pm\) 0.6 & 16 & 17 & 17 & 19 & 18 & 20.3 \(\pm\) 0.6 & 20 & 21 & 20\\
mprime-succumb(35) & 12 & 14 & 11.3 \(\pm\) 1.2 & 10 & 12 & 12 & 21 & 14 & 19.7 \(\pm\) 1.2 & 19 & 19 & 21 & 15 & 14 & 16.7 \(\pm\) 0.6 & 17 & 17 & 16 & 22 & 14 & 20.3 \(\pm\) 0.6 & 20 & 20 & 21\\
mystery-feast(20) & 5 & 5 & 6.3 \(\pm\) 1.2 & 7 & 5 & 7 & 6 & 7 & 6.3 \(\pm\) 0.6 & 7 & 6 & 6 & 7 & 5 & 7.7 \(\pm\) 0.6 & 7 & 8 & 8 & 6 & 5 & 7.3 \(\pm\) 1.5 & 6 & 9 & 7\\
nomystery-fuel(20) & 9 & 10 & 9 \(\pm\) 0 & 9 & 9 & 9 & 9 & 10 & 9.3 \(\pm\) 0.6 & 10 & 9 & 9 & 10 & 10 & 10 \(\pm\) 0 & 10 & 10 & 10 & 10 & 10 & 10 \(\pm\) 0 & 10 & 10 & 10\\
parking-movecc(20) & 0 & 0 & 0 \(\pm\) 0 & 0 & 0 & 0 & 0 & 0 & 0 \(\pm\) 0 & 0 & 0 & 0 & 0 & 0 & 0 \(\pm\) 0 & 0 & 0 & 0 & 0 & 0 & 0 \(\pm\) 0 & 0 & 0 & 0\\
pathways-fuel(30) & 4 & 5 & 4 \(\pm\) 0 & 4 & 4 & 4 & 4 & 5 & 4.3 \(\pm\) 0.6 & 4 & 5 & 4 & 5 & 5 & 4.7 \(\pm\) 0.6 & 5 & 5 & 4 & 5 & 5 & 4.3 \(\pm\) 0.6 & 5 & 4 & 4\\
pipesnt-pushstart(20) & 6 & 7 & 8.3 \(\pm\) 0.6 & 8 & 8 & 9 & 8 & 6 & 10 \(\pm\) 0 & 10 & 10 & 10 & 8 & 8 & 8.3 \(\pm\) 0.6 & 8 & 8 & 9 & 8 & 8 & 10 \(\pm\) 0 & 10 & 10 & 10\\
pipesworld-pushend(20) & 2 & 4 & 2.7 \(\pm\) 0.6 & 3 & 3 & 2 & 4 & 3 & 5.3 \(\pm\) 0.6 & 6 & 5 & 5 & 3 & 4 & 3.7 \(\pm\) 0.6 & 4 & 4 & 3 & 3 & 3 & 5 \(\pm\) 0 & 5 & 5 & 5\\
psr-small-open(20) & 19 & 19 & 19 \(\pm\) 0 & 19 & 19 & 19 & 19 & 19 & 19 \(\pm\) 0 & 19 & 19 & 19 & 19 & 19 & 19 \(\pm\) 0 & 19 & 19 & 19 & 19 & 19 & 19 \(\pm\) 0 & 19 & 19 & 19\\
rovers-fuel(40) & 7 & 9 & 7 \(\pm\) 0 & 7 & 7 & 7 & 8 & 9 & 9 \(\pm\) 0 & 9 & 9 & 9 & 8 & 8 & 8 \(\pm\) 0 & 8 & 8 & 8 & 8 & 8 & 8 \(\pm\) 0 & 8 & 8 & 8\\
scanalyzer-analyze(20) & 3 & 9 & 3 \(\pm\) 0 & 3 & 3 & 3 & 6 & 5 & 5 \(\pm\) 0 & 5 & 5 & 5 & 9 & 9 & 9 \(\pm\) 0 & 9 & 9 & 9 & 9 & 10 & 9.3 \(\pm\) 0.6 & 10 & 9 & 9\\
sokoban-pushgoal(20) & 18 & 18 & 18 \(\pm\) 0 & 18 & 18 & 18 & 18 & 18 & 17.7 \(\pm\) 0.6 & 18 & 17 & 18 & 18 & 18 & 18 \(\pm\) 0 & 18 & 18 & 18 & 18 & 18 & 18 \(\pm\) 0 & 18 & 18 & 18\\
storage-lift(20) & 4 & 4 & 4 \(\pm\) 0 & 4 & 4 & 4 & 5 & 5 & 5 \(\pm\) 0 & 5 & 5 & 5 & 4 & 4 & 4 \(\pm\) 0 & 4 & 4 & 4 & 5 & 4 & 4 \(\pm\) 0 & 4 & 4 & 4\\
tidybot-motion(20) & 14 & 16 & 14.7 \(\pm\) 0.6 & 14 & 15 & 15 & 15 & 15 & 16 \(\pm\) 0 & 16 & 16 & 16 & 16 & 16 & 16 \(\pm\) 0 & 16 & 16 & 16 & 16 & 16 & 16 \(\pm\) 0 & 16 & 16 & 16\\
tpp-fuel(30) & 7 & 11 & 8 \(\pm\) 0 & 8 & 8 & 8 & 10 & 10 & 11 \(\pm\) 0 & 11 & 11 & 11 & 8 & 11 & 8 \(\pm\) 0 & 8 & 8 & 8 & 11 & 10 & 11 \(\pm\) 0 & 11 & 11 & 11\\
woodworking-cut(20) & 2 & 7 & 5.7 \(\pm\) 0.6 & 6 & 6 & 5 & 7 & 5 & 8.7 \(\pm\) 1.5 & 9 & 7 & 10 & 5 & 7 & 7 \(\pm\) 0 & 7 & 7 & 7 & 8 & 5 & 8.3 \(\pm\) 0.6 & 8 & 8 & 9\\
zenotravel-fuel(20) & 7 & 7 & 7 \(\pm\) 0 & 7 & 7 & 7 & 7 & 7 & 7 \(\pm\) 0 & 7 & 7 & 7 & 7 & 7 & 7 \(\pm\) 0 & 7 & 7 & 7 & 7 & 7 & 7 \(\pm\) 0 & 7 & 7 & 7\\
\end{tabular}
\end{center}

 \caption{
 Coverage comparison \textbf{with \lmcut, without $h$-tiebreaking, on 620 zerocost instances}. We highlight the
 best results when the difference between the maximum and the mininum coverage exceeds 2.
 }
 \label{tbl:lmcut-zero-noh}
 }
\end{table}
\begin{table}[htbp]
 {
 \centering
 \setlength{\tabcolsep}{0.1em}
 \begin{center}
\begin{tabular}{|rcccHHH|cccHHHHHHHHHHHHHHH|}
 & \([f,\fifo]\) & \([f,\lifo]\) & \([f,\ro]\) & R & R & R & \([f,\depth,\fifo]\) & \([f,\depth,\lifo]\) & \([f,\depth,\ro]\) & R & R & R & \([f,h,\fifo]\) & \([f,h,\lifo]\) & \([f,h,\ro]\) & R & R & R & \([f,h,\depth,\fifo]\) & \([f,h,\depth,\lifo]\) & \([f,h,\depth,\ro]\) & R & R & R\\
\hline
 & 250 & 315 & 270 \(\pm\) 1 & 269 & 271 & 270 & 310 & 289 & 316 \(\pm\) 1.7 & 317 & 314 & 317 & 295 & 316 & 304 \(\pm\) 0 & 304 & 304 & 304 & 317 & 303 & 324.7 \(\pm\) 2.3 & 326 & 322 & 326\\
\hline
airport-fuel(20) & 5 & 5 & 5 \(\pm\) 0 & 5 & 5 & 5 & 5 & 5 & 5 \(\pm\) 0 & 5 & 5 & 5 & 5 & 5 & 5 \(\pm\) 0 & 5 & 5 & 5 & 5 & 5 & 5 \(\pm\) 0 & 5 & 5 & 5\\
blocks-stack(20) & 20 & 20 & 20 \(\pm\) 0 & 20 & 20 & 20 & 20 & 20 & 20 \(\pm\) 0 & 20 & 20 & 20 & 20 & 20 & 20 \(\pm\) 0 & 20 & 20 & 20 & 20 & 20 & 20 \(\pm\) 0 & 20 & 20 & 20\\
depot-fuel(22) & 5 & 5 & 6 \(\pm\) 0 & 6 & 6 & 6 & 6 & 5 & 6 \(\pm\) 0 & 6 & 6 & 6 & 5 & 5 & 6 \(\pm\) 0 & 6 & 6 & 6 & 6 & 5 & 6 \(\pm\) 0 & 6 & 6 & 6\\
driverlog-fuel(20) & 8 & 9 & 8 \(\pm\) 0 & 8 & 8 & 8 & 9 & 9 & 9 \(\pm\) 0 & 9 & 9 & 9 & 9 & 9 & 9 \(\pm\) 0 & 9 & 9 & 9 & 9 & 9 & 9 \(\pm\) 0 & 9 & 9 & 9\\
elevators-up(20) & 8 & 14 & 8.3 \(\pm\) 0.6 & 8 & 8 & 9 & 9 & 13 & 11.3 \(\pm\) 1.2 & 12 & 12 & 10 & 8 & 14 & 8.3 \(\pm\) 0.6 & 8 & 8 & 9 & 9 & 13 & 11.3 \(\pm\) 1.2 & 12 & 12 & 10\\
floortile-ink(20) & 8 & 8 & 8 \(\pm\) 0 & 8 & 8 & 8 & 7 & 8 & 7.3 \(\pm\) 0.6 & 7 & 7 & 8 & 8 & 8 & 8 \(\pm\) 0 & 8 & 8 & 8 & 7 & 7 & 6.7 \(\pm\) 0.6 & 7 & 7 & 6\\
freecell-move(20) & 5 & 17 & 7 \(\pm\) 1 & 8 & 7 & 6 & 17 & 15 & 17 \(\pm\) 0 & 17 & 17 & 17 & 5 & 17 & 7 \(\pm\) 1 & 8 & 7 & 6 & 17 & 15 & 17 \(\pm\) 0 & 17 & 17 & 17\\
 & 15 & 15 & 15 \(\pm\) 0 & 15 & 15 & 15 & 15 & 15 & 15 \(\pm\) 0 & 15 & 15 & 15 & 15 & 15 & 15 \(\pm\) 0 & 15 & 15 & 15 & 15 & 15 & 15 \(\pm\) 0 & 15 & 15 & 15\\
grid-fuel(5) & 2 & 2 & 2 \(\pm\) 0 & 2 & 2 & 2 & 2 & 2 & 2 \(\pm\) 0 & 2 & 2 & 2 & 2 & 2 & 2 \(\pm\) 0 & 2 & 2 & 2 & 2 & 2 & 2 \(\pm\) 0 & 2 & 2 & 2\\
gripper-move(20) & 8 & 20 & 8 \(\pm\) 0 & 8 & 8 & 8 & 20 & 10 & 18.3 \(\pm\) 0.6 & 18 & 18 & 19 & 20 & 20 & 20 \(\pm\) 0 & 20 & 20 & 20 & 20 & 20 & 20 \(\pm\) 0 & 20 & 20 & 20\\
hiking-fuel(20) & 12 & 13 & 12.3 \(\pm\) 0.6 & 12 & 12 & 13 & 13 & 12 & 12 \(\pm\) 0 & 12 & 12 & 12 & 13 & 13 & 13 \(\pm\) 0 & 13 & 13 & 13 & 13 & 12 & 12.3 \(\pm\) 0.6 & 12 & 13 & 12\\
logistics00-fuel(28) & 16 & 16 & 16 \(\pm\) 0 & 16 & 16 & 16 & 16 & 16 & 16 \(\pm\) 0 & 16 & 16 & 16 & 16 & 16 & 16 \(\pm\) 0 & 16 & 16 & 16 & 16 & 16 & 16 \(\pm\) 0 & 16 & 16 & 16\\
miconic-up(30) & 19 & 30 & 19.7 \(\pm\) 0.6 & 19 & 20 & 20 & 30 & 30 & 30 \(\pm\) 0 & 30 & 30 & 30 & 29 & 30 & 30 \(\pm\) 0 & 30 & 30 & 30 & 30 & 30 & 30 \(\pm\) 0 & 30 & 30 & 30\\
mprime-succumb(35) & 14 & 19 & 15.3 \(\pm\) 0.6 & 15 & 16 & 15 & 24 & 15 & 21.3 \(\pm\) 1.2 & 22 & 20 & 22 & 21 & 19 & 19.7 \(\pm\) 0.6 & 20 & 19 & 20 & 25 & 15 & 23.7 \(\pm\) 1.5 & 24 & 22 & 25\\
mystery-feast(20) & 4 & 4 & 6 \(\pm\) 0 & 6 & 6 & 6 & 4 & 4 & 6 \(\pm\) 0 & 6 & 6 & 6 & 4 & 4 & 6 \(\pm\) 0 & 6 & 6 & 6 & 4 & 4 & 6 \(\pm\) 0 & 6 & 6 & 6\\
nomystery-fuel(20) & 15 & 16 & 16 \(\pm\) 0 & 16 & 16 & 16 & 15 & 16 & 16 \(\pm\) 0 & 16 & 16 & 16 & 16 & 16 & 16 \(\pm\) 0 & 16 & 16 & 16 & 16 & 16 & 16 \(\pm\) 0 & 16 & 16 & 16\\
parking-movecc(20) & 0 & 0 & 0 \(\pm\) 0 & 0 & 0 & 0 & 0 & 0 & 0 \(\pm\) 0 & 0 & 0 & 0 & 0 & 0 & 0 \(\pm\) 0 & 0 & 0 & 0 & 0 & 0 & 0 \(\pm\) 0 & 0 & 0 & 0\\
pathways-fuel(30) & 4 & 4 & 4 \(\pm\) 0 & 4 & 4 & 4 & 4 & 4 & 4 \(\pm\) 0 & 4 & 4 & 4 & 4 & 4 & 4 \(\pm\) 0 & 4 & 4 & 4 & 4 & 4 & 4 \(\pm\) 0 & 4 & 4 & 4\\
pipesnt-pushstart(20) & 3 & 3 & 3.3 \(\pm\) 0.6 & 3 & 4 & 3 & 5 & 3 & 5 \(\pm\) 0 & 5 & 5 & 5 & 3 & 3 & 3.3 \(\pm\) 0.6 & 3 & 4 & 3 & 5 & 3 & 5 \(\pm\) 0 & 5 & 5 & 5\\
pipesworld-pushend(20) & 3 & 9 & 7.7 \(\pm\) 0.6 & 7 & 8 & 8 & 4 & 4 & 8.7 \(\pm\) 0.6 & 9 & 8 & 9 & 5 & 9 & 8 \(\pm\) 0 & 8 & 8 & 8 & 5 & 6 & 9 \(\pm\) 1 & 9 & 8 & 10\\
psr-small-open(20) & 19 & 19 & 19 \(\pm\) 0 & 19 & 19 & 19 & 19 & 19 & 19 \(\pm\) 0 & 19 & 19 & 19 & 19 & 19 & 19 \(\pm\) 0 & 19 & 19 & 19 & 19 & 19 & 19 \(\pm\) 0 & 19 & 19 & 19\\
rovers-fuel(40) & 8 & 8 & 8 \(\pm\) 0 & 8 & 8 & 8 & 8 & 8 & 8 \(\pm\) 0 & 8 & 8 & 8 & 8 & 8 & 8 \(\pm\) 0 & 8 & 8 & 8 & 8 & 8 & 8 \(\pm\) 0 & 8 & 8 & 8\\
scanalyzer-analyze(20) & 9 & 11 & 9 \(\pm\) 0 & 9 & 9 & 9 & 9 & 9 & 8.7 \(\pm\) 0.6 & 9 & 9 & 8 & 11 & 11 & 11 \(\pm\) 0 & 11 & 11 & 11 & 11 & 11 & 11 \(\pm\) 0 & 11 & 11 & 11\\
sokoban-pushgoal(20) & 18 & 18 & 18.3 \(\pm\) 0.6 & 19 & 18 & 18 & 18 & 18 & 17 \(\pm\) 0 & 17 & 17 & 17 & 19 & 19 & 18.3 \(\pm\) 0.6 & 18 & 18 & 19 & 18 & 18 & 18 \(\pm\) 0 & 18 & 18 & 18\\
storage-lift(20) & 4 & 4 & 4 \(\pm\) 0 & 4 & 4 & 4 & 4 & 4 & 4 \(\pm\) 0 & 4 & 4 & 4 & 4 & 4 & 4 \(\pm\) 0 & 4 & 4 & 4 & 4 & 4 & 4 \(\pm\) 0 & 4 & 4 & 4\\
tidybot-motion(20) & 0 & 0 & 0 \(\pm\) 0 & 0 & 0 & 0 & 0 & 0 & 0 \(\pm\) 0 & 0 & 0 & 0 & 0 & 0 & 0 \(\pm\) 0 & 0 & 0 & 0 & 0 & 0 & 0 \(\pm\) 0 & 0 & 0 & 0\\
tpp-fuel(30) & 8 & 10 & 8 \(\pm\) 0 & 8 & 8 & 8 & 11 & 10 & 11 \(\pm\) 0 & 11 & 11 & 11 & 9 & 10 & 9.3 \(\pm\) 0.6 & 9 & 10 & 9 & 11 & 10 & 11 \(\pm\) 0 & 11 & 11 & 11\\
woodworking-cut(20) & 2 & 7 & 7 \(\pm\) 0 & 7 & 7 & 7 & 7 & 6 & 9 \(\pm\) 0 & 9 & 9 & 9 & 7 & 7 & 8 \(\pm\) 0 & 8 & 8 & 8 & 8 & 7 & 9.7 \(\pm\) 1.5 & 10 & 8 & 11\\
zenotravel-fuel(20) & 8 & 9 & 9 \(\pm\) 0 & 9 & 9 & 9 & 9 & 9 & 9.3 \(\pm\) 0.6 & 9 & 9 & 10 & 10 & 9 & 10 \(\pm\) 0 & 10 & 10 & 10 & 10 & 9 & 10 \(\pm\) 0 & 10 & 10 & 10\\
\end{tabular}
\end{center}

 \caption{
 Coverage comparison \textbf{with \mands, without $h$-tiebreaking, on 620 zerocost instances}. We highlight the
 best results when the difference between the maximum and the mininum coverage exceeds 2.
 }
 \label{tbl:mands-zero-noh}
 }
\end{table}



