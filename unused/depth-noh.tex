
%% does not seem important!

\subsection{Depth-Based Tiebreaking Without Considering $h$}

\label{sec:depth-noh}

In \refsec{sec:noh}, we showed that $[f,\lifo]$ tiebreaking
(without considering $h$) is sufficient for the standard IPC benchmarks
-- the performance of $[f,\lifo]$, $[f,h,\lifo]$, and $[f,h,\fifo]$ are
comparable.  In order to see if it also holds for depth-based
tiebreaking, we evaluated the performance of $[f,\depth,\fifo]$
and $[f,\depth,\lifo]$.
\reftbl{tbl:lmcut-zero-noh} and \reftbl{tbl:mands-zero-noh}
 shows that
 $[f,h,\depth,\fifo]$ and  $[f,h,\depth,\lifo]$
 perform comparably to $[f,h,\fifo]$ and $[f,h,\lifo]$.

% Although $[\rd,\ro]$ behaves in a less greedy/depth-first manner than $[f,\lifo]$, 
% it explores nodes with high depth sufficiently often so that even if \lifo behavior (seeking nodes that are far from the plateau entrance) is required, $[\rd,\ro]$ will eventually find the solution.
% Moreover, there are some domains (\pddl{pipesworld-pushend} and \pddl{woodworking-opt11}) where a the more randomized behavior of $[\rd,\ro]$ is advantageous.
% Thus, overall, $[\rd,\ro]$ performs moderately well, and 
% \emph{neither $h$ nor \lifo-behavior is necessary in order to obtain performance that is competitive with the standard
% tiebreaking strategies}.

\begin{table}[htbp]
 {
 \centering
 \setlength{\tabcolsep}{0.1em}
 \input{tables/7-ipc-lmcut-noh}
 \caption{
 Coverage comparison \textbf{with \lmcut, without $h$-tiebreaking, on 1104 standard IPC benchmark instances}. We highlight the
 best results when the difference between the maximum and the mininum coverage exceeds 2.
 }
 \label{tbl:lmcut-ipc-noh}
 }
\end{table}
\begin{table}[htbp]
 {
 \centering
 \setlength{\tabcolsep}{0.1em}
 \input{tables/7-ipc-mands-noh}
 \caption{
 Coverage comparison \textbf{with \mands, without $h$-tiebreaking, on 1104 standard IPC benchmark instances}. We highlight the
 best results when the difference between the maximum and the mininum coverage exceeds 2.
 }
 \label{tbl:mands-ipc-noh}
 }
\end{table}

\begin{table}[htbp]
 {
 \centering
 \setlength{\tabcolsep}{0.1em}
 \begin{center}
\begin{tabular}{|r|cccHHH|cccHHHHHHHHHHHHHHH|}
 & $[f,\fifo]$ & $[f,\lifo]$ & $[f,\ro]$ & R & R & R & $[f,\depth,\fifo]$ & $[f,\depth,\fifo]$ & $[f,\depth,\ro]$ & R & R & R & $[f,h,\fifo]$ & $[f,h,\fifo]$ & $[f,h,\ro]$ & R & R & R & $[f,h,\depth,\fifo]$ & $[f,h,\depth,\fifo]$ & $[f,h,\depth,\ro]$ & R & R & R\\
\hline
 & 227 & 296 & 238.3 $\pm$ 1.5 & 237 & 238 & 240 & 284 & 276 & 294.7 $\pm$ 3.1 & 298 & 292 & 294 & 271 & 294 & 276.7 $\pm$ 1.2 & 276 & 278 & 276 & 299 & 279 & 303.7 $\pm$ 1.5 & 305 & 304 & 302\\
\hline
airport-fuel(20) & 7 & 15 & 7 $\pm$ 0 & 7 & 7 & 7 & 10 & 13 & 10.3 $\pm$ 0.6 & 10 & 10 & 11 & 15 & 13 & 13.7 $\pm$ 0.6 & 13 & 14 & 14 & 14 & 13 & 14 $\pm$ 0 & 14 & 14 & 14\\
blocks-stack(20) & 15 & 17 & 15 $\pm$ 0 & 15 & 15 & 15 & 17 & 18 & 17.7 $\pm$ 0.6 & 18 & 18 & 17 & 17 & 17 & 17 $\pm$ 0 & 17 & 17 & 17 & 17 & 17 & 17 $\pm$ 0 & 17 & 17 & 17\\
depot-fuel(22) & 4 & 6 & 5.3 $\pm$ 0.6 & 5 & 5 & 6 & 6 & 6 & 6 $\pm$ 0 & 6 & 6 & 6 & 6 & 6 & 6 $\pm$ 0 & 6 & 6 & 6 & 6 & 6 & 6 $\pm$ 0 & 6 & 6 & 6\\
driverlog-fuel(20) & 7 & 8 & 7 $\pm$ 0 & 7 & 7 & 7 & 8 & 8 & 8 $\pm$ 0 & 8 & 8 & 8 & 8 & 8 & 8 $\pm$ 0 & 8 & 8 & 8 & 8 & 8 & 8 $\pm$ 0 & 8 & 8 & 8\\
elevators-up(20) & 7 & 13 & 7 $\pm$ 0 & 7 & 7 & 7 & 7 & 9 & 8.7 $\pm$ 0.6 & 9 & 9 & 8 & 7 & 13 & 7 $\pm$ 0 & 7 & 7 & 7 & 7 & 9 & 8.7 $\pm$ 0.6 & 9 & 9 & 8\\
floortile-ink(20) & 8 & 8 & 8 $\pm$ 0 & 8 & 8 & 8 & 8 & 8 & 8 $\pm$ 0 & 8 & 8 & 8 & 8 & 8 & 8.3 $\pm$ 0.6 & 9 & 8 & 8 & 8 & 8 & 8.3 $\pm$ 0.6 & 9 & 8 & 8\\
freecell-move(20) & 4 & 19 & 5 $\pm$ 0 & 5 & 5 & 5 & 17 & 10 & 16.7 $\pm$ 0.6 & 17 & 16 & 17 & 4 & 19 & 5 $\pm$ 0 & 5 & 5 & 5 & 17 & 10 & 16.3 $\pm$ 0.6 & 17 & 16 & 16\\
 & 15 & 15 & 15 $\pm$ 0 & 15 & 15 & 15 & 13 & 15 & 14.7 $\pm$ 0.6 & 15 & 15 & 14 & 15 & 15 & 15 $\pm$ 0 & 15 & 15 & 15 & 15 & 15 & 15 $\pm$ 0 & 15 & 15 & 15\\
grid-fuel(5) & 1 & 1 & 1 $\pm$ 0 & 1 & 1 & 1 & 1 & 1 & 1 $\pm$ 0 & 1 & 1 & 1 & 1 & 1 & 1 $\pm$ 0 & 1 & 1 & 1 & 1 & 1 & 1 $\pm$ 0 & 1 & 1 & 1\\
gripper-move(20) & 7 & 7 & 7 $\pm$ 0 & 7 & 7 & 7 & 7 & 7 & 7 $\pm$ 0 & 7 & 7 & 7 & 7 & 7 & 7 $\pm$ 0 & 7 & 7 & 7 & 7 & 7 & 7 $\pm$ 0 & 7 & 7 & 7\\
hiking-fuel(20) & 8 & 9 & 8 $\pm$ 0 & 8 & 8 & 8 & 9 & 9 & 9 $\pm$ 0 & 9 & 9 & 9 & 9 & 9 & 9 $\pm$ 0 & 9 & 9 & 9 & 9 & 9 & 9 $\pm$ 0 & 9 & 9 & 9\\
logistics00-fuel(28) & 15 & 16 & 15 $\pm$ 0 & 15 & 15 & 15 & 15 & 16 & 15 $\pm$ 0 & 15 & 15 & 15 & 16 & 16 & 16 $\pm$ 0 & 16 & 16 & 16 & 16 & 16 & 15.3 $\pm$ 0.6 & 16 & 15 & 15\\
miconic-up(30) & 10 & 17 & 10 $\pm$ 0 & 10 & 10 & 10 & 19 & 18 & 19.3 $\pm$ 1.2 & 20 & 20 & 18 & 16 & 17 & 16.7 $\pm$ 0.6 & 16 & 17 & 17 & 19 & 18 & 20.3 $\pm$ 0.6 & 20 & 21 & 20\\
mprime-succumb(35) & 12 & 14 & 11.3 $\pm$ 1.2 & 10 & 12 & 12 & 21 & 14 & 19.7 $\pm$ 1.2 & 19 & 19 & 21 & 15 & 14 & 16.7 $\pm$ 0.6 & 17 & 17 & 16 & 22 & 14 & 20.3 $\pm$ 0.6 & 20 & 20 & 21\\
mystery-feast(20) & 5 & 5 & 6.3 $\pm$ 1.2 & 7 & 5 & 7 & 6 & 7 & 6.3 $\pm$ 0.6 & 7 & 6 & 6 & 7 & 5 & 7.7 $\pm$ 0.6 & 7 & 8 & 8 & 6 & 5 & 7.3 $\pm$ 1.5 & 6 & 9 & 7\\
nomystery-fuel(20) & 9 & 10 & 9 $\pm$ 0 & 9 & 9 & 9 & 9 & 10 & 9.3 $\pm$ 0.6 & 10 & 9 & 9 & 10 & 10 & 10 $\pm$ 0 & 10 & 10 & 10 & 10 & 10 & 10 $\pm$ 0 & 10 & 10 & 10\\
parking-movecc(20) & 0 & 0 & 0 $\pm$ 0 & 0 & 0 & 0 & 0 & 0 & 0 $\pm$ 0 & 0 & 0 & 0 & 0 & 0 & 0 $\pm$ 0 & 0 & 0 & 0 & 0 & 0 & 0 $\pm$ 0 & 0 & 0 & 0\\
pathways-fuel(30) & 4 & 5 & 4 $\pm$ 0 & 4 & 4 & 4 & 4 & 5 & 4.3 $\pm$ 0.6 & 4 & 5 & 4 & 5 & 5 & 4.7 $\pm$ 0.6 & 5 & 5 & 4 & 5 & 5 & 4.3 $\pm$ 0.6 & 5 & 4 & 4\\
pipesnt-pushstart(20) & 6 & 7 & 8.3 $\pm$ 0.6 & 8 & 8 & 9 & 8 & 6 & 10 $\pm$ 0 & 10 & 10 & 10 & 8 & 8 & 8.3 $\pm$ 0.6 & 8 & 8 & 9 & 8 & 8 & 10 $\pm$ 0 & 10 & 10 & 10\\
pipesworld-pushend(20) & 2 & 4 & 2.7 $\pm$ 0.6 & 3 & 3 & 2 & 4 & 3 & 5.3 $\pm$ 0.6 & 6 & 5 & 5 & 3 & 4 & 3.7 $\pm$ 0.6 & 4 & 4 & 3 & 3 & 3 & 5 $\pm$ 0 & 5 & 5 & 5\\
psr-small-open(20) & 19 & 19 & 19 $\pm$ 0 & 19 & 19 & 19 & 19 & 19 & 19 $\pm$ 0 & 19 & 19 & 19 & 19 & 19 & 19 $\pm$ 0 & 19 & 19 & 19 & 19 & 19 & 19 $\pm$ 0 & 19 & 19 & 19\\
rovers-fuel(40) & 7 & 9 & 7 $\pm$ 0 & 7 & 7 & 7 & 8 & 9 & 9 $\pm$ 0 & 9 & 9 & 9 & 8 & 8 & 8 $\pm$ 0 & 8 & 8 & 8 & 8 & 8 & 8 $\pm$ 0 & 8 & 8 & 8\\
scanalyzer-analyze(20) & 3 & 9 & 3 $\pm$ 0 & 3 & 3 & 3 & 6 & 5 & 5 $\pm$ 0 & 5 & 5 & 5 & 9 & 9 & 9 $\pm$ 0 & 9 & 9 & 9 & 9 & 10 & 9.3 $\pm$ 0.6 & 10 & 9 & 9\\
sokoban-pushgoal(20) & 18 & 18 & 18 $\pm$ 0 & 18 & 18 & 18 & 18 & 18 & 17.7 $\pm$ 0.6 & 18 & 17 & 18 & 18 & 18 & 18 $\pm$ 0 & 18 & 18 & 18 & 18 & 18 & 18 $\pm$ 0 & 18 & 18 & 18\\
storage-lift(20) & 4 & 4 & 4 $\pm$ 0 & 4 & 4 & 4 & 5 & 5 & 5 $\pm$ 0 & 5 & 5 & 5 & 4 & 4 & 4 $\pm$ 0 & 4 & 4 & 4 & 5 & 4 & 4 $\pm$ 0 & 4 & 4 & 4\\
tidybot-motion(20) & 14 & 16 & 14.7 $\pm$ 0.6 & 14 & 15 & 15 & 15 & 15 & 16 $\pm$ 0 & 16 & 16 & 16 & 16 & 16 & 16 $\pm$ 0 & 16 & 16 & 16 & 16 & 16 & 16 $\pm$ 0 & 16 & 16 & 16\\
tpp-fuel(30) & 7 & 11 & 8 $\pm$ 0 & 8 & 8 & 8 & 10 & 10 & 11 $\pm$ 0 & 11 & 11 & 11 & 8 & 11 & 8 $\pm$ 0 & 8 & 8 & 8 & 11 & 10 & 11 $\pm$ 0 & 11 & 11 & 11\\
woodworking-cut(20) & 2 & 7 & 5.7 $\pm$ 0.6 & 6 & 6 & 5 & 7 & 5 & 8.7 $\pm$ 1.5 & 9 & 7 & 10 & 5 & 7 & 7 $\pm$ 0 & 7 & 7 & 7 & 8 & 5 & 8.3 $\pm$ 0.6 & 8 & 8 & 9\\
zenotravel-fuel(20) & 7 & 7 & 7 $\pm$ 0 & 7 & 7 & 7 & 7 & 7 & 7 $\pm$ 0 & 7 & 7 & 7 & 7 & 7 & 7 $\pm$ 0 & 7 & 7 & 7 & 7 & 7 & 7 $\pm$ 0 & 7 & 7 & 7\\
\end{tabular}
\end{center}

 \caption{
 Coverage comparison \textbf{with \lmcut, without $h$-tiebreaking, on 620 zerocost instances}. We highlight the
 best results when the difference between the maximum and the mininum coverage exceeds 2.
 }
 \label{tbl:lmcut-zero-noh}
 }
\end{table}
\begin{table}[htbp]
 {
 \centering
 \setlength{\tabcolsep}{0.1em}
 \input{tables/7-zero-mands-noh}
 \caption{
 Coverage comparison \textbf{with \mands, without $h$-tiebreaking, on 620 zerocost instances}. We highlight the
 best results when the difference between the maximum and the mininum coverage exceeds 2.
 }
 \label{tbl:mands-zero-noh}
 }
\end{table}



