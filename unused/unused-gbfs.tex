\section{Background: Tiebreaking Strategy for GBFS}

\label{sec:gbfs}

From this section, we discuss and evaluate the tiebreaking strategy of
GBFS for satisficing search.
Satisficing search algorithms sacrifice the solution quality for speed,
and GBFS is one extreme of such variants which completely sacrifices the
bounded quality guarantee like those in \astar.
GBFS is heavily used by
various planning solvers in the Satisficing Track of the International
Planning Competition in order to obtain the first solution of a
planning problem.




% \todo{unrelated -- interesting but too ambitious?}
% 
% Heuristic errors consist of two major components called \emph{precision}
% and \emph{accuracy}. Both terms are defined based on the distribution of
% the heuristic estimates compared to the true distance to the goal $h^*$,
% but has a key difference as follows: \emph{accuracy} accounts for the
% difference of means, while \emph{precision} accounts for the deviation
% from the true distance. The notion was proposed in the early
% literature for investigating the performance of probabilistic heuristic
% function \cite{pearl1984heuristics}, but is largely forgotten in the
% current search community. \citeauthor{pearl1984heuristics}
% concluded that, for satsificing search, the key characteristics which
% determines the performance of inadmissible heuristics is
% \emph{precision}, rather than \emph{accuracy}. Assume we have an
% inadmissible heuristic function which always overestimates the true
% distance $h^*$ by a constant error $e$ --- the guidance provided by
% function $h^*(s)+e$ is not much different from that of $h^*(s)$
% itself. This is because it changes the accuracy but does not change
% the precision of the heuristics.


Alternation OPEN List \cite{RogerH10} is a technique to combine multiple
heuristic functions during the search in order to improve the robustness
of the search algorithm. Nodes are simultaneously stored and sorted into the
multiple independent OPEN lists with different sorting strategies, and
it alternates among the OPEN lists when it expands a new node.
We denote an alternating OPEN list as $\mit{alt}(X_1,X_2,\ldots)$ where
each $X_i$ is a sorting strategy.

$\epsilon$-greedy GBFS \cite{valenzano2014comparison} is a variation of GBFS
which selects a random node in the OPEN list at a certain fixed
probability $\epsilon <1$ given as a parameter. When the random
exploration takes place, the entire nodes in OPEN list are treated equally, and
there is no explicit criteria for characterising the nodes.
This is conceptually equivalent to the weighted version of the alternation
open list using $[h]_{\fifo}$ and $[\ ]_{\ro}$ (no sorting criteria) with weights
$1-\epsilon$ and $\epsilon$.
%% this is not used in this paper
%  We denote such a weighted alternation open
% list as $\mit{alt}(w_1X_1,w_2X_2,\ldots)$ where each $w_i$ is a weight
% for a sorting strategy $X_i$. Now $\epsilon$-greedy GBFS corresponds to
% $\mit{alt}((1-\epsilon)[h,\fifo],\epsilon [\ro])$.

While the exploration phase of $\epsilon$-greedy GBFS simply randomizes
the node expansion, Type-GBFS \cite{xie14type} explicitly tries to
detect and avoid the bias in the OPEN list.  Consider, for example, if
there are millions of nodes with $h=2$ in the open list while there are
only handful of the nodes with $h=5$. Then the effect of
$\epsilon$-greedy diversification will be very limited because a
randomized selection almost always select a node with $h=2$, resulting
in the same behavior as the standard GBFS.

Type-GBFS tries to avoid this bias by categorizing the nodes into
several buckets (called ``type bucket''), each associated with a
specific vector of key values, such as $[g,h]$ for each state.  On each
explorative expansion, the search algorithm selects a random node in a
random bucket, which avoids the cardinality bias --- the concentration
of the nodes into a particular bucket is no longer a problem.

Type-GBFS categorizes the nodes in type buckets, but does not need to
sort the buckets in any particular order. Thus, we denote such a random
selection among buckets as $\brackets{\ldots}$ where $\ldots$ is a
classification criteria, e.g., $\brackets{g,h}$ denotes the type buckets
whose keys are the vector $[g,h]$. This notation is intentionally same
as those we have used for depth-based diversification.

In their paper, Type-GBFS was evaluated under a
configuration that the explorative expansion and the exploitative
expansion (standard GBFS) alternates. The last-resort tiebreaking 
is specified for the explorative expansion (random order), but is
not specified for the exploitative expansion.
Assuming that their implementations are based on the standard Fast
Downward code base (with FIFO last-resort tiebreaking),
we can write this algorithm as $\mit{alt}([h,\fifo], [\brackets{g,h},\ro])$.
% This also corresponds to the case of $\epsilon$-greedy with $\epsilon=0.5$.
% 
They empirically show that this configuration results in larger coverage
and that it expands much broader variety of nodes in the explorative
expansion phase compared to $\epsilon$-greedy GBFS.

\section{Depth-based Tiebreaking for GBFS}

%\citeauthor{Korf1985depth} uses $h$-based tiebreaking in the context of WA*
%\cite{korf1993linear} 
% 
% (g-based tiebreaking, p69, for GBFS:) With all the weight on h, meaning
% that g is used only for tie-breaking among nodes with equal h values,
% 
% (p73:) to break ties in favor of nodes closest to the initial state.  
% ** not sure how/where to put this..


Compared to \astar, to our knowledge, there are currently no
well-established tiebreaking policy for GBFS. GBFS does not use $f$ and
$g$ value during the search process.  The search by GBFS is solely
guided by the heuristic value $h$: It always expands the node with the
smallest $h$ value, and hence the analogy from \astar e.g.\ sorting the
nodes with $f$ and breaking ties according to $h$ is not possible.

As a consequence, except for diversification,
search enhancement for GBFS have been achieved by
modifying the heuristic function itself.  This not only includes the use
of multi-heuristics search, but also, in a much broader term, lazy evaluation,
preferred operators and PLUSONE/ONE cost type can be considered as a
modification to the heuristic function.
 
Lazy evaluation is a technique which derives the heuristic value of a
state from its parent. This sacrifices the informativeness of the
heuristics against the effort of computing each heuristic function.
 
Preferred operator (helpful action in \cite{Hoffmann01}) marks some
operators to be promising when it improved the least cost estimate in the
previous node expansion. Marked nodes are put in a special queue
which is expanded more often, which is equivalent to temporarily
increasing the priority of the particular nodes based on the past search
knowledge.
 
PLUSONE cost type is a technique which adds a cost of 1 to each edge
cost, which also affects the heuristic value. Similarly, ONE cost type
treats all actions to have a unit cost.  ONE cost type is used in
the first GBFS iteration of LAMA, and PLUSONE cost type is used in the
second GBFS iteration of LAMA.

We instead consider the use of the depth-based tiebreaking policy, which
is able to identify the relative location of the current state in the
plateau.  In this section, we evaluate the effect of depth-based
tiebreaking on Eager GBFS, and also compare the effect against various
search enhancements stated above.

\subsection{Comparison against GBFS}

We compared the performance of 
($g$) the standard GBFS
$[h]$,
($G$) GBFS using depth-based diversification
$[h,\depth]$,
($t$) Type-GBFS
$alt([h],\brackets{g,h})$,
($T$) Type-GBFS using depth-based diversification
$alt([h,\depth],\brackets{g,h})$,
($td$) Type-GBFS using depth-based type bucket
$alt([h],\brackets{g,h,d})$,
($Td$) Type-GBFS using both depth-based diversification and depth-based type bucket
$alt([h,\depth],\brackets{g,h,d})$.

Since the effect of depth could be heuristic-dependent, we tested three heuristics as $h$:
FF heuristics\cite{Hoffmann01}, Causal Graph (CG) heuristics \cite{Helmert2006}, Context Enhanced
Additive (CEA) heuristics\cite{helmert2008unifying}.
%% not included --- its is a pseudo heuristics anyways
% , Landmark-Count (LC) heuristics\cite{richter2008landmarks}

% In order to remove the effect of randomness of the algorithm, we
% implemented a deterministic version of Type-based queue for Type-GBFS
% which, instead of selecting a bucket at random, iterates over the
% buckets in a reverse order that each bucket is introduced.
% Similarly, the diversification based on the depth is not randomized but
% is implemented as a loop-based implementation.

Results are shown in \reftbl{eager-results}. Overall, depth offers
improvements to all of CEA, CG and FF heuristics. Thus, we conjecture
that the effect of depth-based diversification is heuristic-independent.

We also tested the effect of lazy (deferred) evaluation of the
heuristic functions. As explained in the previous section, lazy
evaluation can be considered as a form of modifying the heuristic
function by sacrificing the accuracy for the node expansion rate.
As expected, depth-based diversification offers the similar speedup as
in the case of eager evaluation.

Another dimension we should consider in evaluating the depth
diversification is the use of PLUSONE which treats all
action costs as the original value plus 1, or ONE cost type which treats
all actions as the unit cost.

% \newcommand{\htitle}[1]{\multicolumn{#1}{c|}{CEA} & \multicolumn{#1}{|c|}{CG} & \multicolumn{#1}{|c|}{FF}}
\newcommand{\etitle}[2]{ \multicolumn{#1}{|c||}{Eager} & \multicolumn{#1}{|c|#2}{Lazy}}
\newcommand{\htitle}[2]{\multicolumn{#1}{c|}{CEA} & \multicolumn{#1}{|c|}{CG} & \multicolumn{#1}{|c|#2}{FF}}
\newcommand{\titles}[3]{
&\etitle{#1}{#3} \\
\hline
&\htitle{#2}{|} #3 &\htitle{#2}{}\\
\hline
}

\begin{table}[htbp]
 \setlength{\tabcolsep}{0.1em}
 % \setlength{\tabcolsep}{0.05em}
 % \relsize{-1}
\centering
\begin{tabular}{|l*{2}{*{3}{|cc}|}*{2}{*{3}{|cc}|}}
\hline
&\etitle{6}{|} & \etitle{6}{}\\
\hline
&\htitle{2}{|} & \htitle{2}{|} &\htitle{2}{|} & \htitle{2}{}\\
\hline
 & g & G & g & G & g & G & g & G & g & G & g & G & gt & Gt & gt & Gt & gt & Gt & gt & Gt & gt & Gt & gt & Gt\\
\hline
coverage & 189 & \bi{193} & 174 & \bi{183} & 166 & 167 & 139 & \bi{172} & 137 & \bi{144} & 146 & \bi{162} & \bi{203} & 200 & 203 & \bi{210} & \bi{217} & 215 & 157 & \bi{173} & 165 & \bi{179} & 178 & \bi{212}\\
\hline
brmn & 0 & 0 & 0 & 0 & 0 & 0 & 0 & 0 & 0 & 0 & 1 & 0 & 0 & 0 & 0 & 0 & 0 & 1 & 0 & 0 & 0 & 0 & 0 & 0\\
lvtr & 1 & \bi{3} & 3 & \bi{5} & 0 & 0 & 1 & \bi{4} & 1 & \bi{4} & 0 & 0 & 1 & \bi{3} & 3 & 3 & 0 & 0 & 1 & \bi{3} & 1 & \bi{4} & 0 & 0\\
flrt & 4 & 4 & 0 & 0 & 7 & 7 & 6 & 6 & 0 & 0 & \bi{8} & 4 & 7 & 7 & 2 & 2 & 9 & 9 & 6 & 7 & 2 & 2 & 8 & 9\\
nmys & 7 & 7 & 7 & 7 & 6 & \bi{8} & 8 & 8 & 9 & 9 & 5 & \bi{7} & 9 & 9 & \bi{16} & 14 & 16 & 15 & 11 & 11 & 14 & 15 & 15 & 16\\
pnst & 0 & 0 & 0 & 0 & 0 & 0 & 0 & 0 & 0 & 0 & 0 & 0 & 0 & 0 & 0 & 0 & 0 & 0 & 0 & 0 & 0 & 0 & 0 & 0\\
prcp & 15 & 15 & 15 & 15 & 5 & 5 & 12 & 12 & 11 & 11 & 0 & 0 & 15 & 15 & 15 & 15 & 7 & 7 & 11 & 11 & 10 & 10 & 6 & 6\\
prkn & 0 & 1 & 12 & 13 & 15 & 14 & 2 & \bi{4} & 3 & \bi{11} & 14 & \bi{19} & 3 & 2 & 7 & \bi{9} & \bi{12} & 8 & 1 & 0 & 3 & 4 & 15 & 16\\
pgsl & 19 & 19 & 20 & 20 & 19 & 19 & 19 & 19 & 20 & 19 & 20 & 20 & 20 & 20 & 20 & 20 & 20 & 20 & 20 & 19 & 20 & 20 & 20 & 20\\
scnl & 17 & 17 & 17 & 17 & 14 & 14 & 17 & 17 & 17 & 17 & 15 & 14 & 17 & 17 & 17 & 17 & 16 & 16 & 17 & 17 & 17 & 17 & 17 & 17\\
skbn & 12 & 12 & 14 & 14 & 18 & 17 & 10 & 11 & 15 & 15 & 18 & 17 & 15 & 16 & 16 & 17 & 18 & 18 & 16 & 16 & 13 & \bi{16} & 18 & 18\\
tdyb & 16 & 17 & 14 & \bi{18} & 15 & 14 & 16 & 17 & 14 & \bi{17} & 14 & 13 & 16 & 16 & 19 & 19 & 16 & 16 & 16 & 17 & 20 & 20 & 14 & \bi{16}\\
trns & 7 & 7 & 9 & \bi{11} & 0 & 0 & 3 & \bi{5} & 7 & 8 & 0 & 0 & 7 & 7 & 11 & 11 & 0 & 0 & 4 & 5 & 9 & 8 & 0 & 0\\
vstl & 3 & 3 & 3 & 3 & 5 & 5 & 3 & 3 & 3 & 3 & 5 & 4 & 4 & 4 & 4 & 5 & 6 & 5 & 4 & 4 & 5 & 5 & 4 & \bi{6}\\
wdwr & 10 & 10 & 10 & 10 & 12 & 12 & 2 & 3 & 1 & 1 & 10 & 11 & 12 & 11 & 12 & 13 & 15 & 15 & 3 & 4 & 1 & 1 & 7 & \bi{13}\\
\hline
brmn & 0 & 0 & 0 & 0 & 0 & 0 & 0 & 0 & 0 & 0 & 0 & 0 & 0 & 0 & 0 & 0 & 3 & 2 & 0 & 0 & 0 & 0 & 0 & 0\\
cvdv & 6 & 6 & 7 & 7 & 6 & 6 & 4 & \bi{7} & 7 & 7 & 6 & 6 & 6 & 7 & 7 & 7 & 7 & 7 & 7 & 7 & 8 & 8 & 7 & 7\\
chld & 2 & 2 & 2 & 2 & 0 & 0 & 0 & 0 & 0 & 0 & 0 & 0 & \bi{2} & 0 & \bi{3} & 0 & 0 & 0 & 2 & 1 & 6 & 5 & 2 & 2\\
ctyc & 20 & 20 & 0 & \bi{2} & 0 & 0 & 1 & \bi{20} & 0 & 0 & 0 & 0 & 20 & 20 & 1 & 1 & 18 & \bi{20} & 1 & \bi{15} & 0 & 0 & 1 & \bi{10}\\
flrt & 2 & 2 & 0 & 0 & 2 & 2 & 2 & 2 & 0 & 0 & 3 & 3 & \bi{5} & 3 & 2 & 1 & \bi{5} & 3 & 3 & 2 & 2 & 2 & 3 & 3\\
gd-s & 0 & 0 & 0 & 0 & 0 & 0 & 0 & 0 & 0 & 0 & 0 & 0 & 0 & 0 & 0 & 0 & 0 & 0 & 0 & 0 & 0 & 0 & 0 & 0\\
hkng & 17 & 16 & 15 & 14 & 16 & \bi{18} & 17 & 17 & \bi{16} & 13 & 12 & \bi{14} & 16 & 16 & 20 & 20 & 20 & 20 & 16 & 15 & 18 & 19 & 18 & 18\\
mntn & 16 & 16 & 16 & 16 & 11 & 12 & 0 & 0 & 0 & 0 & 4 & \bi{7} & 16 & 16 & 16 & 16 & 13 & 14 & 7 & 7 & 7 & 7 & 9 & 10\\
pnst & 0 & 0 & 0 & 0 & 0 & 0 & 0 & 0 & 0 & 0 & 0 & 0 & 0 & 0 & 0 & 0 & 0 & 0 & 0 & 0 & 0 & 0 & 0 & 0\\
prkn & 0 & 1 & 1 & 1 & 1 & 1 & 0 & 1 & 0 & 1 & 3 & \bi{11} & 0 & 0 & 0 & 0 & 2 & 2 & 0 & 0 & 0 & 0 & 1 & \bi{8}\\
ttrs & 3 & 4 & 2 & 2 & 3 & 3 & 1 & \bi{4} & 2 & 2 & 1 & \bi{3} & 2 & 2 & 6 & \bi{12} & 2 & \bi{4} & 1 & 2 & 2 & \bi{9} & 3 & \bi{6}\\
thgh & 11 & 11 & 4 & 5 & 11 & 10 & 12 & 12 & \bi{6} & 4 & 7 & \bi{9} & 9 & 8 & 5 & 5 & 12 & 13 & 9 & 9 & 5 & 5 & 10 & 11\\
trns & 1 & 0 & \bi{3} & 1 & 0 & 0 & \bi{3} & 0 & \bi{5} & 2 & 0 & 0 & 1 & 1 & 1 & \bi{3} & 0 & 0 & 1 & 1 & 2 & 2 & 0 & 0\\
vstl & 0 & 0 & 0 & 0 & 0 & 0 & 0 & 0 & 0 & 0 & 0 & 0 & 0 & 0 & 0 & 0 & 0 & 0 & 0 & 0 & 0 & 0 & 0 & 0\\
\hline
\end{tabular}
\caption{Results of 5 min, 4Gb experiments, comparing the variants with and without depths:}%
 % (g): GBFS $[h]$,
 % (G): GBFS with depth based diversification $[h]_\{d\}$.
 % Left side shows the result of eager evaluation, and the right side shows
 % the result of lazy evaluation.
 % ,
 % (gt): $alt([h],\brackets{g,h})$,
 % (gT): $alt([h],\brackets{g,h,d})$,
 % (Gt): $alt([h]_{\depth},\brackets{g,h})$,
 % (GT): $alt([h]_{\depth},\brackets{g,h,d})$
 % }
\end{table}

\begin{table}[htbp]
\centering
\setlength{\tabcolsep}{0.1em}
\begin{tabular}{lll|ccc|ccc}
 &  &  & Eager & Eager & Eager & Lazy & Lazy & Lazy\\
 &  &  & IPC7 & IPC8 & sum & IPC7 & IPC8 & sum\\
\hline
ce & g & F & 112 & 70 & 182 & 116 & 54 & 170\\
ce & G & F & 113 & 70 & 183 & 119 & 48 & 167\\
ce & gt & F & 120 & 67 & 187 & 125 & 67 & 192\\
ce & gT & F & 120 & 67 & 187 & 125 & 66 & 191\\
ce & Gt & F & 120 & \bi{77} & 197 & 128 & 62 & 190\\
ce & GT & F & 121 & \bi{77} & 198 & 129 & 62 & 191\\
cg & g & F & 126 & 54 & 180 & 110 & 37 & 147\\
cg & G & F & 132 & 55 & 187 & 118 & 36 & 154\\
cg & gt & F & 139 & 63 & 202 & 122 & 49 & 171\\
cg & gT & F & 138 & 63 & 201 & 121 & 49 & 170\\
cg & Gt & F & \bi{142} & 63 & \bi{205} & 129 & 56 & 185\\
cg & GT & F & \bi{142} & 63 & \bi{205} & 128 & 56 & 184\\
ff & g & F & 110 & 49 & 159 & 110 & 48 & 158\\
ff & G & F & 110 & 54 & 164 & 117 & 45 & 162\\
ff & gt & F & 130 & 71 & 201 & \bi{137} & \bi{75} & \bi{212}\\
ff & gT & F & 129 & 71 & 200 & 136 & \bi{75} & 211\\
ff & Gt & F & 134 & 68 & 202 & 132 & 59 & 191\\
ff & GT & F & 135 & 70 & \bi{205} & 133 & 59 & 192\\
\hline
ce & g & L & 111 & 78 & 189 & 99 & 40 & 139\\
ce & G & L & 115 & 78 & 193 & 109 & 63 & 172\\
ce & gt & L & 126 & 77 & 203 & 110 & 47 & 157\\
ce & gT & L & 127 & 76 & 203 & 114 & 45 & 159\\
ce & Gt & L & 127 & 73 & 200 & 114 & 59 & 173\\
ce & GT & L & 125 & 75 & 200 & 116 & 54 & 170\\
cg & g & L & 124 & 50 & 174 & 101 & 36 & 137\\
cg & G & L & 133 & 50 & 183 & 115 & 29 & 144\\
cg & gt & L & 142 & 61 & 203 & 115 & 50 & 165\\
cg & gT & L & 141 & 57 & 198 & 119 & 41 & 160\\
cg & Gt & L & \bi{145} & 65 & 210 & 122 & 57 & 179\\
cg & GT & L & \bi{145} & 59 & 204 & 122 & 48 & 170\\
ff & g & L & 116 & 50 & 166 & 110 & 36 & 146\\
ff & G & L & 115 & 52 & 167 & 109 & 53 & 162\\
ff & gt & L & 135 & 82 & \bi{217} & 124 & 54 & 178\\
ff & gT & L & 127 & 79 & 206 & 130 & 46 & 176\\
ff & Gt & L & 130 & \bi{85} & 215 & 137 & \bi{75} & \bi{212}\\
ff & GT & L & 133 & 74 & 207 & \bi{140} & 70 & 210\\
\end{tabular}
\end{table}

\begin{table}[htbp]
 \setlength{\tabcolsep}{0.1em}
 % \setlength{\tabcolsep}{0.05em}
 % \relsize{-1}
\centering
\begin{tabular}{|l*{2}{*{3}{|cc}|}*{2}{*{3}{|cc}|}}
\hline
&\etitle{6}{|} & \etitle{6}{}\\
\hline
&\htitle{2}{|} & \htitle{2}{|} &\htitle{2}{|} & \htitle{2}{}\\
\hline
 % & e & e & e & e & e & e & l & l & l & l & l & l & e & e & e & e & e & e & l & l & l & l & l & l\\
 % & ce1 & ce1 & cg1 & cg1 & ff1 & ff1 & ce1 & ce1 & cg1 & cg1 & ff1 & ff1 & ce1 & ce1 & cg1 & cg1 & ff1 & ff1 & ce1 & ce1 & cg1 & cg1 & ff1 & ff1\\
 & g & G & g & G & g & G & g & G & g & G & g & G & gt & Gt & gt & Gt & gt & Gt & gt & Gt & gt & Gt & gt & Gt\\
\hline
IPC7 & 118 & 118 & 140 & \bi{143} & \bi{142} & 139 & 99 & \bi{117} & 113 & \bi{134} & 112 & \bi{143} & 130 & 130 & 153 & 154 & \bi{166} & 157 & 106 & \bi{121} & 124 & \bi{146} & 135 & \bi{156}\\
\hline
brmn & 0 & 0 & 0 & 0 & 0 & 0 & 0 & 0 & 0 & 0 & 1 & 0 & 0 & 0 & 0 & 0 & 10 & 10 & 0 & 0 & 0 & 0 & 2 & 2\\
lvtr & 8 & 8 & 8 & 9 & 10 & 10 & 10 & 10 & 9 & 8 & 8 & \bi{12} & 8 & 8 & 8 & 9 & 10 & 10 & 5 & \bi{8} & 7 & \bi{9} & 4 & \bi{11}\\
flrt & 3 & 3 & 0 & 0 & 3 & 4 & 3 & 3 & 0 & 0 & 3 & 3 & 5 & 5 & 2 & 2 & 8 & 7 & 5 & 5 & 2 & 2 & 6 & 6\\
nmys & 7 & 7 & 7 & 7 & 6 & \bi{8} & 8 & 8 & 9 & 9 & 4 & \bi{7} & 9 & 9 & \bi{16} & 14 & 16 & 15 & 11 & 11 & 14 & 15 & 15 & 16\\
pnst & 13 & 12 & 13 & 13 & 15 & 15 & 0 & \bi{15} & 0 & \bi{14} & 0 & \bi{20} & 11 & 11 & 11 & 11 & 13 & 13 & 0 & \bi{11} & 1 & \bi{12} & 4 & \bi{9}\\
prcp & 20 & 20 & 20 & 19 & 20 & 20 & 11 & 11 & 11 & 11 & 11 & 11 & 18 & 18 & 18 & 19 & 20 & 20 & 12 & 13 & 11 & 12 & 14 & 14\\
prkn & 0 & \bi{2} & 12 & 13 & \bi{15} & 11 & 2 & \bi{4} & 3 & \bi{11} & 14 & \bi{19} & 3 & 2 & 7 & \bi{9} & \bi{12} & 8 & 1 & 0 & 3 & 4 & 15 & 16\\
pgsl & 20 & 20 & 20 & 20 & 19 & 20 & 20 & 20 & 20 & 20 & 20 & 20 & 20 & 20 & 20 & 20 & 20 & 20 & 20 & 20 & 20 & 20 & 20 & 20\\
scnl & 17 & 17 & 17 & 17 & 14 & 14 & 17 & 17 & 17 & 17 & 12 & 13 & 16 & 16 & 17 & 17 & 15 & 15 & 17 & 17 & 17 & 17 & 17 & 17\\
skbn & 3 & 3 & 13 & 12 & 18 & 17 & 3 & 3 & 15 & 14 & 18 & 17 & 10 & 10 & 17 & 17 & 18 & 17 & 10 & 9 & 15 & \bi{17} & 18 & 18\\
tdyb & 16 & 17 & 14 & \bi{18} & \bi{15} & 13 & 16 & 17 & 14 & \bi{17} & 14 & 13 & 16 & 16 & 19 & 19 & 16 & 15 & 16 & 17 & 20 & 20 & 14 & \bi{16}\\
trns & \bi{7} & 5 & 12 & 11 & 0 & 0 & 4 & 5 & \bi{11} & 9 & 0 & 0 & 6 & \bi{8} & 13 & 12 & 0 & 0 & 3 & \bi{5} & 8 & \bi{12} & 0 & 0\\
vstl & 3 & 3 & 3 & 3 & 5 & 5 & 3 & 3 & 3 & 3 & 5 & 4 & 4 & 4 & 4 & 4 & 6 & 5 & 4 & 4 & 5 & 5 & 4 & \bi{6}\\
wdwr & 1 & 1 & 1 & 1 & 2 & 2 & 2 & 1 & 1 & 1 & 2 & \bi{4} & 4 & 3 & 1 & 1 & 2 & 2 & 2 & 1 & 1 & 1 & 2 & \bi{5}\\
\hline
IPC8 & 62 & 62 & 60 & \bi{64} & 73 & 73 & 36 & \bi{50} & 39 & \bi{48} & 54 & \bi{93} & \bi{57} & 55 & 76 & \bi{79} & 80 & 79 & 44 & \bi{53} & 66 & \bi{74} & 66 & \bi{92}\\
\hline
brmn & 0 & 0 & 0 & 0 & 0 & 0 & 0 & 0 & 0 & 0 & 0 & 0 & 0 & 0 & 0 & 0 & \bi{4} & 2 & 0 & 0 & 0 & 0 & 0 & 0\\
cvdv & 6 & 6 & 7 & 7 & 7 & 7 & 4 & \bi{7} & 7 & 7 & 6 & 7 & 6 & 6 & 7 & 7 & 7 & 7 & 6 & 7 & 7 & 8 & 7 & 7\\
chld & 2 & 2 & 2 & 2 & 0 & 0 & 0 & 0 & 0 & 0 & 0 & 0 & \bi{2} & 0 & \bi{3} & 0 & 0 & 0 & 2 & 1 & 6 & 5 & 2 & 2\\
ctyc & 2 & 2 & 0 & 1 & 0 & 0 & 0 & \bi{2} & 0 & 0 & 0 & 0 & 4 & 5 & 1 & 1 & 3 & 3 & 1 & \bi{9} & 0 & 0 & 0 & \bi{7}\\
flrt & 2 & 2 & 0 & 0 & 2 & 2 & 2 & 2 & 0 & 0 & 2 & 2 & 2 & 2 & 2 & 2 & 2 & 2 & 2 & 2 & 1 & 1 & 2 & 2\\
gd-s & 0 & 0 & 0 & 0 & \bi{19} & 17 & 0 & 0 & 0 & 0 & 20 & 20 & 0 & 0 & 8 & 9 & 16 & 15 & 0 & 0 & 11 & 11 & 15 & \bi{17}\\
hkng & 17 & 16 & 15 & 14 & 16 & \bi{18} & 17 & 17 & \bi{16} & 13 & 12 & \bi{14} & 16 & 16 & 20 & 20 & 20 & 20 & 16 & 15 & 18 & 19 & 18 & 18\\
mntn & 16 & 16 & 16 & 16 & 11 & 12 & 0 & 0 & 0 & 0 & 4 & \bi{7} & 16 & 16 & 16 & 16 & 13 & 13 & 7 & 7 & 7 & 7 & 9 & 10\\
pnst & 0 & 0 & 0 & 0 & 4 & 4 & 0 & \bi{3} & 0 & \bi{3} & 0 & \bi{20} & 0 & 0 & 0 & 0 & 0 & 0 & 0 & 0 & 0 & 1 & 0 & \bi{6}\\
prkn & 0 & 2 & 1 & 1 & 2 & 1 & 0 & 1 & 0 & 1 & 3 & \bi{11} & 0 & 0 & 0 & 0 & 1 & 2 & 0 & 0 & 0 & 0 & 1 & \bi{8}\\
ttrs & 4 & 3 & 12 & \bi{15} & 1 & 2 & 1 & \bi{5} & 7 & \bi{17} & 0 & \bi{3} & 1 & 1 & 13 & \bi{15} & 2 & 2 & 1 & 3 & 9 & \bi{15} & 2 & \bi{4}\\
thgh & 12 & 11 & 4 & 5 & 11 & 10 & 12 & 12 & \bi{6} & 4 & 7 & \bi{9} & 9 & 8 & 5 & 5 & 12 & 13 & 9 & 9 & 5 & 5 & 10 & 11\\
trns & 1 & 2 & 3 & 3 & 0 & 0 & 0 & 1 & 3 & 3 & 0 & 0 & 1 & 1 & 1 & \bi{4} & 0 & 0 & 0 & 0 & 2 & 2 & 0 & 0\\
vstl & 0 & 0 & 0 & 0 & 0 & 0 & 0 & 0 & 0 & 0 & 0 & 0 & 0 & 0 & 0 & 0 & 0 & 0 & 0 & 0 & 0 & 0 & 0 & 0\\
\hline
\end{tabular}
\caption{Results obtained by treating each action to have a unit cost
 during heuristic calculation.
 5 min, 4Gb experiments, comparing the variants with
 and without depths:
 % (g): GBFS $[h]$,
 % (G): GBFS with depth based diversification $[h]_\depth$.
 Left side shows the result of eager evaluation, and the right side shows
 the result of lazy evaluation.
 % ,
 % (gt): $alt([h],\brackets{g,h})$,
 % (gT): $alt([h],\brackets{g,h,d})$,
 % (Gt): $alt([h]_{\depth},\brackets{g,h})$,
 % (GT): $alt([h]_{\depth},\brackets{g,h,d})$
 }
\end{table}


% \subsection{Comparison against Lazy Evaluation}

% Although the policy was originally designed for optimising
% \astar search with admissible heuristics, the core idea of depth would
% not be much affected by the difference between \astar and GBFS.

\subsection{Comparison against LAMA}

LAMA \cite{richter2010lama} is a \sota planner which is a winner of
IPC2011. It has numerous search enhancements including Lazy Evaluation,
Multi-heuristic search, Preferred Operators and PLUSONE cost type.

LAMA runs a GBFS followed by WA* with decreasing weights, and works in
an anytime fasion. However, this is due to the context of the
competition setting where the task also includes finding a better plan.
Since this paper focuses on GBFS, we only consider 
the coverage within the resource constraint. Thus, in the following
experiments, we do not run the successive iterations of WA* in LAMA.

The first iteration (GBFS) of LAMA is expressed as
$alt([\mit{ff}],\pref{[\mit{ff}]},[\mit{lc}],\pref{[\mit{lc}]})$, where $\mit{ff}$
denotes the FF heuristics, $\mit{lc}$ denotes the Landmark-Count heuristics,
and $\pref{X}$ denotes the preferred operator queue with sorting
strategy $X$.

We compare this base LAMA with the type-based version of LAMA and their
variants using depths.
We followed the best configuration suggested by \citeauthor{xie14type} (\citeyear{xie14type})
for Type-GBFS, which uses the type vector $\brackets{g,\mit{ff}}$.
Due to the multiple enhancements, the configurations are fairly complicated:

\begin{eqnarray}
  alt([\mit{ff}],\pref{[\mit{ff}]},[\mit{lc}],\pref{[\mit{lc}]},\brackets{g,\mit{ff}}) \\
  alt([\mit{ff}]_\depth,\pref{[\mit{ff}]_\depth},[\mit{lc}]_\depth,\pref{[\mit{lc}]_\depth},\brackets{g,\mit{ff}}) \\
  alt([\mit{ff}],\pref{[\mit{ff}]},[\mit{lc}],\pref{[\mit{lc}]},\brackets{g,\mit{ff},d}) \\
  alt([\mit{ff}]_\depth,\pref{[\mit{ff}]_\depth},[\mit{lc}]_\depth,\pref{[\mit{lc}]_\depth},\brackets{g,\mit{ff},d}) 
\end{eqnarray}

% _\depth

\begin{table}[htbp]
 \setlength{\tabcolsep}{0.2em}
\centering
\begin{tabular}{l|cccc||cccc}
 & F & F & F & F & L & L & L & L\\
 & g & G & gt & Gt & g & G & gt & Gt\\
\hline
IPC7 & \bi{223} & 220 & 224 & 224 & 210 & \bi{214} & \bi{224} & 219\\
\hline
brmn & 15 & 16 & 16 & 16 & 15 & 14 & \bi{16} & 14\\
lvtr & 14 & \bi{17} & 16 & 17 & 18 & 18 & 16 & \bi{18}\\
flrt & 3 & 2 & 4 & 4 & 3 & 3 & 4 & 4\\
nmys & 11 & 10 & 12 & 12 & 6 & \bi{9} & \bi{12} & 8\\
pnst & 20 & 20 & 20 & 20 & 20 & 20 & 20 & 20\\
prcp & \bi{20} & 18 & 16 & 17 & 14 & 14 & 16 & \bi{18}\\
prkn & 20 & 20 & 20 & 20 & 20 & 20 & 20 & 20\\
pgsl & 20 & 19 & 20 & 20 & 20 & 20 & 20 & 20\\
scnl & 17 & 17 & 17 & 17 & 17 & 17 & 17 & 17\\
skbn & 16 & 15 & 16 & 16 & 15 & 15 & 16 & 17\\
tdyb & 16 & 15 & \bi{17} & 15 & 13 & 14 & \bi{17} & 13\\
trns & 11 & 11 & 10 & 10 & 10 & 10 & 10 & 10\\
vstl & 20 & 20 & 20 & 20 & 20 & 20 & 20 & 20\\
wdwr & 20 & 20 & 20 & 20 & 19 & 20 & 20 & 20\\
\hline
IPC8 & 111 & \bi{125} & 112 & \bi{117} & 111 & \bi{121} & 114 & \bi{120}\\
\hline
brmn & 8 & \bi{10} & 9 & 8 & \bi{9} & 6 & 9 & 8\\
cvdv & 7 & 7 & 6 & 7 & 6 & 7 & 6 & 6\\
chld & 0 & \bi{10} & 2 & \bi{6} & 3 & 3 & 2 & 3\\
ctyc & 1 & 0 & 5 & 4 & 2 & \bi{4} & 5 & 5\\
flrt & 2 & 2 & 2 & 2 & 2 & 2 & 2 & 2\\
gd-s & 20 & 20 & 20 & 20 & 20 & 20 & 20 & 20\\
hkng & 15 & \bi{17} & 15 & 15 & 14 & 15 & 15 & 16\\
mntn & 1 & 1 & 6 & 6 & 1 & 1 & 6 & 7\\
pnst & 17 & 17 & 15 & \bi{17} & 16 & \bi{18} & 16 & 15\\
prkn & 9 & 9 & 6 & 6 & 9 & 10 & 7 & 8\\
ttrs & 2 & 2 & 2 & 1 & 2 & \bi{4} & 2 & 2\\
thgh & 14 & 15 & 14 & 15 & 13 & \bi{15} & 14 & \bi{17}\\
trns & 2 & 2 & 1 & \bi{3} & 2 & 3 & 1 & 1\\
vstl & 13 & 13 & \bi{9} & 7 & 12 & 13 & 9 & 10\\
\hline
\end{tabular}
\caption{Lama results.}
\end{table}



\section{Related Work}

\cite{schulte2014balancing}

\cite{auer2002finite}

%% where to put this paragraph?
% KBFS(k) \cite{Korf??} is an early attempt on alleviating the problem of
% GBFS. It expands $k$ nodes at a time in order to prevent the search
% algorithm from getting stuck in the heuristic trap. KBFS provided an
% interesting observation to GBFS --- KBFS(1) is equivalent to GBFS, and
% KBFS($\infty$) is equivalent to Breadth-First Search.

% DBFS

\section{Conclusion}

