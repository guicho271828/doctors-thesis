
We first define some notation and terminology used throughout the rest of the paper.
A \emph{tiebreaking strategy} selects from among nodes with the same $f$-value.
Tiebreaking strategies are denoted as $[\text{criterion}_1, \text{criterion}_2, ..., \text{criterion}_k]$,
which means: If there are multiple nodes with the same $f$-value, first, break ties using $\text{criterion}_1$. 
If there are still multiple nodes remaining, then break ties using $\text{criterion}_2$ and so on, until a single node is selected.
The \emph{first-level tiebreaking policy} of a strategy is
$\text{criterion}_1$, the \emph{second-level tiebreaking policy} is
$\text{criterion}_2$, and so on.
%% the word frontier is no longer used in the later text.
% \emph{final frontier} is the set of open nodes with $f^*$.
