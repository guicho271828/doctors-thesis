\subsection{Theoretical Characteristics of the Depth Distribution}

We give further insight into how the depth is distributed among the
plateau region.

\begin{theo}[Uniformness of Iteration-based Implementation]
 In an implementation of depth-based diversification based on iteration
 from the largest depth to 0, the chance of expanding each node is
 unaffected by the depth of the node, after enough number of iterations.
\end{theo}

We first show that the number of expansion of each depth decreases
linearly.  Assume that each depth buckets never exhausts as a result of
expansion.  If the expansion of a node in the largest depth $D\geq 0$ resulted
in more nodes in the same plateau, then the newly generated nodes have
depth $D+1$.  Also, as we explained in the previous section, the
expansion is diversified by a sequence of iterations from the current
largest depth to 0.  It indicates that when the current maximum depth of
the plateau is $D\geq 0$, the number of iteration happened so far is also $D$.
Therefore, at the end of the $D$'th iteration, each depth $d$ has
 been expanded exactly $D-d$ times, meaning that the chance of
expanding each depth is decreasing as the depth increase.

However, it does not mean that this strategy favors the nodes in the
shallower region.
% This is because the number of nodes in each depth is
% exponential to $d$, which is common in practice (we verify this in the
% later experiments).
 Assume that the plateau region form a tree of width
$w\geq 2$, rather than a graph with indefinite number of successor
nodes.  Under this assumption, each expansion of depth $d$ results in
$w$ new nodes in depth $d+1$. In this assumption, if there are at least
$D$ nodes in depth 0, each depth bucket never exhausts until the end of $D$'th
iteration (when depth 0 becomes empty).

% $D-d$th expansion in the $D$th iteration expands the depth $d$.
At the end of $D$th iteration,
each depth $d$ is expanded $D-d$ times in the preceeding $D$ iterations.
Therefore, the total number of nodes that have been in depth $d+1$, including those
that have been expanded, is $w(D-d)$.
Since depth $d+1$ is expanded $D-(d+1)$ times so far,
each node that has ever been in this depth is given the chance of expansion
$w(D-d)/(D-(d+1))=1-\frac{w}{D-d-1}$, whose second term is
negligeble as $D$ increases.
Therefore, after enough number of iterations, the
chance of expanding each node converges to 1, and is unaffected by the depth of the node.

%% it does not actually mean it converges to 1. if d=D, 1-\frac{w}{D-d-1}=1-\frac{w}{-1}=1+w


\begin{theo}[Ignoring Depth causes Biased Expansion]
 An expansion order which expands all nodes in the plateau, then 
 If we use a simple randomized last-resort strategy
 without depth diversification (e.g.\ $[f,h,\ro]$), the chance of expanding each
 node in depth $d$ decreases exponentially to $d$.
\end{theo}

Assume the same setting as in the previous theorem: Initially we have
$D$ nodes in depth 0 and the plateau has the fixed tree width $w\geq 2$.
Let $n_k(d)$ denote the number of nodes in depth $d$ after $k$th
expansion in the plateau, and
$m_k(d)$ be the number of nodes that have been in a depth $d$, including
those expanded and removed from the bucket.
Also, let $M_k$ and $N_k$ denote the total number of nodes in the platau
including and not including those expanded so far after $k$th
expansion in the plateau, i.e. $M_k=\sum_{d=0}^{D}m_k(d)=D+kw$, $N_k=\sum_{d=0}^{D}n_k(d)=D+k(w-1)$.
Since the node is
selected uniformly random at each expansion, each depth $d$ has a
probability $p_k(d) = n_k(d)/N_k$ of being expanded.


\begin{theo}[Randomization without Depth is NOT Uniform]
 If we use a simple randomized last-resort strategy
 without depth diversification (e.g.\ $[f,h,\ro]$), the chance of expanding each
 node in depth $d$ decreases exponentially to $d$.
\end{theo}

Assume the same setting as in the previous theorem: Initially we have
$D$ nodes in depth 0 and the plateau has the fixed tree width $w\geq 2$.
Let $n_k(d)$ denote the number of nodes in depth $d$ after $k$th
expansion in the plateau, and
$m_k(d)$ be the number of nodes that have been in a depth $d$, including
those expanded and removed from the bucket.
Also, let $M_k$ and $N_k$ denote the total number of nodes in the platau
including and not including those expanded so far after $k$th
expansion in the plateau, i.e. $M_k=\sum_{d=0}^{D}m_k(d)=D+kw$, $N_k=\sum_{d=0}^{D}n_k(d)=D+k(w-1)$.
Since the node is
selected uniformly random at each expansion, each depth $d$ has a
probability $p_k(d) = n_k(d)/N_k$ of being expanded.



The expected value of $n_k(0)$ is therefore represented as

\[
\begin{array}{rll}
 m_k(0) &= D\\
 n_0(0) &= D\\
 n_k(0) &= n_{k-1} -p_{k-1}(0)             \\
        &= n_{k-1}(0) (1-1/N_{k-1})        \\ 
        % &= n_{k-1}(0) (1-1/(D+(k-1)(w-1))) \\
        &= D\cdot \prod_{l=0}^k(1-1/N_l) 
\end{array}
\]

For $d\geq 1$,

\[
\begin{array}{rll}
 m_0(d) &=  n_0(d) = 0                         \\
 n_k(d) &= n_{k-1}(d)+wp_{k-1}(d-1)-p_{k-1}(d) \\
        &= n_{k-1}(d) (1-1/N_{k-1}) + n_{k-1}(d-1) \cdot w/N_{k-1}\\
 n_k(d) -  n_{k-1}(d) (1-1/N_{k-1}) &= n_{k-1}(d-1) \cdot w/N_{k-1}\\
 N_{k-1}/w \cdot n_k(d) + (N_{k-1}/w)(1/N_{k-1}-1) n_{k-1}(d) &= n_{k-1}(d-1) \cdot \\
 N_{k-1}/w \cdot n_k(d) + (1-N_{k-1})/w\cdot n_{k-1}(d) &= n_{k-1}(d-1) \cdot \\
 N_{k-1} \cdot n_k(d) + (1-N_{k-1})\cdot n_{k-1}(d) &= wn_{k-1}(d-1) \cdot \\
 % m_k(d) &= m_{k-1}(d)+ wp_{k-1}(d-1) \\
 %        &= m_0    (d)\quad + w \sum_{l=0}^{k-1} p_l(d-1)\\
 %        &= 0        \qquad \quad   + w \sum_{l=0}^{k-1} n_l(d-1)/N_l\\
\end{array}
\]


% the number of expansion happened in depth $d$ is
% 
% \[
% \begin{array}{rll}
%  x_k(d) = m_k(d+1)/w &= \sum_{l=0}^{k-1} n_l(d)/N_l \\
%  x_k(0) = m_k(0+1)/w &= \sum_{l=0}^{k-1} n_l(0)/N_l \\
% \end{array}
% \]
% 
% now the chance of each node in depth $d$ to be expanded is
% 
% \[
% \begin{array}{rll}
%  c_k(d) = x_k(d)/m_k(d) &= \frac{
%                            \sum_{l=0}^{k-1} n_l(d)/N_l
%                            }{w \sum_{l=0}^{k-1} n_l(d-1)/N_l} \\ 
%  c_k(d-1) = x_k(d-1)/m_k(d-1) &= \frac{
%                            \sum_{l=0}^{k-1} n_l(d-1)/N_l
%                            }{w \sum_{l=0}^{k-1} n_l(d-2)/N_l} \\
% \end{array}
% \]







% and each node in 
% 
% Given $N_k=D+k(w-1)$, the case of $d=0$ is
% 
% \[
% \begin{array}{rl}
%  n_k(0) &= n_{k-1}(0) (1-1/(D+(k-1)(w-1)))                                       \\
%         &= D\cdot \prod_{l=0}^k(1-1/(D+(l-1)(w-1))) 
% \end{array}
% \]
% 
% and this can be shown to be a logarithm. Substitute $k=2^i$:
% 
% \[
% \begin{array}{rll}
%  n_{2^i}(0) &= D\cdot &\prod_{l=0}^{2^i}(1-1/(D+l-1)) \\
%             &= D\cdot &\prod_{0}^{2^{i-1}}(1-1/(D+l-1)) \cdot \\
%             &         &\prod_{2^{i-1}+1}^{2^i}(1-1/(D+l-1)) \\
%             &= n_{2^{i-1}}(0) &\prod_{2^{i-1}+1}^{2^i}(1-1/(D+l-1)) \\
%             &\geq n_{2^{i-1}}(0) &(1-1/(1+2^{i-1}))^{2^{i-1}} \\
%             &\geq n_{2^{i-1}}(0) & 1/e \\
%             &\geq n_{2^0}(0) & 1/e^{i} = n_1(0) / e^i = (D-1)/e^{i}\\
%  % l           &\geq 2^{i-1}+1           &\\
%  % D+l-1       &\geq D+2^{i-1}       &\geq 1+2^{i-1} \\
%  % 1/(D+l-1)   &\leq 1/(D+2^{i-1})   &\leq 1/(1+2^{i-1})\\
%  % 1-1/(D+l-1) &\leq 1-1/(D+2^{i-1}) &\leq 1-1/(1+2^{i-1}) 
% \end{array}
% \]
% 
% Thus
% 
% \[
% \begin{array}{rll}
%  n_{k}(0) &\geq (D-1) /e^{\log_{2} k} \\
%           &= (D-1) \ln 2 / k  \\
%  % l           &\geq 2^{i-1}+1           &\\
%  % D+l-1       &\geq D+2^{i-1}       &\geq 1+2^{i-1} \\
%  % 1/(D+l-1)   &\leq 1/(D+2^{i-1})   &\leq 1/(1+2^{i-1})\\
%  % 1-1/(D+l-1) &\leq 1-1/(D+2^{i-1}) &\leq 1-1/(1+2^{i-1}) 
% \end{array}
% \]
% 
% 
% When $d>1$,
% 
% % \[
% % \begin{array}{lll}
% %  n_k(d)   &= n_{k-1}(d) (1-1/(D+k-1))   &+ n_{k-1}(d-1) \cdot w/(D+k-1) &(d>1) \\
% %  n_k(d-1) &= n_{k-1}(d-1) (1-1/(D+k-1)) &+ n_{k-1}(d-2) \cdot w/(D+k-1) &(d>1) \\
% % \end{array}
% % \]
% 
% Thus we can describe the behavior as 
% 
% % \begin{align*}
% %  \left(
% %  \begin{array}{c}
% %   n(d) \\ n(d-1)
% %  \end{array}
% %   \right) \leftarrow
% %  \left(
% %  \begin{array}{cc}
% %   (1-1/N) & w/N \\
% %   (1-1/N) & w/N 
% %  \end{array}
% %   \right)
% %  \left(
% %  \begin{array}{c}
% %   n(d) \\ n(d-1)
% %  \end{array}
% %   \right)
% % \end{align*}
% 
% 
% The behavior can be described as a markov decision process.
% 
% % Assume that initially we have
% % $D$ nodes in depth 0 and the plateau forms a tree of fixed width 1
% % (indicating star-like topology).
% 
% 
% When we have already expanded $k-1$ nodes in the plateau 
