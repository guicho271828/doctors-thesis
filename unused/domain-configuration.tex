% However we use a Random Order (\ro) criterion, which 
% randomly selects an element from the depth bucket selected by the depth-based tiebreaking.
% This is because the effectiveness of the tiebreaking behavior within a bucket
% can be affected by accidental biases, e.g., names/orders of action schema in the PDDL domain
% definition \cite{vallati2015effective}.
% %Finding the best action ordering is not the scope of this paper.
% Thus, we avoid bias at this level of tiebreaking by using \ro and assess its expected/average
% performance.

% Among \fifo, \lifo and \ro, the natural criterion is Random Order.
% This is because the effectiveness of the third-level tiebreaking behavior
% is affected by the accidental bias in action ordering in the PDDL domain
% definition.  Recent work \cite{vallati2015effective} showed that the
% planner performance is greatly affected by changing and tuning the action ordering
% (and also variable ordering, but it is irrelevant to the tiebreaking behavior). 
% However, finding the best third-level tiebreaking is not the scope of this paper.
% Thus, focusing on \ro and assess its expected/average
% performance is the most reasonable practice to understand the behavior of second-level,
% depth-based tiebreaking.







% is not necessary?

\section{Domain Configuration and Tiebreaking}
%Moved this out of the depth-based evaluation section because we also evaluate [h,fifo] and [h,lifo], and the results show that all are robust wrto action ordering
Recently, \citeauthor{vallati2015effective} showed that the performances of 
satisficing planners were significantly affected by PDDL domain \emph{configurations}, which include the name/ordering of actions, propositions, and objects in the PDDL input file (\citeyear{vallati2015effective}).
% note that vallati et al. did not show that tiebreaking was affected -- they observed that performance of the planners was affected by configuration, and conjectured that the performance variation was due to the effects of configuration on tiebreaking.
They conjectured that performance variations caused
by different domain configurations are due to the impact that the naming/ordering of objects has on tiebreaking.
% in \cite{vallati2015effective} p.7, col2, par#1
In Fast Downward, action names can affect search performance, because FD 
sorts the action schemas according to the dictionary
order of the schema names, which affects the order of applicable ground
actions, which in turn affects the node insertion order into OPEN.

%%%%%%%%%%%%%%%%%%%%%%%%%%%%%%%%%%%%%%%%%%%%%%%%%%%%5

We tested the robustness of the standard $[h,\lifo]$ and $[h,\fifo]$ strategies, as well as $[h,\rd,\ro]$,
with respect to 
biases introduced by domain configuration (action naming) in the PDDL domain definition.
We created 3 different sets of domains in which the
original names of action schema are mangled into random strings. 
%We only reordered the actions because in the FD codebase, the order of propositions has no effect on tiebreaking. --- let's avoid talking about anything other than action names, in case there are other configuration factors.
%  XXXTODO CHECK whether above sentence correctly summarizes the sentence below.
%% In contrast, we continued to use the variable ordering in the original domains
%% because the effect of variable ordering is irrelevant to the tiebreaking
%% criteria.
% %We avoided the effect of action orderings by using the randomized third tiebreaking.
%Each set of action-renamed domains contains all of the benchmark and zerocost domains.   %XX - can be inferred from table
We ran each of the 3 strategies on each
set of mangled domains, three times each with different random seeds,
resulting in 9 runs per strategy (recall that robustness wrto random seed was shown in \refsec{sec:depth-based-evaluation}.)

The results are shown in \reftbl{actionordering-robustness} (We
also included the original 10 runs from  \reftbl{depth}).
We statistically analyzed the results for $[h,\rd,\ro]$ 
to see if any of the 4 sets of domains
significantly outperformed the others.
%In order to test the significance of mean value,
Fligner-Killeen's non-parametric test could not reject the homogeneity of variances
($p=0.75$ for IPC, $p=0.26$ for Zerocost), so
% Since the variances were not significantly different,
we then applied the non-parametric Kruskal-Wallis test,
% to see if there is any difference in the mean values between the original
% population of the sample groups, 
which showed that the mean differences were not significant
($p=0.28$ for IPC, $p=0.44$ for Zerocost),
% We first applied Fligner-Killeen's non-parametric test to see if the sample groups 
% in each set of randomized domains share the same variance, 
% but could not reject the homogeneity of variances ($p=0.74$).
% Since the variances were not significantly different,
% we could then apply the non-parametric Kruskal-Wallis test to see if
% there is any difference in the mean values between the original
% population of the sample groups, which showed that the differences were not significant ($p=0.26$),
i.e., action name mangling did not significantly affect performance.

Thus, in contrast to the results for satisficing search by \cite{vallati2015effective}, 
the effect of action ordering  seems to be relatively weak for cost-optimal search using \astar.
This may be because 
compared to the satisficing, best-first search algorithms evaluated in \cite{vallati2015effective},
the behavior of admissible search is more constrained.
% $[h,\lifo]$, $[h,\fifo]$, and $[h,rd,\ro]$ are
% because of the first-level tiebreaking according to $h$.
%in that all nodes with cost $f$ must be expanded before any nodes with cost larger than $f$.
% this is true for best-first-search in general, by definition.

\begin{table}[tb]
 \setlength{\tabcolsep}{0.2em}
 \centering \relsize{-1}
 \begin{tabular}{|c|c|c||c|}
\hline         
 Domain & $[h,\fifo]$ & $[h,\lifo]$   & \spc{$[h,\rd,\ro]$ \\($n$: number of runs)}    \\
\hline         
 Mangled IPC 1 (1104) &  556 &  564 &  571.7\spm{}0.9 ($n=3$)\\\hline
 Mangled IPC 2 (1104) &  557 &  568 &  571.3\spm{}0.9 ($n=3$)\\\hline
 Mangled IPC 3 (1104) &  557 &  568 &  573.0\spm{}1.6 ($n=3$)\\\hline
 Original IPC (1104) &  558 &  565 &  570.6\spm{}1.5 ($n=10$)\\\hline
 Mangled Zerocost 1 (620) &  256 &  277 &  288.7\spm{}3.7 ($n=3$)\\\hline
 Mangled Zerocost 2 (620) &  256 &  277 &  285.0\spm{}0.8 ($n=3$)\\\hline
 Mangled Zerocost 3 (620) &  256 &  279 &  286.7\spm{}0.9 ($n=3$)\\\hline
 Original Zerocost  (620) &  256 &  279 &  287.2\spm{}2.4 ($n=10$)\\\hline
\end{tabular}

 \caption{Total coverages of $[h,\fifo]$, $[h,\lifo]$
 and $[h,\rd,\ro]$ (with three seeds). Each row represents the original set of
 domains or its three action-mangled variants. The effect
 of action ordering is small enough for $[h,\rd,\ro]$ to
 constantly perform better than the traditional tiebreaking methods.}
 \label{actionordering-robustness}
\end{table}
