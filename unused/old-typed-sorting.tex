
% The diversification based on the depth can be considered as an instance
% of last-resort tiebreaking, since the depth is used \emph{after} the
% sorting strategy selects a set of nodes.
% % 
% We can also view the depth-based diversification as a classification of
% the nodes into type-based buckets with depth as a single key value.
% Thus, we call this framework as \emph{typed-sorting strategy}, a mix of
% sorting strategy and type-based buckets, denoted as $[\mbox{sorting
% strategy}]_{\brackets{\mbox{\small type vector}}}$.
% % 
% For example, \astar using $h$ as the first tiebreaking criteria and
% depth as a diversification method can be expressed as
% $[f,h]_{\brackets{d}}$. 
% % _{\fifo}
% % The reason \fifo is present
% % after $\brackets{d}$ is that when multiple nodes are in the same depth, a
% % single node is selected in a FIFO order.
% % Trivially, $[\ ]_{\brackets{X}}$ is equivalent to $\brackets{X}$ and
% % $[X]_{\brackets{\ }}$ is equivalent to $[X]$.
