Much of the work in the recent search and planning literature has focused on
reducing the size of the effective search space by developing
informative heuristic functions. However, there is a theoretical
limitation in this approach: It is known that the search space grows
exponentially even if we had an optimistic
heuristic function called \emph{almost-perfect} heuristics,
$h^*-c$ \cite{helmert2008good} --- a
theoretical heuristic function which
underestimates the perfect heuristics $h^*$ by a small constant
$c$, and directly computing this value is as difficult as solving the
problem itself.\todo{First 3 paragraphs: scope too broad, move to discussion -- in the intro, let's just focus on ``tiebreaking for A*''}

The fact that even the almost-perfect heuristics result in an exponential
blowup means, therefore, that the key to overcome this limitation is
to develop heuristic-agnostic improvements. This observation has
led to further developments of various techniques such as Symmetry Breaking
\cite{Fox1998,pochter2011exploiting,domshlak2013symmetry} or Partial
Order Reduction \cite{hall2013faster,wehrle2013relative}.

A tiebreaking strategy is
independent from heuristic functions because it does not manipulate the
main heuristics used to guide the search. It is under-investigated
and even ignored in most papers that are published recently.
We break the conventional wisdom on tiebreaking strategies by an
empirical and theoretical analysis of optimal search using \astar.
