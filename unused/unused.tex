Since \astar expands all nodes with $f$-cost less than $f^*$
before expanding any nodes with cost $f^*$, the tiebreaking policy only
affects performance when considering the final frontier (nodes with cost
$f^*$).

Although there has not been much previous \emph{in-depth} work
on tiebreaking policies for admissible search algorithms, 

% dangerous
Through these experiments, we points out a bias in the current set of
benchmark domains and the importance of tiebreaking in
the more practical cost-minimization problems.

% this is not necessary now because we use a deterministic version of
% the depth.
Depth-based tiebreaking is robust, in the sense that it does not rely on 
a particular action ordering in the domain definition.

% this is not necessary either
Although depth is a component of a multi-level tiebreaking strategy, the
depth is the principal factor in determining performance.



% XXX It's doubtful that anybody would believe that machine-level efficiency of LIFO (compared to FIFO) is important in domain-independent planning.
%At the same time, we show that 
%the performance of above \lifo tiebreaking can be explained by its
%depth-first strategy, and the other characteristics of \lifo such as
%machine-level efficiency have little effect on the performance.

%% Above describes the enough detail of the paper structure?

