\section{Background: Tiebreaking Strategies for \astar}

\label{sec:astar-background}

\astar is a standard search algorithm for finding an optimal cost path
on a graph.
Starting from the single initial node, in each iteration, \astar
selects and expands a node $n$ with the lowest $f$-cost in the OPEN
priority queue. The successor nodes are inserted back to OPEN, and $n$
is marked as CLOSED, in order to avoid duplicated evaluations.
\astar returns an optimal solution when $h$ is admissible, i.e., when
$\forall n; h(n) \leq h^*(n)$, where $h^*(n)$ is the optimal distance from $n$ to
the nearest goal.

The best-first order of the expansion is a key to guaranteeing solution optimality. 
The first solution found by the algorithm is guaranteed to have the optimal cost $f=f^*$ because 
all nodes with $f(n) < k$ are already expanded when it starts expanding
the nodes with $f(n) = k$.
Thus, the \emph{effective search space of \astar} is the set of nodes with 
$f(n) \leq f^*$: \astar expands all nodes with $f(n) < f^*$, then
expands \emph{some} of the nodes with $f(n) = f^*$, and
never expands the nodes with $f(n) > f^*$.

If there are multiple nodes with the same $f$-cost, \astar
must implement some tiebreaking strategy (either
explicitly or implicitly) which selects from among these nodes.
The early literature on heuristic search seems to have been mostly agnostic regarding tiebreaking.
The original \astar paper, as well as Nilsson's subsequent textbook 
states: ``Select the open node $n$ whose value $f$
is smallest. Resolve ties arbitrarily, but always in favor of any [goal
node]'' \cite[p.102 Step 2]{hart1968formal}, \cite[p.69]{Nilsson71}.
% Although it is possible to interpret this to imply $h$-based tiebreaking
% since goal nodes are the special case where $h=0$,
% they make no further mention of tiebreaking.
Pearl's textbook on heuristic search specifies that best-first search should ``break ties arbitrarily'' (\citeyear{pearl1984heuristics}, p.48, Step 3), and does not specifically mention tiebreaking for \astar.
To the best of our knowledge, the first explicit mention of a tiebreaking strategy that considers node generation order is by Korf in his analysis of IDA*: ``If \astar employs the tiebreaking rule of 'most-recently generated', it must also expand the same nodes [as IDA*]'', i.e., a \lifo ordering.

In recent years, tiebreaking according to $h$-values has become ``folklore'' in the search community.
\citeauthor{hansen2007anytime} state that ``[i]t is well-known 
that \astar achieves best performance when it breaks ties
in favor of nodes with least h-cost'' \cite{hansen2007anytime}.
\citeauthor{holte2010common} writes ``\astar breaks ties in favor
of larger $g$-values, as is most often done'' \cite[note that since $f=g+h$,
preferring large $g$ is equivalent to preferring smaller $h$]{holte2010common}.
\shortciteauthor{felner2011inconsistent} also assume ``ties are broken in
favor of low h-values'' in describing Bidirectional Pathmax for \astar \citeyear{felner2011inconsistent}.
In their detailed survey/tutorial on efficient \astar implementations,
\shortciteauthor{burns2012implementing} \citeyear{burns2012implementing}
also break ties ``preferring high $g$'' (equivalent to low $h$).
%% this could be moved to later analysis
% They further write: ``The reasoning is that the goal can be found more
% quickly in the final $f$ layer of search''.
Thus, tiebreaking according to $h$-values appears
to be ubiquitous in practice while,
to our knowledge, an in-depth experimental analysis of tiebreaking strategies for \astar is lacking in the literature.

Although the standard practice of tiebreaking according to $h$ might be
sufficient in some domains, further levels of tiebreaking (explicit or
implicit) are required if multiple nodes have the same $f$ as well as
the same $h$ values. To date, the effect of such \emph{default}
tiebreaking was not investigated in depth.
% 
For example, the survey of efficient \astar implementation techniques in
\cite{burns2012implementing} did not explicitly mention the default
tiebreaking, while their library
code\footnote{https://github.com/eaburns/search} uses \lifo
default tiebreaking.
% 
It first breaks ties according to $h$, and then
breaks remaining ties according to a \lifo criterion (most recently
generated nodes first), i.e., $[f,h,\lifo]$.
% 
Although not documented, their choice of a \lifo 2nd-level tiebreaking
criterion appears to be a natural consequence of the fact it can be
trivially and efficiently implemented in their two-level bucket (vector)
implementation of OPEN.
% 
In contrast, the current implementation of the \sota \astar based planner Fast
Downward \cite{Helmert2006}, as well as the work by \cite{RogerH10} uses
a $[f,h,\fifo]$ tiebreaking strategy.
% 
Although we could not find an explanation in the publication nor in the
website, this choice is most likely due to their use of alternating OPEN
lists, in which case the \fifo second-level criterion serves to provide a
limited form of fairness.
% 
Such lack of explanation suggests that this topic has long been out
of focus of the heuristic search literature.
